\expandafter\ifx\csname readornot\endcsname\relax
  \documentclass[uplatex]{jsarticle}
  \usepackage{octopus}
  \usepackage{url}

  \renewcommand{\proofname}{\textsf{証明}}
  \renewcommand{\postpartname}{章}
  \renewcommand{\thesection}{\thepart.\arabic{section}}
  \renewcommand{\thepart}{\arabic{part}}
  \makeatletter\renewcommand{\theequation}{\thesection.\arabic{equation}}\@addtoreset{equation}{section}\makeatother

  \newcommand{\octopuspart}[1]{\newpage\part{#1}\setcounter{section}{0}\vspace{3\baselineskip}}

  \renewcommand{\restriction}[2]{\left. #1 \right|_{#2}}
  \DeclareMathOperator{\dcup}{\dot{\cup}}
  \DeclareMathOperator{\conv}{conv}
  \DeclareMathOperator{\Image}{Im}
  \DeclareMathOperator{\Kernel}{Ker}
  \begin{document}
\fi

\section{テンソルの定義}
\midashi{ベクトル空間}
\begin{teigi}[ベクトル空間]
  略.
\end{teigi}

\begin{rei}
  $\mathbb{R}^{n}$は$\mathbb{R}$上のベクトル空間。
\end{rei}

\midashi{注意.}環においてベクトル空間に対応する概念は\nw{加群 (module)}という。

\begin{teigi}[基底]
  $B \subseteq V$が\nw{基底} $\iff$ 任意の$v \in V$を${\displaystyle v = \sum_{u \in B: \text{有限和}} \alpha_{u} u}$と一意的に表される。
\end{teigi}

$B \subseteq V$が基底であることは,任意の$v \in V$が${\displaystyle v = \sum_{u \in B: \text{有限和}} \alpha_{u} u}$と表すことができて,任意の有限部分集合$B^{\prime}$が一次独立であることと同値。

\begin{hodai}
  ベクトル空間には基底が存在する\footnote{証明にはZornの補題を用いるがここでは省略する。}。
\end{hodai}

ベクトル空間の基底が有限個の元からなるとき,その個数は基底の取り方によらずに等しいので,次のように次元を定義することができる。
\begin{teigi}
  ベクトル空間$V$が$n$次元 $\iff$ $n$個の元からなる基底が存在\par
  ベクトル空間$V$が無限次元 $\iff$ 任意の$n \in \mathbb{N}$に対してベクトル空間$V$が$n$次元でない
\end{teigi}
ベクトル空間$V$の次元が$n$であることを$\dim V = n$と書く。$V$が無限次元であることは$\dim V = \infty$と書く。

\begin{rei}
  $\mathbb{R}^{n}$は$n$次元。
\end{rei}

\begin{rei}
  $\mathbb{R}[x]$を実係数1変数多項式全体の集合とする。これは$\mathbb{R}$ベクトル空間になる。このときの基底は$B = \sets{1, x, x^{2}, \dots}$であり,したがって$\mathbb{R}[x]$は$\mathbb{R}$上の無限次元ベクトル空間である。
\end{rei}

\midashi{注意.} $L^{2}[0,1]$を$[0,1]$上で2乗可積分な関数全体の集合とすると,これは$\mathbb{R}$ベクトル空間である。このとき,任意の$f \in L^{2}[0,1]$は
\begin{align}
  f(x) = \sum_{k=1}^{\infty} \alpha_{k} \sin (2 \pi k x) + \sum_{k=0}^{\infty} \beta_{k} \cos (2 \pi k x)
\end{align}
と一意的に書けるが,$\sets{\sin (2 \pi k x) | k \in \mathbb{N}_{>0}} \cup \sets{\cos (2 \pi k x) | k \in \mathbb{N}}$は上の意味での「基底」ではないことに注意。この基底のことは「\textbf{代数基底}」や「\textbf{Hamel基底}」と呼んで区別する\footnote{\url{https://ja.wikipedia.org/wiki/基底_(線型代数学)}}。

以降は$V$を$n$次元ベクトル空間とする。

\sukima\midashi{双対空間}

\begin{teigi}[線形汎関数]
  $f\colon V\to K$が線形汎関数 $\defines$ 任意の$\alpha,\beta \in K, u,v \in V$に対して
  \begin{align*}
    f(\alpha u+\beta v) = \alpha f(u) + \beta f(v)
  \end{align*}
\end{teigi}
\begin{teigi}[双対空間]
  $V$の\textbf{双対空間}$V^\ast$を,$V^\ast := \{ f\colon V\to K \mid fは線形汎関数 \}$で定義する.
\end{teigi}
\begin{teigi}[双対基底]
  $\{e_1,\cdots,e_n\}$を$V$の基底とする.このとき,$V^\ast$の基底$\{e^1,\cdots,e^n\}$を
  \begin{align*}
    e^j(e_i) = \delta_i^j := \begin{cases}
      1 & i=j \\
      0 & i\neq j
    \end{cases}
  \end{align*}
  を満たす線形汎関数として定義する.
\end{teigi}
\begin{prop}
  $\{e^1,\cdots,e^n\}$は$V^\ast$の基底.
\end{prop}
\begin{proof}
  $f\in V^\ast$に対して
  \begin{align*}
    f = \sum_i f(e_i) e^i
  \end{align*}
  と表すことが出来る.また,
  \begin{align*}
    \sum_i \beta^i e^i = 0
  \end{align*}
  が成り立つとき(右辺の0は零写像を表している),
  \begin{align*}
    \left(\sum_i \beta^i e^i \right) (e_j) & = \beta_j \\
    0(e_j) & = 0 
  \end{align*}
  よって$\beta_j = 0$なので$\{e_1,\cdots,e_n\}$は基底である.
\end{proof}
\begin{corr}
  $\dim V = n$ならば$\dim V^\ast = n$.
\end{corr}
\begin{prop}
  $V^{\ast \ast} \simeq V$
\end{prop}
\begin{proof}
  $v\in V$を$f\in V^\ast$に対して$f\mapsto f(v)$という対応を考えることで,$v$を写像$v\colon V^\ast \to K$とみなす.このとき,
  \begin{align*}
    v(\alpha f + \beta g) & = (\alpha f + \beta g)(v) \\
    & = \alpha f(v) + \beta g(v) \\
    & = \alpha v(f) + \beta v(g)
  \end{align*}
  より$v$は$V^\ast$上の線形汎関数.したがって$V\subseteq V^{\ast \ast}$.また$\dim V = \dim V^\ast = \dim V^{\ast \ast}$より,$V \simeq V^{\ast \ast}$.
\end{proof}

\sukima\midashi{テンソル}

以下では$U,V$を体$K$上の有限次元ベクトル空間とする.
\begin{teigi}[双線形写像]
  $\Phi\colon U\times V\to K$が任意の$\alpha,\alpha',\beta,\beta'\in K,u,u'\in U ,v,v'\in V$に対して
  \begin{align*}
    \Phi(\alpha u + \alpha' u',v) & = \alpha \Phi(u,v) + \alpha' \Phi(u',v) \\
    \Phi(u,\beta v + \beta' v) & = \beta \Phi(u,v) + \beta' \Phi(u,v') 
  \end{align*}
  を満たすとき,$\Phi$を\textbf{双線形写像}という.
\end{teigi}

\begin{rei}
  $U = \mathbb{R}^n , V = \mathbb{R}^m, A \in \mathbb{R}^{n\times m}$とする.$x \in U , y \in V$に対して
  \begin{align*}
    (x,y) \mapsto x^\top A y
  \end{align*}
  という対応は双線形写像になっている(この双線形写像を2次形式とよぶ).
\end{rei}
\begin{remark}
  $\Phi\colon U\times V\to K$を双線形写像とする.$U$の基底$\{e_1,\cdots,e_n\}$,$V$の基底$\{f_1,\cdots,f_m\}$としたとき,$u = \sum_{i} \alpha^i e_i \in U , v = \sum_j \beta^j  f_j \in V$に対して$\Phi(u,v)$は
  \begin{align*}
    \Phi\left(\sum_{i} \alpha^i e_i,\sum_j \beta^j f_j\right) = \sum_{i,j} \alpha^i \beta^j \Phi(e_i,f_j)
  \end{align*}
  となり,二次形式とみなすことができる.
\end{remark}
\begin{teigi}[多重線形写像]
  $V_i\ (i=1,\cdot,k)$を体$K$上の線形空間とする.$\Phi\colon V_1\times \cdots V_k \to K$で各$i\  ( i=1,\cdots,k)$で線形な写像を多重線形写像という.
\end{teigi}

\sukima\midashi{テンソル積の定義(I)}
\begin{teigi}[テンソル積]
  $U$と$V$のテンソル積$U\otimes V$を
  \begin{align*}
    U\otimes V := \{ \Phi \colon U^\ast \times V^\ast \to K \mid \Phi\mbox{は双線形写像} \}
  \end{align*}
  で定義する.

  また,$u\in U$と$v\in V$のテンソル積 $u\otimes v\in U\times V$を
  \begin{align*}
    (u\otimes v)(f,g) = f(u)\cdot g(v)
  \end{align*}
  で定義する.これは双線形写像になっている.
\end{teigi}
\begin{remark}
  $U\otimes V \not \simeq \{ u\otimes v \mid u \in U, v \in V\}$である.
\end{remark}
\begin{prop}
  $\{e_1,\cdots,e_n\}$を$U$の基底,$\{f_1,\cdots,f_m\}$を$V$の基底とする.このとき,$\{e_i\otimes f_j \mid i=1,\cdots,n,j=1,\cdots,m \}$は$U\otimes V$の基底.
\end{prop}
\begin{proof}
  $\{e^1,\cdots,e^n\}$を$U^\ast$の基底,$\{f^1,\cdots,f^m\}$を$V^\ast$の基底とする.
  \begin{align*}
    \sum_{i,j} \alpha^{i,j} e_i \otimes f_j = 0
  \end{align*}
  のとき,
  \begin{align*}
    (e_i\otimes f_j)(e^\nu ,f^\mu) = e_i(e^\nu)\cdot f_j(f^\mu) 
    = \delta_i^\nu \delta_j^\mu = \begin{cases}
      1 & (i,j) = (\nu,\mu) \\
      0 & \mathrm{otherwise}
    \end{cases}
  \end{align*}
  より,
  \begin{align*}
    \left(\sum_{i,j} \alpha^{i,j} e_i \otimes f_j \right) (e^\nu,f^\mu) = 0
  \end{align*}
  よって,$\alpha^{i j} = 0$.
  また,$\Phi \colon U^\ast \times V^\ast \to K$が双線形写像のとき,
  \begin{align*}
    \Phi = \sum_{i,j} \Phi(e^i,f^j) e_i \otimes f_j
  \end{align*}
  と表すことが出来る.
\end{proof}
\begin{corr}
  $\dim U\otimes V = \dim U \times \dim V$
\end{corr}

\sukima\midashi{テンソル積の定義(II)}
\begin{align*}
  U\otimes V := \left\{ \sum_{\mbox{有限和}} \alpha_i u_i \otimes v_i \relmiddle| u_i \in U, v_i \in V, \alpha_i \in K \right\} \mathbin{/} \sim
\end{align*}
ただし$\sim$は
\begin{align*}
  (u+u')\otimes v & \sim u \otimes v + u'\otimes v \\
  u\otimes (v+v') & \sim u\otimes v + u\otimes v' \\
  \alpha(u\otimes v) & \sim (\alpha u)\otimes v \sim u \otimes (\alpha v)
\end{align*}
を満たす同値関係.

\expandafter\ifx\csname readornot\endcsname\relax
  \end{document}
\fi
