\expandafter\ifx\csname readornot\endcsname\relax
  \documentclass[uplatex]{jsarticle}    
  \usepackage{octopus}
  \usepackage{url}
  %%%% コマンド定義専用のtexファイル
\renewcommand{\postpartname}{章}
\renewcommand{\thepart}{\arabic{part}}
\renewcommand{\thesection}{\thepart.\arabic{section}}
\makeatletter\renewcommand{\theequation}{\thesection.\arabic{equation}}\@addtoreset{equation}{section}\makeatother
\newcommand{\octopuspart}[1]{\newpage\part{#1}\setcounter{section}{0}\vspace{3\baselineskip}}
\renewcommand{\restriction}[2]{\left. #1 \right|_{#2}}
\DeclareMathOperator{\dcup}{\dot{\cup}}
\DeclareMathOperator{\conv}{conv}
\DeclareMathOperator{\Image}{Im}
\DeclareMathOperator{\Kernel}{Ker}
\DeclareMathOperator{\diag}{diag}
\DeclareMathOperator{\rank}{rank}
\DeclareMathOperator{\sgn}{sgn}
\DeclareMathOperator{\rot}{rot}
  \usetikzlibrary{calc}
  \begin{document}
\fi

\section{被覆空間}
\begin{teigi}[被覆空間]
  $E,X$を位相空間,$p\colon E \to X$を全射な連続関数とする.任意の$x\in X$について$x$の開近傍$U$が存在し,$p^{-1}(U)$が$E$の互いに交わらない開集合$V_\alpha\ (\alpha \in \Lambda)$で
  \begin{align*}
    p^{-1}(U) = \bigcup_{\alpha \in \Lambda} V_\alpha
  \end{align*}
  と表されているとする.このとき,$\restriction{p}{V_\alpha} \colon V_\alpha \to U$が同相写像であるとき,$p$を\textbf{被覆写像(covering map)},$E$を$X$の\textbf{被覆空間}という.
\end{teigi}

\begin{rei}
  $\mathbb{R}$は$S^1$の被覆空間.
\end{rei}
\begin{proof}
  $p\colon \mathbb{R} \to S^1$として$p(x) = (\cos 2\pi x,\sin 2\pi x)$を考えると,これは全射な連続関数になっている.
\end{proof}

\begin{teigi}[リフト]
  $p\colon E\to X$,$f\colon Y\to X$をそれぞれ連続写像とする.このとき$f$のリフト$\tilde{f}\colon Y\to E$を$p\circ \tilde{f} = f$を満たすものとして定義する.つまり$\tilde{f}$は以下を可換にする.
  \begin{align*}
    \begin{diagram}
      \node{ } \node{E} \arrow{s,r}{p}\\
      \node{Y} \arrow{ne,t}{\tilde{f}} \arrow{e,b}{f} \node{X}
    \end{diagram}
  \end{align*}
\end{teigi}

\begin{teiri}
  $f\colon \left[0,1\right]\to X$をパス,$f(0)=x_0$とする.このとき,$\tilde{x}_0 \in p^{-1}(\{x_0\})$に対して$f$のリフト$\tilde{f}\colon \left[0,1\right]\to E,\tilde{f}(0)=\tilde{x}_0$が一意に存在する.
\end{teiri}
\begin{proof}
  ルベーグ数の補題より,ある開被覆に対して$\delta > 0$が存在し,任意の$x \in \left[0,1\right]$に対してある$U_i \in\mathcal{C}$が存在して$\mathcal{N}(x;\delta) \subseteq U_i$が成り立つ.これより,$x_0$を含む$U_0$の逆像の連結成分$V_0$は一意に決まり,$\tilde{f}\colon \left[s_0,s_1\right]\to V_0$で$p\circ \tilde{f} = \restriction{f}{\left[s_0,s_1\right]}$となる$\tilde{f}$が一意に定まる.これを$\left[s_1,s_2\right],\left[s_2,s_3\right],\cdots$に対して行えばよい.
\end{proof}

\begin{teiri}
  $f_t \colon \left[0,1\right]\to X$をパスのホモトピー,$f_t(0)=x_0$とする.このとき,$\tilde{x}_0 \in p^{-1}(\{x_0\})$に対して$f_t$のリフト$\tilde{f_t}\colon \left[0,1\right]\to E,\tilde{f_t}(0)=\tilde{x}_0$が一意に存在する.
\end{teiri}
\begin{proof}
  上と同様の議論を$\left[0,1\right]\times \left[0,1\right]$に対して行う.
\end{proof}
\begin{teiri}
  $p\colon E\to X$を被覆写像,$\phi\colon \pi_1(X,x_0)\to p^{-1}(x_0)$を
  \begin{align*}
    \phi(\left[f\right]) := \tilde{f}(1)
  \end{align*}
  で定義する.ただし$\tilde{f}$は$f$のリフトで$\tilde{f}(0)=\tilde{x_0}$となるものとする.このとき,$E$が弧状連結なら$\phi$は全射,$E$が単連結なら$\phi$は全単射となる.
\end{teiri}
\begin{proof}
  任意の$z\in p^{-1}(x_0)$に対し$\tilde{x_0}$と$z$を結ぶパス$h$をとれば$\left[p\circ h\right] \in \pi_1(X,x_0)$で$\phi(\left[p\circ h\right]) = h(1) = z$となる.$E$が単連結のとき,$\phi(\left[f\right]) = \phi(\left[f'\right])=z\in p^{-1}(x_0)$.$\tilde{f},\tilde{f'}$は$\tilde{x_0}$と$z$を結ぶパスで,単連結性より$\tilde{f}\simeq \tilde{f'}$.よって$p\cdot \tilde{f} \simeq p\cdot \tilde{f'}$であり,$f\simeq f'$.
\end{proof}
\begin{teiri}
  $\pi_1(S^1) \simeq \mathbb{Z}$
\end{teiri}
\begin{proof}
  $p\colon \mathbb{R}\to S^1 , p(x) = (\cos 2\pi x,\sin 2\pi x)$とし,$x_0=(1,0)$を基点とする基本群を考える.$\mathbb{R}$は単連結であるから,$\phi \colon \pi_1(S^1)\to \mathbb{Z}$は全単射.また,
  \begin{align*}
    \phi(\left[f\right]\cdot \left[g\right]) = \phi(\left[f\right]) + \phi(\left[g\right])
  \end{align*}
  が成り立つので,$\phi$は準同型.よって$\pi_1(S^1)\simeq \mathbb{Z}$
\end{proof}
\begin{midashi}{射影平面$P^2$の基本群}

  $S^2$の元$x$に対して,同値関係$\sim$を$x\sim y \Leftrightarrow y = -x$で定める.この同値関係による同値類を$\left[x\right]$と書く.すると,$P^2 \simeq S^2 \mathbin{/} \sim$と考えることが出来る.ここで$p\colon S^2\to P^2$として$p(x) = \left[x\right]$を考えると,これは被覆写像となる.ここで,$S^2$は単連結であるため,上で定義される$\phi\colon \pi_1(P^2,x_0)\to p^{-1}(x_0)$は全単射.$\left|p^{-1}(x_0)\right| = 2$であるため$\left|\pi_1(P^2,x_0)\right| = 2$.$\pi_1(P^2,x_0)$は群であるので$\pi_1(P^2,x_0) = \pi_1(P^2) \simeq \mathbb{Z}/2\mathbb{Z}$.
\end{midashi}

\expandafter\ifx\csname readornot\endcsname\relax
  \end{document}
\fi
