\expandafter\ifx\csname readornot\endcsname\relax
  \documentclass[uplatex]{jsarticle}
  \usepackage{octopus}
  \usepackage{url}

  \renewcommand{\proofname}{\textsf{証明}}
  \renewcommand{\postpartname}{章}
  \renewcommand{\thesection}{\thepart.\arabic{section}}
  \renewcommand{\thepart}{\arabic{part}}
  \renewcommand{\restriction}[2]{\left. #1 \right|_{#2}}
  \makeatletter\renewcommand{\theequation}{\thesection.\arabic{equation}}\@addtoreset{equation}{section}\makeatother

  \newcommand{\octopuspart}[1]{\newpage\part{#1}\setcounter{section}{0}\vspace{3\baselineskip}}

  \DeclareMathOperator{\dcup}{\dot{\cup}}
  % sub-resume-7.tex
  \begin{document}
\fi

\section{被覆空間}
\begin{teigi}[被覆空間]
  $E,X$を位相空間,$p\colon E \to X$を全射な連続関数とする.任意の$x\in X$について$x$の開近傍$U$が存在し,$p^{-1}(U)$が$E$の互いに交わらない開集合$V_\alpha$で表されているとする.このとき,$\restriction{p}{V_\alpha} \colon V_\alpha \to U$が同相写像であるとき,$p$を被覆写像(covering map),$E$を$X$の被覆空間という.
\end{teigi}

\begin{rei}
  $\mathbb{R}$は$S^1$の被覆空間.
\end{rei}

\begin{teigi}[リフト]
  $p\colon E\to X$,$f\colon Y\to X$をそれぞれ連続写像とする.このとき$f$のリフト$\tilde{f}\colon Y\to E$を$p\circ \tilde{f} = f$を満たすものとして定義する.つまり$\tilde{f}$は以下を可換にする.
  \begin{align*}
    \begin{diagram}
      \node{ } \node{E} \arrow{s,r}{p}\\
      \node{Y} \arrow{ne,t}{\tilde{f}} \arrow{e,b}{f} \node{X}
    \end{diagram}
  \end{align*}
\end{teigi}

\begin{teiri}
  $f\colon \left[0,1\right]\to X$をパス,$f(0)=x_0$とする.このとき,$\tilde{x}_0 \in p^{-1}(\{x_0\})$に対して$f$のリフト$\tilde{f}\colon \left[0,1\right]\to E,\tilde{f}(0)=\tilde{x}_0$が一意に存在する.
\end{teiri}
\begin{proof}
  自明ではない(あとで書く)
\end{proof}

\begin{teiri}
  $f_t \colon \left[0,1\right]\to X$をパスのホモトピー,$f_t(0)=x_0$とする.このとき,$\tilde{x}_0 \in p^{-1}(\{x_0\})$に対して$f_t$のリフト$\tilde{f_t}\colon \left[0,1\right]\to E,\tilde{f_t}(0)=\tilde{x}_0$が一意に存在する.
\end{teiri}
\begin{proof}
  自明ではない
\end{proof}
\begin{teiri}
  $p\colon E\to X$を被覆写像,とする.このとき,$E$が弧状連結なら$\phi$は全射,$E$が単連結なら$\phi$は全単射となる.
\end{teiri}
\begin{proof}
  自明ではない
\end{proof}
\begin{teiri}
  $\pi_1(S^1) \simeq \mathbb{Z}$
\end{teiri}

\expandafter\ifx\csname readornot\endcsname\relax
  \end{document}
\fi
