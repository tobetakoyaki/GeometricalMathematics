\expandafter\ifx\csname readornot\endcsname\relax
  \documentclass[uplatex]{jsarticle}
  \usepackage{octopus}
  \usepackage{url}

  \renewcommand{\proofname}{\textsf{証明}}
  \renewcommand{\postpartname}{章}
  \renewcommand{\thesection}{\thepart.\arabic{section}}
  \renewcommand{\thepart}{\arabic{part}}
  \makeatletter\renewcommand{\theequation}{\thesection.\arabic{equation}}\@addtoreset{equation}{section}\makeatother

  \newcommand{\octopuspart}[1]{\newpage\part{#1}\setcounter{section}{0}\vspace{3\baselineskip}}

  \DeclareMathOperator{\dcup}{\dot{\cup}}
  \begin{document}
\fi

\section{位相空間}

$X$を非空な集合とする。$X$は「\nw{位相}」(topology)を入れて「空間」にする。
「位相」を入れるとは,$X$の開集合族を指定することである。

\begin{teigi}
  以下を満たす開集合系$\mathcal{O} \subseteq 2^{X}$を\nw{位相}という。

  \midashi{O1. } $X \in \mathcal{O}$,$\emptyset \in \mathcal{O}$

  \midashi{O2. } $O_{1}, O_{2}, \cdots, O_{k} \in \mathcal{O} \Longrightarrow O_{1} \cap O_{2} \cap \cdots \cap O_{k} \in \mathcal{O}$

  \midashi{O3. } 任意の集合$\Lambda$に対して$O_{\lambda} \in \mathcal{O}$($\forall \lambda \in \Lambda$) ${\displaystyle \Longrightarrow \bigcup_{\lambda \in \Lambda} O_{\lambda} \in \mathcal{O}}$

  $\mathcal{O}$の元を\nw{開集合}という。
\end{teigi}


\expandafter\ifx\csname readornot\endcsname\relax
  \end{document}
\fi