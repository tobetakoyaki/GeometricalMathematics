\expandafter\ifx\csname readornot\endcsname\relax
  \documentclass[uplatex]{jsarticle}    
  \usepackage{octopus}
  \usepackage{url}
  %%%% コマンド定義専用のtexファイル
\renewcommand{\postpartname}{章}
\renewcommand{\thepart}{\arabic{part}}
\renewcommand{\thesection}{\thepart.\arabic{section}}
\makeatletter\renewcommand{\theequation}{\thesection.\arabic{equation}}\@addtoreset{equation}{section}\makeatother
\newcommand{\octopuspart}[1]{\newpage\part{#1}\setcounter{section}{0}\vspace{3\baselineskip}}
\renewcommand{\restriction}[2]{\left. #1 \right|_{#2}}
\DeclareMathOperator{\dcup}{\dot{\cup}}
\DeclareMathOperator{\conv}{conv}
\DeclareMathOperator{\Image}{Im}
\DeclareMathOperator{\Kernel}{Ker}
\DeclareMathOperator{\diag}{diag}
\DeclareMathOperator{\rank}{rank}
\DeclareMathOperator{\sgn}{sgn}
\DeclareMathOperator{\rot}{rot}
  \usetikzlibrary{calc}
  \begin{document}
\fi

\section{ホモトピー}

\midashi{2つの空間が「同じ形」をしているとはどういうことか?}

\noindent $\xrightarrow{\text{ans}}$位相空間$X,Y$が位相同型 $\defines$ $\exists f \colon X \longrightarrow Y$:連続全単射,$f^{-1}$:連続

例:ドーナツとコーヒーカップ

\noindent $\xrightarrow{\text{ans}}$連続変形で移り合う

例:太字のAと細字のAと円 %図

例:ディスクと一点(明らかにこの2つは位相同型ではない)

このような「空間の連続変形」を定式化したい

\sukima \midashi{\large 変形レトラクション}

$X$:位相空間,$A \subseteq X$とする。

\begin{teigi}[変形レトラクション]
  $X$から$A$への\nw{変形レトラクション(deformation retraction)}とは,
  次の条件を満たす$\sets{f_{t} \colon X \longrightarrow X}_{t \in [0,1]}$のこととする。:
  \begin{itemize}
    \vspace{-0.5\baselineskip}
    \item $f_{0} = \mathrm{id}_{X}$
    \item $f_{1}(X) = A$
    \item $\restriction{f_{t}}{A} = \mathrm{id}_{A}$($\forall t \in [0,1]$)
    \item $F \colon \mapdef{X \times [0,1]}{X}{(x,t)}{f_{t}(x)}$は連続写像
  \end{itemize}
\end{teigi}

\begin{rei}
  $X \subseteq \mathbb{R}^{n}$:凸集合 i.e. $\forall x,y \in X, \quad \forall \lambda \in [0,1], \quad (1-\lambda) x + \lambda y \in X$とする。
  $z \in X$として
  \begin{equation}
    f_{t} \colon \mapdef{X}{X}{x}{(1-t) x + t z} ,\qquad(t \in [0,1])
  \end{equation}
  は,$X$から$\sets{z}$への変形レトラクションである。(凸性のおかげで写像の行き先が常に$X$に入る)
\end{rei}

\begin{rei}
  $X = \mathbb{R}^{n} \setminus \sets{0}$とする。
  \begin{equation}
    f_{t} \colon \mapdef{X}{X}{x}{(1-t)x+t\dfrac{x}{\left\| x \right\|}}
  \end{equation}
  は$\mathbb{R}^{n} \setminus \sets{0}$から$S^{n-1}$への変形レトラクションである。
\end{rei}

\midashi{問題.} $A \subseteq \mathbb{R}^{n}$:閉凸集合として,$\mathbb{R}^{n}$から$A$への変形レトラクションを構成せよ。
({\bf ヒント}:近接点写像)

\sukima \midashi{\large ホモトピー}

$X,Y$:位相空間

\begin{teigi}[ホモトピー]
  $\sets{f_{t}}$:\nw{ホモトピー(homotopy)} $\defines$ 次の条件を満たす$\sets{f_{t} \colon X \longrightarrow Y}_{t \in [0,1]}$のこと:
  \begin{itemize}
    \vspace{-0.5\baselineskip}
    \item $F \colon \mapdef{X \times [0,1]}{Y}{(x,t)}{f_{t}(x)}$が連続
  \end{itemize}

  また,$g \colon X \longrightarrow Y$と$h \colon X \longrightarrow Y$をつなぐホモトピー$\sets{f_{t}}$とは
  上の$\sets{f_{t}}$に関する条件に加えて,$f_{0} = g$,$f_{1} = h$を満たすもののことを指す。
  このとき,$g$と$h$は\nw{ホモトープ(homotope)}または\nw{ホモトピック(homotopic)}といい,$g \simeq h$と書く。
\end{teigi}

\begin{hodai}
  ホモトープの関係$\simeq$は同値関係
\end{hodai}

\begin{proof}
  \begin{itemize}
    \vspace{-0.5\baselineskip}
    \item $g \simeq g$は$f_{t} \equiv g$とすればよい。
    \item $g \simeq h$とすると,このときのホモトピー$f_{t}$に対して$f_{1-t}$を考えれば$h \simeq g$である。
    \item $f \simeq g$かつ$g \simeq h$とする。$f$と$g$を結ぶホモトピーを$\sets{f_{t}}$,$g$と$h$を結ぶホモトピーを$\sets{g_{t}}$とする。
    このとき,$f$と$h$の間に写像の族$\sets{\overline{f}_{t}}$を
    \begin{equation}
      \overline{f}_{t} = \begin{cases}
        f_{2t} & (0 \le t \le 1/2) \\
        g_{2t-1} & (1/2 \le t \le 1)
      \end{cases}
    \end{equation}
    で定める。すると,$\overline{f}_{0} = f$であり$\overline{f}_{1} = h$である。
    このとき,
    \begin{equation}
      \overline{F} \colon \mapdef{X \times [0,1]}{Y}{(x,t)}{\overline{f}_{t}(x)}
    \end{equation}
    は$X \times [0,1/2]$上と$X \times [1/2,1]$上のそれぞれで連続であり,この2つの集合はともに閉集合である。
    \rref{補題}{close.continuous}から, $f$は$X \times [0,1]$上でも連続であり,$f$と$h$を結ぶホモトピーになっている。
  \end{itemize}
\end{proof}

上の補題の証明に用いた事実を明記して証明する。:

\begin{hodai}
  \label{close.continuous}
  $X = A \cup B$,$A,B$:閉とする。$f \colon X \longrightarrow Y$に対して
  $\restriction{f}{A},\restriction{f}{B}$がそれぞれ連続であるならば$f$は$X$上で連続である。
\end{hodai}

\begin{proof}
  $F \subseteq Y$:閉とすると,$f^{-1} (F) = \restriction{f^{-1}}{A}(F) \cup \restriction{f^{-1}}{B}(F)$は有限個の閉集合のunionなので閉集合である。
\end{proof}

\begin{teigi}[ホモトピー類]
  ホモトピーの同値関係による同値類を\nw{ホモトピー類}という。
\end{teigi}

\sukima \midashi{注意.} 変形レトラクションは,恒等写像$\mathrm{id}_{X}$とレトラクション$r$を結ぶホモトピーである。
ここで,\nw{レトラクション}とは次を満たす写像$r \colon X \longrightarrow X$のことである。
\begin{itemize}
  \vspace{-0.5\baselineskip}
  \item $r(X) = A$
  \item $\restriction{r}{A} = \mathrm{id}_{A}$
\end{itemize}

\sukima \midashi{注意.} $Y \subseteq \mathbb{R}^{n}$のとき,$g \colon X \longrightarrow Y$,$h \colon X \longrightarrow Y$を結ぶホモトピーとして,
$(1-t)g+th$(これを\nw{線形ホモトピー}という)を作りたくなるが,
例えば凸性が保証されていないときには$(1-t)g(x) + t h(x) \notin Y$となる可能性があるので気をつける必要がある。

\sukima \midashi{注意.} もし$Y$が弧状連結であれば,$g(x)$と$h(x)$を結ぶパスに沿っていつもホモトピーを作ることができるのだろうか。
例えば$X = Y = S^{1}$として,$\mathrm{id}_{S^{1}}$と$S^{1}$上の一点へのレトラクションを結ぶホモトピーは{\bf できなさそう}である。

\sukima \midashi{\large ホモトピー同値}

\begin{teigi}[ホモトピー同値]
  位相空間$X,Y$が\nw{ホモトピー同値} $\defines$ $\exists f \colon X \longrightarrow Y$:連続,$\exists g \colon Y \longrightarrow X$:連続,
  $f \circ g \simeq \mathrm{id}_{Y}$,$g \circ f \simeq \mathrm{id}_{X}$

  また,このとき$X$と$Y$は\nw{同じホモトピー型をもつ}ともいう。ホモトピー同値であることを$X \simeq Y$と書く。
\end{teigi}

\midashi{注意.} 位相同型ならばホモトピー同値である。

\begin{hodai}
  ホモトピー同値の関係$\simeq$は同値関係
\end{hodai}

\begin{proof}
  ここでは推移律だけを示す。$X \simeq Y$かつ$Y \simeq Z$とする。
  \begin{align}
    &\exists f \colon X \longrightarrow Y, \quad \exists g \colon Y \longrightarrow X, \quad f \circ g \simeq \mathrm{id}_{Y}, \quad g \circ f \simeq \mathrm{id}_{X} \\
    &\exists f' \colon Y \longrightarrow Z, \quad \exists g' \colon Z \longrightarrow Y, \quad f' \circ g' \simeq \mathrm{id}_{Z}, \quad g' \circ f' \simeq \mathrm{id}_{Y}
  \end{align}
  であるから,$\overline{f} = f' \circ f$,$\overline{g} = g \circ g'$と定めれば,これは連続写像であり,
  $\overline{f} \circ \overline{g} = f' \circ f \circ g \circ g' \simeq f' \circ g' \simeq \mathrm{id}_{Z}$
  である。同様に,$\overline{g} \circ \overline{f} \simeq \mathrm{id}_{X}$である。したがって,$X \simeq Z$である。
\end{proof}

\begin{hodai}
  $X$から$A \subseteq X$への変形レトラクションがあるなら$X$と$A$はホモトピー同値である。
\end{hodai}

\begin{proof}
  $f_{t} \colon X \longrightarrow X$:$X$から$A$への変形レトラクションとする。つまり,$f_{0} = \mathrm{id}_{X}$,$f_{1}(X) = A$,$\restriction{f_{t}}{A} = \mathrm{id}_{A}$。
  $g \colon X \longrightarrow A$,$g(x) = f_{1} (x)$,$h \colon A \longrightarrow X$:包含写像とすれば,
  $g \circ h = \mathrm{id}_{A}$,$h \circ g = f_{1} \simeq f_{0} = \mathrm{id}_{X}$($\because f_{t}$:変形レトラクション)
\end{proof}

\begin{corr}
  {\bf 1.} $\mathbb{R}^{n}$,凸集合,1点はすべてホモトピー同値である。

  {\bf 2.} $\mathbb{R}^{n} \setminus \sets{0}$と$S^{n-1}$はホモトピー同値である。
\end{corr}

\begin{teigi}
  1点とホモトピー同値な空間を\nw{可縮}な空間という。
\end{teigi}

\midashi{注意.} 可縮な空間は凸集合の次に基本的な図形であるといえる。

\sukima \midashi{問題.} 超平面$H_{a,b} := \sets{x \in \mathbb{R}^{n} | \left\langle a, x \right\rangle = b}$,
半空間$H_{a,b}^{+} := \sets{x \in \mathbb{R}^{n} | \left\langle a, x \right\rangle \le b}$。
多面体$P$:有限個の半空間の積;${\displaystyle P = \bigcap_{i=1}^{k} H_{a_{i}, b_{i}}^{+}}$,
$P$の面:$P$または$P \subseteq H_{a,b}^{+}$なる$a,b$に対して$P \cap H_{a,b}$と書けるものである。

(非有界な)多面体の有界な面の和集合は可縮であることを示せ。

\begin{rei}
  トーラス$T^{2}$から1点を除く。
  \begin{center}
    \begin{tikzpicture}[baseline=12pt]
      \node at (0,0.5) {$T^{2} \setminus \sets{x}$};
      \node at (1,0.5) {$\simeq$};

      \filldraw[fill opacity=.3, draw=black] (1.5,0) rectangle +(1,1);
      \draw (1.4,0.45) -- (1.5, 0.55) -- (1.6,0.45);
      \draw (2.4,0.45) -- (2.5, 0.55) -- (2.6,0.45);
      \draw (1.9,1.1) -- (2,1) -- (1.9,0.9);
      \draw (2,1.1) -- (2.1,1) -- (2,0.9);
      \draw (1.9,0.1) -- (2,0) -- (1.9,-0.1);
      \draw (2,0.1) -- (2.1,0) -- (2,-0.1);
      \filldraw[fill=white, opacity=1, draw=black] (2,0.5) circle [radius=0.05];
      
      \node at (3,0.5) {$\simeq$};

      \draw (3.5,0) rectangle +(1,1);
      \draw (3.4,0.45) -- (3.5, 0.55) -- (3.6,0.45);
      \draw (4.4,0.45) -- (4.5, 0.55) -- (4.6,0.45);
      \draw (3.9,1.1) -- (4,1) -- (3.9,0.9);
      \draw (4,1.1) -- (4.1,1) -- (4,0.9);
      \draw (3.9,0.1) -- (4,0) -- (3.9,-0.1);
      \draw (4,0.1) -- (4.1,0) -- (4,-0.1);

      \node at (5,0.5) {$\simeq$};

      \draw (6,1) circle [x radius=0.5, y radius=0.2];
      \draw (6,0) circle [x radius=0.5, y radius=0.2];
      \draw (5.5,1) -- (5.5,0);
      \draw (6,0.9) -- (6.1,0.8) -- (6,0.7);
      \draw (6,-0.1) -- (6.1,-0.2) -- (6,-0.3);

      \node at (7,0.5) {$\simeq$};

      \draw (8,0.9) circle [x radius=0.3, y radius=0.4];
      \draw (8,0.1) circle [x radius=0.3, y radius=0.4];
      \filldraw (8,0.5) circle [radius=0.05];
    \end{tikzpicture}
  \end{center}
  $S^{1}$と$S^{1}$を1点で同一視した空間$S^{1} \vee S^{1}$とホモトピー同値。
\end{rei}

\begin{teigi}
  $X$と$Y$の\nw{wedge和} $X \vee Y$を次のように定める。
  $x \in X$,$y \in Y$をとって,
  \begin{equation}
    X \vee Y := X \amalg Y \,/\, x \sim y
  \end{equation}
\end{teigi}

\midashi{問題.} 閉曲面から何点か除いた空間のホモトピー型をいろいろ調べよ。

\begin{teigi}[セル複体(CW複体)]
  \nw{$n$--セル} $\simeq$ $n$次元ディスク$D^{n} = \sets{x \in \mathbb{R}^{n} | \left\| x \right\| \le 1}$
  とする。\nw{セル複体}とは以下のように帰納的につくられる空間をいう。
  \begin{itemize}
    \vspace{-0.5\baselineskip}
    \item $X^{0}$:0--セルからなる離散集合
    \item $X^{n}$:$n$--スケルトン
    
    $X^{n-1}$と$n$--セルの集合$D_{\alpha}^{n}$,$\varphi_{\alpha} \colon S_{\alpha}^{n} = \partial D_{\alpha}^{n} \longrightarrow X^{n-1}$($\alpha \in \Lambda$)を考えて,
    同値関係$\sim$を$x \sim \varphi_{\alpha}(x)$として,$X^{n}$を以下で定める。
    \begin{equation}
      X^{n} := \left( X^{n - 1} \amalg \coprod_{\alpha} D_{\alpha}^{n} \right) / \sim
    \end{equation}
  \end{itemize}
  セル複体の次元を含まれるセルの最大次元で定める。
\end{teigi}

\begin{rei}
  セル複体の例
\end{rei}
%%%%% あとで図を入れる %%%%%

\midashi{問題.} 他にもいろいろなセル複体をつくってみよ。

\begin{hodai}
  $X$:セル複体,$A \subseteq X$:可縮な部分複体とする。$X \sim A$を$A$を1点に潰して得られる空間とする。このとき,
  $X \simeq X / A$である。
\end{hodai}

\midashi{問題.} 証明せよ。

\begin{teigi}
  \nw{グラフ}とは1次元セル複体のことである。
\end{teigi}

\begin{prop}
  $X$:連結なグラフ,$n$:$X$の頂点数,$m$:$X$の枝数(取り付けた1--セルの個数)とする。このとき,
  $X \simeq \underbrace{S^{1} \vee S^{1} \vee \dots \vee S^{1}}_{m-n+1}$である。
\end{prop}

\begin{proof}
  グラフの\nw{全域木}とは,すべての頂点と接続するサイクルを含まないグラフとする。
  全域木の枝数は$n-1$である。
  グラフのうち全域木の部分を1点に潰して得られる空間を考えれば,これは先の全域木に入らない$m-n+1$本の枝がループになる。
\end{proof}

\midashi{参考.} house with two rooms

\expandafter\ifx\csname readornot\endcsname\relax
  \end{document}
\fi