\documentclass[uplatex]{jsarticle}
\usepackage{octopus}
\usepackage{url}
%%%% コマンド定義専用のtexファイル
\renewcommand{\postpartname}{章}
\renewcommand{\thepart}{\arabic{part}}
\renewcommand{\thesection}{\thepart.\arabic{section}}
\makeatletter\renewcommand{\theequation}{\thesection.\arabic{equation}}\@addtoreset{equation}{section}\makeatother
\newcommand{\octopuspart}[1]{\newpage\part{#1}\setcounter{section}{0}\vspace{3\baselineskip}}
\renewcommand{\restriction}[2]{\left. #1 \right|_{#2}}
\DeclareMathOperator{\dcup}{\dot{\cup}}
\DeclareMathOperator{\conv}{conv}
\DeclareMathOperator{\Image}{Im}
\DeclareMathOperator{\Kernel}{Ker}
\DeclareMathOperator{\diag}{diag}
\DeclareMathOperator{\rank}{rank}
\DeclareMathOperator{\sgn}{sgn}
\DeclareMathOperator{\rot}{rot}
\usetikzlibrary{calc}
\newcommand{\readornot}{false}

\begin{document}
\begin{center}{\LARGE \bf 2018年度 幾何数理工学 まとめノート}\end{center}

\sukima \midashi{本講義の内容}
\begin{enumerate}
    \vspace{-0.5\baselineskip}
    \item 位相空間:「近さ」が備わった空間概念.$\mathbb{R}^{n}$の一般化.
    \item 位相幾何:連続変形に対する不変性…
    \begin{itemize}
        \item 基本群
        \item ホモロジー
    \end{itemize}
    \item テンソル:座標変換に対する不変性…
    \begin{itemize}
        \item 3次元ベクトル${\displaystyle \begin{pmatrix}
            v_{1} & v_{2} & v_{3} 
        \end{pmatrix}^{\top}}$と${\displaystyle \begin{pmatrix}
            \pdif{f}{x_{1}} & \pdif{f}{x_{2}} & \pdif{f}{x_{3}}
        \end{pmatrix}^{\top}}$の違いは何か.
    \end{itemize}
\end{enumerate}

\midashi{参考書}
\begin{itemize}
    \vspace{-0.5\baselineskip}
    \item 内田伏一,「集合と位相」\footnote{amazonだと2,600円$+$税.}
    \item Allen Hatcher,「Algebraic Topology」\footnote{\url{https://pi.math.cornell.edu/~hatcher/AT/ATpage.html}}
    \item 伊理正夫,韓太舜,「テンソル解析入門」
\end{itemize}

\renewcommand{\baselinestretch}{0.1}
\tableofcontents
\renewcommand{\baselinestretch}{1.0}

\octopuspart{位相空間}
\expandafter\ifx\csname readornot\endcsname\relax
    \documentclass[uplatex]{jsarticle}    
    \usepackage{octopus}
    \usepackage{url}
    %%%% コマンド定義専用のtexファイル
\renewcommand{\postpartname}{章}
\renewcommand{\thepart}{\arabic{part}}
\renewcommand{\thesection}{\thepart.\arabic{section}}
\makeatletter\renewcommand{\theequation}{\thesection.\arabic{equation}}\@addtoreset{equation}{section}\makeatother
\newcommand{\octopuspart}[1]{\newpage\part{#1}\setcounter{section}{0}\vspace{3\baselineskip}}
\renewcommand{\restriction}[2]{\left. #1 \right|_{#2}}
\DeclareMathOperator{\dcup}{\dot{\cup}}
\DeclareMathOperator{\conv}{conv}
\DeclareMathOperator{\Image}{Im}
\DeclareMathOperator{\Kernel}{Ker}
\DeclareMathOperator{\diag}{diag}
\DeclareMathOperator{\rank}{rank}
\DeclareMathOperator{\sgn}{sgn}
\DeclareMathOperator{\rot}{rot}
    \usetikzlibrary{calc}
    \begin{document}
\fi

\section{距離空間}
\subsection{距離空間}
\begin{teigi}[距離空間]
    $X$を非空な集合,$d \colon X \times X \to \mathbb{R}$を実数値関数とする.次の3つの条件\textbf{D1},\textbf{D2},\textbf{D3}を考える.
    
    \textbf{D1. } $\forall x, y \in X, \quad d(x,y) \ge 0$,\qquad $d(x,y) = 0 \iff x = y$\par
    \textbf{D2. } $\forall x, y \in X, \quad d(x,y) = d(y,x)$\par
    \textbf{D3. } $\forall x, y \in X, \quad d(x,y) + d(y,z) \ge d(x,z)$
    
    $d$が\textbf{D1},\textbf{D2},\textbf{D3}の条件を満たすとき,$d$を$X$上の\nw{距離関数}という.また,$X$あるいは$(X,d)$を\nw{距離空間(metric space)}という.
\end{teigi}

\begin{remark}~\vspace{-0.5\baselineskip}
    \begin{itemize}
        \item \textbf{D2.}は対称性を表し,\textbf{D3.}は三角不等式と呼ばれる.
        \item \textbf{D1.},\textbf{D2.},\textbf{D3.}を満たさないような関数として,例えば${d_{2}}^{2}$(2乗距離)がある.
        これは三角不等式を満たさない.
    \end{itemize}
\end{remark}

\begin{rei}
    $X = \mathbb{R}^{n}$とする.
    \begin{align}
        d_{2} (x,y) &:= \sqrt{\sum_{i=1}^{n} (x_{i} - y_{i})^{2}}, \\
        d_{1} (x,y) &:= \sum_{i=1}^{n} |x_{i} - y_{i}|, \\
        d_{\infty} (x,y) &:= \max_{i} |x_{i} - y_{i}|
    \end{align}
    とすると,これらはどれも$X$上の距離関数である.なお,$(\mathbb{R}^{n}, d_{2})$を\nw{$n$次元Euclid空間}と呼ぶ.
\end{rei}
\begin{proof}
    ここでは$d_{2}$が\textbf{D3.}の三角不等式を満たすことのみを示す.
    \begin{align*}
        {d_{2}}^{2}(x,z)
        &= \sum_{i=1}^{n} (x_{i} - z_{i})^{2} = \sum_{i=1}^{n} (x_{i} - y_{i} + y_{i} - z_{i})^{2} \notag \\
        &= \sum_{i=1}^{n} \left( x_{i} - y_{i} \right)^{2} + \sum_{i=1}^{n} \left( y_{i} - z_{i} \right)^{2} + 2 \sum_{i=1}^{n} \left( x_{i} - y_{i} \right) \left( y_{i} - z_{i} \right) \notag \\
      &\le \sum_{i=1}^{n} \left( x_{i} - y_{i} \right)^{2} + \sum_{i=1}^{n} \left( y_{i} - z_{i} \right)^{2} + 2 \sqrt{\left( \sum_{i=1}^{n} \left( x_{i} - y_{i} \right)^{2} \right) \left( \sum_{i=1}^{n} \left( y_{i} - z_{i} \right)^{2} \right)} \\
        &= \left\{ \sqrt{\sum_{i=1}^{n} \left( x_{i} - y_{i} \right)^{2}} + \sqrt{\sum_{i=1}^{n} \left( y_{i} - z_{i} \right)^{2}} \right\}^{2}
         = \left( d_{2}(x,y) + d_{2} (y,z) \right)^{2}
    \end{align*}
    である.ここで,不等号はCauchy-Schwarzの不等式による.
\end{proof}

\begin{rei}
    $X = \mathcal{C} [a,b]$を区間$[a,b] \subseteq \mathbb{R}$上の連続関数全体の集合とする.
    \begin{align}
        d_{2} (f,g) &:= \sqrt{ \int_{a}^{b} | f(t) - g(t) | ^{2} \dx{t}}, \\
        d_{\infty} (f,g) &:= \sup \sets{ \left| f(t) - g(t) \right| | t \in [a,b]}, \\
        d_{1} (f,g) &:= \int_{a}^{b} \left| f(t) - g(t) \right| \dx{t}
    \end{align}
    とすると,これらはどれも$X$上の距離関数である.
\end{rei}

\begin{proof}
    ここでは$d_{1}$が\textbf{D3.}の三角不等式を満たすことと\textbf{D1.}の零点が対角集合に限られることのみを示す.\par
    \midashi{[D3.について]}
    \begin{align*}
        d_{1} (f,h)
        &= \int_{a}^{b} \left| f(t) - h(t) \right| \dx{t} \\
      &\le \int_{a}^{b} \left( \left| f(t) - g(t) \right| + \left| g(t) - h(t) \right| \right) \dx{t} \\
      &\le \int_{a}^{b} \left| f(t) - g(t) \right| \dx{t} +  \int_{a}^{b} \left| g(t) - h(t) \right| \dx{t} \\
        &= d_{1} (f,g) + d_{1} (g,h).
    \end{align*}
    \midashi{[零点が対角集合に限られることについて]}\par
    $f = g$ならば$d_{1} (f,g) = 0$は明らかであるので,問題はその逆である.

    もし,$f(x) \neq g(x)$ならばある$x_{0} \in [a,b]$で$\left| f(x_{0}) - g(x_{0}) \right|> 0$であり,
    $f-g$の連続性からある$\varepsilon > 0$が存在して,$y \in [x_{0} - \varepsilon, x_{0} + \varepsilon]$に対して
    $\left| f(y) - g(y) \right| > 0$である.
    したがって,
    \begin{equation*}
        \int_{a}^{b} \left| f(x) - g(x) \right| \dx{x} \ge \varepsilon \min_{x_{0} - \varepsilon \le y \le x_{0} + \varepsilon} \left| f(y) - g(y) \right| > 0
    \end{equation*}
    である.
\end{proof}

\begin{rei}
    $X = \sets{0,1}^{n}$とする.
    \begin{equation}
        d_{H}(x,y) = \# \sets{i | x_{i} \neq y_{i}}
    \end{equation}
    で定めると,これは$X$上の距離関数である.この距離関数は\nw{Hamming距離}と呼ばれる.
\end{rei}

\begin{rei}
    $G = (X,E)$を無向グラフとする.
    $d_{G}(x,y)$を$x$から$y$への最短路の長さとして定義すると,
    これは$G$上の距離関数である.
\end{rei}

以下,$(X,d)$を距離空間とする.

\begin{teigi}[内点・外点・境界点・触点]~\par
    \midashi{(1) } $a \in X$,$\varepsilon > 0$とする.集合$N(a,\varepsilon) := \sets{x \in X | d(a,x) < \varepsilon}$を,$a$の\nw{$\varepsilon$-近傍}という.

    \midashi{(2) } $A \subseteq X$の\nw{内点}とは,次の条件を満たす点$x \in X$のこと:
    \begin{equation}
        \exists \varepsilon > 0, \quad N(x,\varepsilon) \subseteq A
    \end{equation}
    また,$A$の内点全体の集合$A^{\circ} = \sets{x \in X | N(x, \varepsilon) \subseteq A}$のことを$A$の\nw{内部}という.

    \midashi{(3) } $A \subseteq X$の\nw{外点}とは,次の条件を満たす点$x \in X$のこと:
    \begin{equation}
        \exists \varepsilon > 0, \quad N(x , \varepsilon ) \cap A = \emptyset
    \end{equation}
    言い換えると,$A$の補集合の内点のこと.
    また,$A$の外点全体の集合$(X \setminus A)^{\circ}$を$A$の\nw{外部}という.

    \midashi{(4) } $A \subseteq X$の\nw{境界点}とは,次の条件を満たす点$x \in X$のこと:
    \begin{equation}
        \forall \varepsilon > 0, \quad N(x, \varepsilon) \cap A \neq \emptyset, \, N(x,\varepsilon) \cap (X \setminus A) \neq \emptyset
    \end{equation}
    また,$A$の境界点全体の集合のことを$A$の\nw{境界}といい,記号$\partial A$で表す.

    \midashi{(5) } $A \subseteq X$の\nw{触点}とは,次の条件を満たす点$x \in X$のこと:
    \begin{equation}
        \forall \varepsilon > 0, \quad N (x,\varepsilon) \cap A \neq \emptyset
    \end{equation}
    また,$A$の触点全体の集合のことを$A$の\nw{閉包}といい,記号$\overline{A}$で表す.
\end{teigi}

\begin{figure}[htbp]
    \centering
    \newcommand{\makeaxis}{
    \draw[thick, ->] (-2.5,0) -- (2.5,0) node [below] {$x$};%x軸
    \draw[thick, ->] (0,-2.5) -- (0,2.5) node [left] {$y$};%y軸
    \coordinate (O) at (0,0);
    \node at (O) [above left] {$a$};
    }
    \centering    
    \begin{tikzpicture}
        \makeaxis
        \draw[dashed] (O) circle (1.5);
        \draw[->] (O) -- node[midway, right]{$\varepsilon$} (45:1.5);
        \node at (-2,2) {$(\mathbb{R}^{2}, d_{2})$};
    \end{tikzpicture}
    \begin{tikzpicture}
        \makeaxis
        \draw[dashed] (1.5,0) -- (0,1.5) -- (-1.5,0) -- (0,-1.5) -- cycle;
        \node at (1.5,0) [below] {$a + \varepsilon$};
        \node at (-1.5,0) [below] {$a - \varepsilon$};
        \node at (-2,2) {$(\mathbb{R}^{2}, d_{1})$};
    \end{tikzpicture}
    \begin{tikzpicture}
        \makeaxis
        \draw[dashed] (-1.5,-1.5) -- (1.5,-1.5) -- (1.5,1.5) -- (-1.5,1.5) -- cycle;
        \node at (-1.5,0) [below left] {$a-\varepsilon$};
        \node at (0,-1.5) [below right] {$a-\varepsilon$};
        \node at (1.5,0) [above right] {$a+\varepsilon$};
        \node at (0,1.5) [above right] {$a+\varepsilon$};
        \node at (-2,2) {$(\mathbb{R}^{2}, d_{\infty})$};
    \end{tikzpicture}
    \caption{距離空間$(\mathbb{R}^{2},d)$上の近傍の違い}
    \label{fig:euclid_dist}
\end{figure}

\begin{figure}[htbp]
    \centering
    \begin{tikzpicture}
        \draw[dashed] (-30:2) arc (-30:210:2);
        \fill [black!20] (0,0) circle [x radius=1.97, y radius=1.97];
        \draw[thick] (210:2) arc (210:330:2);
        \draw[<-] (-50:2.1) -- (-50:2.4) node[right]{\textbf{外点}};
        \draw (20:0.5) node{\textbf{内点}};
        \draw[<-] (160:2.1) -- (160:2.6) node[left]{\textbf{境界点}};
        \draw (45:2.5) node {$A$};
    \end{tikzpicture}
    \caption{$A \subseteq X$の内部,外部,境界}
    \label{fig:in_ex_boundary}
\end{figure}

この定義から,内部,外部,境界点について$X = A^{\circ} \dcup \partial A \dcup (X \setminus A)^{\circ}$と互いに非交な集合に分解できる.
また,閉包について,次の二つが成り立つ.
\begin{align}
    & A ^ { \circ } \subseteq A \subseteq \overline{A}, \\
    & \overline { A } = A ^ { \circ } \cup \partial A.
\end{align}

\begin{teigi}[開集合・閉集合]~\par
    \midashi{(1) } $A \subseteq X \text{:\nw{開集合}}  \defines \;\forall x \in A, \quad \exists \varepsilon > 0, \quad  N(x , \varepsilon ) \subseteq A \iff A = A^{\circ}$\par
    \midashi{(2) } $A \subseteq X \text{:\nw{閉集合}} \defines \; 「\forall x \in X, \quad \forall \varepsilon > 0, \quad  N(x , \varepsilon ) \cap A \neq \emptyset \Longrightarrow x \in A」\iff A = \overline{A}$
\end{teigi}

\begin{hodai}
    $N(x,\varepsilon)$は開集合.特に,$N(x,\varepsilon) = N(x,\varepsilon)^{\circ}$.
\end{hodai}

\begin{proof}
    $y \in N ( x, \varepsilon)$をとる.$\delta := \varepsilon - d ( x , y ) > 0$とすると,
    $z \in N(y, \varepsilon)$に対して,三角不等式より
    \begin{equation*}
        d(x,z) \le d(x,y) + d(y,z) < \varepsilon
    \end{equation*}
である.よって,$N(y,\delta) \subseteq N(x,\varepsilon)$となる.
\end{proof}

\begin{prop}
    $\left( A^{\circ} \right)^{\circ} = A^{\circ}$,$\overline{\overline{A}} = \overline{A}$
\end{prop}

\begin{proof}
    $\left( A^{\circ} \right)^{\circ} \subseteq A^{\circ}$より,$\supseteq$向きを示す.
    $x \in A^{\circ}$とする.
    \begin{equation*}
       \exists \varepsilon > 0, \quad N(x, \varepsilon) \subseteq A
    \end{equation*}
    である.$N(x, \varepsilon) = N(x, \varepsilon)^{\circ} \subseteq A^{\circ}$,
    つまり$N(x, \varepsilon) \subseteq A^{\circ}$である.
    $x \in \left( A^{\circ} \right)^{\circ}$である.

    $\overline{\overline{A}} \subseteq \overline{A}$より,$\subseteq$向きを示す.
    $x \in \overline{\overline{A}}$とする.
    \begin{equation}
        \forall \varepsilon > 0, \quad N(x,\varepsilon) \cap \overline{A} \neq \emptyset
    \end{equation}
    である.$y \in N(x, \varepsilon) \cap \overline{A}$,$\delta := \varepsilon - d(x,y) > 0$とする.
    $y \in \overline{A}$より$N(y, \delta) \cap A \neq \emptyset$である.
    $N(y, \delta) \subseteq N(x,\varepsilon)$より,上と同じロジックで
    $\emptyset \neq N(y, \delta) \cap A \subseteq N(x, \varepsilon) \cap A$である.よって,$x \in \overline{A}$である.
\end{proof}

\begin{prop}~\par
    \midashi{(1) } $A$:開 $\Longrightarrow$ $X \setminus A$:閉\par
    \midashi{(2) } $A$:閉 $\Longrightarrow$ $X \setminus A$:開
\end{prop}

\begin{proof}
    \begin{align*}
        A\text{が開でない} & \Longleftrightarrow \exists x \in A, \quad \forall \varepsilon > 0, \quad N(x, \varepsilon) \not\subseteq A \\
        & \Longleftrightarrow \exists x \in A, \quad \forall \varepsilon > 0, \quad N(x, \varepsilon) \cap (X \setminus A) \neq \emptyset \\
        & \Longleftrightarrow X \setminus A \text{が閉でない}
    \end{align*}
    ここで2行目の同値性は,$x$は$X \setminus A$の触点であるが$X \setminus A$の元でないことから従う.
\end{proof}

開集合全体の集合を\nw{開集合系}といい,記号$\mathcal{O}$で表す.
閉集合全体の集合を\nw{閉集合系}といい,記号$\mathfrak{A}$で表す.

\begin{prop}
    \label{prop:open_set_axiom}
    開集合系$\mathcal{O}$について以下が成り立つ.\par
    \midashi{(1) } $X \in \mathcal{O}$,$\emptyset \in \mathcal{O}$\par
    \midashi{(2) } $O_{1}, O_{2}, \cdots, O_{k} \in \mathcal{O} \Longrightarrow O_{1} \cap O_{2} \cap \cdots \cap O_{k} \in \mathcal{O}$\par
    \midashi{(3) } 任意の集合$\Lambda$に対して$O_{\lambda} \in \mathcal{O}$($\forall \lambda \in \Lambda$) ${\displaystyle \Longrightarrow \bigcup_{\lambda \in \Lambda} O_{\lambda} \in \mathcal{O}}$
\end{prop}

\begin{proof}
    \midashi{(1) } obvious.

    \midashi{(2) } $x \in O_{1} \cap O_{2} \cap \dots \cap O_{k}$をとる.$i = 1,2,\dots,k$に対して$\varepsilon_{i} > 0$が存在して
    \begin{equation*}
        N(x_{i}, \varepsilon_{i}) \subseteq O_{i}
    \end{equation*}
    である.${\displaystyle \varepsilon := \min_{i} \varepsilon_{i}}$とすると,
    $N(x, \varepsilon) \subseteq O_{1} \cap O_{2} \cap \dots \cap O_{k}$である.
    $x$は$O_{1} \cap O_{2} \cap \dots \cap O_{k}$の内点である.よって,$\left( O_{1} \cap O_{2} \cap \dots \cap O_{k} \right)^{\circ} = O_{1} \cap O_{2} \cap \dots \cap O_{k}$である.

    \midashi{(3) } ${x \in \displaystyle \bigcup_{\lambda \in \Lambda} O_{\lambda}}$とすると,
    $\exists \lambda \in \Lambda, \quad x \in O_{\lambda}$である.${\displaystyle \exists \varepsilon > 0, \quad N(x, \varepsilon) \subseteq O_{\lambda} \subseteq \bigcup_{\lambda} O_{\lambda}}$とする.
\end{proof}

\begin{prop}
    閉集合系$\mathfrak{A}$について以下が成り立つ.\par
    \midashi{(1) } $X \in \mathfrak{A}$,$\emptyset \in \mathfrak{A}$\par
    \midashi{(2) } $A_{1}, A_{2}, \cdots, A_{k} \in \mathfrak{A} \Longrightarrow A_{1} \cup A_{2} \cup \cdots \cup A_{k} \in \mathfrak{A}$\par
    \midashi{(3) } 任意の集合$\Lambda$に対して$A_{\lambda} \in \mathfrak{A}$($\forall \lambda \in \Lambda$) ${\displaystyle \Longrightarrow \bigcap_{\lambda \in \Lambda} A_{\lambda} \in \mathfrak{A}}$
\end{prop}

\begin{proof}
    \midashi{(1) } $\mathfrak{A} = \sets{ X \setminus O | O \in \mathcal{O}}$に注意すると明らか.

    \midashi{(2) } $A_{i} = X \setminus O_{i}$とかくと,
    \begin{equation*}
        A_{1} \cup A_{2} \cup \dots \cup A_{k}
        = (X \setminus O_{1}) \cup (X \setminus O_{2}) \cup \dots \cup (X \setminus O_{k})
        = X \setminus (O_{1} \cap O_{2} \cap \dots \cap O_{k}) \in \mathfrak{A}
    \end{equation*}
    である.

    \midashi{(3) } ${\displaystyle \bigcap_{\lambda \in \Lambda} A_{\lambda} = \bigcap_{\lambda \in \Lambda} (X \setminus O_{\lambda}) = X \setminus \bigcup_{\lambda \in \Lambda} O_{\lambda}}$であるから従う.
\end{proof}

収束性,連続性なども距離空間で定義できる.

\begin{teigi}[収束性・連続性]
    \midashi{(1) } $a_{i} \in X$($i=1,2,\dots$):\nw{収束}とは,
    \begin{equation}
        \lim_{i \to \infty} a_{i} = a \defines \lim_{i \to \infty} d(a, a_{i}) = 0
    \end{equation}
    
    \midashi{(2) } $f:X \longrightarrow \mathbb{R}$が$a \in X$で\nw{連続}とは,
    \begin{equation}
        \forall \varepsilon > 0, \quad \exists \delta > 0, d(a,x) < \delta \Longrightarrow \left| f(a) - f(x) \right| < \varepsilon
    \end{equation}
\end{teigi}

では,$X$上の異なる距離$d$,$d'$の下で連続性や収束性も変わるのか.

$\longrightarrow$位相空間の考え方へ

実際,開集合系を与える(空間に「位相」を与える)ことによって,
連続性や収束性が定義でき,
開集合系が同じとき,連続性や収束性も等しくなる.

\begin{rei}
    $(\mathbb{R}^{n},d_{2})$と$(\mathbb{R}^{n},d_{\infty})$が定義する開集合系は等しい.(位相は等しい)
    \begin{proof}
        $A$:$d_{2}$の下で開集合とする.
        $x \in A$をとる.$\exists \varepsilon > 0, \quad N_{2}(x,\varepsilon) \subseteq A$である.
        $\exists c > 0, \quad N_{\infty} (x, c \varepsilon) \subseteq N_{2} (x, \varepsilon) \subseteq A$である.
        $A$は$d_{\infty}$の下でも開.逆も同様に言える. 
    \end{proof}
    結局,連続性や収束性はどちらの距離で考えても同じになる.
\end{rei}

\begin{rei}
    $\mathbb{R}^{n}$上のノルム$\left\| \cdot \right\|$とは,
    \begin{itemize}
        \vspace{-0.5\baselineskip}
        \item $\left\| v \right\| \ge 0$,\qquad $\left\| v \right\| = 0 \Longleftrightarrow v = 0$
        \item $\forall a \in \mathbb{R}$,$v \in \mathbb{R}^{n}$,$\left\| av \right\| = \left| a \right| \left\| v \right\|$
        \item $\forall u,v \in \mathbb{R}^{n}$,$\left\| u + v \right\| \le \left\| u \right\| + \left\| v \right\|$
        \vspace{-0.5\baselineskip}
    \end{itemize}
    を満たすものである.このとき,$d(x,y) := \left\| x - y \right\|$で定義するとこれは距離関数である.
\end{rei}

\midashi{知っておいてほしいこと}
\begin{itemize}
    \vspace{-0.5\baselineskip}
    \item $\mathbb{R}^{n}$上のどんなノルムを定める「位相」は等しい.
    \item 無限次元では,そうはいかない(\textbf{らしい}).
\end{itemize}

\expandafter\ifx\csname readornot\endcsname\relax
  \end{document}
\fi % 距離空間
\expandafter\ifx\csname readornot\endcsname\relax
  \documentclass[uplatex]{jsarticle}
  \usepackage{octopus}
  \usepackage{url}

  \renewcommand{\proofname}{\textsf{証明}}
  \renewcommand{\postpartname}{章}
  \renewcommand{\thesection}{\thepart.\arabic{section}}
  \renewcommand{\thepart}{\arabic{part}}
  \makeatletter\renewcommand{\theequation}{\thesection.\arabic{equation}}\@addtoreset{equation}{section}\makeatother
  
  \newcommand{\octopuspart}[1]{\newpage\part{#1}\setcounter{section}{0}\vspace{3\baselineskip}}
  
  \DeclareMathOperator{\dcup}{\dot{\cup}}
  \begin{document}
  \fi
\renewcommand{\thesubsection}{\thepart.\arabic{section}.\arabic{subsection}}
  
\section{位相空間}

$X$を非空な集合とする。$X$は「\nw{位相}」(topology)を入れて「空間」にする。
「位相」を入れるとは,$X$の開集合族を指定することである。

\begin{teigi}[位相]
  以下を満たす開集合系$\mathcal{O} \subseteq 2^{X}$を\nw{位相}という。

  \midashi{O1. } $X \in \mathcal{O}$,$\emptyset \in \mathcal{O}$

  \midashi{O2. } $O_{1}, O_{2}, \cdots, O_{k} \in \mathcal{O} \Longrightarrow O_{1} \cap O_{2} \cap \cdots \cap O_{k} \in \mathcal{O}$

  \midashi{O3. } 任意の集合$\Lambda$に対して$O_{\lambda} \in \mathcal{O}$($\forall \lambda \in \Lambda$) ${\displaystyle \Longrightarrow \bigcup_{\lambda \in \Lambda} O_{\lambda} \in \mathcal{O}}$

  $\mathcal{O}$の元を\nw{開集合}という。また,開集合族$\mathcal{O}$が指定された集合$X$を\nw{位相空間}という。
  これを$(X,\mathcal{O})$で表すこともある。
\end{teigi}

\begin{rei}
  \begin{itemize}
    \item \nw{離散位相} $\mathcal{O} = 2^{X}$
    \item \nw{密着位相} $\mathcal{O} = \sets{\emptyset, X}$
    \item 距離空間$(X,d)$;$\mathcal{O} := \sets{O \subseteq X | \forall x \in O, \quad \exists \varepsilon > 0, \quad N(x, \varepsilon) \subseteq O}$
    (これは距離空間のところで定義した開集合の集まり)
  \end{itemize}
\end{rei}

距離空間のときに定義したいくつかの用語を位相空間の言葉で記述し直す。

\begin{itemize}
  \vspace{-0.5\baselineskip}
  \item $A \subseteq X$の内部$A^{\circ}$
  \begin{equation}
    A^{\circ} := \bigcup \sets{O \in \mathcal{O} | O \subseteq A} \in \mathcal{O}
  \end{equation}
  これは$A$に含まれる最大の開集合のこと。$A$の内点$x$とは$A^{\circ}$の元のこと。
  \item 閉集合とは,ある開集合の補集合になっているものとする。
  すると,閉集合族$\mathfrak{A}$は以下を満たす。:

  \midashi{A1. } $X \in \mathfrak{A}$,$\emptyset \in \mathfrak{A}$

  \midashi{A2. } $F_{1}, F_{2}, \dots, F_{k} \in \mathfrak{A} \Longrightarrow F_{1} \cup F_{2} \cup \dots \cup F_{k} \in \mathfrak{A}$

  \midashi{A3. } 任意の集合$\Lambda$に対して$F_{\lambda} \in \mathfrak{A} $($\lambda \in \Lambda$)$\Longrightarrow {\displaystyle \bigcap_{\lambda} F_{\lambda} \in \mathfrak{A}}$

  \item $A$の閉包$\overline{A}$
  \begin{equation}
    \overline{A} := \bigcap \sets{F \in \mathfrak{A} | A \subseteq F} \in \mathfrak{A}
  \end{equation}
  これは$A$を含む最小の閉集合のこと。$A$の触点$x$とは$\overline{A}$の元のこと。
  \item $N$が$x \in X$の近傍であるとは,$x \in N^{\circ}$となること,すなわち,
  \begin{equation}
    \exists O \in \mathcal{O}, \quad x \in O \subseteq N
  \end{equation}
  となること。特に$N$が開集合であるとき,$N$を$x$の開近傍という。
  \item $\mathcal{N}(x)$:$x$の近傍全体
  \vspace{-0.5\baselineskip}
\end{itemize}

位相の与え方にはいろいろある。

\begin{enumerate}
  \item {\bf (A1)},{\bf (A2)},{\bf (A3)}を満たす集合族$\mathfrak{A}$(閉集合族)を指定する。
  \item 各$A \subseteq X$に$A^{\circ}$を対応させる写像$2^{X} \longrightarrow 2^{X}$を指定する。(\nw{開核作用子})
  \item 各$A \subseteq X$に$\overline{A}$を対応させる写像$2^{X} \longrightarrow 2^{X}$を指定する。(\nw{閉包作用子})
  \item 各点$x$に$\mathcal{N}(x)$を対応させる写像$X \longrightarrow 2^{2^{X}}$を指定する。
\end{enumerate}

ただし,指定する写像はどんなものでもいいわけではない。
それぞれある条件を満たすような写像に限られる。
例えば,閉包作用子$\tau \colon 2^{X} \longrightarrow 2^{X}$は次を満たす必要がある。
\begin{itemize}
  \vspace{-0.5\baselineskip}
  \item $\tau (\emptyset) = \emptyset$
  \item $A \subseteq \tau(A)$
  \item $\tau (A \cup B) = \tau (A) \cup \tau (B)$
  \item $\tau (\tau (A)) = \tau (A)$
  \vspace{-0.5\baselineskip}
\end{itemize}

これらから自然に開集合族が決まる。
例えば,閉包作用子に対しては,$A = \tau (A)$となるような$A$を閉集合と定める。
詳細は演習。

\sukima \midashi{\large 連続写像}

\renewcommand{\arraystretch}{0.8}
\vspace{0.2\baselineskip}
\begin{tabular}{ll@{\,}l@{\,}l}
  位相空間 & $X$, & $\mathcal{O}_{X}$:開集合族, & $\mathfrak{A}_{X}$:閉集合族 \\
           & $Y$, & $\mathcal{O}_{Y}$:開集合族, & $\mathfrak{A}_{Y}$:閉集合族
\end{tabular}
\renewcommand{\arraystretch}{1}

\begin{teigi}[連続性]
  写像$f \colon X \longrightarrow Y$が$x \in X$で\nw{連続} $\defines$ 
  「$\forall N$:$f(x)$の近傍 $\Longrightarrow$ $f^{-1}(N)$は$x$の近傍」
\end{teigi}

\midashi{距離空間の連続性との同値性}
\begin{align*}
  x \in X \text{で連続} & \Longleftrightarrow \forall \varepsilon > 0, \quad \exists \delta > 0, \quad d_{X} (x,y) < \delta \Longrightarrow d_{Y}(f(x), f(y)) < \varepsilon \\
  & \Longleftrightarrow \forall \varepsilon > 0, \quad \exists \delta > 0, \quad N(x,\delta) \subseteq f^{-1} (N(f(x), \varepsilon)) \\
  & \Longleftrightarrow f(x) \text{の任意の近傍の逆像は} x \text{の近傍}
\end{align*}

\begin{teiri}
  以下の{\bf (1)},{\bf (2)},{\bf (3)}は同値。:
  
  \midashi{(1)} $f$は$X$の各点で連続

  \midashi{(2)} $\forall O \in \mathcal{O}_{Y}, \quad f^{-1}(O) \in \mathcal{O}_{X}$

  \midashi{(3)} $\forall F \in \mathfrak{A}_{Y}, \quad f^{-1}(F) \in \mathfrak{A}_{X}$
\end{teiri}

\begin{proof}
  \midashi{(1) $\Longrightarrow$ (2):} $O \in \mathcal{O}_{Y}$を任意にとり,$x \in f^{-1}(O)$とする。$O$は$f(x)$の開近傍である。
  連続性の定義からある$x$の開近傍$O_{x}$が存在して,$O_{x} \subseteq f^{-1} (O)$である。
  \begin{equation}
    f^{-1} (O) = \bigcup_{x \in f^{-1}(O)} \sets{x} \subseteq \bigcup_{x \in f^{-1}(O)} O_{x} \subseteq f^{-1}(O)
  \end{equation}
  であるから,${\displaystyle f^{-1}(O) = \bigcup_{x \in f^{-1} (O)} O_{x} \in \mathcal{O}_{X}}$である。

  \midashi{(2) $\Longrightarrow$ (3):} $F \in \mathfrak{A}_{Y}$を任意にとる。$Y \setminus F \in \mathcal{O}_{Y}$であり,
  したがって,$f^{-1} (Y \setminus F) = X \setminus f^{-1}(F) \in \mathcal{O}_{X}$である。
  よって,$f^{-1} (F) \in \mathfrak{A}_{X}$である。

  \midashi{(3) $\Longrightarrow$ (2):} $O \in \mathcal{O}_{Y}$を任意にとる。$Y \setminus O \in \mathfrak{A}_{Y}$であり,
  したがって,$f^{-1} (Y \setminus O) = X \setminus f^{-1} (O) \in \mathfrak{A}_{X}$である。
  よって,$f^{-1} (O) \in \mathcal{O}_{X}$である。

  \midashi{(2) $\Longrightarrow$ (1):} $x \in X$を任意にとる。$N$を$f(x)$の近傍として,その内部を$O := N^{\circ}$とすれば$f(x) \in O$である。
  {\bf (2)}より$f^{-1}(O) \in \mathcal{O}_{X}$であってこれは$x$の開近傍である。
  $x \in f^{-1} (O) \subseteq f^{-1} (N)$である。
\end{proof}

\begin{teigi}[同相]
  \midashi{(1)} $f \colon X \longrightarrow Y$が\nw{連続} $\defines$ 上の{\bf (1)}から{\bf (3)}のどれかの条件を満たす
  
  \midashi{(2)} $f \colon X \longrightarrow Y$が\nw{同相}(homeomorphic) $\defines$ $f$:連続全単射 かつ $f^{-1} \colon Y \longrightarrow X$:連続
  
  $X$と$Y$の間に同相写像が存在するとき,$X$と$Y$は\nw{同相}あるいは\nw{位相同型}といい,
  記号$X \simeq Y$で表す。
\end{teigi}

\midashi{演習:}全単射連続写像であって同相でないものの例を与えよ。

\begin{teigi}[位相の強弱]
  $X$上の2つの位相$\mathcal{O}$,$\mathcal{O}'$を考える。
  $\mathcal{O} \subseteq \mathcal{O}'$のとき,
  「$\mathcal{O}$は$\mathcal{O}'$より\nw{弱い}位相である」,「$\mathcal{O}'$は$\mathcal{O}$より\nw{強い}位相である」という。
\end{teigi}

つまり,$\mathcal{O}$で連続な写像は,$\mathcal{O}'$でも連続な写像になる。
{\footnotesize「使ったことない」「弱い強い混乱するわロクでもない概念」などと。}

\sukima \midashi{\large いろいろな位相}

\newcommand{\dtimes}{\times\mkern-16mu\times}

\begin{teigi}[誘導位相]
  $X$:集合,$(Y,\mathcal{O}_{Y})$:位相空間,$f \colon X \longrightarrow Y$とする。
  \begin{equation}
    \mathcal{O}_{X} := \sets{f^{-1}(O) | O \in \mathcal{O}_{Y}}
  \end{equation}
  を$f$による\nw{誘導位相}という。
\end{teigi}

$f^{-1} (O \cup O') = f^{-1} (O) \cup f^{-1} (O')$,
$f^{-1} (O \cap O') = f^{-1} (O) \cap f^{-1} (O')$に注意すると,位相の定義の条件を満たすことは理解できる。

\begin{teigi}[相対位相]
  $(X, \mathcal{O}_{X})$:位相空間,$Z \subseteq X$:部分集合。
  \begin{equation}
    \mathcal{O}_{Z} := \sets{O \cap Z | O \in \mathcal{O}_{X}}
  \end{equation}
  とすると,$(Z, \mathcal{O}_{Z})$は$(X, \mathcal{O}_{X}$の部分(位相)空間をなす。
  この位相を\nw{相対位相}という。
\end{teigi}

相対位相は包含写像$Z \hookrightarrow X$による誘導位相でもある。

\sukima \midashi{直積位相}

$(X, \mathcal{O}_{X})$,$(Y, \mathcal{O}_{Y})$:位相空間として,
直積$X \times Y = \sets{(x,y) | x \in X, y \in Y}$に位相を与えたい。しかし,
\begin{equation}
  \mathcal{B} := \sets{O \times O' | O \in \mathcal{O}_{X}, O' \in \mathcal{O}_{Y}}
\end{equation}
は開集合の条件を満たさない。そこで$\mathcal{B}$が「生成する」位相$\mathcal{O}_{X} \dtimes \mathcal{O}_{Y}$を
\begin{equation}
  \mathcal{O}_{X} \dtimes \mathcal{O}_{Y} := \sets{ \bigcup_{W \in \mathcal{B}'} W | \mathcal{B}' \subseteq \mathcal{B}}
\end{equation}
で定めると,$(X \times Y, \mathcal{O}_{X} \dtimes \mathcal{O}_{Y})$は位相空間になる。

3つ以上の位相空間に対しても同様にして直積位相を定めることができる。

\begin{rei}[$(\mathbb{R}^{2}, d_{2})$の位相]
  $x$の$d_{2}$の下での$\varepsilon$-近傍に含まれる$d_{\infty}$の下での$\varepsilon'$-近傍をとることができる。
  $\mathbb{R}^{2}$の開集合は$d_{\infty}$の近傍たちの和集合で書くことができる。
  $d_{\infty}$の$\varepsilon'$-近傍はすべて$\mathcal{B}$の元である。
  よって,$\mathbb{R}^{2}$の位相は$\mathbb{R} \times \mathbb{R}$の直積位相に等しい。
\end{rei}

\sukima \midashi{商位相}

2つの位相空間を貼り合わせて新たな位相空間を作りたい。
% 貼り合わせる図 if needed

\begin{itemize}
  \vspace{-0.5\baselineskip}
  \item まず$X$と$Y$の直和$X \amalg Y$を考える。$X \amalg Y$に位相を入れる。
  $\mathcal{O}_{X \amalg Y} := \sets{O \cup O' | O \in \mathcal{O}_{X}, O' \in \mathcal{O}_{Y}}$
  \item 同一視したい点たちを同値関係$\sim$で同一視して,商集合$(X \amalg Y) / \sim$を作る。
  \item 商集合に商位相を入れる。
  \vspace{-0.5\baselineskip}
\end{itemize}

位相空間$(X,\mathcal{O})$,$\sim$:$X$上の同値関係,
$X / \sim$:商集合,$\varphi \colon X \longrightarrow X / \sim$:自然な射影(自身を代表元とする同値類を返す)
\begin{equation}
  {\mathcal{O} / \sim} := \sets{H \subseteq {X / \sim} | \varphi^{-1}(H) \subseteq \mathcal{O}}
\end{equation}
$({X / \sim}, {\mathcal{O} / \sim})$は位相空間になる。
%よくわからない絵

\sukima \midashi{\large 重要な位相空間の例}

$\mathbb{R}^{n}$:Euclid空間(位相はEuclid距離から)

\begin{itemize}
  \vspace{-0.5\baselineskip}
  \item $S^{n}$:$n$次元球面
  \begin{equation}
    S^{n} := \sets{x \in \mathbb{R}^{n+1} | \sum_{j=1}^{n+1} {x_{j}}^{2} = 1}
  \end{equation}
  位相は$\mathbb{R}^{n+1}$からの相対位相とする。

  \item $D^{n}$:$n$次元ディスク
  \begin{equation}
    D^{n} := \sets{x \in \mathbb{R}^{n} | \sum_{j=1}^{n} {x_{j}}^{2} \le 1}
  \end{equation}

  \begin{center}
    $D^{2} \simeq$
    \begin{tikzpicture}[baseline=12pt] \filldraw[fill opacity=.3,draw = black] (0,0) rectangle (1,1); \end{tikzpicture}
      , \quad $S^{2} \simeq {D^{2} / \sim} \simeq$
    \begin{tikzpicture}[baseline=12pt]
      \filldraw[fill opacity=.3, draw = black] (0,0) rectangle (1,1);
      \draw (0.55,-0.1) -- (0.45, 0) -- (0.55,0.1);
      \draw (0.9,0.45) -- (1, 0.55) -- (1.1,0.45);
      \draw (-0.1,0.6) -- (0, 0.5) -- (0.1,0.6);
      \draw (-0.1,0.5) -- (0, 0.4) -- (0.1,0.5);
      \draw (0.4,1.1) -- (0.5,1) -- (0.4,0.9);
      \draw (0.5,1.1) -- (0.6,1) -- (0.5,0.9);
    \end{tikzpicture}
  \end{center}

  \item $P^{n}$:射影空間
  \begin{equation}
    P^{n} := (\mathbb{R}^{n+1} \setminus \sets{0}) / \sim
  \end{equation}
  このときの同値関係は
  \begin{equation}
    (x_{1},x_{2},\dots,x_{n+1}) \sim (y_{1},y_{2},\dots,y_{n+1}) \defines \exists \alpha \in \mathbb{R}, \quad (x_{1},x_{2},\dots,x_{n+1}) = \alpha (y_{1},y_{2},\dots,y_{n+1})
  \end{equation}
  で定めている。
  位相は$\mathbb{R}^{n+1}$の部分空間$+$商位相で定める。
  ちなみに$S_{+}^{n}$は上半球面として
  \begin{align}
    P^{n} &\simeq S^{n} / (\text{対蹠点を同一視}) \\
    &\simeq S_{+}^{n} / (\text{対蹠点を同一視}) \\
    &\simeq D^{n-1} / (\text{対蹠点を同一視})
  \end{align}
  が成り立つ。
  %図

  \item $T^{n}$:$n$次元トーラス
  \begin{equation}
    T^{n} := \underbrace{S^{1} \times S^{1} \times \dots \times S^{1}}_{n}
  \end{equation}
  位相は直積位相で定める。

  \item M\"obiusの輪
  \item Kleinの壺
  \vspace{-0.5\baselineskip}
\end{itemize}
% 適宜図を挿入

\sukima
\begin{teigi}(Hausdorff空間)
  $X$:位相空間。
  \vspace{-0.5\baselineskip}
  \begin{equation*}
    \begin{array}{lll}
      X \colon \text{\nw{Hausdorff空間}} & \defines & 「\forall x,y \in X, \quad x \neq y \Longrightarrow \exists U \in \mathcal{O}, \quad
      \exists V \in \mathcal{O}, \quad x \in U, \quad y \in V, \quad U \cap V = \emptyset」
    \end{array}
  \end{equation*}
\end{teigi}

\begin{prop}
  距離空間はHausdorff空間である。
\end{prop}

\begin{teigi}[多様体]
  $M$:Hausdorff空間。
  \vspace{-0.5\baselineskip}
  \begin{equation*}
    \begin{array}{lll}
      M \colon \text{\nw{$n$次元多様体}} & \defines & \forall p \in M, \quad \exists U \colon \text{$p$の開近傍}, \quad
      \exists U' \colon \mathbb{R}^{n} \text{の開集合}, \quad
      \exists \varphi \colon U \longrightarrow U' \subseteq \mathbb{R}^{n}, \quad \varphi \colon \text{同相写像}
    \end{array}
  \end{equation*}
\end{teigi}

\begin{rei}
  $S^{n}$,$P^{n}$,$T^{n}$は多様体。
\end{rei}

\midashi{演習:}$f \colon \mathbb{R}^{n} \longrightarrow \mathbb{R}$に対して
$\sets{x \in \mathbb{R}^{n} | f(x) = 0}$が多様体となるのはどんなときか。

\begin{rei}[単体複体]
  $S_{\lambda}$($\lambda \in \Lambda$):単体(これも位相空間の一種)の集合
  \begin{equation}
    X := \coprod_{\lambda \in \Lambda} S_{\lambda} = \bigcup_{\lambda \in \Lambda} S_{\lambda} \times \sets{\lambda}
  \end{equation}
  として,ここに同値関係$\sim$を貼り合わせの同一視の関係として定める。
  商空間$K = X / \sim$で全体の単体複体を表す。
\end{rei}

\expandafter\ifx\csname readornot\endcsname\relax
  \end{document}
\fi
 % 位相空間
\expandafter\ifx\csname readornot\endcsname\relax
  \documentclass[uplatex]{jsarticle}
  \usepackage{octopus}
  \usepackage{url}

  \renewcommand{\proofname}{\textsf{証明}}
  \renewcommand{\postpartname}{章}
  \renewcommand{\thesection}{\thepart.\arabic{section}}
  \renewcommand{\thepart}{\arabic{part}}
  \makeatletter\renewcommand{\theequation}{\thesection.\arabic{equation}}\@addtoreset{equation}{section}\makeatother

  \newcommand{\octopuspart}[1]{\newpage\part{#1}\setcounter{section}{0}\vspace{3\baselineskip}}

  \DeclareMathOperator{\dcup}{\dot{\cup}}
  \begin{document}
\fi

\section{連結性}
$(X, \mathcal{O})$を位相空間とする。

\sukima \midashi{復習}(省略)

\begin{teigi}[連結]
  \begin{flushleft}
    \begin{tabular}{llll}
      $X$:\nw{連結} & $\defines$ & $\mathcal{O} \cap \mathfrak{A} = \sets{\emptyset, X}$ & ($X$上開かつ閉なる集合は$\emptyset$,$X$のみ)\\
      $A \subseteq X$:連結 & $\defines$ & $A$が$X$の部分位相空間として連結である
    \end{tabular}
  \end{flushleft}
\end{teigi}

すなわち,$X$が連結{\bf でない}ことが次の二つの同値な条件で特徴づけられる。:
\begin{center}
  $\exists O \in \mathcal{O} \colon \quad O \neq \emptyset, \quad X \setminus O \in \mathcal{O}$ \\
  $X$は非交で非空な開集合の和集合に分解される
\end{center}

\begin{prop}
  $f \colon X \longrightarrow Y$:連続,$A \subseteq X$:連結のとき,$f(A)$:連結
\end{prop}

「連結性」は連続写像によって「不変」な性質であり,
位相的な性質であるといえる。(同相写像によって不変)

\begin{proof}
  $B \subseteq f(A)$を開かつ閉な集合とする。
  \begin{align}
    & \exists G: Y\text{上開}, \quad \exists F: Y\text{上閉}, \quad B = G \cap f(A) = F \cap f(A) \\
    \to & f^{-1}(B) = f^{-1} (G) \cap f^{-1}(f(A)) = f^{-1}(F) \cap f^{-1}(f(A)) \\
    \to & f^{-1}(B) \cap A = f^{-1}(G) \cap A = f^{-1}(F) \cap A
  \end{align}
  $f$の連続性より$f^{-1}(G)$:開,$f^{-1}(F)$:閉。
  よって,相対位相の定義より,$f^{-1}(G) \cap A$:$A$上開,$f^{-1}(F) \cap A$:$A$上閉。
  $f^{-1}(B) \cap A$は$A$の相対位相で開かつ閉。
  $A$の連結性より$f^{-1}(B) \cap A = \emptyset \: \text{or} \: A$である。

  $f^{-1} (B) \cap A = \emptyset$なら$B = \emptyset$,
  $f^{-1} (B) \cap A = A$なら$f(f^{-1}(B)) \cap f(A) = f(A)$であるから$B = f(A)$である。
  つまり,$f(A)$上で開かつ閉な集合は$\emptyset$か$f(A)$であるので,$f(A)$は連結である。
\end{proof}

\begin{prop}
  $A, B \subseteq X$に対し,$A \subseteq B \subseteq \overline{A}$であるとする。
  このとき,$A$:連結$\Longrightarrow B$:連結である。
  特に,連結な集合の閉包は連結である。
\end{prop}

\begin{proof}
  $B' \subseteq B$:$B$上開かつ閉とする。
  \begin{equation}
    \exists G: X\text{上開},\quad \exists F: X\text{上閉}, \quad B' = G \cap B = F \cap B
  \end{equation}
  であり,$A \subseteq B$より
  $A \cap B' = G \cap A = F \cap A$である。よって,$A \cap B'$は$A$上開かつ閉である。
  よって,$A$の連結性から$A \cap B' = \emptyset \: \text{or} \: A$である。

  $A \cap B' = \emptyset$のとき,$G \cap A = \emptyset$であり,$G \cap \overline{A} = \emptyset$,よって,$B' = \emptyset$である。

  $A \cap B' = A$のとき,$F \supseteq A$であるから,$F \supseteq \overline{A} \supseteq B$であり,$B' = F \cap B = B$である。

  これより$B$の連結性が従う。
\end{proof}

\begin{prop}
  $A,B$:連結,$A \cap B \neq \emptyset$とする。このとき$A \cup B$:連結である。
\end{prop}

\begin{proof}
  $N \subseteq A \cap B$:$A \cup B$上開かつ閉とする。
  \begin{equation}
    \exists G: X\text{上開},\quad \exists F: X\text{上閉}, \quad N = G \cap (A \cup B) = F \cap (A \cup B)
  \end{equation}
  特に$N \cap A = G \cap A = F \cap A$であり,$N \cap A$は$A$上開かつ閉。
  $N \cap B = G \cap B = F \cap B$であり,$N \cap B$は$B$上開かつ閉。
  よって,$N \cap A = \emptyset \: \text{or} \: A$,$N \cap B = \emptyset \: \text{or} \: B$である。

  もし$N \cap A = \emptyset$ならば$A \cap B \neq \emptyset$より$N \cap B = B$となることはない。よって,$N = \emptyset$である。

  もし$N \cap A = A$ならば$N \supseteq A \supseteq A \cap B \neq \emptyset$より$N \cap B = B$である。よって,$N = A \cup B$である。
\end{proof}

\expandafter\ifx\csname readornot\endcsname\relax
  \end{document}
\fi % 連結性
\expandafter\ifx\csname readornot\endcsname\relax
  \documentclass[uplatex]{jsarticle}
  \usepackage{octopus}
  \usepackage{url}

  \renewcommand{\proofname}{\textsf{証明}}
  \renewcommand{\postpartname}{章}
  \renewcommand{\thesection}{\thepart.\arabic{section}}
  \renewcommand{\thepart}{\arabic{part}}
  \makeatletter\renewcommand{\theequation}{\thesection.\arabic{equation}}\@addtoreset{equation}{section}\makeatother

  \newcommand{\octopuspart}[1]{\newpage\part{#1}\setcounter{section}{0}\vspace{3\baselineskip}}

  \DeclareMathOperator{\dcup}{\dot{\cup}}
  \begin{document}
\fi

\section{コンパクト性}
$(X, \mathcal{O})$:位相空間,$A \subseteq X$とする。

\begin{teigi}[被覆]
  $\mathcal{C} \subseteq 2^{X}$:$A$の\nw{被覆} $\defines$ ${\displaystyle A \subseteq \bigcup_{C \in \mathcal{C}} C}$

  特に,$\mathcal{C} \subseteq \mathcal{O}$のとき,$\mathcal{C}$を\nw{開被覆}という。
\end{teigi}

\begin{teigi}[コンパクト]
  \midashi{(1)} $A \subseteq X$:\nw{コンパクト} $\defines$ $\forall \mathcal{C}$:$A$の開被覆,$\exists O_{1}, \dots, O_{k} \in \mathcal{C}$,${\displaystyle A \subseteq \bigcup_{j=1}^{k} O_{j}}$

  \midashi{(2)} $X$がコンパクトのとき,$(X, \mathcal{O})$をコンパクト空間という。
\end{teigi}

$A \subseteq X$がコンパクトであることは,標語的に「任意の開被覆は有限部分開被覆を含む」ということができる。
また,$A \subseteq X$がコンパクトであることは,$(A, \mathcal{O}_{A})$がコンパクト空間であることと同値である。
ただし,$\mathcal{O}_{A}$は相対位相。

\begin{hodai}
  $A_{1}, \dots, A_{k} \subseteq X$ :コンパクト $\Longrightarrow$ $A_{1} \cup \dots \cup A_{k}$:コンパクト
\end{hodai}

\begin{hodai}
  $(X, \mathcal{O})$:コンパクト $\Longrightarrow$ $A \in \mathfrak{A}$:コンパクト
\end{hodai}

\begin{proof}
  $A$の開被覆と$X \setminus A$が$X$の開被覆になっていることから従う。
\end{proof}

\expandafter\ifx\csname readornot\endcsname\relax
  \end{document}
\fi % コンパクト性

\octopuspart{位相幾何}
連続変形に対する不変性

\expandafter\ifx\csname readornot\endcsname\relax
  \documentclass[uplatex]{jsarticle}
  \usepackage{octopus}
  \usepackage{url}

  \renewcommand{\proofname}{\textsf{証明}}
  \renewcommand{\postpartname}{章}
  \renewcommand{\thesection}{\thepart.\arabic{section}}
  \renewcommand{\thepart}{\arabic{part}}
  \renewcommand{\restriction}[2]{\left. #1 \right|_{#2}}
  \makeatletter\renewcommand{\theequation}{\thesection.\arabic{equation}}\@addtoreset{equation}{section}\makeatother

  \newcommand{\octopuspart}[1]{\newpage\part{#1}\setcounter{section}{0}\vspace{3\baselineskip}}

  \DeclareMathOperator{\dcup}{\dot{\cup}}
  \begin{document}
\fi

\section{ホモトピー}

\midashi{2つの空間が「同じ形」をしているとはどういうことか?}

$\xrightarrow{\text{ans}}$位相空間$X,Y$が位相同型 $\defines$ $\exists f \colon X \longrightarrow Y$:連続全単射,$f^{-1}$:連続

例:ドーナツとコーヒーカップ

$\xrightarrow{\text{ans}}$連続変形で移り合う

例:太字のAと細字のAと丸 %図

例:ディスクと一点(明らかにこの2つは位相同型ではない)

このような「空間の連続変形」を定式化したい

\sukima \midashi{\large 変形レトラクション}

$X$:位相空間,$A \subseteq X$とする。

\begin{teigi}[変形レトラクション]
  $X$から$A$への\nw{変形レトラクション(deformation retraction)}とは,
  次の条件を満たす$\sets{f_{t} \colon X \longrightarrow X}_{t \in [0,1]}$のこととする。:
  \begin{itemize}
    \vspace{-0.5\baselineskip}
    \item $f_{0} = \mathrm{id}_{X}$
    \item $f_{1}(X) = A$
    \item $\restriction{f_{t}}{A} = \mathrm{id}_{A}$($\forall t \in [0,1]$)
    \item $\renewcommand{\arraystretch}{0.8} \begin{array}{c@{\:}c@{\:}c@{\:}c}
               & X \times [0,1] & \longrightarrow & X \\
      F \colon & \vin           &                 & \vin \\
               & (x,t)          & \longmapsto     & f_{t}(x)
    \end{array} \renewcommand{\arraystretch}{1.3}$は連続写像
  \end{itemize}
\end{teigi}

\expandafter\ifx\csname readornot\endcsname\relax
  \end{document}
\fi % ホモトピー
\expandafter\ifx\csname readornot\endcsname\relax
 \documentclass{jsarticle}
 \begin{document}
\fi

\section{基本群}

\expandafter\ifx\csname readornot\endcsname\relax
  \end{document}
\fi % 基本群
\expandafter\ifx\csname readornot\endcsname\relax
 \documentclass{jsarticle}
 \begin{document}
\fi

\section{被覆空間}

\expandafter\ifx\csname readornot\endcsname\relax
  \end{document}
\fi % 被覆空間
\expandafter\ifx\csname readornot\endcsname\relax
  \documentclass[uplatex]{jsarticle}
  \usepackage{octopus}
  \usepackage{url}

  \renewcommand{\proofname}{\textsf{証明}}
  \renewcommand{\postpartname}{章}
  \renewcommand{\thesection}{\thepart.\arabic{section}}
  \renewcommand{\thepart}{\arabic{part}}
  \makeatletter\renewcommand{\theequation}{\thesection.\arabic{equation}}\@addtoreset{equation}{section}\makeatother

  \newcommand{\octopuspart}[1]{\newpage\part{#1}\setcounter{section}{0}\vspace{3\baselineskip}}

  \DeclareMathOperator{\dcup}{\dot{\cup}}
  \DeclareMathOperator{\conv}{conv}
  \begin{document}
\fi

\section{ホモロジー}

\midashi{\large ホモロジー群の導入}\\
\renewcommand{\arraystretch}{1.0}
\begin{tabular}{ll}
  基本群 & ・「2次元の穴」が検出できる \\
  & ・非可換 \\
  & ・計算は一般に難しい \\
  ホモロジー群 & ・「高次元の穴」が検出できる \\
  & ・可換 \\
  & ・計算が比較的容易(計算機にも相性が良い)
\end{tabular}
\renewcommand{\arraystretch}{1.3}

\sukima \midashi{\large 準備:単体}

\begin{teigi}[アフィン独立]
  $v_{0}, v_{1}, \dots, v_{k} \in \mathbb{R}^{n}$が\nw{アフィン独立} $\defines$
  ${\displaystyle \sum_{i=0}^{k} \lambda_{i} v_{i} = 0}$ かつ ${\displaystyle \sum_{i=0}^{k} \lambda_{i} = 0}$ならば,$\lambda_{0} = \lambda_{1} = \dots = \lambda_{k} = 0$
\end{teigi}

アフィン独立であることは,
\begin{center}
  $v_{1} - v_{0}, v_{2} - v_{0}, \dots, v_{k} - v_{0}$が一次独立
\end{center}
であることと同値である。

\begin{rei}
  $\mathbb{R}^{2}$上で,

  \begin{center}
    \begin{tikzpicture}
      \filldraw (0,0) circle [radius=0.05] node[left] {$v_{0}$};
      \filldraw (1.5,-0.5) circle [radius=0.05] node[left] {$v_{1}$};
      \filldraw (0.7,1.2) circle [radius=0.05] node[left] {$v_{2}$};
      \node at (0.7,-2) {アフィン独立};

      \filldraw (5,-1) circle [radius=0.05] node[left] {$v_{0}$};
      \filldraw (5.9,0.2) circle [radius=0.05] node[left] {$v_{1}$};
      \filldraw (6.8,1.4) circle [radius=0.05] node[left] {$v_{2}$};
      \draw[dashed] (4.7,-1.4) -- (7.1,1.8);
      \node at (5.9,-2) {アフィン独立でない};

      \filldraw (11.5,0.3) circle [radius=0.05] node[above right] {$v_{0}$};
      \filldraw (10.5,-0.5) circle [radius=0.05] node[below] {$v_{1}$};
      \filldraw (12.5,-0.5) circle [radius=0.05] node[below] {$v_{2}$};
      \filldraw (11.5,1.3) circle [radius=0.05] node[above] {$v_{3}$};
      \node at (11.5,-2) {アフィン独立でない};
    \end{tikzpicture}
  \end{center}
\end{rei}

\begin{teigi}[$n$次元単体]
  $\Delta^{n}$:\nw{$n$次元単体(simplex)} $\defines$ 
  アフィン独立な$v_{0},v_{1},\dots,v_{n}$が存在して次を満たすような$\Delta^{n}$のこと:
  \begin{equation}
    \Delta^{n} = \conv \sets{v_{0}, v_{1}, \dots, v_{n}} = \sets{\sum_{i=0}^{n} \lambda_{i} v_{i} | \sum_{i=0}^{n} \lambda_{i} = 1, \quad \lambda_{i} \ge 0}
  \end{equation}
\end{teigi}

$n$次元単体を\nw{$n$--単体}ともいう。
また,定義中に登場した$\conv\sets{v_{0}, \dots, v_{n}}$は\nw{凸包}を表しており,
\nw{凸結合}と呼ばれる${\displaystyle \sum_{i=0}^{n} \lambda_{i} = 1}$かつ$\lambda_{i} \ge 0$を満たす$\lambda_{i}$らによって
${\displaystyle \sum_{i=0}^{n} \lambda_{i} v_{i}}$と表される点全体のなす集合のことをいう。

\begin{rei}
  0--単体から3--単体までを以下に示す。
  \begin{center}
    \begin{tikzpicture}
      % 0-単体
      \filldraw (0,0) circle [radius=0.05];
      % 1-単体
      \filldraw (1.9, -0.3) circle [radius=0.05] -- (2.1,0.3) circle [radius=0.05];
      % 2-単体
      \filldraw[fill=gray!50] (3.7, -0.3) -- (4.3, -0.2) -- (4,0.3) -- (3.7, -0.3) -- cycle;
      \fill (3.7, -0.3) circle [radius=0.05] (4.3,-0.2) circle [radius=0.05] (4,0.3) circle [radius=0.05];
      \filldraw[fill=gray!50, opacity=0.5,draw=black, opacity=1] (5.5,-0.3) -- (6.3,-0.5) -- (6.5,0.2) -- (5.9,0.5) -- (5.5,-0.3) -- cycle;
      % 3-単体
      \draw (6.3,-0.5) -- (5.9,0.5);
      \draw[dashed] (5.5,-0.3) -- (6.5,0.2);
      \fill (5.5,-0.3) circle [radius=0.05] (6.3,-0.5) circle [radius=0.05] (6.5,0.2) circle [radius=0.05] (5.9,0.5) circle [radius=0.05];
      \node at (0,-1) {$\Delta^{0}$};
      \node at (2,-1) {$\Delta^{1}$};
      \node at (4,-1) {$\Delta^{2}$};
      \node at (6,-1) {$\Delta^{3}$};
    \end{tikzpicture}
  \end{center}
\end{rei}

以下,頂点$v_{0}, v_{1}, \dots, v_{n}$による単体を$[v_{0}, v_{1}, \dots, v_{n}]$と書く。

また,\nw{面} $[v_{0}, \dots, \hat{v_{i}}, \dots, v_{n}]$を
\begin{equation}
  [v_{0}, \dots, \hat{v_{i}}, \dots, v_{n}] := [v_{0}, \dots, v_{i-1}, v_{i+1}, \dots, v_{n}]
\end{equation}
で定める。

\begin{center}
  \begin{tikzpicture}
    \filldraw[fill=gray!50] (-0.5, -0.5) node[left]{$v_{0}$} -- (0.5, -0.4) node[right]{$v_{1}$} -- (0,0.5) node[above]{$v_{2}$} -- (-0.5, -0.5) -- cycle;
    \fill (-0.5, -0.5) circle [radius=0.05] (0.5,-0.4) circle [radius=0.05] (0,0.5) circle [radius=0.05];
    \draw[<-] node at (0,-1) {$[v_{0}, v_{1}, v_{2}]$} (0.4,0.1) -- (0.8,0.4) node[right] {面$[v_{1},v_{2}]$};
    \filldraw[fill=gray!50, opacity=0.5,draw=black, opacity=1] (7.3,-0.8) node[below]{$v_{0}$} -- (7.7,0) node[right]{$v_{1}$} -- (6.9,0.5) node[above]{$v_{3}$} -- (7.3,-0.8) -- cycle;
    \draw (6.3,-0.4) node[left]{$v_{2}$} -- (7.3,-0.8) (6.3,-0.4) -- (6.9,0.5);
    \draw[dashed] (6.3,-0.4) -- (7.7,0);
    \fill (6.3,-0.4) circle [radius=0.05] (7.3,-0.8) circle [radius=0.05] (7.7,0) circle [radius=0.05] (6.9,0.5) circle [radius=0.05];
    \draw[<-] (7.4,0) -- (8,0.5) node[right]{面$[v_{0}, v_{1}, v_{3}]$};
  \end{tikzpicture}
\end{center}

\expandafter\ifx\csname readornot\endcsname\relax
  \end{document}
\fi % ホモロジー
\expandafter\ifx\csname readornot\endcsname\relax
  \documentclass[uplatex]{jsarticle}
  \usepackage{octopus}
  \usepackage{url}
  \usetikzlibrary{calc}

  \renewcommand{\proofname}{\textsf{証明}}
  \renewcommand{\postpartname}{章}
  \renewcommand{\thesection}{\thepart.\arabic{section}}
  \renewcommand{\thepart}{\arabic{part}}
  \makeatletter\renewcommand{\theequation}{\thesection.\arabic{equation}}\@addtoreset{equation}{section}\makeatother

  \newcommand{\octopuspart}[1]{\newpage\part{#1}\setcounter{section}{0}\vspace{3\baselineskip}}

  \renewcommand{\restriction}[2]{\left. #1 \right|_{#2}}
  \DeclareMathOperator{\dcup}{\dot{\cup}}
  \DeclareMathOperator{\conv}{conv}
  \DeclareMathOperator{\Image}{Im}
  \DeclareMathOperator{\Kernel}{Ker}
  \begin{document}
\fi

\section{ホモロジーの計算}

%%%%%%%%%% 図 Δ複体構造の例 X = トーラス %%%%%%%%%%  %% 文章を入れたら消してください
\begin{center}
	\begin{tikzpicture}
		% 本図
		%% 面を塗る
		\fill[red!20] (0,0) -- (2,2) -- (0,2) -- cycle;
		\fill[blue!20] (0,0) -- (2,2) -- (2,0) -- cycle;
		%% 点の名前もろもろ
		\fill \point{(0,0)}{below left}{$v$} \point{(2,0)}{below right}{$v$} \point{(0,2)}{above left}{$v$} \point{(2,2)}{above right}{$v$};
		\node at (1,0) [below]{$b$} node at (0,1) [left]{$a$} node at (1,2) [above]{$b$} node at (2,1) [right]{$a$} node at (1,1) [left]{$c$};
		\node at (0.5,1.5){$U$} node at (1.5,0.5){$L$};
		%% 矢印つき
		\midwayarrow{(0,0)}{(0,2)} \midwayarrow{(0,0)}{(2,0)} \midwayarrow{(0,2)}{(2,2)} \midwayarrow{(2,0)}{(2,2)} \midwayarrow{(0,0)}{(2,2)}
		% 取り巻き作り
		%% 点
		\fill \point{(-1.5,2.2)}{}{};
		\draw[->] (-1.4,2.1) to[bend right] node[midway,above]{$\sigma_{v}$} (-0.1,1.9);
		%% 辺
		\fill \point{(3.5,1)}{}{} \point{(3.5,2)}{}{} \point{(1.25,-1.5)}{}{} \point{(2.25,-1.5)}{}{} \point{(-0.25,-1.5)}{}{} \point{(0.75,-1.5)}{}{};
		\midwayarrow{(3.5,1)}{(3.5,2)} \midwayarrow{(1.25,-1.5)}{(2.25,-1.5)} \midwayarrow{(-0.25,-1.5)}{(0.75,-1.5)}
		\draw[->] (3.4,1.5) to[bend left] node[midway,above]{$\sigma_{a}$} (2.1,1.5);
		\draw[->] (1.75,-1.4) to[bend right] node[midway,right]{$\sigma_{b}$} (1.5,-0.1);
		\draw[->] (0.25,-1.4) to[bend left] node[midway,left]{$\sigma_{c}$} (0.5,-0.1);
		%% 面
		\fill[red!20] (-3,-2.5) -- (-1,-0.5) -- (-3,-0.5) -- cycle;
		\fill[blue!20] (2.5,-2) -- (4.5,0) -- (4.5,-2) -- cycle;
		\fill \point{(-3,-2.5)}{}{} \point{(-1,-0.5)}{}{} \point{(-3,-0.5)}{}{} \point{(2.5,-2)}{}{} \point{(4.5,0)}{}{} \point{(4.5,-2)}{}{};
		\midwayarrow{(-3,-2.5)}{(-1,-0.5)} \midwayarrow{(-3,-2.5)}{(-3,-0.5)} \midwayarrow{(-3,-0.5)}{(-1,-0.5)} \midwayarrow{(2.5,-2)}{(4.5,0)} \midwayarrow{(2.5,-2)}{(4.5,-2)} \midwayarrow{(4.5,-2)}{(4.5,0)}
		\draw[->] (-1.5,-0.4) -- node[pos=.3, above]{$\sigma_{U}$} (0.4,0.9);
		\draw[->] (4,-0.25) -- node[pos=.3, above]{$\sigma_{L}$} (1.8,0.5);
	\end{tikzpicture}
\end{center}
%%%%%%%%%% %% 文章を入れたら消してください

%%%%%%%%%% 図 Δ複体構造の例 X = P^{2} %%%%%%%%%%  %% 文章を入れたら消してください
\begin{center}
	\begin{tikzpicture}
		\fill[red!20] (0,0) -- (2,2) -- (0,2) -- cycle;
		\fill[blue!20] (0,0) -- (2,2) -- (2,0) -- cycle;
		\fill \point{(0,0)}{below left}{$v$} \point{(2,0)}{below right}{$w$} \point{(0,2)}{above left}{$w$} \point{(2,2)}{above right}{$v$};
		\node at (1,0) [below]{$b$} node at (0,1) [left]{$a$} node at (1,2) [above]{$b$} node at (2,1) [right]{$a$} node at (1,1) [left]{$c$};
		\node at (0.5,1.5){$U$} node at (1.5,0.5){$L$};
		\midwayarrow{(0,0)}{(2,0)} \midwayarrow{(0,0)}{(0,2)} \midwayarrow{(0,0)}{(2,2)} \midwayarrow{(2,2)}{(2,0)} \midwayarrow{(2,2)}{(0,2)}
	\end{tikzpicture}
\end{center}
%%%%%%%%%% %% 文章を入れたら消してください

%%%%%%%%%% 図 抽象的単体的複体 %%%%%%%%%%  %% 文章を入れたら消してください
\begin{center}
	\begin{tikzpicture}
		\filldraw[fill=black!30] (0,0) -- (1.4,-1) -- (1.4,1) -- cycle;
		\draw (1.4,1) -- (2.5,0) -- (3.5,0);
		\draw (1.4,-1) -- (2.5,0);
		\fill \point{(0,0)}{left}{0} \point{(1.4,-1)}{below left}{1} \point{(1.4,1)}{above left}{2} \point{(2.5,0)}{below right}{3} \point{(3.5,0)}{below}{4};
	\end{tikzpicture}
\end{center}
%%%%%%%%%% %% 文章を入れたら消してください

%%%%%%%%%% 図 単体的ホモロジーの例 P^{2} %%%%%%%%%%  %% 文章を入れたら消してください
\begin{center}
	\begin{tikzpicture}
		\fill[red!20] (0,0) -- (2,2) -- (0,2) -- cycle;
		\fill[blue!20] (0,0) -- (2,2) -- (2,0) -- cycle;
		\fill \point{(0,0)}{below left}{$v$} \point{(2,0)}{below right}{$w$} \point{(0,2)}{above left}{$w$} \point{(2,2)}{above right}{$v$};
		\node at (1,0) [below]{$b$} node at (0,1) [left]{$a$} node at (1,2) [above]{$b$} node at (2,1) [right]{$a$} node at (1,1) [left]{$c$};
		\node at (0.5,1.5){$U$} node at (1.5,0.5){$L$};
		\midwayarrow{(0,0)}{(2,0)} \midwayarrow{(0,0)}{(0,2)} \midwayarrow{(0,0)}{(2,2)} \midwayarrow{(2,2)}{(2,0)} \midwayarrow{(2,2)}{(0,2)}
	\end{tikzpicture}
\end{center}

$\Delta_{0} = \sets{n_{v} \tikz[baseline=0] \fill \point{(0,0.1)}{below}{\footnotesize $v$};
+ n_{w} \tikz[baseline=0] \fill \point{(0,0.1)}{below}{\footnotesize $w$}; | n_{v}, n_{w} \in \mathbb{Z}}$

$\Delta_{1} = \sets{n_{a} \,
\begin{tikzpicture}[baseline=-2.5]
	\fill \point{(0,0.3)}{}{} \point{(0,-0.3)}{}{};
	\midwayarrow{(0,0.3)}{(0,-0.3)};
	\node at (0,0) [right]{\footnotesize $a$};
\end{tikzpicture}
+ n_{b} \,
\begin{tikzpicture}[baseline=-2.5]
	\fill \point{(0,0)}{}{} \point{(0.6,0)}{}{};
	\midwayarrow{(0,0)}{(0.6,0)};
	\node at (0.3,0) [below]{\footnotesize $b$};
\end{tikzpicture} | n_{a}, n_{b} \in \mathbb{Z}}$

$\Delta_{2} = \sets{n_{U} \,
\begin{tikzpicture}[baseline=-2.5]
	\fill[red!20] (0,-0.3) -- (0.6,0.3) -- (0,0.3) -- cycle;
	\fill \point{(0,-0.3)}{}{} \point{(0.6,0.3)}{}{} \point{(0,0.3)}{}{};
	\midwayarrow{(0,-0.3)}{(0.6,0.3)}; \midwayarrow{(0.6,0.3)}{(0,0.3)}; \midwayarrow{(0,-0.3)}{(0,0.3)};
	\node at (0.18,0.13){\footnotesize $U$};
\end{tikzpicture}
+ n_{L} \,
\begin{tikzpicture}[baseline=-2.5]
	\fill[blue!20] (0,-0.3) -- (0.6,-0.3) -- (0.6,0.3) -- cycle;
	\fill \point{(0,-0.3)}{}{} \point{(0.6,-0.3)}{}{} \point{(0.6,0.3)}{}{};
	\midwayarrow{(0,-0.3)}{(0.6,-0.3)}; \midwayarrow{(0.6,0.3)}{(0.6,-0.3)}; \midwayarrow{(0,-0.3)}{(0.6,0.3)};
	\node at (0.42,-0.13){\footnotesize $L$};
\end{tikzpicture} | n_{U}, n_{L} \in \mathbb{Z}}$

2-チェインの例:$3\, \begin{tikzpicture}[baseline=-2.5]
	\fill[red!20] (0,-0.3) -- (0.6,0.3) -- (0,0.3) -- cycle;
	\fill \point{(0,-0.3)}{}{} \point{(0.6,0.3)}{}{} \point{(0,0.3)}{}{};
	\midwayarrow{(0,-0.3)}{(0.6,0.3)}; \midwayarrow{(0.6,0.3)}{(0,0.3)}; \midwayarrow{(0,-0.3)}{(0,0.3)};
	\node at (0.18,0.13){\footnotesize $U$};
\end{tikzpicture}
-2 \,
\begin{tikzpicture}[baseline=-2.5]
	\fill[blue!20] (0,-0.3) -- (0.6,-0.3) -- (0.6,0.3) -- cycle;
	\fill \point{(0,-0.3)}{}{} \point{(0.6,-0.3)}{}{} \point{(0.6,0.3)}{}{};
	\midwayarrow{(0,-0.3)}{(0.6,-0.3)}; \midwayarrow{(0.6,0.3)}{(0.6,-0.3)}; \midwayarrow{(0,-0.3)}{(0.6,0.3)};
	\node at (0.42,-0.13){\footnotesize $L$};
\end{tikzpicture}$
%%%%%%%%%% %% 文章を入れたら消してください

%%%%%%%%%% 図 ホモロジー群の定義の後の図式 %%%%%%%%%%  %% 文章を入れたら消してください
\begin{center}
	\begin{tikzpicture}
	  \draw[->] (0,0.5) --node[midway,left]{$i:包含写像$} (0,-0.5);
	  \draw[->] (3,0.5) --node[midway,left]{$i$} (3,-0.5);
	  \draw[->] (6,0.5) --node[midway,left]{$i$} (6,-0.5);
  
	  \draw[->] (-2,1) --node[midway,above]{$\partial_{n+2}^{\Delta}$} (-1,1);
	  \draw[->] (1,1) --node[midway,above]{$\partial_{n+1}^{\Delta}$} (2,1);
	  \draw[->] (4,1) --node[midway,above]{$\partial_{n}^{\Delta}$} (5,1);
	  \draw[->] (7,1) --node[midway,above]{$\partial_{n-1}^{\Delta}$} (8,1);
  
	  \draw[->] (-2,-1) --node[midway,below]{$\partial_{n+2}$} (-1,-1);
	  \draw[->] (1,-1) --node[midway,below]{$\partial_{n+1}$} (2,-1);
	  \draw[->] (4,-1) --node[midway,below]{$\partial_{n}$} (5,-1);
	  \draw[->] (7,-1) --node[midway,below]{$\partial_{n-1}$} (8,-1);
  
	  \node at (-3,1) {$\cdots$} node at (0,1) {$\Delta_{n+1}$} node at (3,1) {$\Delta_{n}$} node at (6,1) {$\Delta_{n-1}$} node at (9,1) {$\cdots$};
	  \node at (-3,-1) {$\cdots$} node at (0,-1) {$C_{n+1}$} node at (3,-1) {$C_{n}$} node at (6,-1) {$C_{n-1}$} node at (9,-1) {$\cdots$};
	\end{tikzpicture}
\end{center}
%%%%%%%%%% %% 文章を入れたら消してください

%%%%%%%%%% 図 Cor. Hn(X) = 0 の後の例の図 %%%%%%%%%%  %% 文章を入れたら消してください
\begin{tikzpicture}
	\draw (0,0) circle [radius=1];
	\draw[<-] (0.1,-1.1) to[bend left] (0.7,-1.7);
	\fill \point{(0,-1)}{below}{$r$} \point{(0.8,-1.8)}{below}{$r$} \point{(2,-0.8)}{}{} \point{(2,0.8)}{}{};
	\node at (2,0) [right]{$e$} node at (0,1) [above right]{$e$} node at (0,1){\rotatebox{90}{\footnotesize $\blacktriangle$}};
	\midwayarrow{(2,-0.8)}{(2,0.8)};
	\draw[<-] (1.25,0) -- (1.75,0);
\end{tikzpicture}

\begin{tikzpicture}
  \filldraw[fill=black!20] (0,0) circle [radius=1];
	\fill \point{(90:1)}{above}{$u$} \point{(-30:1)}{below right}{$v$} \point{(210:1)}{below left}{$w$};
	\node at (30:1){\rotatebox{210}{\footnotesize $\blacktriangle$}} node at (270:1){\rotatebox{90}{\footnotesize $\blacktriangle$}} node at (150:1){\rotatebox{150}{\footnotesize $\blacktriangle$}};
  \draw[<-] (1.5,0) -- (2.5,0);
  \fill[black!20] ($(4,-0.25) + (90:1)$) -- ($(4,-0.25) + (-30:1)$) -- ($(4,-0.25) + (210:1)$) -- cycle;
	\fill \point{($(4,-0.25) + (90:1)$)}{above}{$u$} \point{($(4,-0.25) + (-30:1)$)}{below right}{$v$} \point{($(4,-0.25) + (210:1)$)}{below left}{$w$};
	\midwayarrow{($(4,-0.25) + (90:1)$)}{($(4,-0.25) + (-30:1)$)} \midwayarrow{($(4,-0.25) + (-30:1)$)}{($(4,-0.25) + (210:1)$)} \midwayarrow{($(4,-0.25) + (90:1)$)}{($(4,-0.25) + (210:1)$)}
\end{tikzpicture}
%%%%%%%%%% %% 文章を入れたら消してください




%%%%%%%%%% 図 Cor. Hn(X) = 0 の後の例の図 S^{2} %%%%%%%%%%  %% 文章を入れたら消してください
\begin{tikzpicture}
  %% 面を塗る
  \fill[red!20] (0,0) -- (2,2) -- (0,2) -- cycle;
  \fill[blue!20] (0,0) -- (2,2) -- (2,0) -- cycle;
  %% 点の名前もろもろ
  \fill \point{(0,0)}{below left}{$u$} \point{(2,0)}{below right}{$v$} \point{(0,2)}{above left}{$v$} \point{(2,2)}{above right}{$w$};
  \node at (1,0) [below]{$b$} node at (0,1) [left]{$b$} node at (1,2) [above]{$a$} node at (2,1) [right]{$a$} node at (1,1) [left]{$c$};
  \node at (0.5,1.5){$U$} node at (1.5,0.5){$L$};
  %% 矢印つき
  \midwayarrow{(0,0)}{(0,2)} \midwayarrow{(0,0)}{(2,0)} \midwayarrow{(0,2)}{(2,2)} \midwayarrow{(2,0)}{(2,2)} \midwayarrow{(0,0)}{(2,2)}
\end{tikzpicture}
%%%%%%%%%% %% 文章を入れたら消してください




%%%%%%%%%% 図 Cor. Hn(X) = 0 の後の例の図 トーラス %%%%%%%%%%  %% 文章を入れたら消してください
\begin{tikzpicture}    
  %% 面を塗る
  \fill[red!20] (0,0) -- (2,2) -- (0,2) -- cycle;
  \fill[blue!20] (0,0) -- (2,2) -- (2,0) -- cycle;
  %% 点の名前もろもろ
  \fill \point{(0,0)}{below left}{$v$} \point{(2,0)}{below right}{$v$} \point{(0,2)}{above left}{$v$} \point{(2,2)}{above right}{$v$};
  \node at (1,0) [below]{$b$} node at (0,1) [left]{$a$} node at (1,2) [above]{$b$} node at (2,1) [right]{$a$} node at (1,1) [left]{$c$};
  \node at (0.5,1.5){$U$} node at (1.5,0.5){$L$};
  %% 矢印つき
  \midwayarrow{(0,0)}{(0,2)} \midwayarrow{(0,0)}{(2,0)} \midwayarrow{(0,2)}{(2,2)} \midwayarrow{(2,0)}{(2,2)} \midwayarrow{(0,0)}{(2,2)}
\end{tikzpicture}
%%%%%%%%%% %% 文章を入れたら消してください

%%%%%%%%%% 図 X = P^{2} %%%%%%%%%%  %% 文章を入れたら消してください
\begin{tikzpicture}
\fill[red!20] (0,0) -- (2,2) -- (0,2) -- cycle;
\fill[blue!20] (0,0) -- (2,2) -- (2,0) -- cycle;
\fill \point{(0,0)}{below left}{$u$} \point{(2,0)}{below right}{$v$} \point{(0,2)}{above left}{$v$} \point{(2,2)}{above right}{$u$};
\node at (1,0) [below]{$b$} node at (0,1) [left]{$a$} node at (1,2) [above]{$b$} node at (2,1) [right]{$a$} node at (1,1) [left]{$c$};
\node at (0.5,1.5){$U$} node at (1.5,0.5){$L$};
\midwayarrow{(0,0)}{(2,0)} \midwayarrow{(0,0)}{(0,2)} \midwayarrow{(0,0)}{(2,2)} \midwayarrow{(2,2)}{(2,0)} \midwayarrow{(2,2)}{(0,2)}
\end{tikzpicture}
%%%%%%%%%% %% 文章を入れたら消してください

\expandafter\ifx\csname readornot\endcsname\relax
  \end{document}
\fi % ホモロジーの計算 1
\expandafter\ifx\csname readornot\endcsname\relax
  \documentclass[uplatex]{jsarticle}
  \usepackage{octopus}
  \usepackage{url}
  \usetikzlibrary{calc}

  \renewcommand{\proofname}{\textsf{証明}}
  \renewcommand{\postpartname}{章}
  \renewcommand{\thesection}{\thepart.\arabic{section}}
  \renewcommand{\thepart}{\arabic{part}}
  \makeatletter\renewcommand{\theequation}{\thesection.\arabic{equation}}\@addtoreset{equation}{section}\makeatother

  \newcommand{\octopuspart}[1]{\newpage\part{#1}\setcounter{section}{0}\vspace{3\baselineskip}}

  \renewcommand{\restriction}[2]{\left. #1 \right|_{#2}}
  \DeclareMathOperator{\dcup}{\dot{\cup}}
  \DeclareMathOperator{\conv}{conv}
  \DeclareMathOperator{\Image}{Im}
  \DeclareMathOperator{\Kernel}{Ker}
  \DeclareMathOperator{\diag}{diag}
  \DeclareMathOperator{\rank}{rank}
  \begin{document}
  \fi

\section{ホモロジーの計算 2}
\newcommand{\BigZero}{\kern3pt \hbox{\huge \strut 0}}

\begin{teigi}[自由$\mathbb{Z}$加群]
  $G$が\nw{自由$\mathbb{Z}$加群}であるとは,$\mathbb{Z}$上の加群$G$が,
  $G \simeq \mathbb{Z}^{r}$であり,かつ$r$個の元$e_{1}, e_{2}, \dots, e_{r}$によって$g \in G$が一意に
  ${\displaystyle g = \sum_{i=1}^{r} n_{i} e_{i}}$とかけるものとする。
\end{teigi}

この$e_{1}, e_{2}, \dots, e_{r}$を$\mathbb{Z}$基底といい,$r$をランクと呼ぶ。

\begin{teigi}[ユニモジュラ行列]
  $Q$:$r$次正方整数行列とする。

  $Q$:\nw{ユニモジュラ(unimodular)} $\defines$ $\det Q = \pm 1$
\end{teigi}

ユニモジュラ行列の定義について次のことが成り立つ。
\renewcommand{\arraystretch}{1}
\begin{center}
  \begin{tabular}{lcl}
    $Q$:ユニモジュラ & $\defines$ & $\det Q = \pm 1$ \\
    & $\Longleftrightarrow$ & $e_{1}, e_{2}, \dots, e_{r}$が$\mathbb{Z}$基底ならば
    ${\displaystyle e_{i}' = \sum_{j=1}^{r} Q_{ij} e_{j}}$($i = 1, \dots, r$)も$\mathbb{Z}$基底である 
  \end{tabular}
\end{center}

このことから,ユニモジュラ行列は,自由$\mathbb{Z}$加群の基底の変換行列であることが従う。
なお,$Q$は定義から正則行列であるが,逆行列$Q^{-1}$を余因子行列で書けば分かるように,$Q^{-1}$も整数行列であってユニモジュラ行列である。
したがって,${\displaystyle g = \sum_{i=1}^{n} n_{i} e_{i}}$ならば
\begin{equation}
  g = \sum_{j=1}^{r} \sum_{i=1}^{r} n_{i} (Q^{-1})_{ji} \, e_{j}'
\end{equation}
と新たな基底の下での表現を得ることができる。

\begin{teiri}[Smith標準形(単因子標準形)]
  (正方とは限らない)整数行列$A$は,あるユニモジュラ行列$P,Q$を用いて
  \begin{equation}
    PAQ = \left( \begin{array}{cccc|c}
      \alpha_{1} &            &        &            & \multirow{4}{*}{\BigZero} \\
                 & \alpha_{2} &        &            & \\
                 &            & \ddots &            & \\
                 &            &        & \alpha_{k} & \\ \hline
      \multicolumn{4}{c|}{\BigZero} & \BigZero
    \end{array} \right)
  \end{equation}
  の形にできる。ここで,$\alpha_{1} | \alpha_{2} | \dots | \alpha_{k}$($\alpha_{i} > 0$)である。
  また,$\alpha_{1}, \dots, \alpha_{k}$は一意に定まり,これを$A$の\nw{単因子}と呼ぶ。
\end{teiri}

\begin{proof}
  以下に示すような基本変形を繰り返せば右辺のような形になる。:
  \begin{itemize}
    \vspace{-0.5\baselineskip}
    \item[(1)] $i$行(列)を$-1$倍する。
    \begin{equation}
      \begin{pmatrix}
        1 & \\
          & \ddots \\
          &        & -1 \\
          &        &    & \ddots \\ 
          &        &    &        & 1
      \end{pmatrix}
    \end{equation}
    \item[(2)] 行(列)の入れ替え
    \begin{equation}
      \left( \begin{array}{ccccccccccc}
        1 \\
          & \ddots \\
          &        & 1 \\
          &        &   & 0      & \hdots & \hdots & \hdots & 1      \\ 
          &        &   & \vdots & 1      &        &        & \vdots \\
          &        &   & \vdots &        & \ddots &        & \vdots \\
          &        &   & \vdots &        &        & 1      & \vdots \\
          &        &   & 1      & \hdots & \hdots & \hdots & 0      \\
          &        &   &        &        &        &        &        & 1 \\
          &        &   &        &        &        &        &        &   & \ddots \\
          &        &   &        &        &        &        &        &   &        & 1 
      \end{array} \right)
    \end{equation}
    \item[(3)] $i$行(列)を$c$倍して$j$行(列)に足す
    \begin{equation}
      \begin{pmatrix}
        1 & \\
          & \ddots \\
          &        & 1      \\
          &        & \vdots & \ddots \\
          &        & c      & \hdots & 1 \\ 
          &        &        &        &   & \ddots \\
          &        &        &        &   &        & 1
      \end{pmatrix}
    \end{equation}
  \end{itemize}
  これらの変形は直下に示したユニモジュラ行列による変換として表現できるので定理が従う。
\end{proof}

\begin{rei}
  実際にSmith標準形を求めてみる。
  \begin{center}
    $\begin{array}{ccc}
      \begin{pmatrix}
        -3 & 4 & 4 \\ -2 & 2 & -4 \\ 4 & -4 & 4
      \end{pmatrix} & \xrightarrow[\text{絶対値最小の非零要素を$(1,1)$成分へ}]{(1),(2)}
      & \begin{pmatrix}
        2 & -2 & 4 \\ -3 & 4 & 4 \\ 4 & -4 & 4
      \end{pmatrix} \\ & \xrightarrow[\text{1行,1列に$a_{11}$で割り切れないものがあればそれを「割った余り」にする}]{(3); 2 \text{行目} + 2 \times 1 \text{行目}}
      & \begin{pmatrix}
        2 & -2 & 4 \\ 1 & 0 & 12 \\ 4 & -4 & 4
      \end{pmatrix} \\ & \xrightarrow[\text{絶対値最小の非零要素を$(1,1)$成分へ}]{(1)}
      & \begin{pmatrix}
        1 & 0 & 12 \\ 2 & -2 & 4 \\ 4 & -4 & 4
      \end{pmatrix} \\ & \xrightarrow[\text{1行,1列がすべて$a_{11}$で割り切れる場合は割って0にする}]{(3)}
      & \begin{pmatrix}
        1 & 0 & 0 \\ 0 & -2 & -20 \\ 0 & -4 & -44
      \end{pmatrix} \\ & \xrightarrow[\text{右下の$2 \times 2$の小行列に対して同じアルゴリズムを適用}]{(1),(3)}
      & \begin{pmatrix}
        1 & 0 & 0 \\ 0 & 2 & -20 \\ 0 & 0 & -4
      \end{pmatrix} \\ & \xrightarrow{(1),(3)}
      & \begin{pmatrix}
        1 & 0 & 0 \\ 0 & 2 & 0 \\ 0 & 0 & 4
      \end{pmatrix}
    \end{array}$
  \end{center}
  辿り着いた行列がSmith標準形である。
\end{rei}

すなわち,
\begin{itemize}
  \item 一番左の行と列に左上の要素で割り切れない要素があるときには,
  $a_{11}$で割った余りになるように変形した上でその要素を左上に移動する。
  \item 一番左の行と列に左上の要素で割り切れない要素がないときには,
  その行と列を1にするように変形し,サイズが一回り小さい行列に対して更に変形を続けていく。
\end{itemize}
の手順を踏むことでSmith標準形に辿り着く。

\begin{hodai}
  \label{hodai:homology3.unimod}
  $BA = O$とし,$A$の単因子を$\alpha_{1} | \alpha_{2} | \dots | \alpha_{k}$,$B$の単因子を$\beta_{1} | \beta_{2} | \dots | \beta_{\ell}$とする。
  このとき,あるユニモジュラ行列$P,Q,R$が存在して,
  \begin{equation}
    R^{-1} B P = \begin{pmatrix}
      O_{k' \times k'} & O_{k' \times \ell} \\ O_{\ell \times k'} & \diag (\beta_{1}, \dots, \beta_{\ell})
    \end{pmatrix}, \quad
    P^{-1} A Q = \begin{pmatrix}
      \diag (\alpha_{1}, \dots, \alpha_{k}) & O \\ O & O
    \end{pmatrix}
  \end{equation}
  となる。ただし,$O_{m \times n}$で$m \times n$の大きさの零行列を表すとし,$k' \ge k$とする。
\end{hodai}

\begin{proof}
  まず,$A$をSmith標準形にする。$\tilde{P},Q$をユニモジュラ行列として,
  \begin{equation}
    \tilde{P}^{-1}AQ = \left( \begin{array}{cccc|c}
      \alpha_{1} &            &        &            & \multirow{4}{*}{\BigZero} \\
                 & \alpha_{2} &        &            & \\
                 &            & \ddots &            & \\
                 &            &        & \alpha_{k} & \\ \hline
      \multicolumn{4}{c|}{\BigZero} & \BigZero
    \end{array} \right)
  \end{equation}
  $BAQ = B\tilde{P} \tilde{P}^{-1} AQ = O$より,$B \tilde{P}$の形状は第1行から第$k$行までが零行列となっている。
  そこで,$B \tilde{P}$の第$k+1$行目以降の小行列に対して,Smith標準形をつくる。
  すると,$R,S$をユニモジュラ行列として次のようになる。
  \begin{equation}
    R B \tilde{P} \begin{pmatrix}
      I_{k} & O \\ O & S
    \end{pmatrix} = 
    \begin{array}{l}
      \hspace{1em}\overbrace{}^{k} \\
      \left( \begin{array}{c|c|c}
                &          & \multirow{4}{*}{\BigZero} \\
                &          &                           \\
                &          &                           \\
        \BigZero & \BigZero &                           \\ \cline{3-3}
                &          & \begin{array}{ccc} \beta_{1} & & \\ & \ddots & \\ & & \beta_{\ell} \end{array} \\
      \end{array} \right) \\
      \hspace{1em}\underbrace{\hspace{4em}}_{k'}
  \end{array}
  \end{equation}
  いま,
  \begin{equation*}
    P = \tilde{P} \begin{pmatrix}
      I_{k} & O \\ O & S
    \end{pmatrix}
  \end{equation*}
  とおけば,
  \begin{equation}
    P^{-1} A Q = \begin{pmatrix}
      I_{k} & O \\ O & S^{-1}
    \end{pmatrix}
    \tilde{P}^{-1} A Q =
    \left( \begin{array}{cccc|c}
      \alpha_{1} &            &        &            & \multirow{4}{*}{\BigZero} \\
                 & \alpha_{2} &        &            & \\
                 &            & \ddots &            & \\
                 &            &        & \alpha_{k} & \\ \hline
      \multicolumn{4}{c|}{\BigZero} & \BigZero
    \end{array} \right) 
  \end{equation}
  であり,
  \begin{equation}
    R^{-1} B P = \begin{array}{l}
      \hspace{1em}\overbrace{\hspace{4em}}^{k'} \\
      \left( \begin{array}{c|c|c}
                &          & \multirow{4}{*}{\BigZero} \\
                &          &                           \\
                &          &                           \\
        \BigZero & \BigZero &                           \\ \cline{3-3}
                &          & \begin{array}{ccc} \beta_{1} & & \\ & \ddots & \\ & & \beta_{\ell} \end{array} \\
      \end{array} \right)
    \end{array}
  \end{equation}
  である。
\end{proof}

\sukima \midashi{\large ホモロジーの計算}

$X$:位相空間,$\Delta$-複体構造,$c_{n}$:$n$-単体の個数(有限とする),とする。
このとき,
\begin{center}
  \begin{tabular}{l@{\,}c@{\,}l}
    $\Delta_{n}$ & $:=$ & $n$-チェイン全体($n$-単体の整数結合) \\
    & $\sim$ & ランク$c_{n}$の自由$\mathbb{Z}$加群,特に$n$-単体$\sigma_{1},\sigma_{2}, \dots, \sigma_{c_{n}}$は$\Delta_{n}$の$\mathbb{Z}$基底 \\
    & $\simeq$ & $\mathbb{Z}^{c_{n}}$
  \end{tabular}
\end{center}
である。よって,境界作用素$\partial_{n} \colon \Delta_{n} \longrightarrow \Delta_{n-1}$は$\mathbb{Z}^{c_{n}} \longrightarrow \mathbb{Z}^{c_{n-1}}$と見なすことができ,
これを行列によって表示できる。
\begin{align}
  & \partial_{n} \colon \mapdef{\Delta_{n}}{\Delta_{n-1}}{\displaystyle \sum_{i=1}^{c_{n}} t_{i} \sigma_{i}}{\displaystyle \sum_{i=1}^{c_{n}} t_{i} \partial \sigma_{i}} \\
  & \partial_{n}^{*} \colon \mapdef{\mathbb{Z}^{c_{n}}}{\mathbb{Z}^{c_{n-1}}}{
    \begin{pmatrix}
      \sigma_{1} & \sigma_{2} & \cdots & \sigma_{c_{n}}
    \end{pmatrix}
    \begin{pmatrix}
      t_{1} \\ t_{2} \\ \vdots \\ t_{c_{n}}
    \end{pmatrix}
  }{
    \begin{pmatrix}
      \pi_{1} & \pi_{2} & \cdots & \pi_{c_{n-1}}
    \end{pmatrix} A
    \begin{pmatrix}
      t_{1} \\ t_{2} \\ \vdots \\ t_{c_{n}}
    \end{pmatrix}
  }
\end{align}
ここで,$c_{n-1} \times c_{n}$次行列$A$は$\partial_{n}$の行列表示であり,
$\partial \sigma_{j}$における$\pi_{i}$の係数を$(i,j)$成分にもつものとする。

\begin{rei}[射影平面の場合]
  $X = P^{2}$:2次元射影空間(射影平面)とする。
  \begin{center}
		\begin{tikzpicture}
			\fill[red!20] (0,0) -- (2,2) -- (0,2) -- cycle;
			\fill[blue!20] (0,0) -- (2,2) -- (2,0) -- cycle;
			\fill \point{(0,0)}{below left}{$u$} \point{(2,0)}{below right}{$v$} \point{(0,2)}{above left}{$v$} \point{(2,2)}{above right}{$u$};
			\node at (1,0) [below]{$b$} node at (0,1) [left]{$a$} node at (1,2) [above]{$b$} node at (2,1) [right]{$a$} node at (1,1) [left]{$c$};
			\node at (0.5,1.5){$U$} node at (1.5,0.5){$L$};
			\midwayarrow{(0,0)}{(2,0)} \midwayarrow{(0,0)}{(0,2)} \midwayarrow{(0,0)}{(2,2)} \midwayarrow{(2,2)}{(2,0)} \midwayarrow{(2,2)}{(0,2)}
		\end{tikzpicture}
  \end{center}
  \begin{itemize}
    \item $\Delta_{2} \simeq \sets{\begin{pmatrix}
      t_{U} & t_{L} 
    \end{pmatrix}^{\top} | t_{U}, t_{L} \in \mathbb{Z}} = \mathbb{Z}^{2}$
    \item $\Delta_{1} \simeq \sets{\begin{pmatrix}
      t_{a} & t_{b} & t_{c} 
    \end{pmatrix}^{\top} | t_{a}, t_{b}, t_{c} \in \mathbb{Z}} = \mathbb{Z}^{3}$
    \item $\Delta_{0} \simeq \sets{\begin{pmatrix}
      t_{u} & t_{v} 
    \end{pmatrix}^{\top} | t_{u}, t_{v} \in \mathbb{Z}} = \mathbb{Z}^{2}$
  \end{itemize}
  このとき,境界作用素の行列表示はそれぞれ,
  \begin{equation}
    A_{\partial_{1}} = \begin{array}{c@{\,}c}
      & \begin{array}{ccc}
        a & b & c
      \end{array} \\
      \begin{array}{c}
        u \\ v
      \end{array} & 
      \begin{pmatrix}
        -1 & 1 & 0 \\
        -1 & 1 & 0
      \end{pmatrix}  
    \end{array}
    , \quad 
    A_{\partial_{2}} = \begin{array}{c@{\,}c}
      & \begin{array}{cc}
        U & L
      \end{array} \\
      \begin{array}{c}
        a \\ b \\ c
      \end{array} &
      \begin{pmatrix}
      -1 & 1 \\
      1 & -1 \\
      1 & 1
    \end{pmatrix}
  \end{array}
  \end{equation}
  である。
\end{rei}

以上を踏まえて$H_{n} = \Kernel \partial_{n} / \Image \partial_{n+1}$を計算する流れについて述べる。

\sukima\midashi{注意する事実}
\begin{itemize}
  \item $\Delta_{n+1} \xrightarrow{\partial_{n+1}} \Delta_{n} \xrightarrow{\partial_{n}} \Delta_{n-1}$について$\partial_{n} \circ \partial_{n+1} = 0$である。
  \item 行列表示で言えば,$\partial_{n}$の行列表示を$A_{n}$として,
  \begin{equation}
    A_{n} A_{n+1} = O
  \end{equation}
  \item \rref{補題}{hodai:homology3.unimod}より,あるユニモジュラ行列$P,Q,R$が存在して
  \begin{align}
    P^{-1} A_{n+1} Q &=
    \left( \begin{array}{ccc|c}
      \alpha_{1} &        &                  & \multirow{3}{*}{\BigZero} \\
                 & \ddots &                  & \\
                 &        & \alpha_{k_{n+1}} & \\ \hline
      \multicolumn{3}{c|}{\BigZero} & \BigZero
    \end{array} \right), \\
    \label{eq:homology3.calc}
    R^{-1} A_{n} P &=
    \left( \begin{array}{c|ccc}
      \BigZero                  & \multicolumn{3}{c}{\BigZero} \\ \hline
      \multirow{3}{*}{\BigZero} & \beta_{1} &        & \\
                                &           & \ddots & \\
                                &           &        & \beta_{k_{n}} \\
    \end{array} \right)
  \end{align}
  とできる。なお,式\eqref{eq:homology3.calc}において上2つの零行列の縦のサイズは$c_{n} - k_{n}$である。
\end{itemize}

これを使って計算していく。

\begin{equation}
  \begin{array}{c@{\:}c@{\:}c@{\:}c@{\:}c}
    \begin{pmatrix}
      \rho_{1} & \cdots & \rho_{c_{n+1}}
    \end{pmatrix}
    \begin{pmatrix}
      s_{1} \\ \vdots \\ s_{c_{n+1}}
    \end{pmatrix}
    & \longmapsto &
    \begin{pmatrix}
      \sigma_{1} & \cdots & \sigma_{c_{n}}
    \end{pmatrix}
    A_{n+1}
    \begin{pmatrix}
      s_{1} \\ \vdots \\ s_{c_{n+1}}
    \end{pmatrix} \\
    \rotatebox{270}{$\in$} & & \rotatebox{270}{$\in$} \\
    \Delta_{n+1} & \xrightarrow{\partial_{n+1}} & \Delta_{n} & \xrightarrow{\partial_{n}} & \Delta_{n-1} \\
    & & \vin & & \vin \\
    & & \begin{pmatrix}
      \sigma_{1} & \cdots & \sigma_{c_{n}}
    \end{pmatrix}
    \begin{pmatrix}
      t_{1} \\ \vdots \\ t_{c_{n}}
    \end{pmatrix}
    & \longmapsto &
    \begin{pmatrix}
      \pi_{1} & \cdots & \pi_{c_{n-1}}
    \end{pmatrix}
    A_{n}
    \begin{pmatrix}
      t_{1} \\ \vdots \\ t_{c_{n}}
    \end{pmatrix} \\
  \end{array}
\end{equation}
において,先の\textbf{注意する事実}で述べた行列$P,Q,R$を用いて
\begin{equation}
  \left\{ \begin{split}
    \bm{\rho}' &= Q^{\top} \bm{\rho}, \\
    \bm{\sigma}' &= P^{\top} \bm{\sigma}, \\
    \bm{\pi}' &= R^{\top} \bm{\pi}, \\
    \bm{s}' &= Q^{-1} \bm{s}, \\
    \bm{t}' &= P^{-1} \bm{t}
  \end{split} \right.
\end{equation}
と変換すると
\begin{equation}
  \begin{array}{c@{\:}c@{\:}c@{\:}c@{\:}c}
    \begin{pmatrix}
      \rho'_{1} & \cdots & \rho'_{c_{n+1}}
    \end{pmatrix}
    \begin{pmatrix}
      s'_{1} \\ \vdots \\ s'_{c_{n+1}}
    \end{pmatrix}
    & \longmapsto &
    \begin{pmatrix}
      \sigma'_{1} & \cdots & \sigma'_{c_{n}}
    \end{pmatrix}
    P^{-1}A_{n+1}Q
    \begin{pmatrix}
      s'_{1} \\ \vdots \\ s'_{c_{n+1}}
    \end{pmatrix} \\
    \rotatebox{270}{$\in$} & & \rotatebox{270}{$\in$} \\
    \Delta_{n+1} & \xrightarrow{\partial_{n+1}} & \Delta_{n} & \xrightarrow{\partial_{n}} & \Delta_{n-1} \\
    & & \vin & & \vin \\
    & & \begin{pmatrix}
      \sigma'_{1} & \cdots & \sigma'_{c_{n}}
    \end{pmatrix}
    \begin{pmatrix}
      t'_{1} \\ \vdots \\ t'_{c_{n}}
    \end{pmatrix}
    & \longmapsto &
    \begin{pmatrix}
      \pi'_{1} & \cdots & \pi'_{c_{n-1}}
    \end{pmatrix}
    R^{-1}A_{n}P
    \begin{pmatrix}
      t'_{1} \\ \vdots \\ t'_{c_{n}}
    \end{pmatrix} \\
  \end{array}
\end{equation}
となって,ここに
\begin{align}
  P^{-1} A_{n+1} Q &=
    \left( \begin{array}{ccc|c}
      \alpha_{1} &        &                  & \multirow{3}{*}{\BigZero} \\
                 & \ddots &                  & \\
                 &        & \alpha_{k_{n+1}} & \\ \hline
      \multicolumn{3}{c|}{\BigZero} & \BigZero
    \end{array} \right), \\
    R^{-1} A_{n} P &=
    \left( \begin{array}{c|ccc}
      \BigZero                  & \multicolumn{3}{c}{\BigZero} \\ \hline
      \multirow{3}{*}{\BigZero} & \beta_{1} &        & \\
                                &           & \ddots & \\
                                &           &        & \beta_{k_{n}} \\
    \end{array} \right)
\end{align}
である。
すると,これより直ちに
\begin{equation}
  \Image \partial_{n+1} = \begin{pmatrix}
    \mathbb{Z} \alpha_{1} \\ \vdots \\ \mathbb{Z} \alpha_{k_{n+1}} \\ 0 \\ \vdots \\ 0
  \end{pmatrix}, \quad 
  \Kernel \partial_{n} = \begin{pmatrix}
    \mathbb{Z} \\ \vdots \\ \mathbb{Z} \\ 0 \\ \vdots \\ 0
  \end{pmatrix}
\end{equation}
と計算される。ここに,$\Kernel \partial_{n}$の$\mathbb{Z}$の数は$c_{n} - k_{n}$であることに注意。
いま,さらに
\begin{equation}
  \alpha_{1} = \alpha_{2} = \dots = \alpha_{\ell-1} = 1
\end{equation}
の関係が成り立っているとすると,$\mathbb{Z} / \mathbb{Z} = 0$に注意すれば
\begin{equation}
  H_{n} = \Kernel \partial_{n} / \Image \partial_{n+1}
  = \begin{pmatrix}
    \mathbb{Z} / \alpha_{1} \mathbb{Z} \\ \vdots \\ \mathbb{Z} / \alpha_{k_{n+1}} \mathbb{Z} \\ \mathbb{Z} \\ \vdots \\ \mathbb{Z} \\ 0 \\ \vdots \\ 0
  \end{pmatrix}
  = \mathbb{Z} / \alpha_{\ell} \mathbb{Z} \oplus \dots \oplus \mathbb{Z} / \alpha_{k_{n+1}} \mathbb{Z} \oplus \mathbb{Z}^{c_{n} - k_{n+1} - k_{n}}
\end{equation}
となる。
このうち,

\begin{equation}
  \mathbb{Z} / \alpha_{\ell} \mathbb{Z} \oplus \dots \oplus \mathbb{Z} / \alpha_{k_{n+1}} \mathbb{Z}
\end{equation}
の部分を\nw{ねじれ成分}と呼ぶ。
また,$H_{n}$の\nw{ランク},$\rank H_{n}$を
\begin{equation}
  \rank H_{n} := c_{n} - k_{n} - k_{n+1}
\end{equation}
で定める。つまり,境界作用素のSmith標準形を求めていけば,ホモロジー群$H_{n}$を決定することができる。

\sukima \midashi{\large Euler標数}

\begin{teigi}
  次の式\eqref{eq:homology3.Euler}で定まる量$\chi (X)$を$X$の\nw{Euler標数(Euler characteristic)}という。
  \begin{equation}
    \label{eq:homology3.Euler}
    \chi (X) := \sum_{n=0}^{\dim X} (-1)^{n} c_{n}
  \end{equation}
  ただし,$c_{n}$は$X$に含まれる$n$-単体の個数である。
\end{teigi}

\begin{prop}
  ${\displaystyle \chi (X) = \sum_{n=0}^{\dim X} (-1)^{n} \rank H_{n} (X)}$
\end{prop}

\begin{proof}
  定義から$\rank H_{n} (X) = c_{n} - k_{n} - k_{n+1}$であったので,
  \begin{equation}
    \sum_{n=0}^{\dim X} (-1)^{n} \rank H_{n} (X) = \sum_{n=0}^{\dim X} (-1)^{n} (c_{n} - k_{n} - k_{n+1}) = \sum_{n=0}^{\dim X} (-1)^{n} c_{n} = \chi (X)
  \end{equation}
\end{proof}

\midashi{注意.} ホモロジー群は\textbf{位相不変量}であり,複体構造に依存しない。
したがって,Euler標数も複体構造に依存せずに決まる。

\begin{rei}[射影平面の例]
  $\Delta_{3} \xrightarrow{\partial_{3}} \Delta_{2} \xrightarrow{\partial_{2}} \Delta_{1} \xrightarrow{\partial_{1}} \Delta_{0}$において,境界作用素の行列表示とSmith標準形はそれぞれ,
  \begin{align}
    A_{\partial_{2}} = \begin{array}{c@{\,}c}
      & \begin{array}{ccc}
        a & b & c
      \end{array} \\
      \begin{array}{c}
        U \\ L
      \end{array} & 
      \begin{pmatrix}
        -1 & 1 & 1 \\
        1 & -1 & 1
      \end{pmatrix}  
    \end{array}
    , \quad 
    A_{\partial_{1}} = \begin{array}{c@{\,}c}
      & \begin{array}{cc}
        u & v
      \end{array} \\
      \begin{array}{c}
        a \\ b \\ c
      \end{array} &
      \begin{pmatrix}
      -1 & 1 \\
      -1 & 1 \\
      0 & 0
    \end{pmatrix}
  \end{array}, \\
  \mathrm{Smith}(A_{\partial_{2}}) = \begin{pmatrix}
    1 & 0 & 0 \\ 0 & 2 & 0
  \end{pmatrix}, \quad
  \mathrm{Smith}(A_{\partial_{1}}) = \begin{pmatrix}
    0 & 0 \\
    0 & 0 \\
    1 & 0
  \end{pmatrix}
  \end{align}
  となる。これより,
  \begin{equation}
    H_{2}(P^{2}) = 0, \quad H_{1}(P^{2}) = \mathbb{Z} / 2 \mathbb{Z}, \quad H_{0}(P^{2}) = \mathbb{Z}
  \end{equation}
  であり,Euler標数は
  \begin{equation}
    \chi (P^{2}) = 1
  \end{equation}
  と計算される。
\end{rei}

\begin{rei}[2次元球面の例]
  複体構造の図は
  \begin{center}
		\begin{tikzpicture}
		%% 面を塗る
		\fill[red!20] (0,0) -- (2,2) -- (0,2) -- cycle;
		\fill[blue!20] (0,0) -- (2,2) -- (2,0) -- cycle;
		%% 点の名前もろもろ
		\fill \point{(0,0)}{below left}{$u$} \point{(2,0)}{below right}{$v$} \point{(0,2)}{above left}{$v$} \point{(2,2)}{above right}{$w$};
		\node at (1,0) [below]{$b$} node at (0,1) [left]{$b$} node at (1,2) [above]{$a$} node at (2,1) [right]{$a$} node at (1,1) [left]{$c$};
		\node at (0.5,1.5){$U$} node at (1.5,0.5){$L$};
		%% 矢印つき
		\midwayarrow{(0,0)}{(0,2)} \midwayarrow{(0,0)}{(2,0)} \midwayarrow{(0,2)}{(2,2)} \midwayarrow{(2,0)}{(2,2)} \midwayarrow{(0,0)}{(2,2)}
		\end{tikzpicture}
  \end{center}
  であるから,
  \begin{equation}
    \chi (S^{2}) = c_{0} - c_{1} + c_{2} = 3 - 3 + 2 = 2
  \end{equation}
  と求まる。因みに
  \begin{equation}
    H_{0}(S^{2}) = \mathbb{Z}, \quad H_{1}(S^{2}) = 0, \quad H_{2}(S^{2}) = \mathbb{Z}
  \end{equation}
\end{rei}

\expandafter\ifx\csname readornot\endcsname\relax
  \end{document}
\fi % ホモロジーの計算 2

\octopuspart{テンソル}
座標変換に対する不変性

\expandafter\ifx\csname readornot\endcsname\relax
  \documentclass[uplatex]{jsarticle}    
  \usepackage{octopus}
  \usepackage{url}
  %%%% コマンド定義専用のtexファイル
\renewcommand{\postpartname}{章}
\renewcommand{\thepart}{\arabic{part}}
\renewcommand{\thesection}{\thepart.\arabic{section}}
\makeatletter\renewcommand{\theequation}{\thesection.\arabic{equation}}\@addtoreset{equation}{section}\makeatother
\newcommand{\octopuspart}[1]{\newpage\part{#1}\setcounter{section}{0}\vspace{3\baselineskip}}
\renewcommand{\restriction}[2]{\left. #1 \right|_{#2}}
\DeclareMathOperator{\dcup}{\dot{\cup}}
\DeclareMathOperator{\conv}{conv}
\DeclareMathOperator{\Image}{Im}
\DeclareMathOperator{\Kernel}{Ker}
\DeclareMathOperator{\diag}{diag}
\DeclareMathOperator{\rank}{rank}
\DeclareMathOperator{\sgn}{sgn}
\DeclareMathOperator{\rot}{rot}
  \usetikzlibrary{calc}
  \begin{document}
\fi

\section{テンソルの定義}
\midashi{ベクトル空間}
\begin{teigi}[ベクトル空間]
  略.
\end{teigi}

\begin{rei}
  $\mathbb{R}^{n}$は$\mathbb{R}$上のベクトル空間。
\end{rei}

\midashi{注意.}環においてベクトル空間に対応する概念は\nw{加群 (module)}という。

\begin{teigi}[基底]
  $B \subseteq V$が\nw{基底} $\iff$ 任意の$v \in V$を${\displaystyle v = \sum_{u \in B: \text{有限和}} \alpha_{u} u}$と一意的に表される。
\end{teigi}

$B \subseteq V$が基底であることは,任意の$v \in V$が${\displaystyle v = \sum_{u \in B: \text{有限和}} \alpha_{u} u}$と表すことができて,任意の有限部分集合$B^{\prime}$が一次独立であることと同値。

\begin{hodai}
  ベクトル空間には基底が存在する\footnote{証明にはZornの補題を用いるがここでは省略する。}。
\end{hodai}

ベクトル空間の基底が有限個の元からなるとき,その個数は基底の取り方によらずに等しいので,次のように次元を定義することができる。
\begin{teigi}
  ベクトル空間$V$が$n$次元 $\iff$ $n$個の元からなる基底が存在\par
  ベクトル空間$V$が無限次元 $\iff$ 任意の$n \in \mathbb{N}$に対してベクトル空間$V$が$n$次元でない
\end{teigi}
ベクトル空間$V$の次元が$n$であることを$\dim V = n$と書く。$V$が無限次元であることは$\dim V = \infty$と書く。

\begin{rei}
  $\mathbb{R}^{n}$は$n$次元。
\end{rei}

\begin{rei}
  $\mathbb{R}[x]$を実係数1変数多項式全体の集合とする。これは$\mathbb{R}$ベクトル空間になる。このときの基底は$B = \sets{1, x, x^{2}, \dots}$であり,したがって$\mathbb{R}[x]$は$\mathbb{R}$上の無限次元ベクトル空間である。
\end{rei}

\midashi{注意.} $L^{2}[0,1]$を$[0,1]$上で2乗可積分な関数全体の集合とすると,これは$\mathbb{R}$ベクトル空間である。このとき,任意の$f \in L^{2}[0,1]$は
\begin{align}
  f(x) = \sum_{k=1}^{\infty} \alpha_{k} \sin (2 \pi k x) + \sum_{k=0}^{\infty} \beta_{k} \cos (2 \pi k x)
\end{align}
と一意的に書けるが,$\sets{\sin (2 \pi k x) | k \in \mathbb{N}_{>0}} \cup \sets{\cos (2 \pi k x) | k \in \mathbb{N}}$は上の意味での「基底」ではないことに注意。この基底のことは「\textbf{代数基底}」や「\textbf{Hamel基底}」と呼んで区別する\footnote{\url{https://ja.wikipedia.org/wiki/基底_(線型代数学)}}。

以降は$V$を$n$次元ベクトル空間とする。

\sukima\midashi{双対空間}

\begin{teigi}[線形汎関数]
  $f\colon V\to K$が線形汎関数 $\defines$ 任意の$\alpha,\beta \in K, u,v \in V$に対して
  \begin{align*}
    f(\alpha u+\beta v) = \alpha f(u) + \beta f(v)
  \end{align*}
\end{teigi}
\begin{teigi}[双対空間]
  $V$の\textbf{双対空間}$V^\ast$を,$V^\ast := \{ f\colon V\to K \mid fは線形汎関数 \}$で定義する.
\end{teigi}
\begin{teigi}[双対基底]
  $\{e_1,\cdots,e_n\}$を$V$の基底とする.このとき,$V^\ast$の基底$\{e^1,\cdots,e^n\}$を
  \begin{align*}
    e^j(e_i) = \delta_i^j := \begin{cases}
      1 & i=j \\
      0 & i\neq j
    \end{cases}
  \end{align*}
  を満たす線形汎関数として定義する.
\end{teigi}
\begin{prop}
  $\{e^1,\cdots,e^n\}$は$V^\ast$の基底.
\end{prop}
\begin{proof}
  $f\in V^\ast$に対して
  \begin{align*}
    f = \sum_i f(e_i) e^i
  \end{align*}
  と表すことが出来る.また,
  \begin{align*}
    \sum_i \beta^i e^i = 0
  \end{align*}
  が成り立つとき(右辺の0は零写像を表している),
  \begin{align*}
    \left(\sum_i \beta^i e^i \right) (e_j) & = \beta_j \\
    0(e_j) & = 0 
  \end{align*}
  よって$\beta_j = 0$なので$\{e_1,\cdots,e_n\}$は基底である.
\end{proof}
\begin{corr}
  $\dim V = n$ならば$\dim V^\ast = n$.
\end{corr}
\begin{prop}
  $V^{\ast \ast} \simeq V$
\end{prop}
\begin{proof}
  $v\in V$を$f\in V^\ast$に対して$f\mapsto f(v)$という対応を考えることで,$v$を写像$v\colon V^\ast \to K$とみなす.このとき,
  \begin{align*}
    v(\alpha f + \beta g) & = (\alpha f + \beta g)(v) \\
    & = \alpha f(v) + \beta g(v) \\
    & = \alpha v(f) + \beta v(g)
  \end{align*}
  より$v$は$V^\ast$上の線形汎関数.したがって$V\subseteq V^{\ast \ast}$.また$\dim V = \dim V^\ast = \dim V^{\ast \ast}$より,$V \simeq V^{\ast \ast}$.
\end{proof}

\sukima\midashi{テンソル}

以下では$U,V$を体$K$上の有限次元ベクトル空間とする.
\begin{teigi}[双線形写像]
  $\Phi\colon U\times V\to K$が任意の$\alpha,\alpha',\beta,\beta'\in K,u,u'\in U ,v,v'\in V$に対して
  \begin{align*}
    \Phi(\alpha u + \alpha' u',v) & = \alpha \Phi(u,v) + \alpha' \Phi(u',v) \\
    \Phi(u,\beta v + \beta' v) & = \beta \Phi(u,v) + \beta' \Phi(u,v') 
  \end{align*}
  を満たすとき,$\Phi$を\textbf{双線形写像}という.
\end{teigi}

\begin{rei}
  $U = \mathbb{R}^n , V = \mathbb{R}^m, A \in \mathbb{R}^{n\times m}$とする.$x \in U , y \in V$に対して
  \begin{align*}
    (x,y) \mapsto x^\top A y
  \end{align*}
  という対応は双線形写像になっている(この双線形写像を2次形式とよぶ).
\end{rei}
\begin{remark}
  $\Phi\colon U\times V\to K$を双線形写像とする.$U$の基底$\{e_1,\cdots,e_n\}$,$V$の基底$\{f_1,\cdots,f_m\}$としたとき,$u = \sum_{i} \alpha^i e_i \in U , v = \sum_j \beta^j  f_j \in V$に対して$\Phi(u,v)$は
  \begin{align*}
    \Phi\left(\sum_{i} \alpha^i e_i,\sum_j \beta^j f_j\right) = \sum_{i,j} \alpha^i \beta^j \Phi(e_i,f_j)
  \end{align*}
  となり,二次形式とみなすことができる.
\end{remark}
\begin{teigi}[多重線形写像]
  $V_i\ (i=1,\cdot,k)$を体$K$上の線形空間とする.$\Phi\colon V_1\times \cdots V_k \to K$で各$i\  ( i=1,\cdots,k)$で線形な写像を多重線形写像という.
\end{teigi}

\sukima\midashi{テンソル積の定義(I)}
\begin{teigi}[テンソル積]
  $U$と$V$のテンソル積$U\otimes V$を
  \begin{align*}
    U\otimes V := \{ \Phi \colon U^\ast \times V^\ast \to K \mid \Phi\mbox{は双線形写像} \}
  \end{align*}
  で定義する.

  また,$u\in U$と$v\in V$のテンソル積 $u\otimes v\in U\times V$を
  \begin{align*}
    (u\otimes v)(f,g) = f(u)\cdot g(v)
  \end{align*}
  で定義する.これは双線形写像になっている.
\end{teigi}
\begin{remark}
  $U\otimes V \not \simeq \{ u\otimes v \mid u \in U, v \in V\}$である.
\end{remark}
\begin{prop}
  $\{e_1,\cdots,e_n\}$を$U$の基底,$\{f_1,\cdots,f_m\}$を$V$の基底とする.このとき,$\{e_i\otimes f_j \mid i=1,\cdots,n,j=1,\cdots,m \}$は$U\otimes V$の基底.
\end{prop}
\begin{proof}
  $\{e^1,\cdots,e^n\}$を$U^\ast$の基底,$\{f^1,\cdots,f^m\}$を$V^\ast$の基底とする.
  \begin{align*}
    \sum_{i,j} \alpha^{i,j} e_i \otimes f_j = 0
  \end{align*}
  のとき,
  \begin{align*}
    (e_i\otimes f_j)(e^\nu ,f^\mu) = e_i(e^\nu)\cdot f_j(f^\mu) 
    = \delta_i^\nu \delta_j^\mu = \begin{cases}
      1 & (i,j) = (\nu,\mu) \\
      0 & \mathrm{otherwise}
    \end{cases}
  \end{align*}
  より,
  \begin{align*}
    \left(\sum_{i,j} \alpha^{i,j} e_i \otimes f_j \right) (e^\nu,f^\mu) = 0
  \end{align*}
  よって,$\alpha^{i j} = 0$.
  また,$\Phi \colon U^\ast \times V^\ast \to K$が双線形写像のとき,
  \begin{align*}
    \Phi = \sum_{i,j} \Phi(e^i,f^j) e_i \otimes f_j
  \end{align*}
  と表すことが出来る.
\end{proof}
\begin{corr}
  $\dim U\otimes V = \dim U \times \dim V$
\end{corr}

\sukima\midashi{テンソル積の定義(II)}
\begin{align*}
  U\otimes V := \left\{ \sum_{\mbox{有限和}} \alpha_i u_i \otimes v_i \relmiddle| u_i \in U, v_i \in V, \alpha_i \in K \right\} \mathbin{/} \sim
\end{align*}
ただし$\sim$は
\begin{align*}
  (u+u')\otimes v & \sim u \otimes v + u'\otimes v \\
  u\otimes (v+v') & \sim u\otimes v + u\otimes v' \\
  \alpha(u\otimes v) & \sim (\alpha u)\otimes v \sim u \otimes (\alpha v)
\end{align*}
を満たす同値関係.

\expandafter\ifx\csname readornot\endcsname\relax
  \end{document}
\fi
 % テンソルの定義
\expandafter\ifx\csname readornot\endcsname\relax
  \documentclass[uplatex]{jsarticle}
  \usepackage{octopus}
  \usepackage{url}

  \renewcommand{\proofname}{\textsf{証明}}
  \renewcommand{\postpartname}{章}
  \renewcommand{\thesection}{\thepart.\arabic{section}}
  \renewcommand{\thepart}{\arabic{part}}
  \makeatletter\renewcommand{\theequation}{\thesection.\arabic{equation}}\@addtoreset{equation}{section}\makeatother

  \newcommand{\octopuspart}[1]{\newpage\part{#1}\setcounter{section}{0}\vspace{3\baselineskip}}

  \renewcommand{\restriction}[2]{\left. #1 \right|_{#2}}
  \DeclareMathOperator{\dcup}{\dot{\cup}}
  \DeclareMathOperator{\conv}{conv}
  \DeclareMathOperator{\Image}{Im}
  \DeclareMathOperator{\Kernel}{Ker}
  \begin{document}
\fi

\section{テンソル解析 1}
$V$を$\mathbb{R}$上の$n$次元ベクトル空間とし,基底を$\sets{e_{1}, \dots, e_{n}}$とする。このとき,任意の$V$の元$v$は${\displaystyle v = \sum_{i=1}^{n} v^{i}e_{i}}$と書ける。
つまり,基底の組を決めると,任意の$v \in V$は$n$次元ベクトル$(v^{1} , \dots, v^{n})^{\top}$とみなせる。つまり,$V \simeq \mathbb{R}^{n}$である。

\begin{teigi}
  \nw{座標系} $\defines$ 基底の組$(e_{\kappa} \,|\, \kappa = 1,\dots, n)$
\end{teigi}

$v \in V$に対して座標系$(e_{\kappa} \,|\, \kappa = 1,\dots, n)$による表示を$v^{\kappa}$,座標系$(e_{\kappa^{\prime}} \,|\, \kappa^{\prime} = 1^{\prime},\dots, n^{\prime})$による表示を$v^{\kappa^{\prime}}$と表すことにする。

\begin{hodai}[変換則]
  \label{lem:12-1-2}
  ${\displaystyle e_{\kappa} = \sum_{\kappa^{\prime} = 1^{\prime}}^{n^{\prime}} A_{\kappa}^{\kappa^{\prime}} e_{\kappa^{\prime}}}$なら,
  \begin{align}
    v^{\kappa^{\prime}} = \sum_{\kappa=1}^{n} A_{\kappa}^{\kappa^{\prime}} v^{\kappa} =: A_{\kappa}^{\kappa^{\prime}} v^{\kappa}.
  \end{align}
\end{hodai}
最後の和の記号$\sum$を省略する書き方を「\nw{Einsteinの縮約記法}」といい,上と下に同じ添字が出てきた場合,その文字については和をとることにする。
\begin{proof}
  $v = v^{\kappa}e_{\kappa} = v^{\kappa^{\prime}}e_{\kappa^{\prime}}$と2通りの表示を考えると,$v^{\kappa} e_{\kappa} = v^{\kappa} A_{\kappa}^{\kappa^{\prime}} e_{\kappa^{\prime}}.
  $である。これより$v^{\kappa^{\prime}} = A_{\kappa}^{\kappa^{\prime}}v^{\kappa}$を得る。
\end{proof}

\begin{teigi}
  座標系から$\mathbb{R}^{n}$への対応を\nw{(一般化された)ベクトル}という。特に,\textbf{補題\ref{lem:12-1-2}}に示された変換則に従うベクトルを\nw{反変ベクトル}という。
\end{teigi}

$V^{*}$を$V$の双対空間,つまり,$V^{*} := \sets{f \colon V \longrightarrow \mathbb{R} | f \text{は線形写像}}$とする。$V$の座標系を$(e_{\kappa})$,その双対基底を$(e^{\kappa})$とする\footnote{すなわち,$e^{\kappa}(e_{\lambda}) = \delta^{\kappa}_{\lambda}$が成り立つ。}。このとき,$f \in V^{*}$は双対基底によって${\displaystyle f = \sum_{\kappa} f_{\kappa} e^{\kappa}}$と書き表すことができる。

\begin{hodai}[変換則]
  \label{lem:12-1-4}
  $V$の座標系$(e_{\kappa})$と$(e_{\kappa^{\prime}})$について,${\displaystyle f = \sum_{\kappa} f_{\kappa} e^{\kappa} = \sum_{\kappa^{\prime}} f_{\kappa^{\prime}} e^{\kappa^{\prime}}}$を考える。$e_{\kappa^{\prime}} = A_{\kappa^{\prime}}^{\kappa} e_{\kappa}$であるとき,
  \begin{align}
    f_{\kappa^{\prime}} = A_{\kappa^{\prime}}^{\kappa} f_{\kappa}.
  \end{align}
\end{hodai}

\begin{proof}
  $f_{\kappa^{\prime}} = f(e_{\kappa^{\prime}}) = f(A_{\kappa^{\prime}}^{\kappa}e_{\kappa}) = A_{\kappa^{\prime}}^{\kappa} f(e_{\kappa}) = A_{\kappa^{\prime}}^{\kappa} f_{\kappa}$である。$f$が線形写像であることに注意。
\end{proof}

\begin{teigi}
  \textbf{補題\ref{lem:12-1-4}}に示された変換則に従うベクトルを\nw{共変ベクトル}という。
\end{teigi}

\begin{remark}
  $V$は反変ベクトル空間であり,$V^{*}$は共変ベクトル空間である。
\end{remark}

\begin{teigi}
  座標系に$\mathbb{R}$の元を対応させる写像であって,いかなる座標系においても同じ値に写すものを\nw{スカラー}という。
\end{teigi}

\begin{hodai}
  $v^{\kappa}$を反変ベクトル,$f_{\kappa}$を共変ベクトルとする。このとき,$f_{\kappa} v^{\kappa}$はスカラー。
\end{hodai}

\begin{proof}
  基底が$e_{\kappa} = A_{\kappa}^{\kappa^{\prime}} e_{\kappa^{\prime}}$と変換されるとする。このとき,
  \begin{align}
    f_{\kappa} v^{\kappa} = f_{\kappa} A^{\kappa}_{\kappa^{\prime}} v^{\kappa^{\prime}} = f_{\kappa^{\prime}} v^{\kappa^{\prime}}.
  \end{align}
  よって,$f_{\kappa} v^{\kappa}$は座標系によらず不変である。
\end{proof}

\sukima \midashi{物理的イメージ}\par
\noindent\textbullet\; 反変ベクトルの例\par
\qquad  
\begin{minipage}[t]{0.15\columnwidth}
  位置ベクトル,
  \begin{align*}
    x = \vect{\mathrm{OP}}
  \end{align*}
\end{minipage}
\vspace{0.05\baselineskip}
\begin{minipage}[t]{0.15\columnwidth}
  速度ベクトル,
  \begin{align*}
    \odif{x}{t}
  \end{align*}
\end{minipage}
\vspace{0.05\baselineskip}
\begin{minipage}[t]{0.15\columnwidth}
  加速度ベクトル
  \begin{align*}
    \odif{^{2}x}{t^{2}}
  \end{align*} 
\end{minipage}

\sukima\noindent\textbullet\; 共変ベクトルの例\par
\qquad  
\begin{minipage}[t]{0.3\columnwidth}
  超平面の法線ベクトル$a_{\kappa}$,
  \begin{align*}
    a_{\kappa} x^{\kappa} = 0.
  \end{align*}
\end{minipage}
\vspace{0.05\baselineskip}
\begin{minipage}[t]{0.3\columnwidth}
  勾配ベクトル,
  \begin{align*}
    \left( \pdif{U}{x^{\kappa}} \right)
  \end{align*}
  {\footnotesize
  \begin{align*}
    \pdif{U}{x^{\kappa^{\prime}}} = \pdif{x^{\kappa}}{x^{\kappa^{\prime}}} \pdif{U}{x^{\kappa}} = A_{\kappa^{\prime}}^{\kappa} \pdif{U}{x^{\kappa}}
  \end{align*}
  による。
  }
\end{minipage}
\vspace{0.05\baselineskip}
\begin{minipage}[t]{0.3\columnwidth}
  力\par
  {\footnotesize
  力が位置エネルギー$U$の勾配であること,力と変位の内積である仕事がスカラー量になることから力は共変ベクトルとみなすのが自然。
  }
\end{minipage}

\expandafter\ifx\csname readornot\endcsname\relax
  \end{document}
\fi % テンソル解析 1
\expandafter\ifx\csname readornot\endcsname\relax
 \documentclass{jsarticle}
 \begin{document}
\fi

\section{反変性・共変性・スカラー・混合テンソル}

\expandafter\ifx\csname readornot\endcsname\relax
  \end{document}
\fi % テンソル解析 2

\end{document}
