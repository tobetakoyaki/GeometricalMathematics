\documentclass[uplatex]{jsarticle}
\usepackage{octopus}
\usepackage{url}
\usetikzlibrary{calc}

\renewcommand{\proofname}{\textsf{証明}}
\renewcommand{\postpartname}{章}
\renewcommand{\thepart}{\arabic{part}}
\renewcommand{\thesection}{\thepart.\arabic{section}}
\makeatletter\renewcommand{\theequation}{\thesection.\arabic{equation}}\@addtoreset{equation}{section}\makeatother

\newcommand{\octopuspart}[1]{\newpage\part{#1}\setcounter{section}{0}\vspace{3\baselineskip}}
\newcommand{\readornot}{false}

\renewcommand{\restriction}[2]{\left. #1 \right|_{#2}}
\DeclareMathOperator{\dcup}{\dot{\cup}}
\DeclareMathOperator{\conv}{conv}
\DeclareMathOperator{\Image}{Im}
\DeclareMathOperator{\Kernel}{Ker}


\begin{document}
\begin{center}{\LARGE \bf 幾何数理工学 ノート}\end{center}

このノートは,2018年度幾何数理工学のまとめノートです。

\midashi{本講義の内容}
\begin{enumerate}
    \item 位相空間:「近さ」が備わった空間概念。$\mathbb{R}^{n}$の一般化。
    \item 位相幾何:連続変形に対する不変性。
    \begin{itemize}
        \item 基本群
        \item ホモロジー
    \end{itemize}
    \item テンソル:座標変換に対する不変性。
    \begin{itemize}
        \item 3次元ベクトル${\displaystyle \begin{pmatrix}
            v_{1} & v_{2} & v_{3} 
        \end{pmatrix}^{\top}}$と${\displaystyle \begin{pmatrix}
            \pdif{f}{x_{1}} & \pdif{f}{x_{2}} & \pdif{f}{x_{3}}
        \end{pmatrix}^{\top}}$の違いは何か。
    \end{itemize}
\end{enumerate}

\midashi{参考書}
\begin{itemize}
    \item 内田伏一,「集合と位相」\footnote{amazonだと2,808円(税込み)。}
    \item Allen Hatcher,「Algebraic Topology」\footnote{\url{https://pi.math.cornell.edu/~hatcher/AT/ATpage.html}}
    \item 伊理正夫,韓太舜,「テンソル解析入門」\footnote{楽天だと2,097円(税込み)。}
\end{itemize}

\renewcommand{\baselinestretch}{0.1}
\tableofcontents
\renewcommand{\baselinestretch}{1.0}

\octopuspart{位相空間}
\expandafter\ifx\csname readornot\endcsname\relax
  \documentclass[uplatex]{jsarticle}
  \usepackage{octopus}
  \usepackage{url}
  \usepackage{tikz}

  \renewcommand{\proofname}{\textsf{証明}}
  \renewcommand{\postpartname}{章}
  \renewcommand{\thesection}{\thepart.\arabic{section}}
  \renewcommand{\thepart}{\arabic{part}}
  \makeatletter\renewcommand{\theequation}{\thesection.\arabic{equation}}\@addtoreset{equation}{section}\makeatother

  \newcommand{\octopuspart}[1]{\newpage\part{#1}\setcounter{section}{0}\vspace{3\baselineskip}}

  \DeclareMathOperator{\dcup}{\dot{\cup}}
  \begin{document}
\fi

\section{距離空間}

\begin{teigi}[距離空間]
    $X$を非空な集合,$d: \, X \times X \to \mathbb{R}$を実数値関数とする。次の三つの条件{\bf D1},{\bf D2},{\bf D3}を考える。
    
    \midashi{D1. } $\forall x, y \in X, \quad d(x,y) \ge 0$,\qquad $d(x,y) = 0 \quad \Longleftrightarrow \quad x = y$
    
    \midashi{D2. } $\forall x, y \in X, \quad d(x,y) = d(y,x)$
    
    \midashi{D3. } $\forall x, y \in X, \quad d(x,y) + d(y,z) \ge d(x,z)$
    
    $d$が{\bf D1},{\bf D2},{\bf D3}の条件を満たすとき,$d$を$X$上の\nw{距離関数}という。

    また,$X$あるいは$(X,d)$を\nw{距離空間(metric space)}という。
\end{teigi}

{\bf D2.}は対称性を表し,{\bf D3.}は三角不等式と呼ばれる。

{\bf D1.},{\bf D2.},{\bf D3.}を満たさないような関数として,例えば${d_{2}}^{2}$(2乗距離)がある。
これは三角不等式を満たさない。

\begin{rei}
    $X = \mathbb{R}^{n}$とする。
    \begin{align}
        d_{2} (x,y) &:= \sqrt{\sum_{i=1}^{n} (x_{i} - y_{i})^{2}}, \\
        d_{1} (x,y) &:= \sum_{i=1}^{n} |x_{i} - y_{i}|, \\
        d_{\infty} (x,y) &:= \max_{i} |x_{i} - y_{i}|
    \end{align}
    とすると,これらはどれも$X$上の距離関数である。なお,$(\mathbb{R}^{n}, d_{2})$を\nw{$n$次元Euclid空間}と呼ぶ。
\end{rei}
\begin{proof}
    ここでは$d_{2}$が{\bf D3.}の三角不等式を満たすことのみを示す。
    \begin{align}
        {d_{2}}^{2}(x,z)
        &= \sum_{i=1}^{n} (x_{i} - z_{i})^{2} = \sum_{i=1}^{n} (x_{i} - y_{i} + y_{i} - z_{i})^{2} \notag \\
        &= \sum_{i=1}^{n} \left( x_{i} - y_{i} \right)^{2} + \sum_{i=1}^{n} \left( y_{i} - z_{i} \right)^{2} + 2 \sum_{i=1}^{n} \left( x_{i} - y_{i} \right) \left( y_{i} - z_{i} \right) \notag \\
      &\le \sum_{i=1}^{n} \left( x_{i} - y_{i} \right)^{2} + \sum_{i=1}^{n} \left( y_{i} - z_{i} \right)^{2} + 2 \sqrt{\left( \sum_{i=1}^{n} \left( x_{i} - y_{i} \right)^{2} \right) \left( \sum_{i=1}^{n} \left( y_{i} - z_{i} \right)^{2} \right)} \label{eq:1.1:Cauchy}\\
        &= \left\{ \sqrt{\sum_{i=1}^{n} \left( x_{i} - y_{i} \right)^{2}} + \sqrt{\sum_{i=1}^{n} \left( y_{i} - z_{i} \right)^{2}} \right\}^{2}
         = \left( d_{2}(x,y) + d_{2} (y,z) \right)^{2}
    \end{align}
    である。ここで,式\eqref{eq:1.1:Cauchy}に至る変形にはCauchy-Schwarzの不等式が用いられている。
\end{proof}

\begin{rei}
    $X = \mathcal{C} [a,b]$を区間$[a,b] \subseteq \mathbb{R}$上の連続関数全体の集合とする。
    \begin{align}
        d_{2} (f,g) &:= \sqrt{ \int_{a}^{b} | f(t) - g(t) | ^{2} \dx{t}}, \\
        d_{\infty} (f,g) &:= \sup \sets{ \left| f(t) - g(t) \right| | t \in [a,b]}, \\
        d_{1} (f,g) &:= \int_{a}^{b} \left| f(t) - g(t) \right| \dx{t}
    \end{align}
    とすると,これらはどれも$X$上の距離関数である。
\end{rei}

\begin{proof}
    ここでは$d_{1}$が{\bf D3.}の三角不等式を満たすことと{\bf D1.}の零点が対角集合に限られることのみを示す。

    \midashi{{\bf D3.}について}
    \begin{align*}
        d_{1} (f,h)
        &= \int_{a}^{b} \left| f(t) - h(t) \right| \dx{t} \\
      &\le \int_{a}^{b} \left( \left| f(t) - g(t) \right| + \left| g(t) - h(t) \right| \right) \dx{t} \\
      &\le \int_{a}^{b} \left| f(t) - g(t) \right| \dx{t} +  \int_{a}^{b} \left| g(t) - h(t) \right| \dx{t} \\
        &= d_{1} (f,g) + d_{1} (g,h)
    \end{align*}
    で従う。

    \midashi{零点が対角集合に限られることについて}

    $f = g$ならば$d_{1} (f,g) = 0$は明らかであるので,問題はその逆である。

    もし,$f(x) \neq g(x)$ならばある$x_{0} \in [a,b]$で$\left| f(x_{0}) - g(x_{0}) \right|> 0$であり,
    $f-g$の連続性からある$\varepsilon > 0$が存在して,$y \in [x_{0} - \varepsilon, x_{0} + \varepsilon]$に対して
    $\left| f(y) - g(y) \right| > 0$である。
    したがって,
    \begin{equation*}
        \int_{a}^{b} \left| f(x) - g(x) \right| \dx{x} \ge \varepsilon \min_{x_{0} - \varepsilon \le y \le x_{0} + \varepsilon} \left| f(y) - g(y) \right| > 0
    \end{equation*}
    である。
\end{proof}

\begin{rei}
    $X = \sets{0,1}^{n}$とする。
    \begin{equation}
        d_{H}(x,y) = \# \sets{i | x_{i} \neq y_{i}}
    \end{equation}
    で定めると,これは$X$上の距離関数である。この距離関数は\nw{Hamming距離}と呼ばれる。
\end{rei}

\begin{rei}
    $G = (X,E)$を無向グラフとする。
    $d_{G}(x,y)$を$x$から$y$への最短路の長さとして定義すると,
    これは$G$上の距離関数である。
\end{rei}

以下,$(X,d)$を距離空間とする。

\begin{teigi}[内点・外点・境界点・触点]
    \midashi{1. } $a \in X$,$\varepsilon > 0$とする。集合$N(a,\varepsilon) := \sets{x \in X | d(a,x) < \varepsilon}$を,$a$の\nw{$\varepsilon$-近傍}という。

    \midashi{2. } $A \subseteq X$の\nw{内点}とは,次の条件を満たす点$x \in X$のこと:
    \begin{equation}
        \exists \varepsilon > 0, \quad N(x,\varepsilon) \subseteq A
    \end{equation}
    また,$A$の内点全体の集合$A^{\circ} = \sets{x \in X | N(x, \varepsilon) \subseteq A}$のことを$A$の\nw{内部}という。

    \midashi{3. } $A \subseteq X$の\nw{外点}とは,次の条件を満たす点$x \in X$のこと:
    \begin{equation}
        \exists \varepsilon > 0, \quad N(x , \varepsilon ) \cap A = \emptyset
    \end{equation}
    言い換えると,$A$の補集合の内点のこと。
    また,$A$の外点全体の集合$(X \setminus A)^{\circ}$を$A$の\nw{外部}という。

    \midashi{4. } $A \subseteq X$の\nw{境界点}とは,次の条件を満たす点$x \in X$のこと:
    \begin{equation}
        \forall \varepsilon > 0, \quad N(x, \varepsilon) \cap A \neq \emptyset, \, N(x,\varepsilon) \cap (X \setminus A) \neq \emptyset
    \end{equation}
    また,$A$の境界点全体の集合のことを$A$の\nw{境界}といい,記号$\partial A$で表す。

    \midashi{5.} $A \subseteq X$の\nw{触点}とは,次の条件を満たす点$x \in X$のこと:
    \begin{equation}
        \forall \varepsilon > 0, \quad N (x,\varepsilon) \cap A \neq \emptyset
    \end{equation}
    また,$A$の触点全体の集合のことを$A$の\nw{閉包}といい,記号$\overline{A}$で表す。
\end{teigi}

\begin{figure}[htbp]
    \centering
    \newcommand{\makeaxis}{
    \draw[thick, ->] (-2.5,0) -- (2.5,0) node [below] {$x$};%x軸
    \draw[thick, ->] (0,-2.5) -- (0,2.5) node [left] {$y$};%y軸
    \coordinate (O) at (0,0);
    \node at (O) [above left] {$a$};
    }
    \centering    
    \begin{tikzpicture}
        \makeaxis
        \draw[dashed] (O) circle (1.5);
        \draw[->] (O) -- node[midway, right]{$\varepsilon$} (45:1.5);
        \node at (-2,2) {$(\mathbb{R}^{2}, d_{2})$};
    \end{tikzpicture}
    \begin{tikzpicture}
        \makeaxis
        \draw[dashed] (1.5,0) -- (0,1.5) -- (-1.5,0) -- (0,-1.5) -- cycle;
        \node at (1.5,0) [below] {$a + \varepsilon$};
        \node at (-1.5,0) [below] {$a - \varepsilon$};
        \node at (-2,2) {$(\mathbb{R}^{2}, d_{1})$};
    \end{tikzpicture}
    \begin{tikzpicture}
        \makeaxis
        \draw[dashed] (-1.5,-1.5) -- (1.5,-1.5) -- (1.5,1.5) -- (-1.5,1.5) -- cycle;
        \node at (-1.5,0) [below left] {$a-\varepsilon$};
        \node at (0,-1.5) [below right] {$a-\varepsilon$};
        \node at (1.5,0) [above right] {$a+\varepsilon$};
        \node at (0,1.5) [above right] {$a+\varepsilon$};
        \node at (-2,2) {$(\mathbb{R}^{2}, d_{\infty})$};
    \end{tikzpicture}
    \caption{距離空間$(\mathbb{R}^{2},d)$上の近傍の違い}
    \label{fig:euclid_dist}
\end{figure}

\begin{figure}[htbp]
    \centering
    \begin{tikzpicture}
        \draw[dashed] (-30:2) arc (-30:210:2);
        \fill [black!20] (0,0) circle [x radius=1.97, y radius=1.97];
        \draw[thick] (210:2) arc (210:330:2);
        \draw[<-] (-50:2.1) -- (-50:2.4) node[right] {\bf 外点};
        \draw (20:0.5) node {\bf 内点};
        \draw[<-] (160:2.1) -- (160:2.6) node[left] {\bf 境界点};
        \draw (45:2.5) node {$A$};
    \end{tikzpicture}
    \caption{$A \subseteq X$の内部,外部,境界}
    \label{fig:in_ex_boundary}
\end{figure}

この定義から,内部,外部,境界点について
\begin{equation}
    X = A^{\circ} \dcup \partial A \dcup (X \setminus A)^{\circ}
\end{equation}
と互いに非交な集合に分解できる。
また,閉包について,次の二つが成り立つ。
\begin{align}
    & A ^ { \circ } \subseteq A \subseteq \overline { A } \\
    & \overline { A } = A ^ { \circ } \cup \partial A
\end{align}

\begin{teigi}[開集合・閉集合]
    \begin{equation*}
        \begin{array}{lcccl}
            A \subseteq X \text{:\nw{開集合}} & \defines & \forall x \in A, \quad \exists \varepsilon > 0, \quad  N(x , \varepsilon ) \subseteq A & \Longleftrightarrow & A = A^{\circ} \\
            A \subseteq X \text{:\nw{閉集合}} & \defines & 「\forall x \in X, \quad \forall \varepsilon > 0, \quad  N(x , \varepsilon ) \cap A \neq \emptyset \Longrightarrow x \in A」& \Longleftrightarrow & A = \overline{A} \\
        \end{array}
    \end{equation*}
\end{teigi}

\begin{hodai}
    $N(x,\varepsilon)$は開集合。特に,$N(x,\varepsilon) = N(x,\varepsilon)^{\circ}$。
\end{hodai}

\begin{proof}
    $y \in N ( x, \varepsilon)$をとる。$\delta := \varepsilon - d ( x , y ) > 0$とすると,
    $z \in N(y, \varepsilon)$に対して,三角不等式より
    \begin{equation}
        d(x,z) \le d(x,y) + d(y,z) < \varepsilon
    \end{equation}
である。よって,$N(y,\delta) \subseteq N(x,\varepsilon)$となる。
\end{proof}

\begin{prop}
    $\left( A^{\circ} \right)^{\circ} = A^{\circ}$,$\overline{\overline{A}} = \overline{A}$
\end{prop}

\begin{proof}
    $\left( A^{\circ} \right)^{\circ} \subseteq A^{\circ}$より,$\supseteq$向きを示す。
    $x \in A^{\circ}$とする。
    \begin{equation}
       \exists \varepsilon > 0, \quad N(x, \varepsilon) \subseteq A
    \end{equation}
    である。$N(x, \varepsilon) = N(x, \varepsilon)^{\circ} \subseteq A^{\circ}$,
    つまり$N(x, \varepsilon) \subseteq A^{\circ}$である。
    $x \in \left( A^{\circ} \right)^{\circ}$である。

    $\overline{\overline{A}} \subseteq \overline{A}$より,$\subseteq$向きを示す。
    $x \in \overline{\overline{A}}$とする。
    \begin{equation}
        \forall \varepsilon > 0, \quad N(x,\varepsilon) \cap \overline{A} \neq \emptyset
    \end{equation}
    である。$y \in N(x, \varepsilon) \cap \overline{A}$,$\delta := \varepsilon - d(x,y) > 0$とする。
    $y \in \overline{A}$より$N(y, \delta) \cap A \neq \emptyset$である。
    $N(y, \delta) \subseteq N(x,\varepsilon)$より,上と同じロジックで
    $\emptyset \neq N(y, \delta) \cap A \subseteq N(x, \varepsilon) \cap A$である。よって,$x \in \overline{A}$である。
\end{proof}

\begin{prop}
    \midashi{1. } $A$:開 $\Longrightarrow$ $X \setminus A$:閉

    \midashi{2. } $A$:閉 $\Longrightarrow$ $X \setminus A$:開
\end{prop}

\begin{proof}
    \begin{align*}
        A\text{が開でない} & \Longleftrightarrow \exists x \in A, \quad \forall \varepsilon > 0, \quad N(x, \varepsilon) \not\subseteq A \\
        & \Longleftrightarrow \exists x \in A, \quad \forall \varepsilon > 0, \quad N(x, \varepsilon) \cap (X \setminus A) \neq \emptyset \\
        & \Longleftrightarrow X \setminus A \text{が閉でない}
    \end{align*}
    ここで2行目の同値性は,$x$は$X \setminus A$の触点であるが$X \setminus A$でないことから従う。
\end{proof}

開集合全体の集合を\nw{開集合系}といい,記号$\mathcal{O}$で表す。
閉集合全体の集合を\nw{閉集合系}といい,記号$\mathfrak{A}$で表す。

\begin{prop}
    \label{prop:open_set_axiom}
    開集合系$\mathcal{O}$について以下が成り立つ。

    \midashi{1. } $X \in \mathcal{O}$,$\emptyset \in \mathcal{O}$

    \midashi{2. } $O_{1}, O_{2}, \cdots, O_{k} \in \mathcal{O} \Longrightarrow O_{1} \cap O_{2} \cap \cdots \cap O_{k} \in \mathcal{O}$

    \midashi{3. } 任意の集合$\Lambda$に対して$O_{\lambda} \in \mathcal{O}$($\forall \lambda \in \Lambda$) ${\displaystyle \Longrightarrow \bigcup_{\lambda \in \Lambda} O_{\lambda} \in \mathcal{O}}$
\end{prop}

\begin{proof}
    \midashi{1. } obvious.

    \midashi{2. } $x \in O_{1} \cap O_{2} \cap \dots \cap O_{k}$をとる。$i = 1,2,\dots,k$に対して$\varepsilon_{i} > 0$が存在して
    \begin{equation}
        N(x_{i}, \varepsilon_{i}) \subseteq O_{i}
    \end{equation}
    である。${\displaystyle \varepsilon := \min_{i} \varepsilon_{i}}$とすると,
    $N(x, \varepsilon) \subseteq O_{1} \cap O_{2} \cap \dots \cap O_{k}$である。
    $x$は$O_{1} \cap O_{2} \cap \dots \cap O_{k}$の内点である。よって,$\left( O_{1} \cap O_{2} \cap \dots \cap O_{k} \right)^{\circ} = O_{1} \cap O_{2} \cap \dots \cap O_{k}$である。

    \midashi{3. } ${x \in \displaystyle \bigcup_{\lambda \in \Lambda} O_{\lambda}}$とすると,
    $\exists \lambda \in \Lambda, \quad x \in O_{\lambda}$である。${\displaystyle \exists \varepsilon > 0, \quad N(x, \varepsilon) \subseteq O_{\lambda} \subseteq \bigcup_{\lambda} O_{\lambda}}$とする。
\end{proof}

\begin{prop}
    閉集合系$\mathfrak{A}$について以下が成り立つ。
    
    \midashi{1. } $X \in \mathfrak{A}$,$\emptyset \in \mathfrak{A}$

    \midashi{2. } $A_{1}, A_{2}, \cdots, A_{k} \in \mathfrak{A} \Longrightarrow A_{1} \cup A_{2} \cup \cdots \cup A_{k} \in \mathfrak{A}$

    \midashi{3. } 任意の集合$\Lambda$に対して$A_{\lambda} \in \mathfrak{A}$($\forall \lambda \in \Lambda$) ${\displaystyle \Longrightarrow \bigcap_{\lambda \in \Lambda} A_{\lambda} \in \mathfrak{A}}$
\end{prop}

\begin{proof}
    \midashi{1. } $\mathfrak{A} = \sets{ X \setminus O | O \in \mathcal{O}}$に注意すると明らか。

    \midashi{2. } $A_{i} = X \setminus O_{i}$とかくと,
    \begin{equation}
        A_{1} \cup A_{2} \cup \dots \cup A_{k}
        = (X \setminus O_{1}) \cup (X \setminus O_{2}) \cup \dots \cup (X \setminus O_{k})
        = X \setminus (O_{1} \cap O_{2} \cap \dots \cap O_{k}) \in \mathfrak{A}
    \end{equation}
    である。

    \midashi{3. } ${\displaystyle \bigcap_{\lambda \in \Lambda} A_{\lambda} = \bigcap_{\lambda \in \Lambda} (X \setminus O_{\lambda}) = X \setminus \bigcup_{\lambda \in \Lambda} O_{\lambda}}$であるから従う。
\end{proof}

収束性,連続性なども距離空間で定義できる。

\begin{teigi}[収束性・連続性]
    \midashi{1. } $a_{i} \in X$($i=1,2,\dots$):\nw{収束}とは,
    \begin{equation}
        \lim_{i \to \infty} a_{i} = a \defines \lim_{i \to \infty} d(a, a_{i}) = 0
    \end{equation}
    
    \midashi{2. } $f:X \longrightarrow \mathbb{R}$が$a \in X$で\nw{連続}とは,
    \begin{equation}
        \forall \varepsilon > 0, \quad \exists \delta > 0, d(a,x) < \delta \Longrightarrow \left| f(a) - f(x) \right| < \varepsilon
    \end{equation}
\end{teigi}

では,$X$上の異なる距離$d$,$d'$の下で連続性や収束性も変わるのか。

$\longrightarrow$位相空間の考え方へ

実際,開集合系を与える(空間に「位相」を与える)ことによって,
連続性や収束性が定義でき,
開集合系が同じとき,連続性や収束性も等しくなる。

\begin{rei}
    $(\mathbb{R}^{n},d_{2})$と$(\mathbb{R}^{n},d_{\infty})$が定義する開集合系は等しい。(位相は等しい)
    \begin{proof}
        $A$:$d_{2}$の下で開集合とする。
        $x \in A$をとる。$\exists \varepsilon > 0, \quad N_{2}(x,\varepsilon) \subseteq A$である。
        $\exists c > 0, \quad N_{\infty} (x, c \varepsilon) \subseteq N_{2} (x, \varepsilon) \subseteq A$である。
        $A$は$d_{\infty}$の下でも開。逆も同様に言える。 
    \end{proof}
    結局,連続性や収束性はどちらの距離で考えても同じになる。
\end{rei}

\begin{rei}
    $\mathbb{R}^{n}$上のノルム$\left\| \cdot \right\|$とは,
    \begin{itemize}
        \vspace{-0.5\baselineskip}
        \item $\left\| v \right\| \ge 0$,\qquad $\left\| v \right\| = 0 \Longleftrightarrow v = 0$
        \item $\forall a \in \mathbb{R}$,$v \in \mathbb{R}^{n}$,$\left\| av \right\| = \left| a \right| \left\| v \right\|$
        \item $\forall u,v \in \mathbb{R}^{n}$,$\left\| u + v \right\| \le \left\| u \right\| + \left\| v \right\|$
        \vspace{-0.5\baselineskip}
    \end{itemize}
    を満たすものである。このとき,$d(x,y) := \left\| x - y \right\|$で定義すると距離関数である。
\end{rei}

\midashi{知っておいてほしいこと}
\begin{itemize}
    \vspace{-0.5\baselineskip}
    \item $\mathbb{R}^{n}$上のどんなノルムを定める「位相」は等しい。
    \item 無限次元では,そうはいかない({\bf らしい})。
\end{itemize}

\expandafter\ifx\csname readornot\endcsname\relax
  \end{document}
\fi % 距離空間
\expandafter\ifx\csname readornot\endcsname\relax
  \documentclass[uplatex]{jsarticle}
  \usepackage{octopus}
  \usepackage{url}

  \renewcommand{\proofname}{\textsf{証明}}
  \renewcommand{\postpartname}{章}
  \renewcommand{\thesection}{\thepart.\arabic{section}}
  \renewcommand{\thepart}{\arabic{part}}
  \makeatletter\renewcommand{\theequation}{\thesection.\arabic{equation}}\@addtoreset{equation}{section}\makeatother
  
  \newcommand{\octopuspart}[1]{\newpage\part{#1}\setcounter{section}{0}\vspace{3\baselineskip}}
  
  \DeclareMathOperator{\dcup}{\dot{\cup}}
  \begin{document}
  \fi
\renewcommand{\thesubsection}{\thepart.\arabic{section}.\arabic{subsection}}
  
\section{位相空間}

\midashi{\large 位相空間}

$X$を非空な集合とする。$X$に「\nw{位相(topology)}」を入れて「空間」にする。
「位相」を入れるとは,$X$の開集合族を指定することである。

\begin{teigi}[位相]
  以下を満たす開集合系$\mathcal{O} \subseteq 2^{X}$を\nw{位相}という。

  \midashi{O1. } $X \in \mathcal{O}$,$\emptyset \in \mathcal{O}$

  \midashi{O2. } $O_{1}, O_{2}, \cdots, O_{k} \in \mathcal{O} \Longrightarrow O_{1} \cap O_{2} \cap \cdots \cap O_{k} \in \mathcal{O}$

  \midashi{O3. } 任意の集合$\Lambda$に対して$O_{\lambda} \in \mathcal{O}$($\forall \lambda \in \Lambda$) ${\displaystyle \Longrightarrow \bigcup_{\lambda \in \Lambda} O_{\lambda} \in \mathcal{O}}$

  $\mathcal{O}$の元を\nw{開集合}という。また,開集合族$\mathcal{O}$が指定された集合$X$を\nw{位相空間}という。
  これを$(X,\mathcal{O})$で表すこともある。
\end{teigi}

\begin{rei}
  \begin{itemize}
    \item \nw{離散位相} $\mathcal{O} = 2^{X}$
    \item \nw{密着位相} $\mathcal{O} = \sets{\emptyset, X}$
    \item 距離空間$(X,d)$;$\mathcal{O} := \sets{O \subseteq X | \forall x \in O, \quad \exists \varepsilon > 0, \quad N(x, \varepsilon) \subseteq O}$
    (これは距離空間のところで定義した開集合の集まり)
  \end{itemize}
\end{rei}

距離空間のときに定義したいくつかの用語を位相空間の言葉で記述し直す。

\begin{itemize}
  \vspace{-0.5\baselineskip}
  \item $A \subseteq X$の内部$A^{\circ}$
  \begin{equation}
    A^{\circ} := \bigcup \sets{O \in \mathcal{O} | O \subseteq A} \in \mathcal{O}
  \end{equation}
  これは$A$に含まれる最大の開集合のこと。$A$の内点$x$とは$A^{\circ}$の元のこと。
  \item 閉集合とは,ある開集合の補集合になっているものとする。
  すると,閉集合族$\mathfrak{A}$は以下を満たす。:

  \midashi{A1. } $X \in \mathfrak{A}$,$\emptyset \in \mathfrak{A}$

  \midashi{A2. } $F_{1}, F_{2}, \dots, F_{k} \in \mathfrak{A} \Longrightarrow F_{1} \cup F_{2} \cup \dots \cup F_{k} \in \mathfrak{A}$

  \midashi{A3. } 任意の集合$\Lambda$に対して$F_{\lambda} \in \mathfrak{A} $($\lambda \in \Lambda$)$\Longrightarrow {\displaystyle \bigcap_{\lambda} F_{\lambda} \in \mathfrak{A}}$

  \item $A$の閉包$\overline{A}$
  \begin{equation}
    \overline{A} := \bigcap \sets{F \in \mathfrak{A} | A \subseteq F} \in \mathfrak{A}
  \end{equation}
  これは$A$を含む最小の閉集合のこと。$A$の触点$x$とは$\overline{A}$の元のこと。
  \item $N$が$x \in X$の近傍であるとは,$x \in N^{\circ}$となること,すなわち,
  \begin{equation}
    \exists O \in \mathcal{O}, \quad x \in O \subseteq N
  \end{equation}
  となること。特に$N$が開集合であるとき,$N$を$x$の開近傍という。
  \item $\mathcal{N}(x)$:$x$の近傍全体
\end{itemize}

位相の与え方にはいろいろある。
\begin{enumerate}
  \item {\bf (A1)},{\bf (A2)},{\bf (A3)}を満たす集合族$\mathfrak{A}$(閉集合族)を指定する。
  \item 各$A \subseteq X$に$A^{\circ}$を対応させる写像$2^{X} \longrightarrow 2^{X}$を指定する。(\nw{開核作用子})
  \item 各$A \subseteq X$に$\overline{A}$を対応させる写像$2^{X} \longrightarrow 2^{X}$を指定する。(\nw{閉包作用子})
  \item 各点$x$に$\mathcal{N}(x)$を対応させる写像$X \longrightarrow 2^{2^{X}}$を指定する。
\end{enumerate}

ただし,指定する写像はどんなものでもいいわけではない。
それぞれある条件を満たすような写像に限られる。
例えば,閉包作用子$\tau \colon 2^{X} \longrightarrow 2^{X}$は次を満たす必要がある。
\begin{itemize}
  \vspace{-0.5\baselineskip}
  \item $\tau (\emptyset) = \emptyset$
  \item $A \subseteq \tau(A)$
  \item $\tau (A \cup B) = \tau (A) \cup \tau (B)$
  \item $\tau (\tau (A)) = \tau (A)$
  \vspace{-0.5\baselineskip}
\end{itemize}

これらから自然に開集合族が決まる。
例えば,閉包作用子に対しては,$A = \tau (A)$となるような$A$を閉集合と定める。
詳細は演習。

\sukima \midashi{\large 連続写像}

\renewcommand{\arraystretch}{1}
\begin{tabular}{ll@{\,}l@{\,}l}
  位相空間 & $X$, & $\mathcal{O}_{X}$:開集合族, & $\mathfrak{A}_{X}$:閉集合族 \\
          & $Y$, & $\mathcal{O}_{Y}$:開集合族, & $\mathfrak{A}_{Y}$:閉集合族
\end{tabular}
\renewcommand{\arraystretch}{1.3}

\begin{teigi}[連続性]
  写像$f \colon X \longrightarrow Y$が$x \in X$で\nw{連続} $\defines$ 
  「$\forall N$:$f(x)$の近傍 $\Longrightarrow$ $f^{-1}(N)$は$x$の近傍」
\end{teigi}

\midashi{距離空間の連続性との同値性}
\begin{align*}
  x \in X \text{で連続} & \Longleftrightarrow \forall \varepsilon > 0, \quad \exists \delta > 0, \quad d_{X} (x,y) < \delta \Longrightarrow d_{Y}(f(x), f(y)) < \varepsilon \\
  & \Longleftrightarrow \forall \varepsilon > 0, \quad \exists \delta > 0, \quad N(x,\delta) \subseteq f^{-1} (N(f(x), \varepsilon)) \\
  & \Longleftrightarrow f(x) \text{の任意の近傍の逆像は} x \text{の近傍}
\end{align*}

\begin{teiri}
  \label{isou.renzoku}
  以下の{\bf (1)},{\bf (2)},{\bf (3)}は同値。:
  
  \midashi{(1)} $f$は$X$の各点で連続

  \midashi{(2)} $\forall O \in \mathcal{O}_{Y}, \quad f^{-1}(O) \in \mathcal{O}_{X}$

  \midashi{(3)} $\forall F \in \mathfrak{A}_{Y}, \quad f^{-1}(F) \in \mathfrak{A}_{X}$
\end{teiri}

\begin{proof}
  \midashi{(1) $\Longrightarrow$ (2):} $O \in \mathcal{O}_{Y}$を任意にとり,$x \in f^{-1}(O)$とする。$O$は$f(x)$の開近傍である。
  連続性の定義からある$x$の開近傍$O_{x}$が存在して,$O_{x} \subseteq f^{-1} (O)$である。
  \begin{equation}
    f^{-1} (O) = \bigcup_{x \in f^{-1}(O)} \sets{x} \subseteq \bigcup_{x \in f^{-1}(O)} O_{x} \subseteq f^{-1}(O)
  \end{equation}
  であるから,${\displaystyle f^{-1}(O) = \bigcup_{x \in f^{-1} (O)} O_{x} \in \mathcal{O}_{X}}$である。

  \midashi{(2) $\Longrightarrow$ (3):} $F \in \mathfrak{A}_{Y}$を任意にとる。$Y \setminus F \in \mathcal{O}_{Y}$であり,
  したがって,$f^{-1} (Y \setminus F) = X \setminus f^{-1}(F) \in \mathcal{O}_{X}$である。
  よって,$f^{-1} (F) \in \mathfrak{A}_{X}$である。

  \midashi{(3) $\Longrightarrow$ (2):} $O \in \mathcal{O}_{Y}$を任意にとる。$Y \setminus O \in \mathfrak{A}_{Y}$であり,
  したがって,$f^{-1} (Y \setminus O) = X \setminus f^{-1} (O) \in \mathfrak{A}_{X}$である。
  よって,$f^{-1} (O) \in \mathcal{O}_{X}$である。

  \midashi{(2) $\Longrightarrow$ (1):} $x \in X$を任意にとる。$N$を$f(x)$の近傍として,その内部を$O := N^{\circ}$とすれば$f(x) \in O$である。
  {\bf (2)}より$f^{-1}(O) \in \mathcal{O}_{X}$であってこれは$x$の開近傍である。
  $x \in f^{-1} (O) \subseteq f^{-1} (N)$である。
\end{proof}

\begin{teigi}[連続・同相]
  \midashi{(1)} $f \colon X \longrightarrow Y$が\nw{連続} $\defines$ \rref{定理}{isou.renzoku}の条件{\bf (1)}から{\bf (3)}のどれかの条件を満たす
  
  \midashi{(2)} $f \colon X \longrightarrow Y$が\nw{同相(homeomorphic)} $\defines$ $f$:連続全単射 かつ $f^{-1} \colon Y \longrightarrow X$:連続
  
  $X$と$Y$の間に同相写像が存在するとき,$X$と$Y$は\nw{同相}あるいは\nw{位相同型}といい,
  記号$X \simeq Y$で表す。
\end{teigi}

\midashi{演習.} 全単射連続写像であって同相でないものの例を与えよ。

\begin{teigi}[位相の強弱]
  $X$上の2つの位相$\mathcal{O}$,$\mathcal{O}'$を考える。
  $\mathcal{O} \subseteq \mathcal{O}'$のとき,
  「$\mathcal{O}$は$\mathcal{O}'$より\nw{弱い}位相である」,「$\mathcal{O}'$は$\mathcal{O}$より\nw{強い}位相である」という。
\end{teigi}

つまり,$\mathcal{O}$で連続な写像は,$\mathcal{O}'$でも連続な写像になる。
{\footnotesize「使ったことない」「弱い強い混乱するわロクでもない概念」などと。}

\sukima \midashi{\large いろいろな位相}

\newcommand{\dtimes}{\times\mkern-16mu\times}

\begin{teigi}[誘導位相]
  $X$:集合,$(Y,\mathcal{O}_{Y})$:位相空間,$f \colon X \longrightarrow Y$とする。
  \begin{equation}
    \mathcal{O}_{X} := \sets{f^{-1}(O) | O \in \mathcal{O}_{Y}}
  \end{equation}
  を$f$による\nw{誘導位相}という。
\end{teigi}

$f^{-1} (O \cup O') = f^{-1} (O) \cup f^{-1} (O')$,
$f^{-1} (O \cap O') = f^{-1} (O) \cap f^{-1} (O')$に注意すると,位相の定義の条件を満たすことは理解できる。

\begin{teigi}[相対位相]
  $(X, \mathcal{O}_{X})$:位相空間,$Z \subseteq X$:部分集合。
  \begin{equation}
    \mathcal{O}_{Z} := \sets{O \cap Z | O \in \mathcal{O}_{X}}
  \end{equation}
  とすると,$(Z, \mathcal{O}_{Z})$は$(X, \mathcal{O}_{X}$の部分(位相)空間をなす。
  この位相を\nw{相対位相}という。
\end{teigi}

相対位相は包含写像$Z \hookrightarrow X$による誘導位相でもある。

\sukima \midashi{直積位相}

$(X, \mathcal{O}_{X})$,$(Y, \mathcal{O}_{Y})$:位相空間として,
直積$X \times Y = \sets{(x,y) | x \in X, y \in Y}$に位相を与えたい。しかし,
\begin{equation}
  \mathcal{B} := \sets{O \times O' | O \in \mathcal{O}_{X}, O' \in \mathcal{O}_{Y}}
\end{equation}
は開集合の条件を満たさない。そこで$\mathcal{B}$が「生成する」位相$\mathcal{O}_{X} \dtimes \mathcal{O}_{Y}$を
\begin{equation}
  \mathcal{O}_{X} \dtimes \mathcal{O}_{Y} := \sets{ \bigcup_{W \in \mathcal{B}'} W | \mathcal{B}' \subseteq \mathcal{B}}
\end{equation}
で定めると,$(X \times Y, \mathcal{O}_{X} \dtimes \mathcal{O}_{Y})$は位相空間になる。

3つ以上の位相空間に対しても同様にして直積位相を定めることができる。

\begin{rei}[$(\mathbb{R}^{2}, d_{2})$の位相]
  $x$の$d_{2}$の下での$\varepsilon$-近傍に含まれる$d_{\infty}$の下での$\varepsilon'$-近傍をとることができる。
  $\mathbb{R}^{2}$の開集合は$d_{\infty}$の近傍たちの和集合で書くことができる。
  $d_{\infty}$の$\varepsilon'$-近傍はすべて$\mathcal{B}$の元である。
  よって,$\mathbb{R}^{2}$の位相は$\mathbb{R} \times \mathbb{R}$の直積位相に等しい。
\end{rei}

\sukima \midashi{商位相}

2つの位相空間を貼り合わせて新たな位相空間を作りたい。
\begin{center}  
  \begin{tikzpicture}
    \draw (150:0.5) arc [start angle = 30, end angle = 330, radius = 1.5] -- cycle;
    \draw (30:0.5) arc [start angle = 150, end angle = -150, radius = 1.5] -- cycle;
    \draw[<->] (-0.3,-0.5) -- (0.3,-0.5);
    \node at (-1.75,-0.5){$X$} node at (1.75,-0.5){$Y$};
    \draw[->] (3.75,-0.5) -- (4.75,-0.5);
    \draw (8,0.25) arc [start angle = 30, end angle = 330, radius = 1.5];
    \draw (8,0.25) arc [start angle = 150, end angle = -150, radius = 1.5];
    \draw[dashed] (8,0.25) -- (8,-1.25);
    \node at (6.75,-0.5){$X$} node at (9.25,-0.5){$Y$};
  \end{tikzpicture}
\end{center}
% 貼り合わせる図 if needed

\begin{itemize}
  \vspace{-0.5\baselineskip}
  \item まず$X$と$Y$の直和$X \amalg Y$を考える。$X \amalg Y$に位相を入れる。
  $\mathcal{O}_{X \amalg Y} := \sets{O \cup O' | O \in \mathcal{O}_{X}, O' \in \mathcal{O}_{Y}}$
  \item 同一視したい点たちを同値関係$\sim$で同一視して,商集合$(X \amalg Y) / \sim$を作る。
  \item 商集合に商位相を入れる。
  \vspace{-0.5\baselineskip}
\end{itemize}

位相空間$(X,\mathcal{O})$,$\sim$:$X$上の同値関係,
$X / \sim$:商集合,$\varphi \colon X \longrightarrow X / \sim$:自然な射影(自身を代表元とする同値類を返す)
\begin{equation}
  {\mathcal{O} / \sim} := \sets{H \subseteq {X / \sim} | \varphi^{-1}(H) \subseteq \mathcal{O}}
\end{equation}
$({X / \sim}, {\mathcal{O} / \sim})$は位相空間になる。

\sukima \midashi{\large 重要な位相空間の例}

$\mathbb{R}^{n}$:Euclid空間(位相はEuclid距離から)

\begin{itemize}
  \vspace{-0.5\baselineskip}
  \item $S^{n}$:$n$次元球面
  \begin{equation}
    S^{n} := \sets{x \in \mathbb{R}^{n+1} | \sum_{j=1}^{n+1} {x_{j}}^{2} = 1}
  \end{equation}
  位相は$\mathbb{R}^{n+1}$からの相対位相とする。

  \item $D^{n}$:$n$次元ディスク
  \begin{equation}
    D^{n} := \sets{x \in \mathbb{R}^{n} | \sum_{j=1}^{n} {x_{j}}^{2} \le 1}
  \end{equation}

  \begin{center}
    $D^{2} \simeq$
    \begin{tikzpicture}[baseline=12pt] \filldraw[fill=black!30,draw = black] (0,0) rectangle (1,1); \end{tikzpicture}
      , \quad $S^{2} \simeq {D^{2} / \sim} \simeq$
    \begin{tikzpicture}[baseline=12pt]
      \filldraw[fill=black!30] (0,0) rectangle (1,1);
      \draw (0.55,-0.1) -- (0.45, 0) -- (0.55,0.1);
      \draw (0.9,0.45) -- (1, 0.55) -- (1.1,0.45);
      \draw (-0.1,0.6) -- (0, 0.5) -- (0.1,0.6);
      \draw (-0.1,0.5) -- (0, 0.4) -- (0.1,0.5);
      \draw (0.4,1.1) -- (0.5,1) -- (0.4,0.9);
      \draw (0.5,1.1) -- (0.6,1) -- (0.5,0.9);
    \end{tikzpicture}
  \end{center}

  \item $P^{n}$:射影空間
  \begin{equation}
    P^{n} := (\mathbb{R}^{n+1} \setminus \sets{0}) / \sim
  \end{equation}
  このときの同値関係は
  \begin{equation}
    (x_{1},x_{2},\dots,x_{n+1}) \sim (y_{1},y_{2},\dots,y_{n+1}) \defines \exists \alpha \in \mathbb{R}, \quad (x_{1},x_{2},\dots,x_{n+1}) = \alpha (y_{1},y_{2},\dots,y_{n+1})
  \end{equation}
  で定めている。
  位相は$\mathbb{R}^{n+1}$の部分空間$+$商位相で定める。
  ちなみに$S_{+}^{n}$は上半球面として
  \begin{align}
    P^{n} &\simeq S^{n} / (\text{対蹠点を同一視}) \\
    &\simeq S_{+}^{n} / (\text{対蹠点を同一視}) \\
    &\simeq D^{n-1} / (\text{対蹠点を同一視})
  \end{align}
  が成り立つ。
  
  このことを$n=1$のときで図示する。
  \begin{center}
    \begin{tikzpicture}
      \draw[->] (-1.6,0) -- (1.6,0);
      \draw[->] (0,-1.6) -- (0,1.6);
      \draw (0,0) circle [radius=1.25];
      \draw[dashed] (60:1.25) -- (240:1.25);
      \filldraw (60:1.25) circle [radius=0.05] (240:1.25) circle [radius=0.05];
      \node at (2,0){$\simeq$};
      \node at (0,-2) {$P^{2} \simeq S^{2} / (対蹠点を同一視)$};
    \end{tikzpicture}
    \begin{tikzpicture}
      \draw[->] (-1.6,0) -- (1.6,0);
      \draw[->] (0,-1.6) -- (0,1.6);
      \draw (1.25,0) arc [start angle=0, end angle=180, radius=1.25];
      \filldraw (1.25,0) circle [radius=0.05] (-1.25,0) circle [radius=0.05];
      \draw[<-] (-1.25,-0.25) -- (0.25,-1) node[right]{同一視};
      \draw[<-] (1.25,-0.25) -- (1.25,-0.6);
      \node at (2,0){$\simeq$};
      \node at (0,-2) {$S^{2}_{+} / (対蹠点を同一視)$};
    \end{tikzpicture}
    \begin{tikzpicture}
      \draw[white] (0,-1.6) -- (0,1.6); %位置合わせ
      \draw (0,0) circle [radius=1.25];
      \node at (0,-2) {$S^{1}$};
    \end{tikzpicture}
  \end{center}

  \item $T^{n}$:$n$次元トーラス
  \begin{equation}
    T^{n} := \underbrace{S^{1} \times S^{1} \times \dots \times S^{1}}_{n}
  \end{equation}
  位相は直積位相で定める。

  \item M\"obiusの輪

  \begin{tikzpicture}[baseline=12pt]
    \filldraw[fill opacity=.3, draw = black] (0,0) rectangle (2,1);
    \draw (1.9,0.55) -- (2, 0.45) -- (2.1,0.55);
    \draw (-0.1,0.45) -- (0, 0.55) -- (0.1,0.45);
  \end{tikzpicture}

  \item Kleinの壺

  \begin{tikzpicture}[baseline=12pt]
    \filldraw[fill opacity=.3, draw = black] (0,0) rectangle (2,1);
    \draw (1.9,0.55) -- (2, 0.45) -- (2.1,0.55);
    \draw (-0.1,0.55) -- (0, 0.45) -- (0.1,0.55);
    \draw (0.9,1.1) -- (1,1) -- (0.9,0.9);
    \draw (1,1.1) -- (1.1,1) -- (1,0.9);
    \draw (1,0.1) -- (0.9,0) -- (1,-0.1);
    \draw (1.1,0.1) -- (1,0) -- (1.1,-0.1);
  \end{tikzpicture}
  \vspace{-0.5\baselineskip}
\end{itemize}
% 適宜図を挿入

\sukima
\begin{teigi}(Hausdorff空間)
  $X$:位相空間。
  \vspace{-0.5\baselineskip}
  \begin{equation*}
    \begin{array}{lll}
      X \colon \text{\nw{Hausdorff空間}} & \defines & 「\forall x,y \in X, \quad x \neq y \Longrightarrow \exists U \in \mathcal{O}, \quad
      \exists V \in \mathcal{O}, \quad x \in U, \quad y \in V, \quad U \cap V = \emptyset」
    \end{array}
  \end{equation*}
\end{teigi}

\begin{prop}
  距離空間はHausdorff空間である。
\end{prop}

\begin{teigi}[多様体]
  $M$:Hausdorff空間。
  \vspace{-0.5\baselineskip}
  \begin{equation*}
    \begin{array}{lll}
      M \colon \text{\nw{$n$次元多様体}} & \defines & \forall p \in M, \quad \exists U \colon \text{$p$の開近傍}, \quad
      \exists U' \colon \mathbb{R}^{n} \text{の開集合}, \quad
      \exists \varphi \colon U \longrightarrow U' \subseteq \mathbb{R}^{n}, \quad \varphi \colon \text{同相写像}
    \end{array}
  \end{equation*}
\end{teigi}

\begin{rei}
  $S^{n}$,$P^{n}$,$T^{n}$は多様体。
\end{rei}

\midashi{演習:}$f \colon \mathbb{R}^{n} \longrightarrow \mathbb{R}$に対して
$\sets{x \in \mathbb{R}^{n} | f(x) = 0}$が多様体となるのはどんなときか。

\begin{rei}[単体複体]
  $S_{\lambda}$($\lambda \in \Lambda$):単体(これも位相空間の一種)の集合
  \begin{equation}
    X := \coprod_{\lambda \in \Lambda} S_{\lambda} = \bigcup_{\lambda \in \Lambda} S_{\lambda} \times \sets{\lambda}
  \end{equation}
  として,ここに同値関係$\sim$を貼り合わせの同一視の関係として定める。
  商空間$K = X / \sim$で全体の単体複体を表す。
\end{rei}

\expandafter\ifx\csname readornot\endcsname\relax
  \end{document}
\fi
 % 位相空間
\expandafter\ifx\csname readornot\endcsname\relax
  \documentclass[uplatex]{jsarticle}    
  \usepackage{octopus}
  \usepackage{url}
  %%%% コマンド定義専用のtexファイル
\renewcommand{\postpartname}{章}
\renewcommand{\thepart}{\arabic{part}}
\renewcommand{\thesection}{\thepart.\arabic{section}}
\makeatletter\renewcommand{\theequation}{\thesection.\arabic{equation}}\@addtoreset{equation}{section}\makeatother
\newcommand{\octopuspart}[1]{\newpage\part{#1}\setcounter{section}{0}\vspace{3\baselineskip}}
\renewcommand{\restriction}[2]{\left. #1 \right|_{#2}}
\DeclareMathOperator{\dcup}{\dot{\cup}}
\DeclareMathOperator{\conv}{conv}
\DeclareMathOperator{\Image}{Im}
\DeclareMathOperator{\Kernel}{Ker}
\DeclareMathOperator{\diag}{diag}
\DeclareMathOperator{\rank}{rank}
\DeclareMathOperator{\sgn}{sgn}
\DeclareMathOperator{\rot}{rot}
  \usetikzlibrary{calc}
  \begin{document}
\fi

\section{連結性}
$(X, \mathcal{O})$を位相空間とする。

\sukima \midashi{復習}(省略)

\begin{teigi}[連結]
  \begin{flushleft}
    \begin{tabular}{llll}
      $X$:\nw{連結} & $\defines$ & $\mathcal{O} \cap \mathfrak{A} = \sets{\emptyset, X}$ & ($X$上開かつ閉なる集合は$\emptyset$,$X$のみ)\\
      $A \subseteq X$:連結 & $\defines$ & $A$が$X$の部分位相空間として連結である
    \end{tabular}
  \end{flushleft}
\end{teigi}

すなわち,$X$が連結{\bf でない}ことが次の二つの同値な条件で特徴づけられる。:
\begin{center}
  $\exists O \in \mathcal{O} \colon \quad O \neq \emptyset, \quad X \setminus O \in \mathcal{O}$ \\
  $X$は非交で非空な開集合の和集合に分解される
\end{center}

\begin{prop}
  $f \colon X \longrightarrow Y$:連続,$A \subseteq X$:連結のとき,$f(A)$:連結
\end{prop}

「連結性」は連続写像によって「不変」な性質であり,
位相的な性質であるといえる。(同相写像によって不変)

\begin{proof}
  $B \subseteq f(A)$を開かつ閉な集合とする。
  \begin{align}
    & \exists G: Y\text{上開}, \quad \exists F: Y\text{上閉}, \quad B = G \cap f(A) = F \cap f(A) \\
    \to & f^{-1}(B) = f^{-1} (G) \cap f^{-1}(f(A)) = f^{-1}(F) \cap f^{-1}(f(A)) \\
    \to & f^{-1}(B) \cap A = f^{-1}(G) \cap A = f^{-1}(F) \cap A
  \end{align}
  $f$の連続性より$f^{-1}(G)$:開,$f^{-1}(F)$:閉。
  よって,相対位相の定義より,$f^{-1}(G) \cap A$:$A$上開,$f^{-1}(F) \cap A$:$A$上閉。
  $f^{-1}(B) \cap A$は$A$の相対位相で開かつ閉。
  $A$の連結性より$f^{-1}(B) \cap A = \emptyset \: \text{or} \: A$である。

  $f^{-1} (B) \cap A = \emptyset$なら$B = \emptyset$,
  $f^{-1} (B) \cap A = A$なら$f(f^{-1}(B)) \cap f(A) = f(A)$であるから$B = f(A)$である。
  つまり,$f(A)$上で開かつ閉な集合は$\emptyset$か$f(A)$であるので,$f(A)$は連結である。
\end{proof}

\begin{prop}
  $A, B \subseteq X$に対し,$A \subseteq B \subseteq \overline{A}$であるとする。
  このとき,$A$:連結$\Longrightarrow B$:連結である。
  特に,連結な集合の閉包は連結である。
\end{prop}

\begin{proof}
  $B' \subseteq B$:$B$上開かつ閉とする。
  \begin{equation}
    \exists G: X\text{上開},\quad \exists F: X\text{上閉}, \quad B' = G \cap B = F \cap B
  \end{equation}
  であり,$A \subseteq B$より
  $A \cap B' = G \cap A = F \cap A$である。よって,$A \cap B'$は$A$上開かつ閉である。
  よって,$A$の連結性から$A \cap B' = \emptyset \: \text{or} \: A$である。

  $A \cap B' = \emptyset$のとき,$G \cap A = \emptyset$であり,$G \cap \overline{A} = \emptyset$,よって,$B' = \emptyset$である。

  $A \cap B' = A$のとき,$F \supseteq A$であるから,$F \supseteq \overline{A} \supseteq B$であり,$B' = F \cap B = B$である。

  これより$B$の連結性が従う。
\end{proof}

\begin{prop}
  \label{prop:renketu.union}
  $A,B$:連結,$A \cap B \neq \emptyset$とする。このとき$A \cup B$:連結である。
\end{prop}

\begin{proof}
  $N \subseteq A \cup B$:$A \cup B$上開かつ閉とする。
  \begin{equation}
    \exists G: X\text{上開},\quad \exists F: X\text{上閉}, \quad N = G \cap (A \cup B) = F \cap (A \cup B)
  \end{equation}
  特に$N \cap A = G \cap A = F \cap A$であり,$N \cap A$は$A$上開かつ閉。
  $N \cap B = G \cap B = F \cap B$であり,$N \cap B$は$B$上開かつ閉。
  よって,$N \cap A = \emptyset \: \text{or} \: A$,$N \cap B = \emptyset \: \text{or} \: B$である。

  もし$N \cap A = \emptyset$ならば$A \cap B \neq \emptyset$より$N \cap B = B$となることはない。よって,$N = \emptyset$である。

  もし$N \cap A = A$ならば$N \supseteq A \supseteq A \cap B \neq \emptyset$より$N \cap B = B$である。よって,$N = A \cup B$である。
\end{proof}

同様に次も成立する。

\begin{prop}
  \label{prop:renketu.union.gen}
  $\forall \lambda \in \Lambda, A_{\lambda}$:連結,
  ${\displaystyle \bigcap_{\lambda \in \Lambda} A_{\lambda} \neq \emptyset}$とする。
  このとき,${\displaystyle \bigcup_{\lambda \in \Lambda} A_{\lambda}}$:連結である。
\end{prop}

\begin{teigi}[連結成分]
  関係$\sim$を
  \begin{equation}
    x \sim y \quad \defines \quad \exists A \subseteq X \text{:連結}, \quad x \in A, \quad y \in A
  \end{equation}
  で定めるとこれは同値関係である。この同値関係による同値類を\nw{連結成分}という。
  $C_{x}$で$x$を含む連結成分を表すとする。
\end{teigi}

\begin{proof}
  \midashi{$\sim$が同値関係をなすこと}

  $\sets{x}$は連結であるから,$x \sim x$である。
  また,明らかに$x \sim y$ならば$y \sim x$である。
  $x \sim y$かつ$y \sim z$とする。このとき,ある連結な集合$A,B$が存在して,
  $x, y \in A$かつ$y, z \in B$である。いま,$y \in A \cap B \neq \emptyset$であるから,
  \rref{命題}{prop:renketu.union}より$A \cup B$は連結である。
  $x, z \in A \cup B$であるから,$x \sim z$である。
\end{proof}

\begin{prop}
  $C_{x}$は$x$を含む最大の連結部分集合である。
  また,$C_{x}$は閉集合である。
\end{prop}

\begin{proof}
  $A$を$x$を含む連結な集合とする。このとき,すべての$y \in A$に対して$y \sim x$であるから,$A \subseteq C_{x}$である。
  よって,${\displaystyle \bigcup \sets{A | x \in A \text{かつ} Aは連結}} \subseteq C_{x}$であり,結局,
  ${\displaystyle \bigcup \sets{A | x \in A \text{かつ} Aは連結}} = C_{x}$である。
  これより\rref{命題}{prop:renketu.union.gen}から$C_{x}$は連結である。
  よって,$C_{x}$の最大性が従う。また,最大性から
  $\overline{C_{x}} = C_{x}$となり,$C_{x}$は閉集合である。
\end{proof}

\begin{prop}
  $X,Y$:連結とする。このとき直積位相のもとで$X \times Y$:連結である。
\end{prop}

\begin{proof}
  $(x,y), (x', y') \in X \times Y$をとる。$C_{(x,y)} = C_{(x',y')}$を示せばよい。
  いま,$Y$:連結ならば$\sets{x} \times Y$:連結である。
  \begin{quote}
    背理法による。すなわち,$\sets{x} \times Y$が連結でないとする。
    このとき,ある非空非交な$\sets{x} \times Y$上の開集合$A,B$が存在して,$\sets{x} \times Y = A \cup B$である。
    このときさらにある非空な$Y$上の開集合$A',B'$が存在して
    $A = \sets{(x,y) | y \in A'} = \sets{x} \times A'$,
    $B = \sets{(x,y) | y \in B'} = \sets{x} \times B'$である。
    よって,$A' \cup B' = Y$となり$Y$は連結でない。これは矛盾。
  \end{quote}
  同様に$X$:連結ならば$X \times \sets{y'}$:連結である。
  したがって,連結成分を定義したときの同値関係を$\sim$として
  $(x,y) \sim (x, y') \sim (x',y')$となる。よって,$C_{(x,y)} = C_{(x',y')}$である。
\end{proof}

\begin{prop}[中間値の定理]
  $X$:連結,$f \colon X \longrightarrow \mathbb{R}$:連続,
  $x,y \in X$ならば$f(x) < f(y)$とする。
  このとき,
  \begin{equation}
    \forall \alpha \in \mathbb{R}, \quad f(x) < \alpha < f(y) \Longrightarrow \exists z \in X, \quad f(z) = \alpha
  \end{equation}
\end{prop}

\begin{proof}
  背理法による。$f^{-1} (\alpha) = \emptyset$とする。
  $M := f^{-1}((- \infty, \alpha))$は$f$の連続性から開集合である。
  また,$f^{-1}((- \infty, \alpha]) = f^{-1}((- \infty, \alpha)) \cup f^{-1}(\alpha) = f^{-1}((- \infty, \alpha))$は$f$の連続性から閉集合である。
  ここで,$x \in M$であるが$y \notin M$であるので,これは$X$の連結性に矛盾する。
\end{proof}

\begin{hodai}
  $a<b$として,$\left[ a,b \right]$は連結。
\end{hodai}

\begin{proof}
  背理法による。
  $O, O'$:開,$O \cap O' \neq \emptyset$,$\left[ a,b \right] = O \cup O'$とする。
  $a \in O$としても一般性を欠かない。
  $c = \sup \sets{x| \left[ a, x \right] \subseteq O}$とすると,$a < c < b$であり,
  任意の$\varepsilon > 0$に対して$c - \varepsilon \notin O'$,$c + \varepsilon \notin O$である。

  いま$c \in O \cup O'$であるが,$c \in O'$とすると,$O'$が開集合であることと$c - \varepsilon \notin O'$であることが矛盾。
  $c \in O$とすると,$O$が開集合であることと$c + \varepsilon \notin O$であることが矛盾。
  よって,$\left[ a,b \right]$は連結。
\end{proof}

\begin{teigi}[弧状連結性]
  $X$:\nw{弧状連結} $\defines$ $\forall x, y \in X, \quad \exists f \colon \left[ 0, 1 \right] \longrightarrow X$:連続,$f(0) = x$,$f(1) = y$

  $A \subseteq X$:弧状連結 $\defines$ $A$が部分位相空間として連結
\end{teigi}

\begin{prop}
  弧状連結ならば連結である。
\end{prop}

\begin{proof}
  $f([0,1])$は連結であるので,任意の$x,y \in X$に対してそれらを含む連結な集合が存在する。
  よって,$C_{x} = C_{y}$であり,$x,y$の任意性より$C_{x} = X$である。
\end{proof}

\midashi{注意}:$[0,1]$が連結であることは自明ではない。

\midashi{注意}:一般に「連結ならば弧状連結」は成立しない。

\begin{prop}
  $A \subseteq \mathbb{R}^{n}$:開集合について,
  「$A$:弧状連結 $\Longleftrightarrow$ $A$:連結」が成立する。
\end{prop}

\begin{proof}
  \midashi{$\Longleftarrow$:} $x \in A$,$O$:$x$とパスで結ばれる$A$の点全体の集合とする。
  $O$は開集合である。
  \begin{quote}
    $A$は開集合であるから,任意の$y \in O$に対してある$\delta > 0$が存在して$N(y,\delta) \subseteq A$である。
    $N(y,\delta)$に属する任意の点$z$は$y$とパスで結ばれるので,$x$と$z$もパスで結ばれる。
    よって$N(y,\delta) \subseteq O$である。
  \end{quote}
  また,$A \setminus O$も開集合である。
  \begin{quote}
    $A$は開集合であるから,任意の$y \in A \setminus O$に対して,ある$\delta > 0$が存在して$N(y,\delta) \subseteq A$である。
    $O \cap N(y,\delta) = \emptyset$でなければ$y$と$x$はパスで結ばれてしまう。
    よって,$N(y,\delta) \subseteq A \setminus O$である。
  \end{quote}
  いま,$A$は連結であるから$A = O$である。よって$A$は弧状連結である。
\end{proof}

\begin{prop}
  $X,Y$:弧状連結 $\Longrightarrow$ $X \times Y$:弧状連結
\end{prop}

\begin{proof}
  $(x, y) , (\tilde{x}, \tilde{y}) \in X \times Y$を任意にとる。
  このとき,$X$の連結性から$(\tilde{x}, \tilde{y})$と$(x, \tilde{y})$はパスで結ばれる。
  また,$Y$の連結性から$(x, \tilde{y})$と$(x,y)$はパスで結ばれる。
  したがって,$(x,y)$と$(\tilde{x}, \tilde{y})$はパスで結ばれる。
  これより,$X \times Y$は弧状連結である。
\end{proof}

\midashi{演習:}上のいくつかの{\bf 命題}について,「連結」を「弧状連結」に変えても成立するかいなかを議論せよ。

\expandafter\ifx\csname readornot\endcsname\relax
  \end{document}
\fi % 連結性
\expandafter\ifx\csname readornot\endcsname\relax
  \documentclass[uplatex]{jsarticle}    
  \usepackage{octopus}
  \usepackage{url}
  %%%% コマンド定義専用のtexファイル
\renewcommand{\postpartname}{章}
\renewcommand{\thepart}{\arabic{part}}
\renewcommand{\thesection}{\thepart.\arabic{section}}
\makeatletter\renewcommand{\theequation}{\thesection.\arabic{equation}}\@addtoreset{equation}{section}\makeatother
\newcommand{\octopuspart}[1]{\newpage\part{#1}\setcounter{section}{0}\vspace{3\baselineskip}}
\renewcommand{\restriction}[2]{\left. #1 \right|_{#2}}
\DeclareMathOperator{\dcup}{\dot{\cup}}
\DeclareMathOperator{\conv}{conv}
\DeclareMathOperator{\Image}{Im}
\DeclareMathOperator{\Kernel}{Ker}
\DeclareMathOperator{\diag}{diag}
\DeclareMathOperator{\rank}{rank}
\DeclareMathOperator{\sgn}{sgn}
\DeclareMathOperator{\rot}{rot}
  \usetikzlibrary{calc}
  \begin{document}
\fi

\section{コンパクト性}
$(X, \mathcal{O})$:位相空間,$A \subseteq X$とする。

\begin{teigi}[被覆]
  $\mathcal{C} \subseteq 2^{X}$:$A$の\nw{被覆} $\defines$ ${\displaystyle A \subseteq \bigcup_{C \in \mathcal{C}} C}$

  特に,$\mathcal{C} \subseteq \mathcal{O}$のとき,$\mathcal{C}$を\nw{開被覆}という。
\end{teigi}

\begin{teigi}[コンパクト]
  \midashi{(1)} $A \subseteq X$:\nw{コンパクト} $\defines$ $\forall \mathcal{C}$:$A$の開被覆,$\exists O_{1}, \dots, O_{k} \in \mathcal{C}$,${\displaystyle A \subseteq \bigcup_{j=1}^{k} O_{j}}$

  \midashi{(2)} $X$がコンパクトのとき,$(X, \mathcal{O})$をコンパクト空間という。
\end{teigi}

$A \subseteq X$がコンパクトであることは,標語的に「任意の開被覆は有限部分開被覆を含む」ということができる。
また,$A \subseteq X$がコンパクトであることは,$(A, \mathcal{O}_{A})$がコンパクト空間であることと同値である。
ただし,$\mathcal{O}_{A}$は相対位相。

\begin{hodai}
  $A_{1}, \dots, A_{k} \subseteq X$ :コンパクト $\Longrightarrow$ $A_{1} \cup \dots \cup A_{k}$:コンパクト
\end{hodai}

\begin{hodai}
  $(X, \mathcal{O})$:コンパクト $\Longrightarrow$ $A \in \mathfrak{A}$:コンパクト
\end{hodai}

\begin{proof}
  $A$の開被覆と$X \setminus A$が$X$の開被覆になっていることから従う。
\end{proof}

\begin{hodai}
  $X$:コンパクト,$f \colon X \longrightarrow Y$:連続 $\Longrightarrow$ $f(X)$:コンパクト
\end{hodai}

\begin{proof}
  $\mathcal{C}$:$f(X)$の開被覆とすると,$f^{-1}(\mathcal{C}) := \sets{f^{-1} (O) | O \in \mathcal{C}}$は$f$の連続性から$f^{-1}(\mathcal{C}) \subseteq \mathcal{O}$を満たし,
  これは$X$の開被覆である。
  $X$のコンパクト性から,ある$O_{1}, \dots, O_{k} \in \mathcal{C}$が存在して,
  $f^{-1}(O_{1}) \cap \dots \cap f^{-1}(O_{k}) = X$を満たす。
  このとき,$f(X) \subseteq O_{1} \cap \dots \cap O_{k}$である。
\end{proof}

\begin{hodai}
  $(X, d)$:距離空間とする。
  $A \subseteq X$:コンパクト $\Longrightarrow$ $A$は有界な閉集合
\end{hodai}

\begin{proof}
  \midashi{[有界性]} $x \in X$とする。開集合族$\sets{N(x,r) | r \in \mathbb{N}}$は$A$の被覆である。
  したがってコンパクト性から有限個の$r_{1} < r_{2} < \dots < r_{k}$が存在して$\sets{N(x,r_{i})}_{i=1}^{k}$は$A$の開被覆である。
  よって,$A \subseteq N(x,r_{k})$であり,$A$は有界。

  \midashi{[閉性]} $X \setminus A$が開集合であることを示す。$y \in X \setminus A$とする。
  ${\displaystyle \sets{X \setminus \overline{N \left( y, \frac{1}{r} \right)} | r \in \mathbb{N}}}$は$A$の開被覆である。
  したがってコンパクト性から有限個の$r_{1} < r_{2} < \dots < r_{k}$が存在して${\displaystyle \sets{X \setminus \overline{N \left( y,\frac{1}{r_{i}} \right)}}_{i=1}^{k}}$は$A$の開被覆である。
  よって,${\displaystyle A \subseteq X \setminus \overline{N \left( y,\frac{1}{r_{1}} \right)}}$であり,
  ${\displaystyle N \left( y, \frac{1}{r_{1}} \right) \subseteq X \setminus A}$であるから,$X \setminus A$は開集合である。
\end{proof}

\begin{corr}
  $X$:コンパクト,$f \colon X \longrightarrow \mathbb{R}$:連続とする。
  このとき,${\displaystyle \max_{x \in X} f(x)}$,${\displaystyle \min_{x \in X} f(x)}$が存在する。
\end{corr}

\begin{proof}
  $f(X)$はコンパクトであり有界閉集合であるから。
\end{proof}

\midashi{演習} 有界閉集合がコンパクトにならない距離空間の例を挙げよ。

\begin{hodai}[Lebesgue数の補題]
  $(X,d)$:コンパクトな距離空間,$\mathcal{C}$:$X$の開被覆とする。このとき
  \begin{equation}
    \exists \delta > 0, \quad \forall x \in X, \quad \exists O \in \mathcal{C}, \quad N(x, \delta) \subseteq O
  \end{equation}
  を満たす。
\end{hodai}

\begin{proof}
  $X$のコンパクト性から有限な部分開被覆$\sets{O_{i}}_{i=1}^{k} \subseteq \mathcal{C}$が存在する。
  \begin{equation}
    f(x) := \frac{1}{k} \sum_{i=1}^{k} d(x, X \setminus O_{i})
  \end{equation}
  とすると$f$は連続である。被覆性から常に$f(x) > 0$である。
  すなわち${\displaystyle \delta := \min_{x \in X} f(x) > 0}$である。よって,$\forall x \in X, \quad f(x) \ge \delta$である。
  これより,すべての$x \in X$に対してある$i \in \sets{1,\dots,k}$が存在して$d(x, X \setminus O_{i}) \ge \delta$である。
  よって,$N(x, \delta) \subseteq O_{i}$である。
\end{proof}

\begin{hodai}[Heine-Borelの被覆定理]
  $\mathbb{R}$の閉区間$I:=[a,b] \subseteq \mathbb{R}$はコンパクト
\end{hodai}

\begin{proof}
  $\mathcal{C}$:$[a,b]$の開被覆,$I':= \sets{x \in I | \mathcal{C}\text{は}[a,x]\text{の有限開被覆を含む}}$とする。
  明らかに$a \in I'$であるから$I' \neq \emptyset$である。
  $I' \subseteq I = [a,b]$であるから$c := \sup I' \le b$である。
  $\exists C_{0} \in \mathcal{C}, \quad c \in C_{0}$であり$\exists \varepsilon > 0, \quad [c-\varepsilon, c+\varepsilon] \subseteq C_{0}$である。
  $c = \sup I'$であるから,ある有限開被覆$\mathcal{C}'$が存在して$\mathcal{C}' \subseteq \mathcal{C}$であり${\displaystyle [a,c-\varepsilon] \subseteq \bigcup_{C' \in \mathcal{C}'} C'}$である。
  このとき,${\displaystyle [a,c+\varepsilon] \subseteq C_{0} \cup \bigcup_{C' \in \mathcal{C}'} C'}$である。
  これより$[a,c+\varepsilon]$には有限部分開被覆が存在する。以上より$c = b$となる。
\end{proof}

\begin{prop}
  $X,Y$:コンパクト $\Longrightarrow$ $X \times Y$:コンパクト
\end{prop}

\begin{proof}
  直積位相$\mathcal{O}_{X \times Y} = \sets{\bigcup B | B \subseteq \mathcal{O}_{X} \times \mathcal{O}_{Y}}$を考える。
  $\mathcal{C}$:$X \times Y$の開被覆とすると,
  \begin{equation}
    \mathcal{C} = \sets{\bigcup B_{\lambda} | B_{\lambda} \subseteq \mathcal{O}_{X} \times \mathcal{O}_{Y}} = \sets{U_{\lambda} \times V_{\lambda} | U_{\lambda} \in \mathcal{O}_{X}, V_{\lambda} \in \mathcal{O}_{Y}, \lambda \in \Lambda}
  \end{equation}
  としてよい。すべての$x \in X$に対して$\sets{x} \times Y$は$Y$と同相であり,これはコンパクトである。
  よって$\exists \Lambda(x)$:$\Lambda$の有限部分集合,${\displaystyle \sets{x} \times Y \subseteq \bigcup_{\lambda \in \Lambda(x)} U_{\lambda} \times V_{\lambda}}$となる。
  また${\displaystyle U_{x} := \bigcap_{\lambda \in \Lambda(x)} U_{\lambda} \in \mathcal{O}_{X}}$であり,${\displaystyle U_{x} \times Y \subseteq \bigcup_{\lambda \in \Lambda(x)} U_{\lambda} \times V_{\lambda}}$である。
  すると$\sets{U_{x}}_{x \in X}$は$X$の開被覆である。
  $X$のコンパクト性から有限個の$x_{1},\dots,x_{n}$が存在して$U_{x_{1}} \cup \dots \cup U_{x_{n}} = X$である。
  よって,${\displaystyle X \times Y = \bigcup_{k=1}^{n} U_{x_{k}} \times Y = \bigcup_{k=1}^{n} \bigcup_{\lambda \in \Lambda(x_{k})} U_{\lambda} \times V_{\lambda}}$であって,$X \times Y$はコンパクトである。
\end{proof}

\begin{teiri}[Tychonoffの定理]
  $X_{\lambda}$:コンパクト($\lambda \in \Lambda$) $\Longrightarrow$ ${\displaystyle \prod_{\lambda} X_{\lambda}}$:コンパクト
\end{teiri}

\begin{prop}
  $X \subseteq \mathbb{R}^{n}$に対して次が成り立つ。
  \begin{center}
    $X$:コンパクト $\Longleftrightarrow$ $X$:有界閉集合
  \end{center}
\end{prop}

\begin{proof}
  $\Longrightarrow$向きは既に示した。

  $\Longrightarrow$向きについて示す。
  $X \subseteq [a_{1}, b_{1}] \times [a_{2}, b_{2}] \times \dots \times [a_{n}, b_{n}] \subseteq \mathbb{R}^{n}$と書くことができる。
  各$[a_{i},b_{i}]$はコンパクトであり,その直積$[a_{1}, b_{1}] \times [a_{2}, b_{2}] \times \dots \times [a_{n}, b_{n}]$もコンパクトである。
  コンパクト集合の部分閉集合はコンパクトであるから$X$はコンパクトである。
\end{proof}

\sukima \midashi{\large コンパクトハウスドルフ空間}

\begin{prop}
  $(X, \mathcal{O})$:Hausdorffとする。
  $A \subseteq X$:コンパクト,$x \in X \setminus A$とすると,
  \begin{equation}
    \exists U, V \in \mathcal{O}, \quad A \subseteq U, \quad x \in V, \quad U \cap V = \emptyset
  \end{equation}
\end{prop}

\begin{proof}
  $X$のHausdorff性から,任意の$a \in A$に対して
  \begin{equation}
    \exists U_{a}, V_{a} \in \mathcal{O}, \quad a \in U_{a}, \quad x \in V_{a}, \quad U_{a} \cap V_{a} = \emptyset
  \end{equation}
  である。$\sets{U_{a} | a \in A}$は$A$の開被覆であるが,コンパクト性からある$a_{1}, \dots, a_{k} \in A$が存在して
  $U := U_{a_{1}} \cup \dots \cup U_{a_{k}} \supseteq A$である。
  $x \in V_{a} \in \mathcal{O}$であるから$i=1, \dots, k$に対して$x \notin \overline{U_{a_{i}}}$である。

  $\overline{U} = \overline{U_{a_{1}}} \cup \overline{U_{a_{2}}} \cup \dots \cup \overline{U_{a_{k}}}$であり,
  $V := X \setminus \overline{U}$とおくとこれは開集合であり,かつ$U \cap V = \emptyset$である。また,$x \in V$である。 
\end{proof}

\begin{corr}
  $X$:Hausdorff,$A \subseteq X$:コンパクトとする。このとき$A$は閉集合である。
\end{corr}

\begin{proof}
  $X \setminus A$の各点$x$は$V \subseteq X \setminus A$となる開近傍をもつので。
\end{proof}

\begin{corr}
  $X$:コンパクト,$Y$:Hausdorff,$f \colon X \longrightarrow Y$:連続全単射とする。このとき,$f$:同相写像。
\end{corr}

\begin{proof}
  逆写像の連続性を示すため,「任意の$X$上の閉集合$A$に対して$f(A)$が$Y$上の閉集合であること」を示す。
  コンパクト空間上の閉集合である$A$はコンパクトであり,コンパクト集合の連続写像による像である$f(A)$はコンパクトである。
  Hausdorff空間上のコンパクト集合は閉集合であるから$f(A)$は$Y$上閉である。
\end{proof}

\begin{teigi}[局所コンパクト]
  $X$:\nw{局所コンパクト}  $\defines$ 任意の$x \in X$に対してある$x$のコンパクトな近傍が存在する
\end{teigi}

\begin{prop}
  コンパクト $\Longrightarrow$ 局所コンパクト
\end{prop}

\begin{proof}
  それ自身をコンパクトな近傍としてとればよい。
\end{proof}

\sukima \midashi{\large 閉曲面}

\begin{teigi}[閉曲面]
  \nw{閉曲面} $\defines$ コンパクトな2次元多様体
\end{teigi}

$S$:閉曲面 $\Longrightarrow$ 有限個の2次元の座標近傍がとれて,これらで$S$を覆い尽くせる。

\begin{rei}
  \begin{itemize}
    \item 球面$S^{2}$(有界閉集合なのでコンパクト)
    \item トーラス$S^{1} \times S^{1}$(コンパクト空間の直積)
    \item 射影平面$P^{2}$($S^{2}/\sim$なので$S^{2}$の連続像)
  \end{itemize}
\end{rei}

\midashi{連結な閉曲面とはどんなものか}
\begin{itemize}
  \vspace{-0.5\baselineskip}
  \item 有限個の多角形をその辺に沿って貼り合わせて得られる。
  \item 1つの辺はちょうど2つの多角形の共通辺。($\leftarrow$多様体の定義から)
\end{itemize}

\begin{rei}
  % なんか楽しそうな絵がある
\end{rei}

\sukima \midashi{\large 分類定理}

(連結な閉曲面) $\simeq$ (穴が$k$個の浮き輪) $\simeq$ (射影平面を$k'$個とりつけたもの)

穴の個数$k$を\nw{種数(genus)}という。

\begin{rei}[種数]
  球の種数は0,トーラスの種数は1である。
\end{rei}

\midashi{演習}これを証明せよ。

\expandafter\ifx\csname readornot\endcsname\relax
  \end{document}
\fi % コンパクト性

\octopuspart{位相幾何}
連続変形に対する不変性

\expandafter\ifx\csname readornot\endcsname\relax
  \documentclass[uplatex]{jsarticle}    
  \usepackage{octopus}
  \usepackage{url}
  %%%% コマンド定義専用のtexファイル
\renewcommand{\postpartname}{章}
\renewcommand{\thepart}{\arabic{part}}
\renewcommand{\thesection}{\thepart.\arabic{section}}
\makeatletter\renewcommand{\theequation}{\thesection.\arabic{equation}}\@addtoreset{equation}{section}\makeatother
\newcommand{\octopuspart}[1]{\newpage\part{#1}\setcounter{section}{0}\vspace{3\baselineskip}}
\renewcommand{\restriction}[2]{\left. #1 \right|_{#2}}
\DeclareMathOperator{\dcup}{\dot{\cup}}
\DeclareMathOperator{\conv}{conv}
\DeclareMathOperator{\Image}{Im}
\DeclareMathOperator{\Kernel}{Ker}
\DeclareMathOperator{\diag}{diag}
\DeclareMathOperator{\rank}{rank}
\DeclareMathOperator{\sgn}{sgn}
\DeclareMathOperator{\rot}{rot}
  \usetikzlibrary{calc}
  \begin{document}
\fi

\section{ホモトピー}

\midashi{2つの空間が「同じ形」をしているとはどういうことか?}

\noindent $\xrightarrow{\text{ans}}$位相空間$X,Y$が位相同型 $\defines$ $\exists f \colon X \longrightarrow Y$:連続全単射,$f^{-1}$:連続

例:ドーナツとコーヒーカップ

\noindent $\xrightarrow{\text{ans}}$連続変形で移り合う

例:太字のAと細字のAと円 %図

例:ディスクと一点(明らかにこの2つは位相同型ではない)

このような「空間の連続変形」を定式化したい

\sukima \midashi{\large 変形レトラクション}

$X$:位相空間,$A \subseteq X$とする。

\begin{teigi}[変形レトラクション]
  $X$から$A$への\nw{変形レトラクション(deformation retraction)}とは,
  次の条件を満たす$\sets{f_{t} \colon X \longrightarrow X}_{t \in [0,1]}$のこととする。:
  \begin{itemize}
    \vspace{-0.5\baselineskip}
    \item $f_{0} = \mathrm{id}_{X}$
    \item $f_{1}(X) = A$
    \item $\restriction{f_{t}}{A} = \mathrm{id}_{A}$($\forall t \in [0,1]$)
    \item $F \colon \mapdef{X \times [0,1]}{X}{(x,t)}{f_{t}(x)}$は連続写像
  \end{itemize}
\end{teigi}

\begin{rei}
  $X \subseteq \mathbb{R}^{n}$:凸集合 i.e. $\forall x,y \in X, \quad \forall \lambda \in [0,1], \quad (1-\lambda) x + \lambda y \in X$とする。
  $z \in X$として
  \begin{equation}
    f_{t} \colon \mapdef{X}{X}{x}{(1-t) x + t z} ,\qquad(t \in [0,1])
  \end{equation}
  は,$X$から$\sets{z}$への変形レトラクションである。(凸性のおかげで写像の行き先が常に$X$に入る)
\end{rei}

\begin{rei}
  $X = \mathbb{R}^{n} \setminus \sets{0}$とする。
  \begin{equation}
    f_{t} \colon \mapdef{X}{X}{x}{(1-t)x+t\dfrac{x}{\left\| x \right\|}}
  \end{equation}
  は$\mathbb{R}^{n} \setminus \sets{0}$から$S^{n-1}$への変形レトラクションである。
\end{rei}

\midashi{問題.} $A \subseteq \mathbb{R}^{n}$:閉凸集合として,$\mathbb{R}^{n}$から$A$への変形レトラクションを構成せよ。
({\bf ヒント}:近接点写像)

\sukima \midashi{\large ホモトピー}

$X,Y$:位相空間

\begin{teigi}[ホモトピー]
  $\sets{f_{t}}$:\nw{ホモトピー(homotopy)} $\defines$ 次の条件を満たす$\sets{f_{t} \colon X \longrightarrow Y}_{t \in [0,1]}$のこと:
  \begin{itemize}
    \vspace{-0.5\baselineskip}
    \item $F \colon \mapdef{X \times [0,1]}{Y}{(x,t)}{f_{t}(x)}$が連続
  \end{itemize}

  また,$g \colon X \longrightarrow Y$と$h \colon X \longrightarrow Y$をつなぐホモトピー$\sets{f_{t}}$とは
  上の$\sets{f_{t}}$に関する条件に加えて,$f_{0} = g$,$f_{1} = h$を満たすもののことを指す。
  このとき,$g$と$h$は\nw{ホモトープ(homotope)}または\nw{ホモトピック(homotopic)}といい,$g \simeq h$と書く。
\end{teigi}

\begin{hodai}
  ホモトープの関係$\simeq$は同値関係
\end{hodai}

\begin{proof}
  \begin{itemize}
    \vspace{-0.5\baselineskip}
    \item $g \simeq g$は$f_{t} \equiv g$とすればよい。
    \item $g \simeq h$とすると,このときのホモトピー$f_{t}$に対して$f_{1-t}$を考えれば$h \simeq g$である。
    \item $f \simeq g$かつ$g \simeq h$とする。$f$と$g$を結ぶホモトピーを$\sets{f_{t}}$,$g$と$h$を結ぶホモトピーを$\sets{g_{t}}$とする。
    このとき,$f$と$h$の間に写像の族$\sets{\overline{f}_{t}}$を
    \begin{equation}
      \overline{f}_{t} = \begin{cases}
        f_{2t} & (0 \le t \le 1/2) \\
        g_{2t-1} & (1/2 \le t \le 1)
      \end{cases}
    \end{equation}
    で定める。すると,$\overline{f}_{0} = f$であり$\overline{f}_{1} = h$である。
    このとき,
    \begin{equation}
      \overline{F} \colon \mapdef{X \times [0,1]}{Y}{(x,t)}{\overline{f}_{t}(x)}
    \end{equation}
    は$X \times [0,1/2]$上と$X \times [1/2,1]$上のそれぞれで連続であり,この2つの集合はともに閉集合である。
    \rref{補題}{close.continuous}から, $f$は$X \times [0,1]$上でも連続であり,$f$と$h$を結ぶホモトピーになっている。
  \end{itemize}
\end{proof}

上の補題の証明に用いた事実を明記して証明する。:

\begin{hodai}
  \label{close.continuous}
  $X = A \cup B$,$A,B$:閉とする。$f \colon X \longrightarrow Y$に対して
  $\restriction{f}{A},\restriction{f}{B}$がそれぞれ連続であるならば$f$は$X$上で連続である。
\end{hodai}

\begin{proof}
  $F \subseteq Y$:閉とすると,$f^{-1} (F) = \restriction{f^{-1}}{A}(F) \cup \restriction{f^{-1}}{B}(F)$は有限個の閉集合のunionなので閉集合である。
\end{proof}

\begin{teigi}[ホモトピー類]
  ホモトピーの同値関係による同値類を\nw{ホモトピー類}という。
\end{teigi}

\sukima \midashi{注意.} 変形レトラクションは,恒等写像$\mathrm{id}_{X}$とレトラクション$r$を結ぶホモトピーである。
ここで,\nw{レトラクション}とは次を満たす写像$r \colon X \longrightarrow X$のことである。
\begin{itemize}
  \vspace{-0.5\baselineskip}
  \item $r(X) = A$
  \item $\restriction{r}{A} = \mathrm{id}_{A}$
\end{itemize}

\sukima \midashi{注意.} $Y \subseteq \mathbb{R}^{n}$のとき,$g \colon X \longrightarrow Y$,$h \colon X \longrightarrow Y$を結ぶホモトピーとして,
$(1-t)g+th$(これを\nw{線形ホモトピー}という)を作りたくなるが,
例えば凸性が保証されていないときには$(1-t)g(x) + t h(x) \notin Y$となる可能性があるので気をつける必要がある。

\sukima \midashi{注意.} もし$Y$が弧状連結であれば,$g(x)$と$h(x)$を結ぶパスに沿っていつもホモトピーを作ることができるのだろうか。
例えば$X = Y = S^{1}$として,$\mathrm{id}_{S^{1}}$と$S^{1}$上の一点へのレトラクションを結ぶホモトピーは{\bf できなさそう}である。

\sukima \midashi{\large ホモトピー同値}

\begin{teigi}[ホモトピー同値]
  位相空間$X,Y$が\nw{ホモトピー同値} $\defines$ $\exists f \colon X \longrightarrow Y$:連続,$\exists g \colon Y \longrightarrow X$:連続,
  $f \circ g \simeq \mathrm{id}_{Y}$,$g \circ f \simeq \mathrm{id}_{X}$

  また,このとき$X$と$Y$は\nw{同じホモトピー型をもつ}ともいう。ホモトピー同値であることを$X \simeq Y$と書く。
\end{teigi}

\midashi{注意.} 位相同型ならばホモトピー同値である。

\begin{hodai}
  ホモトピー同値の関係$\simeq$は同値関係
\end{hodai}

\begin{proof}
  ここでは推移律だけを示す。$X \simeq Y$かつ$Y \simeq Z$とする。
  \begin{align}
    &\exists f \colon X \longrightarrow Y, \quad \exists g \colon Y \longrightarrow X, \quad f \circ g \simeq \mathrm{id}_{Y}, \quad g \circ f \simeq \mathrm{id}_{X} \\
    &\exists f' \colon Y \longrightarrow Z, \quad \exists g' \colon Z \longrightarrow Y, \quad f' \circ g' \simeq \mathrm{id}_{Z}, \quad g' \circ f' \simeq \mathrm{id}_{Y}
  \end{align}
  であるから,$\overline{f} = f' \circ f$,$\overline{g} = g \circ g'$と定めれば,これは連続写像であり,
  $\overline{f} \circ \overline{g} = f' \circ f \circ g \circ g' \simeq f' \circ g' \simeq \mathrm{id}_{Z}$
  である。同様に,$\overline{g} \circ \overline{f} \simeq \mathrm{id}_{X}$である。したがって,$X \simeq Z$である。
\end{proof}

\begin{hodai}
  $X$から$A \subseteq X$への変形レトラクションがあるなら$X$と$A$はホモトピー同値である。
\end{hodai}

\begin{proof}
  $f_{t} \colon X \longrightarrow X$:$X$から$A$への変形レトラクションとする。つまり,$f_{0} = \mathrm{id}_{X}$,$f_{1}(X) = A$,$\restriction{f_{t}}{A} = \mathrm{id}_{A}$。
  $g \colon X \longrightarrow A$,$g(x) = f_{1} (x)$,$h \colon A \longrightarrow X$:包含写像とすれば,
  $g \circ h = \mathrm{id}_{A}$,$h \circ g = f_{1} \simeq f_{0} = \mathrm{id}_{X}$($\because f_{t}$:変形レトラクション)
\end{proof}

\begin{corr}
  {\bf 1.} $\mathbb{R}^{n}$,凸集合,1点はすべてホモトピー同値である。

  {\bf 2.} $\mathbb{R}^{n} \setminus \sets{0}$と$S^{n-1}$はホモトピー同値である。
\end{corr}

\begin{teigi}
  1点とホモトピー同値な空間を\nw{可縮}な空間という。
\end{teigi}

\midashi{注意.} 可縮な空間は凸集合の次に基本的な図形であるといえる。

\sukima \midashi{問題.} 超平面$H_{a,b} := \sets{x \in \mathbb{R}^{n} | \left\langle a, x \right\rangle = b}$,
半空間$H_{a,b}^{+} := \sets{x \in \mathbb{R}^{n} | \left\langle a, x \right\rangle \le b}$。
多面体$P$:有限個の半空間の積;${\displaystyle P = \bigcap_{i=1}^{k} H_{a_{i}, b_{i}}^{+}}$,
$P$の面:$P$または$P \subseteq H_{a,b}^{+}$なる$a,b$に対して$P \cap H_{a,b}$と書けるものである。

(非有界な)多面体の有界な面の和集合は可縮であることを示せ。

\begin{rei}
  トーラス$T^{2}$から1点を除く。
  \begin{center}
    \begin{tikzpicture}[baseline=12pt]
      \node at (0,0.5) {$T^{2} \setminus \sets{x}$};
      \node at (1,0.5) {$\simeq$};

      \filldraw[fill opacity=.3, draw=black] (1.5,0) rectangle +(1,1);
      \draw (1.4,0.45) -- (1.5, 0.55) -- (1.6,0.45);
      \draw (2.4,0.45) -- (2.5, 0.55) -- (2.6,0.45);
      \draw (1.9,1.1) -- (2,1) -- (1.9,0.9);
      \draw (2,1.1) -- (2.1,1) -- (2,0.9);
      \draw (1.9,0.1) -- (2,0) -- (1.9,-0.1);
      \draw (2,0.1) -- (2.1,0) -- (2,-0.1);
      \filldraw[fill=white, opacity=1, draw=black] (2,0.5) circle [radius=0.05];
      
      \node at (3,0.5) {$\simeq$};

      \draw (3.5,0) rectangle +(1,1);
      \draw (3.4,0.45) -- (3.5, 0.55) -- (3.6,0.45);
      \draw (4.4,0.45) -- (4.5, 0.55) -- (4.6,0.45);
      \draw (3.9,1.1) -- (4,1) -- (3.9,0.9);
      \draw (4,1.1) -- (4.1,1) -- (4,0.9);
      \draw (3.9,0.1) -- (4,0) -- (3.9,-0.1);
      \draw (4,0.1) -- (4.1,0) -- (4,-0.1);

      \node at (5,0.5) {$\simeq$};

      \draw (6,1) circle [x radius=0.5, y radius=0.2];
      \draw (6,0) circle [x radius=0.5, y radius=0.2];
      \draw (5.5,1) -- (5.5,0);
      \draw (6,0.9) -- (6.1,0.8) -- (6,0.7);
      \draw (6,-0.1) -- (6.1,-0.2) -- (6,-0.3);

      \node at (7,0.5) {$\simeq$};

      \draw (8,0.9) circle [x radius=0.3, y radius=0.4];
      \draw (8,0.1) circle [x radius=0.3, y radius=0.4];
      \filldraw (8,0.5) circle [radius=0.05];
    \end{tikzpicture}
  \end{center}
  $S^{1}$と$S^{1}$を1点で同一視した空間$S^{1} \vee S^{1}$とホモトピー同値。
\end{rei}

\begin{teigi}
  $X$と$Y$の\nw{wedge和} $X \vee Y$を次のように定める。
  $x \in X$,$y \in Y$をとって,
  \begin{equation}
    X \vee Y := X \amalg Y \,/\, x \sim y
  \end{equation}
\end{teigi}

\midashi{問題.} 閉曲面から何点か除いた空間のホモトピー型をいろいろ調べよ。

\begin{teigi}[セル複体(CW複体)]
  \nw{$n$--セル} $\simeq$ $n$次元ディスク$D^{n} = \sets{x \in \mathbb{R}^{n} | \left\| x \right\| \le 1}$
  とする。\nw{セル複体}とは以下のように帰納的につくられる空間をいう。
  \begin{itemize}
    \vspace{-0.5\baselineskip}
    \item $X^{0}$:0--セルからなる離散集合
    \item $X^{n}$:$n$--スケルトン
    
    $X^{n-1}$と$n$--セルの集合$D_{\alpha}^{n}$,$\varphi_{\alpha} \colon S_{\alpha}^{n} = \partial D_{\alpha}^{n} \longrightarrow X^{n-1}$($\alpha \in \Lambda$)を考えて,
    同値関係$\sim$を$x \sim \varphi_{\alpha}(x)$として,$X^{n}$を以下で定める。
    \begin{equation}
      X^{n} := \left( X^{n - 1} \amalg \coprod_{\alpha} D_{\alpha}^{n} \right) / \sim
    \end{equation}
  \end{itemize}
  セル複体の次元を含まれるセルの最大次元で定める。
\end{teigi}

\begin{rei}
  セル複体の例
\end{rei}
%%%%% あとで図を入れる %%%%%

\midashi{問題.} 他にもいろいろなセル複体をつくってみよ。

\begin{hodai}
  $X$:セル複体,$A \subseteq X$:可縮な部分複体とする。$X \sim A$を$A$を1点に潰して得られる空間とする。このとき,
  $X \simeq X / A$である。
\end{hodai}

\midashi{問題.} 証明せよ。

\begin{teigi}
  \nw{グラフ}とは1次元セル複体のことである。
\end{teigi}

\begin{prop}
  $X$:連結なグラフ,$n$:$X$の頂点数,$m$:$X$の枝数(取り付けた1--セルの個数)とする。このとき,
  $X \simeq \underbrace{S^{1} \vee S^{1} \vee \dots \vee S^{1}}_{m-n+1}$である。
\end{prop}

\begin{proof}
  グラフの\nw{全域木}とは,すべての頂点と接続するサイクルを含まないグラフとする。
  全域木の枝数は$n-1$である。
  グラフのうち全域木の部分を1点に潰して得られる空間を考えれば,これは先の全域木に入らない$m-n+1$本の枝がループになる。
\end{proof}

\midashi{参考.} house with two rooms

\expandafter\ifx\csname readornot\endcsname\relax
  \end{document}
\fi % ホモトピー
\expandafter\ifx\csname readornot\endcsname\relax
  \documentclass[uplatex]{jsarticle}    
  \usepackage{octopus}
  \usepackage{url}
  %%%% コマンド定義専用のtexファイル
\renewcommand{\postpartname}{章}
\renewcommand{\thepart}{\arabic{part}}
\renewcommand{\thesection}{\thepart.\arabic{section}}
\makeatletter\renewcommand{\theequation}{\thesection.\arabic{equation}}\@addtoreset{equation}{section}\makeatother
\newcommand{\octopuspart}[1]{\newpage\part{#1}\setcounter{section}{0}\vspace{3\baselineskip}}
\renewcommand{\restriction}[2]{\left. #1 \right|_{#2}}
\DeclareMathOperator{\dcup}{\dot{\cup}}
\DeclareMathOperator{\conv}{conv}
\DeclareMathOperator{\Image}{Im}
\DeclareMathOperator{\Kernel}{Ker}
\DeclareMathOperator{\diag}{diag}
\DeclareMathOperator{\rank}{rank}
\DeclareMathOperator{\sgn}{sgn}
\DeclareMathOperator{\rot}{rot}
  \usetikzlibrary{calc}
  \begin{document}
\fi

\definecolor{mygreen}{rgb}{0,0.75,0}
\section{基本群}

\midashi{\large これからの展望}

\begin{center}
  \begin{tikzpicture}
    \draw (0,0) circle [x radius=1, y radius=0.5] node {空間 $X$};
    \draw (0,-2) circle [x radius=1, y radius=0.5] node {空間 $Y$};
    \node at (3.5,0) {群$\pi_{1}(X)$};
    \node at (3.5,-2) {群$\pi_{1}(Y)$};
    \node[left] at (0,-1) {ホモトピー同値\hspace{1zw}\rotatebox[origin=c]{-90}{$\simeq$}};
    \node[right] at (3.5,-1) {\rotatebox[origin=c]{-90}{$\simeq$}\hspace{1zw}群同型};
    \draw[->] (1.5,0) -- (2.5,0);
    \draw[->] (1.5,-2) -- (2.5,-2);
  \end{tikzpicture}
\end{center}
となるような空間と群の対応を構成したい。

\sukima \midashi{\large 基本群}

$X$:位相空間
\begin{teigi}[パス]
  \nw{パス(path)} $\defines$ $f \colon [0,1] \longrightarrow X$:連続
\end{teigi}

\begin{teigi}[ホモトピー]
  $\sets{f_{t}}$:パスの\nw{ホモトピー} $\defines$ 次の条件を満たす$\sets{f_{t} \colon [0,1] \longrightarrow X}_{t \in [0,1]}$のこと:
  \begin{itemize}
    \vspace{-0.5\baselineskip}
    \item $\forall t \in [0,1],\quad f_{t}(0) = x_{0},\quad f_{t}(1) = x_{1}$
    \item $F \colon \mapdef{[0,1] \times [0,1]}{X}{(s,t)}{f_{t}(s)}$が連続
  \end{itemize}
\end{teigi}

\begin{teigi}[ホモトープ]
  2つのパス$f', f''$が\nw{ホモトープ} $\defines$ $f', f''$をつなぐホモトピー$\sets{f_{t}}$であって,
  $f_{0} = f'$,$f_{1} = f''$であるものが存在する

  このことを$f' \simeq f''$と書く。
\end{teigi}

\begin{prop}
  ホモトープの$\simeq$は同値関係である。
\end{prop}

\begin{proof}
  パスのホモトープは普通のホモトープの特別な場合であるから,そのときの同値性から従う。
\end{proof}

ホモトープの同値関係$\simeq$によるパス$f$の同値類$[f]$を\nw{ホモトピー類}という。

\begin{teigi}[パスの積(合成)]
  $f,g \colon [0,1] \longrightarrow X$:パス,$f(1) = g(0)$を満たすとする。
  パスの\nw{合成} $f \cdot g \colon [0,1] \longrightarrow X$を次で定義する。
  \begin{equation}
    f \cdot g (s) := \begin{cases}
      f(2s) & \left( 0 \le s \le \dfrac{1}{2} \right) \\
      g(2s-1) & \left( \dfrac{1}{2} \le s \le 1 \right)
    \end{cases}
  \end{equation}
\end{teigi}

\sukima \midashi{注意.} $f' \simeq f$,$g' \simeq g$ならば$f' \cdot g' \simeq f \cdot g$である。

なぜならば,$f$と$f'$を結ぶホモトピー$\sets{f_{t}}$,$g$と$g'$を結ぶホモトピー$\sets{g_{t}}$に対して
$\left( f \cdot g \right)_{t} := f_{t} \cdot g_{t}$とすれば,これが$f' \cdot g'$と$f \cdot g$を結ぶホモトピーになるから。

\begin{teigi}[ループ]
  $x_{0}$を\nw{基点(basepoint)とするループ(loop)} $\defines$ パス$f \colon [0,1] \longrightarrow X$であって,$f(0) = f(1) = x_{0}$
\end{teigi}

ループに対するホモトープの同値関係(パスのそれと同じように定められる)による同値類を考える。
集合$\pi_{1}(X, x_{0})$を次で定める。
\begin{equation}
  \pi_{1}(X, x_{0}) := \sets{[f] | f \text{:$x_{0}$を基点とするループ}}
\end{equation}

\begin{prop}
  $\pi_{1}(X,x_{0})$は積$[f] \cdot [g] := [f \cdot g]$の下で
  \footnote{もう少し正確にいうと,$\pi_{1}(X,x_{0})$のホモトピー類$F,G$の積$F \cdot G$を
  $F = [f]$,$G = [g]$となるようなループ$f,g$をとって,
  $F \cdot G := [f \cdot g]$と定義した。}\
  群になる。
\end{prop}

\begin{proof}
  \midashi{[積の定義のwell-defined性]} 

  まず,$x_{0}$を基点とするループの合成も$x_{0}$を基点とするループであることに注意する。
  $f' \in [f]$,$g' \in [g]$をとって$[f' \cdot g'] = [f \cdot g]$を示せばよい。
  これは$f' \simeq f$,$g' \simeq g$であるから,先の{\bf 注意.}より$f' \cdot g' \simeq f \cdot g$であるから従う。

  \sukima \midashi{[準備]}
  
  $\varphi \colon [0,1] \longrightarrow [0,1]$,$\varphi(0) = 0$,$\varphi(1) = 1$とする。
  このとき,ホモトピー$(f \circ \varphi)_{t} := f \circ \left( (1-t) \varphi + t \, \mathrm{id} \right)$を考えれば,
  これは$f \circ \varphi$と$f$を結ぶホモトピーであり,$(f \circ \varphi)_{0} = f \circ \varphi$,$(f \circ \varphi)_{1} = f$である。
  よって$[f \circ \varphi] = [f]$である。

  この$\varphi$を適切に定めることでパスの進むスピードを変換する。

  \sukima \midashi{[単位元の存在]}

  $c \colon [0,1] \longrightarrow X$を$c(s) \equiv x_{0}$(定ループ)で定める。$[c]$は単位元になる。
  \begin{quote}
    $\because)$ $f$を$x_{0}$を基点とするループとして,$c \cdot f$を考える。
    時刻$s$を
    \begin{center}
      \begin{tikzpicture}
        \filldraw (0,0) circle (0.05);
        \filldraw (0,1) circle (0.05);
        \filldraw (0,2) circle (0.05);
        \filldraw (2,0) circle (0.05);
        \filldraw (2,2) circle (0.05);
        \draw[->] (0,0) node[left]{0} -- (0,1) node[left]{1/2} -- (0,2) node[left]{1} -- (0,2.2);
        \draw[->] (2,0) node[right]{0} -- (2,2) node[right]{1} -- (2,2.2);
        \draw[dashed] (0,0) -- (2,0);
        \draw[dashed] (0,1) -- (2,0);
        \draw[dashed] (0,2) -- (2,2);
        \draw[->, mygreen] (0.3,1.8) -- (1.7,1.8);
        \draw[->, mygreen] (0.3,1.2) -- (1.7,0.5);
        \draw[->, mygreen] (0.3,0.75) -- (1.7,0.05);
        \draw[->, mygreen] (0.3,-0.1) -- (1.7,-0.1);
        \node at (1,1.5){\textcolor{mygreen}{$\varphi$}};
      \end{tikzpicture}
    \end{center}
    のように変換する。(つまり時刻1/2まで立ち止まってから2倍のスピードで移動する。)
    このとき,$f \circ \varphi = c \cdot f$である。
    よって,$[f] = [f \circ \varphi] = [c \cdot f] = [c] \cdot [f]$である。
    $[f] \cdot [c] = [f]$も同様。
  \end{quote}

  \sukima \midashi{[結合律]}
  
  $[f \cdot (g \cdot h)] = [(f \cdot g) \cdot h]$を示せばよい。
  時刻のスピードを
  \begin{center}
    \begin{tikzpicture}
      \filldraw (0,0) circle (0.05);
      \filldraw (0,1) circle (0.05);
      \filldraw (0,1.5) circle (0.05);
      \filldraw (0,2) circle (0.05);
      \filldraw (2,0) circle (0.05);
      \filldraw (2,0.5) circle (0.05);
      \filldraw (2,1) circle (0.05);
      \filldraw (2,2) circle (0.05);
      \draw[->] (0,0) node[left]{0} -- (0,1) node[left]{1/2} -- (0,1.5) node[left]{3/4} -- (0,2) node[left]{1} -- (0,2.2);
      \draw[->] (2,0) node[right]{0} -- (2,0.5) node[right]{1/4} -- (2,1) node[right]{1/2} -- (2,2) node[right]{1} -- (2,2.2);
      \draw[dashed] (0,0) -- (2,0);
      \draw[dashed] (0,1) -- (2,0.5);
      \draw[dashed] (0,1.5) -- (2,1);
      \draw[dashed] (0,2) -- (2,2);
      \draw[->, mygreen] (0.3,1.9) -- (1.7,1.9);
      \draw[->, mygreen] (0.3,1.6) -- (1.7,1.25);
      \draw[->, mygreen] (0.3,1.2) -- (1.7,0.85);
      \draw[->, mygreen] (0.3,0.75) -- (1.7,0.4);
      \draw[->, mygreen] (0.3,0.1) -- (1.7,0.1);
      \node at (1,-0.25){\textcolor{mygreen}{$\varphi$}};
      \node at (0.2,0.5){$f$};
      \node at (0.2,1.25){$g$};
      \node at (0.2,1.75){$h$};
      \node at (1.8,0.25){$f$};
      \node at (1.8,0.75){$g$};
      \node at (1.8,1.5){$h$};
    \end{tikzpicture}
  \end{center}
  のように変換する$\varphi$を考える。このとき,$(f \cdot (g \cdot h)) \circ \varphi = (f \cdot g) \cdot h$となる。
  したがって,$[f \cdot (g \cdot h)] = [(f \cdot g) \cdot h]$である。

  \sukima \midashi{[逆元]}
  
  $f \colon [0,1] \longrightarrow X$に対し,$\overline{f} \colon [0,1] \longrightarrow X$を
  $\overline{f}(s) := f(1-s)$で定義すると,$[f][\overline{f}] = [\overline{f}][f] = [c]$である。

  \begin{quote}
    $\because)$ $f \cdot \overline{f} \simeq c$を示す。
    $f_{t}$と$g_{t}$を
    \begin{align}
      & f_{t}(s) := \begin{cases}
        f(s) & (0 \le s \le t) \\
        f(t) & (t \le s \le 1)
      \end{cases}, \\
      & g_{t}(s) := \begin{cases}
        \overline{f}(1-t) = f(t) & (0 \le s \le 1-t) \\
        \overline{f}(s) & (1-t \le s \le 1)
      \end{cases}
    \end{align}
    で定義すると,$f_{t} \cdot g_{t}$は$f \cdot \overline{f}$と$c$を結ぶホモトピーになる。
  \end{quote}
\end{proof}

\begin{teigi}[基本群]
  $\pi_{1}(X,x_{0})$を,$x_{0}$を基点とする$X$の\nw{基本群(fundamental group)}という。
\end{teigi}

\sukima \midashi{\large 基点のとりかえ}

\renewcommand{\arraystretch}{1.0}
\begin{tabular}{lll}
  $x_{0}, x_{1} \in X$, & $h \colon [0,1] \longrightarrow X$,$h(0) = x_{0}$,$h(1) = x_{1}$ &($x_{0}$から$x_{1}$へのパス)\\
  & $\overline{h} \colon [0,1] \longrightarrow X$,$\overline{h}(s) = h(1-s)$ &($x_{1}$から$x_{0}$へのパス)\\
  \multicolumn{2}{l}{$f$:$x_{0}$を基点とするループ} \\
  \multicolumn{2}{l}{$\Rightarrow h \cdot f \cdot \overline{h}$:$x_{0}$を基点とするループ}
\end{tabular}
\renewcommand{\arraystretch}{1.3}

\begin{tikzpicture}
	\filldraw (0,0) circle [radius=0.05];
	\filldraw (2,0.8) circle [radius=0.05];
	\draw (0,0) node[above]{$x_{1}$} .. controls (1,0.35) .. (2,0.8) node[above left]{$x_{0}$};
	\draw (3,1) circle [x radius = 1, y radius = 0.8, rotate=15];
	\draw[->] (0.7,0.5) -- (1.5,0.8);
	\draw[<-] (0.8,0.1) -- (1.6,0.4);
	\draw (2.4,1.5) -- (2.4,1.6) -- (2.3,1.6);
	\draw (3.7,0.4) -- (3.6,0.4) -- (3.6,0.5);
	\draw (4,1.8) node{$f$};
	\draw (1,0.9) node{$h$};
	\draw (1.5,0) node{$\overline{h}$};
\end{tikzpicture}

いま,ホモトピー$\sets{f_{t}}$によって$f' \simeq f$であるとすると,
ホモトピー$\sets{h \cdot f_{t} \cdot \overline{h}}$によって$h \cdot f \cdot \overline{h} \simeq h \cdot f' \cdot \overline{h}$である。
したがって,
\begin{equation}
  \beta_{h} \colon \mapdef{\pi_{1}(X,x_{0})}{\pi_{1}(X,x_{1})}{\left[ f \right]}{\left[ h \cdot f \cdot \overline{h} \, \right]}
\end{equation}
はwell-definedになる。

\begin{prop}
  $\beta_{h}$は$\pi_{1}(X,x_{0})$から$\pi_{1}(X,x_{1})$への同型写像
\end{prop}

\begin{proof}
  \midashi{[準同型性]}
  \begin{equation}
    \beta_{h}([f] \cdot [g]) = [h \cdot f \cdot g \cdot \overline{h}] = [h \cdot f \cdot \overline{h} \cdot h \cdot g \cdot \overline{h}] = [h \cdot f \cdot \overline{h}] \cdot [h \cdot g \cdot \overline{h}] = \beta_{h}([f]) \cdot \beta_{h}([g])
  \end{equation}

  \sukima \midashi{[全単射性]}

  $\beta_{\overline{h}} \colon \pi_{1}(X, x_{0}) \longrightarrow \pi_{1}(X, x_{1})$を考える。
  \begin{equation}
    \beta_{\overline{h}} \circ \beta_{h} ([f]) = \beta_{\overline{h}} ([h \cdot f \cdot \overline{h}]) = [\overline{h} \cdot h \cdot f \cdot \overline{h} \cdot h] = [f]
  \end{equation}
  であるから$\beta_{\overline{h}} \circ \beta_{h} = \mathrm{id}$である。同様に$\beta_{h} \circ \beta_{\overline{h}} = \mathrm{id}$である。
  よって$\beta_{\overline{h}}$が$\beta_{h}$の逆写像である。
\end{proof}

\begin{prop}
  $X$が弧状連結ならば,任意の点$x_{0} \in X$に対して$x_{0}$を基点とする基本群$\pi_{1}(X,x_{0})$はすべて同型。
\end{prop}

そこで,$X$が弧状連結のときには基本群を単に$\pi_{1}(X)$と書く。

\begin{rei}
  $X \subseteq \mathbb{R}^{n}$:凸集合とする。任意のループ$f$は定ループ$c$とホモトープであるから,$\pi_{1}(X) = \sets{e}$(自明な群)である。
  同様に,$X$が可縮空間であるときも$\pi_{1}(X) = \sets{e}$である。
\end{rei}

\begin{teigi}[単連結]
  $X$:\nw{単連結(simply-connected)} $\defines$ $X$:弧状連結 かつ $\pi_{1}(X) = \sets{e}$
\end{teigi}

\begin{prop}
  $X$:単連結 $\Longleftrightarrow$ 「$\forall x, y \in X$,$f, f'$:$x$から$y$へのパス $\Longrightarrow$ $f \simeq f'$」
\end{prop}

\begin{proof}
  $x = y$のときを考えれば$\Longleftarrow$向きは明らかである。
  $\Longrightarrow$向きについては,$f \cdot \overline{f'}$がループであり$\pi_{1}(X) = \sets{e}$であることに注意すれば$f \simeq f \cdot \overline{f'} \cdot f' \simeq f'$である。 
\end{proof}

\begin{teigi}
  $\varphi \colon X \longrightarrow Y$:連続,$y_{0} = \varphi(x_{0})$とする。写像$\varphi_{*}$を次のように定める。
  \begin{equation}
    \varphi_{*} \colon \mapdef{\pi_{1}(X,x_{0})}{\pi_{1}(Y,y_{0})}{\left[ f \right]}{\left[ \varphi \circ f \right]}
  \end{equation}
\end{teigi}

\begin{hodai}
  $\varphi_{*}$はwell-definedであり準同型
\end{hodai}

\begin{proof}
  \midashi{[well-definedness]}

  ホモトピー$\sets{f_{t}}$によって$f \simeq f'$であるならば,ホモトピー$\sets{\varphi \circ f_{t}}$によって$\varphi \circ f \simeq \varphi \circ f'$である。

  \sukima \midashi{[準同型性]}
  \begin{equation}
    \varphi_{*} ([f] \cdot [g]) = \varphi_{*} ([f \cdot g]) = [ \varphi \circ (f \cdot g)] = [(\varphi \circ f) \cdot (\varphi \circ g)] = [\varphi \circ f] \cdot [\varphi \circ g] = \varphi_{*}([f]) \cdot \varphi_{*} ([g])
  \end{equation}
\end{proof}

\begin{hodai}
  \midashi{(1)} $\varphi \colon X \longrightarrow Y$,$\phi \colon Y \longrightarrow Z$に対して,$(\phi \circ \varphi)_{*} = \phi_{*} \circ \varphi_{*}$である。

  \midashi{(2)} $(\mathrm{id}_{X})_{*} = \mathrm{id}_{\pi_{1}(X,x_{0})}$
\end{hodai}

\begin{proof}
  \midashi{(1)} $(\phi \circ \varphi)_{*}([f]) = [\phi \circ \varphi \circ f] = \phi_{*}([\varphi \circ f]) = (\phi_{*} \circ \varphi_{*}) ([f])$

  \midashi{(2)} 明らか
\end{proof}

\begin{teiri}
  $\varphi \colon X \longrightarrow Y$:同相 $\Longrightarrow$ $\varphi_{*} \colon \pi_{1} (X, x_{0}) \longrightarrow \pi_{1} (Y, \varphi (x_{0}))$:同型
\end{teiri}

\begin{proof}
  $\phi_{*}$が準同型写像であることは既に示したことに注意する。

  いま,$\varphi^{-1} \circ \varphi = \mathrm{id}_{X}$から$(\varphi^{-1})_{*} \circ \varphi_{*} = \mathrm{id}_{\pi_{1}(X,x_{0})}$である。
  また,$\varphi \circ \varphi^{-1} = \mathrm{id}_{Y}$から$\varphi_{*} \circ (\varphi^{-1})_{*} = \mathrm{id}_{\pi_{1}(Y,\varphi(x_{0}))}$である。
  よって,$\varphi_{*}$は同型写像。
\end{proof}

\begin{corr}
  $\pi_{1}(X) \not\simeq \pi_{1}(Y)$なら$X$と$Y$は同相でない
\end{corr}

\begin{teiri}
  \label{kihongun.homequiv}
  $\varphi \colon X \longrightarrow Y$がホモトピー同値写像なら
  $\varphi_{*} \colon \pi_{1} (X, x_{0}) \longrightarrow \pi_{1} (Y, \varphi(x_{0}))$は同型写像
\end{teiri}

この\rref{定理}{kihongun.homequiv}の証明に際して,次の補題を用いる。

\begin{hodai}
  $\rho, \rho' \colon X \longrightarrow Y$がホモトピー$\sets{\rho_{t}}$によってホモトピー同値であり,
  $h \colon t \longmapsto \rho_{t}(x_{0})$:パス,とする。
  このとき,$\rho_{*} = \beta_{h} \circ \rho'_{*}$である。
  \begin{tikzpicture}
    \filldraw (0,0) circle [radius=0.05];
    \draw (0,0) node[below]{$x_{0}$} .. controls (2.5,0) and (0,2) .. (0,0);
    \node at (0.6,1.2) {$f$};
    \draw[->] (1.5,0) -- (2,0);
    \node at (1.75,0) [below]{$\rho_{t}$};
    \filldraw (2.8,1) circle [radius=0.05];
    \draw (2.8,1) node[above=8, left=-3]{$\rho'(x_{0})$} .. controls (4.5,0) and (4.3,2) .. (2.8,1);
    \filldraw (2.5,0) circle [radius=0.05];
    \draw (2.5,0) .. controls (4.2,-1) and (4,1) .. (2.5,0);
    \filldraw (2.3,-1) circle [radius=0.05];
    \draw (2.3,-1) node[below=8, left=-3]{$\rho(x_{0})$} .. controls (4,-2) and (3.8,0) .. (2.3,-1);
    \draw (2.3,-1) .. controls (2.5,0) .. (2.8,1);
    \node at (2.3,0.4) {$h$};
    \node at (4.5,1.3) {$\rho' \circ f$};
    \node at (4.2,0.3) {$\rho_{t} \circ f$};
    \node at (4,-0.7) {$\rho \circ f$};
  \end{tikzpicture}
\end{hodai}

\begin{proof}
  \begin{equation}
    (\beta_{h} \circ \rho'_{*}) ([f]) = \beta_{h} ([\rho' \circ f]) = [h \cdot \rho' \circ f \cdot \overline{h}] = [\rho \circ f] = \rho_{*} ([f])
  \end{equation}
  による。なお,後ろから2つ目の等号はホモトピー$\sets{h_{t} \cdot \rho_{t} \circ f \cdot \overline{h}_{t}}$による。
\end{proof}

\begin{proof}[{\bf \rref{定理}{kihongun.homequiv}の証明}]
  ホモトピー$\sets{\rho_{t}}$によって$\phi \circ \varphi \simeq \mathrm{id}$となるならば,あるパス$h$が存在して$\phi_{*} \circ \varphi_{*} = \beta_{h}$となる。
  特に写像$\phi_{*} \circ \varphi_{*}$は$\pi_{1}(X, x_{0})$と$\pi_{1}(X, (\phi \circ \varphi)(x_{0}))$の間の同型写像である。このことから$\phi_{*}$は全射,$\varphi_{*}$は単射である。
  同様に$\varphi_{*} \circ \phi_{*}$は同型写像であり,$\varphi_{*}$は全射,$\phi_{*}$は単射である。よって$\varphi_{*}$は全単射である。
\end{proof}

\sukima \midashi{\large $S^{n}$の基本群について}

\begin{teiri}
  $n \ge 2$に対して,$\pi_{1}(S^{n}) = \sets{e}$である。つまり,2次元以上の球面は単連結である。
\end{teiri}

\begin{proof}
  $f$を$x_{0} \in S^{n}$を基点とするループ,とする。

  このとき,$y \in S^{n}$であって$y \neq x_{0}$,$y \notin f([0,1])$を満たす点をとることができる。({\bf 注意.}へ)

  いま,$S^{n} \setminus \sets{y} \simeq \mathbb{R}^{n}$なので,$f$を$\mathbb{R}^{n}$のループとみると,
  これは1点にホモトピー同値である。
\end{proof}

\midashi{注意.} このような条件を満たす$y$がとれることは直感的には明らかであるが,本来は証明が必要な事柄である。

\midashi{問題.} では,これを証明せよ。

\sukima \midashi{注意.} $S^{1}$では$y \notin f([0,1])$をとることができない。

\sukima \midashi{\large Poincar\'e予想}\footnote{Перельман(Perelman)により証明された}
\begin{teiri}
  単連結なコンパクト3次元多様体は$S^{3}$に同相である
\end{teiri}

このことは2次以上のコンパクト多様体についても同様なことがいえる。

\sukima \midashi{\large $S^{1}$の基本群について}

ループ$f$が円周を$n$回「まわり」,$g$が円周を$m$回「まわる」とする。このとき,$f \cdot g$は円周を$n + m$回「まわる」といえる。
この写像$\pi_{1}(S^{1}) \longrightarrow \mathbb{Z}$は同型写像を与えるのではないかと思える。

このことをあとできちんと示す。

\begin{teiri}
  $\pi_{1}(S^{1}) \simeq \mathbb{Z}$
\end{teiri}

証明は,被覆空間の考え方を用いるので,後に回す。

\sukima \midashi{\large 直積の基本群}

\begin{teiri}
  $\pi_{1} (X \times Y, (x_{0}, y_{0})) \simeq \pi_{1}(X, x_{0}) \times \pi_{1}(Y, y_{0})$
\end{teiri}

\begin{proof}
  $f$:$(x_{0}, y_{0})$を基点とする$X \times Y$上のループ,とする。
  $f \colon [0,1] \longrightarrow X \times Y$であるから

  \renewcommand{\arraystretch}{1.0}
  \begin{tabular}{lll}
    $f = (g,h)$, & $g \colon [0,1] \longrightarrow X$, & $g(0) = g(1) = x_{0}$, \\
    & $h \colon [0,1] \longrightarrow Y$, & $h(0) = h(1) = y_{0}$
  \end{tabular}
  \renewcommand{\arraystretch}{1.3}

  と表せる。$g,h$は$X,Y$上のループである。
  
  写像
  \begin{equation}
    \varphi \colon \mapdef{\pi_{1}(X \times Y, (x_{0}, y_{0}))}{\pi_{1}(X, x_{0}) \times \pi_{1}(Y, y_{0})}{\left[ f \right]}{(\left[ g \right], \left[ h \right])}  
  \end{equation}
  は準同型写像である。(well-defined性も準同型性も示せる)

  同様に
  \begin{equation}
    \overline{\varphi} \colon \mapdef{\pi_{1}(X, x_{0}) \times \pi_{1}(Y, y_{0})}{\pi_{1}(X \times Y, (x_{0}, y_{0}))}{(\left[ g \right], \left[ h \right])}{\left[ (g,h) \right]}
  \end{equation}
  を考える。
  
  すると,$\overline{\varphi} \circ \varphi = \mathrm{id}$,$\varphi \circ \overline{\varphi} = \mathrm{id}$であり,$\simeq$が成り立つ。
\end{proof}

\sukima \midashi{\large トーラスの基本群}
\begin{corr}
  $\pi_{1}(T^{2}) \simeq \mathbb{Z} \times \mathbb{Z}$
\end{corr}

\begin{proof}
  $T^{2} = S^{1} \times S^{1}$であることから従う。
\end{proof}

\sukima \midashi{\large wedge和の基本群}

$X \vee Y := X \sqcup Y / \sim$,$\sim$を$x \in X$と$y \in Y$を同一視するような同値関係($x \sim y$)として定める。
$\pi_{1}(X)$と$\pi_{1}(Y)$から$\pi_{1}(X \vee Y)$が定まるであろうと考えられる。

\begin{teigi}[準備:群の自由積]
  $G,G'$:群とする。群の\nw{自由積(free product)}を
  \begin{equation}
    G * G' := \sets{g_{1}g_{2} \cdots g_{n} | n \ge 0, \quad g_{i} \in G \, \text{or} \, G', \quad g_{i} \neq e}
  \end{equation}
  とする。但し,$g_{1} \cdots g_{i} g_{i+1} \cdots g_{n}$が,$g_{i}, g_{i+1} \in G$または$g_{i}, g_{i+1} \in G'$であって$g_{i} \cdot g_{i+1} = h$であるならば
  これを$g_{1} \cdots g_{i-1} h g_{i+2} \cdots g_{n}$と同じ元であるとみなす。$h = e$ならばさらに$g_{1} \cdots g_{i-1} g_{i+2} \cdots g_{n}$と同じものであると見なす。
  ここで,積$\cdot$は列の連結:
  \begin{equation}
    ( g_{1} g_{2} \cdots g_{n}) \cdot (h_{1} h_{2} \cdots h_{m}) := g_{1} g_{2} \cdots g_{n} h_{1} h_{2} \cdots h_{m}
  \end{equation}
  で定める。
\end{teigi}

\begin{prop}
  $G * G'$は積$\cdot$により群をなす。単位元は空列である。
\end{prop}

この{\bf 命題}は自明ではないが,証明は演習とする。

\begin{teiri}
  \label{kihongun.Seifelt}
  $\pi_{1} (X \vee Y) \simeq \pi_{1} (X) * \pi_{1} (Y)$
\end{teiri}

\begin{corr}
  $\pi_{1} (S^{1} \vee S^{1} \vee \dots \vee S^{1}) \simeq \mathbb{Z} * \mathbb{Z} * \dots * \mathbb{Z}$
\end{corr}

意味を考えれば{\bf 確かにそんな気がする}。

\sukima \midashi{演習.} \rref{定理}{kihongun.Seifelt}はSeifelt-van Kampenの定理と呼ばれるものの特殊ケースである。
この定理について調べ(証明して),いろいろな空間・曲面の基本群を計算せよ。

\expandafter\ifx\csname readornot\endcsname\relax
  \end{document}
\fi % 基本群
\expandafter\ifx\csname readornot\endcsname\relax
  \documentclass[uplatex]{jsarticle}    
  \usepackage{octopus}
  \usepackage{url}
  %%%% コマンド定義専用のtexファイル
\renewcommand{\postpartname}{章}
\renewcommand{\thepart}{\arabic{part}}
\renewcommand{\thesection}{\thepart.\arabic{section}}
\makeatletter\renewcommand{\theequation}{\thesection.\arabic{equation}}\@addtoreset{equation}{section}\makeatother
\newcommand{\octopuspart}[1]{\newpage\part{#1}\setcounter{section}{0}\vspace{3\baselineskip}}
\renewcommand{\restriction}[2]{\left. #1 \right|_{#2}}
\DeclareMathOperator{\dcup}{\dot{\cup}}
\DeclareMathOperator{\conv}{conv}
\DeclareMathOperator{\Image}{Im}
\DeclareMathOperator{\Kernel}{Ker}
\DeclareMathOperator{\diag}{diag}
\DeclareMathOperator{\rank}{rank}
\DeclareMathOperator{\sgn}{sgn}
\DeclareMathOperator{\rot}{rot}
  \usetikzlibrary{calc}
  \begin{document}
\fi

\section{被覆空間}
\begin{teigi}[被覆空間]
  $E,X$を位相空間,$p\colon E \to X$を全射な連続関数とする.任意の$x\in X$について$x$の開近傍$U$が存在し,$p^{-1}(U)$が$E$の互いに交わらない開集合$V_\alpha\ (\alpha \in \Lambda)$で
  \begin{align*}
    p^{-1}(U) = \bigcup_{\alpha \in \Lambda} V_\alpha
  \end{align*}
  と表されているとする.このとき,$\restriction{p}{V_\alpha} \colon V_\alpha \to U$が同相写像であるとき,$p$を\textbf{被覆写像(covering map)},$E$を$X$の\textbf{被覆空間}という.
\end{teigi}

\begin{rei}
  $\mathbb{R}$は$S^1$の被覆空間.
\end{rei}
\begin{proof}
  $p\colon \mathbb{R} \to S^1$として$p(x) = (\cos 2\pi x,\sin 2\pi x)$を考えると,これは全射な連続関数になっている.
\end{proof}

\begin{teigi}[リフト]
  $p\colon E\to X$,$f\colon Y\to X$をそれぞれ連続写像とする.このとき$f$のリフト$\tilde{f}\colon Y\to E$を$p\circ \tilde{f} = f$を満たすものとして定義する.つまり$\tilde{f}$は以下を可換にする.
  \begin{align*}
    \begin{diagram}
      \node{ } \node{E} \arrow{s,r}{p}\\
      \node{Y} \arrow{ne,t}{\tilde{f}} \arrow{e,b}{f} \node{X}
    \end{diagram}
  \end{align*}
\end{teigi}

\begin{teiri}
  $f\colon \left[0,1\right]\to X$をパス,$f(0)=x_0$とする.このとき,$\tilde{x}_0 \in p^{-1}(\{x_0\})$に対して$f$のリフト$\tilde{f}\colon \left[0,1\right]\to E,\tilde{f}(0)=\tilde{x}_0$が一意に存在する.
\end{teiri}
\begin{proof}
  ルベーグ数の補題より,ある開被覆に対して$\delta > 0$が存在し,任意の$x \in \left[0,1\right]$に対してある$U_i \in\mathcal{C}$が存在して$\mathcal{N}(x;\delta) \subseteq U_i$が成り立つ.これより,$x_0$を含む$U_0$の逆像の連結成分$V_0$は一意に決まり,$\tilde{f}\colon \left[s_0,s_1\right]\to V_0$で$p\circ \tilde{f} = \restriction{f}{\left[s_0,s_1\right]}$となる$\tilde{f}$が一意に定まる.これを$\left[s_1,s_2\right],\left[s_2,s_3\right],\cdots$に対して行えばよい.
\end{proof}

\begin{teiri}
  $f_t \colon \left[0,1\right]\to X$をパスのホモトピー,$f_t(0)=x_0$とする.このとき,$\tilde{x}_0 \in p^{-1}(\{x_0\})$に対して$f_t$のリフト$\tilde{f_t}\colon \left[0,1\right]\to E,\tilde{f_t}(0)=\tilde{x}_0$が一意に存在する.
\end{teiri}
\begin{proof}
  上と同様の議論を$\left[0,1\right]\times \left[0,1\right]$に対して行う.
\end{proof}
\begin{teiri}
  $p\colon E\to X$を被覆写像,$\phi\colon \pi_1(X,x_0)\to p^{-1}(x_0)$を
  \begin{align*}
    \phi(\left[f\right]) := \tilde{f}(1)
  \end{align*}
  で定義する.ただし$\tilde{f}$は$f$のリフトで$\tilde{f}(0)=\tilde{x_0}$となるものとする.このとき,$E$が弧状連結なら$\phi$は全射,$E$が単連結なら$\phi$は全単射となる.
\end{teiri}
\begin{proof}
  任意の$z\in p^{-1}(x_0)$に対し$\tilde{x_0}$と$z$を結ぶパス$h$をとれば$\left[p\circ h\right] \in \pi_1(X,x_0)$で$\phi(\left[p\circ h\right]) = h(1) = z$となる.$E$が単連結のとき,$\phi(\left[f\right]) = \phi(\left[f'\right])=z\in p^{-1}(x_0)$.$\tilde{f},\tilde{f'}$は$\tilde{x_0}$と$z$を結ぶパスで,単連結性より$\tilde{f}\simeq \tilde{f'}$.よって$p\cdot \tilde{f} \simeq p\cdot \tilde{f'}$であり,$f\simeq f'$.
\end{proof}
\begin{teiri}
  $\pi_1(S^1) \simeq \mathbb{Z}$
\end{teiri}
\begin{proof}
  $p\colon \mathbb{R}\to S^1 , p(x) = (\cos 2\pi x,\sin 2\pi x)$とし,$x_0=(1,0)$を基点とする基本群を考える.$\mathbb{R}$は単連結であるから,$\phi \colon \pi_1(S^1)\to \mathbb{Z}$は全単射.また,
  \begin{align*}
    \phi(\left[f\right]\cdot \left[g\right]) = \phi(\left[f\right]) + \phi(\left[g\right])
  \end{align*}
  が成り立つので,$\phi$は準同型.よって$\pi_1(S^1)\simeq \mathbb{Z}$
\end{proof}
\begin{midashi}{射影平面$P^2$の基本群}

  $S^2$の元$x$に対して,同値関係$\sim$を$x\sim y \Leftrightarrow y = -x$で定める.この同値関係による同値類を$\left[x\right]$と書く.すると,$P^2 \simeq S^2 \mathbin{/} \sim$と考えることが出来る.ここで$p\colon S^2\to P^2$として$p(x) = \left[x\right]$を考えると,これは被覆写像となる.ここで,$S^2$は単連結であるため,上で定義される$\phi\colon \pi_1(P^2,x_0)\to p^{-1}(x_0)$は全単射.$\left|p^{-1}(x_0)\right| = 2$であるため$\left|\pi_1(P^2,x_0)\right| = 2$.$\pi_1(P^2,x_0)$は群であるので$\pi_1(P^2,x_0) = \pi_1(P^2) \simeq \mathbb{Z}/2\mathbb{Z}$.
\end{midashi}

\expandafter\ifx\csname readornot\endcsname\relax
  \end{document}
\fi
 % 被覆空間
\expandafter\ifx\csname readornot\endcsname\relax
  \documentclass[uplatex]{jsarticle}    
  \usepackage{octopus}
  \usepackage{url}
  %%%% コマンド定義専用のtexファイル
\renewcommand{\postpartname}{章}
\renewcommand{\thepart}{\arabic{part}}
\renewcommand{\thesection}{\thepart.\arabic{section}}
\makeatletter\renewcommand{\theequation}{\thesection.\arabic{equation}}\@addtoreset{equation}{section}\makeatother
\newcommand{\octopuspart}[1]{\newpage\part{#1}\setcounter{section}{0}\vspace{3\baselineskip}}
\renewcommand{\restriction}[2]{\left. #1 \right|_{#2}}
\DeclareMathOperator{\dcup}{\dot{\cup}}
\DeclareMathOperator{\conv}{conv}
\DeclareMathOperator{\Image}{Im}
\DeclareMathOperator{\Kernel}{Ker}
\DeclareMathOperator{\diag}{diag}
\DeclareMathOperator{\rank}{rank}
\DeclareMathOperator{\sgn}{sgn}
\DeclareMathOperator{\rot}{rot}
  \usetikzlibrary{calc}
  \begin{document}
\fi

\section{ホモロジー}

\midashi{\large ホモロジー群の導入}\\
\renewcommand{\arraystretch}{1.0}
\begin{tabular}{ll}
  基本群 & ・「2次元の穴」が検出できる \\
  & ・非可換 \\
  & ・計算は一般に難しい \\
  ホモロジー群 & ・「高次元の穴」が検出できる \\
  & ・可換 \\
  & ・計算が比較的容易(計算機にも相性が良い)
\end{tabular}
\renewcommand{\arraystretch}{1.3}

\sukima \midashi{\large 準備:単体}

\begin{teigi}[アフィン独立]
  $v_{0}, v_{1}, \dots, v_{k} \in \mathbb{R}^{n}$が\nw{アフィン独立} $\defines$
  ${\displaystyle \sum_{i=0}^{k} \lambda_{i} v_{i} = 0}$ かつ ${\displaystyle \sum_{i=0}^{k} \lambda_{i} = 0}$ならば,$\lambda_{0} = \lambda_{1} = \dots = \lambda_{k} = 0$
\end{teigi}

アフィン独立であることは,
\begin{center}
  $v_{1} - v_{0}, v_{2} - v_{0}, \dots, v_{k} - v_{0}$が一次独立
\end{center}
であることと同値である。

\begin{rei}
  $\mathbb{R}^{2}$上で,

  \begin{center}
    \begin{tikzpicture}
      \filldraw (0,0) circle [radius=0.05] node[left] {$v_{0}$};
      \filldraw (1.5,-0.5) circle [radius=0.05] node[left] {$v_{1}$};
      \filldraw (0.7,1.2) circle [radius=0.05] node[left] {$v_{2}$};
      \node at (0.7,-2) {アフィン独立};

      \filldraw (5,-1) circle [radius=0.05] node[left] {$v_{0}$};
      \filldraw (5.9,0.2) circle [radius=0.05] node[left] {$v_{1}$};
      \filldraw (6.8,1.4) circle [radius=0.05] node[left] {$v_{2}$};
      \draw[dashed] (4.7,-1.4) -- (7.1,1.8);
      \node at (5.9,-2) {アフィン独立でない};

      \filldraw (11.5,0.3) circle [radius=0.05] node[above right] {$v_{0}$};
      \filldraw (10.5,-0.5) circle [radius=0.05] node[below] {$v_{1}$};
      \filldraw (12.5,-0.5) circle [radius=0.05] node[below] {$v_{2}$};
      \filldraw (11.5,1.3) circle [radius=0.05] node[above] {$v_{3}$};
      \node at (11.5,-2) {アフィン独立でない};
    \end{tikzpicture}
  \end{center}
\end{rei}

\begin{teigi}[$n$次元単体]
  $\Delta^{n}$:\nw{$n$次元単体(simplex)} $\defines$ 
  アフィン独立な$v_{0},v_{1},\dots,v_{n}$が存在して次を満たすような$\Delta^{n}$のこと:
  \begin{equation}
    \Delta^{n} = \conv \sets{v_{0}, v_{1}, \dots, v_{n}} = \sets{\sum_{i=0}^{n} \lambda_{i} v_{i} | \sum_{i=0}^{n} \lambda_{i} = 1, \quad \lambda_{i} \ge 0}
  \end{equation}
\end{teigi}

$n$次元単体を\nw{$n$--単体}ともいう。
また,定義中に登場した$\conv\sets{v_{0}, \dots, v_{n}}$は\nw{凸包}を表しており,
\nw{凸結合}と呼ばれる${\displaystyle \sum_{i=0}^{n} \lambda_{i} = 1}$かつ$\lambda_{i} \ge 0$を満たす$\lambda_{i}$らによって
${\displaystyle \sum_{i=0}^{n} \lambda_{i} v_{i}}$と表される点全体のなす集合のことをいう。

\begin{rei}
  0--単体から3--単体までを以下に示す。
  \begin{center}
    \begin{tikzpicture}
      % 0-単体
      \filldraw (0,0) circle [radius=0.05];
      % 1-単体
      \filldraw (1.9, -0.3) circle [radius=0.05] -- (2.1,0.3) circle [radius=0.05];
      % 2-単体
      \filldraw[fill=gray!50] (3.7, -0.3) -- (4.3, -0.2) -- (4,0.3) -- (3.7, -0.3) -- cycle;
      \fill (3.7, -0.3) circle [radius=0.05] (4.3,-0.2) circle [radius=0.05] (4,0.3) circle [radius=0.05];
      \filldraw[fill=gray!50, opacity=0.5,draw=black, opacity=1] (5.5,-0.3) -- (6.3,-0.5) -- (6.5,0.2) -- (5.9,0.5) -- (5.5,-0.3) -- cycle;
      % 3-単体
      \draw (6.3,-0.5) -- (5.9,0.5);
      \draw[dashed] (5.5,-0.3) -- (6.5,0.2);
      \fill (5.5,-0.3) circle [radius=0.05] (6.3,-0.5) circle [radius=0.05] (6.5,0.2) circle [radius=0.05] (5.9,0.5) circle [radius=0.05];
      \node at (0,-1) {$\Delta^{0}$};
      \node at (2,-1) {$\Delta^{1}$};
      \node at (4,-1) {$\Delta^{2}$};
      \node at (6,-1) {$\Delta^{3}$};
    \end{tikzpicture}
  \end{center}
\end{rei}

以下,頂点$v_{0}, v_{1}, \dots, v_{n}$による単体を$[v_{0}, v_{1}, \dots, v_{n}]$と書く。

また,\nw{面}$[v_{0}, \dots, \hat{v_{i}}, \dots, v_{n}]$を
\begin{equation}
  [v_{0}, \dots, \hat{v_{i}}, \dots, v_{n}] := [v_{0}, \dots, v_{i-1}, v_{i+1}, \dots, v_{n}]
\end{equation}
で定める。

\begin{center}
  \begin{tikzpicture}
    \filldraw[fill=gray!50] (-0.5, -0.5) node[left]{$v_{0}$} -- (0.5, -0.4) node[right]{$v_{1}$} -- (0,0.5) node[above]{$v_{2}$} -- (-0.5, -0.5) -- cycle;
    \fill (-0.5, -0.5) circle [radius=0.05] (0.5,-0.4) circle [radius=0.05] (0,0.5) circle [radius=0.05];
    \draw[<-] node at (0,-1) {$[v_{0}, v_{1}, v_{2}]$} (0.4,0.1) -- (0.8,0.4) node[right] {面$[v_{1},v_{2}]$};
    \filldraw[fill=gray!50, opacity=0.5,draw=black, opacity=1] (7.3,-0.8) node[below]{$v_{0}$} -- (7.7,0) node[right]{$v_{1}$} -- (6.9,0.5) node[above]{$v_{3}$} -- (7.3,-0.8) -- cycle;
    \draw (6.3,-0.4) node[left]{$v_{2}$} -- (7.3,-0.8) (6.3,-0.4) -- (6.9,0.5);
    \draw[dashed] (6.3,-0.4) -- (7.7,0);
    \fill (6.3,-0.4) circle [radius=0.05] (7.3,-0.8) circle [radius=0.05] (7.7,0) circle [radius=0.05] (6.9,0.5) circle [radius=0.05];
    \draw[<-] (7.4,0) -- (8,0.5) node[right]{面$[v_{0}, v_{1}, v_{3}]$};
  \end{tikzpicture}
\end{center}

\sukima \midashi{\large 向き付けられた単体}

\nw{向き付けられた単体}は,頂点の順番が指定されている単体のこと。
$[v_{0}, v_{1}, \dots, v_{n}]$は頂点に$v_{0} \to v_{1} \to \dots \to v_{n}$の順番が指定されているものとする。

\begin{center}
	\begin{tikzpicture}
		\fill \point{(0,0)}{left}{$v_{0}$} \point{(0,2.5)}{left}{$v_{1}$};
		\midwayarrow{(0,0)}{(0,2.5)};
	\end{tikzpicture}
	\qquad
	\begin{tikzpicture}
		\fill \point{(0,0)}{below left}{$v_{0}$} \point{(2.5,0)}{below right}{$v_{1}$} \point{(1.25,2)}{right}{$v_{2}$};
		\midwayarrow{(0,0)}{(2.5,0)} \midwayarrow{(2.5,0)}{(1.25,2)} \midwayarrow{(0,0)}{(1.25,2)}
		\midwayarrow[red]{(0.4,0.2)}{(2.1,0.2)} \midwayarrow[red]{(2.1,0.2)}{(1.3,1.5)}
	\end{tikzpicture}
	\qquad
	\begin{tikzpicture}
		\coordinate (v0) at (2,-0.5) coordinate (v1) at (2.5,1) coordinate (v2) at (0,0) coordinate (v3) at (1,2);
		\fill \point{(v2)}{left}{$v_{2}$} \point{(v0)}{below}{$v_{0}$} \point{(v1)}{right}{$v_{1}$} \point{(v3)}{right}{$v_{3}$};
		\midwayarrow{(v0)}{(v1)} \midwayarrow{(v2)}{(v0)} \midwayarrow{(v0)}{(v3)} \midwayarrow{(v3)}{(v1)} \midwayarrow[dashed]{(v2)}{(v1)} \midwayarrow{(v2)}{(v3)}
		\midwayarrow[red]{(0.2,-0.2)}{(1.8,-0.6)} \midwayarrow[red]{(1.8,-0.6)}{(0.8,1.9)} \midwayarrow[red]{(0.8,1.9)}{(2.3,0.9)}
	\end{tikzpicture}
\end{center}

このとき,面にも向きが誘導される。すなわち,面$[v_{0}, \dots, \hat{v_{i}}, \dots, v_{n}]$には
頂点に$v_{0} \to v_{1} \to \dots \to v_{i-1} \to v_{i+1} \to \dots \to v_{n}$の順番が指定される。

するとさらに,向き付けられた$n$--単体どうしの間に次のような自然な全単射が定まる。
\begin{equation}
  \mapdef{[v_{0}, \dots, v_{n}]}{[u_{0}, \dots, u_{n}]}{\displaystyle \sum_{i=0}^{n} \lambda_{i} v_{i}}{\displaystyle \sum_{i=0}^{n} \lambda_{i} u_{i}}
\end{equation}

\begin{teigi}[境界,内部]
  単体の\nw{境界} $\partial \Delta^{n}$と\nw{内部}$\mathring{\Delta}^{n}$をそれぞれ次で定める。
  \begin{align}
    & \partial \Delta^{n} := \bigcup_{i=0}^{n} [v_{0}, \dots, \hat{v_{i}}, \dots, v_{n}], \\
    & \mathring{\Delta}^{n} := \Delta^{n} \setminus \partial \Delta^{n}
  \end{align}
\end{teigi}

\sukima \midashi{\large 特異ホモロジー}

$X$:位相空間

\begin{teigi}[特異$n$-単体]
  $\sigma$:\nw{特異$n$-単体(singular n-simplex)} $\defines$ $\sigma$:向き付けられた$n$-単体から$X$への連続写像
\end{teigi}

ものすごい特異的なものも許される。
例えば,$X$を一点集合にとったりKleinの壺にとったりしてもよい。

\begin{teigi}
  特異$n$-単体$\sigma \colon [v_{0}, \dots, v_{n}] \longrightarrow X$の\nw{面}を
  $\restriction{\sigma}{[v_{0}, \dots,\hat{v_{i}}, \dots, v_{n}]}$とする。
  これは特異$n-1$-単体である。
\end{teigi}

\begin{teigi}[特異$n$-チェイン]
  \nw{特異$n$-チェイン} $\defines$ 特異$n$-単体たちの有限な形式的整数結合
\end{teigi}

すなわち,有限個の$\alpha$以外では$n_{\alpha}=0$であるような$\sets{n_{\alpha} \in \mathbb{Z}}_{\alpha}$を用いて
\begin{equation}
  \sum_{\alpha} n_{\alpha} \sigma_{\alpha}
\end{equation}
と表されるものを特異$n$-チェインという。

特異$n$-チェイン全体のなす集合を$C_{n}$とする。
\begin{equation}
  C_{n} := \sets{\sigma | \sigma\text{:特異$n$-チェイン}}
\end{equation}
ここに,
\begin{equation}
    \sum_{\alpha} n_{\alpha} \sigma_{\alpha} + \sum_{\alpha} n'_{\alpha} \sigma_{\alpha} := \sum_{\alpha} (n_{\alpha} + n'_{\alpha}) \sigma_{\alpha}
\end{equation}
で演算を定めると,
単位元を${\displaystyle \sum_{\alpha} 0 \cdot \sigma := 0}$として$C_{n}$は可換群になる。

\begin{teigi}[境界作用素]
  \nw{境界作用素}$\partial_{n} \colon C_{n} \longrightarrow C_{n-1}$を次にようにして定める。
  \begin{itemize}
    \vspace{-0.5\baselineskip}
    \item $n$-単体$\sigma \colon [v_{0},\dots,v_{n}] \longrightarrow X$に対し,
    \begin{equation}
      \partial_{n} \sigma := \sum_{i=0}^{n} (-1)^{i} \restriction{\sigma}{[v_{0}, \dots, \hat{v_{i}}, \dots, v_{n}]}
    \end{equation}
    \item $n$-チェインに対し,
    \begin{equation}
      \partial_{n} \left( \sum_{\alpha} n_{\alpha} \sigma_{\alpha} \right) := \sum_{\alpha} n_{\alpha} \partial_{n} \sigma_{\alpha}
    \end{equation}
  \end{itemize}
\end{teigi}

イメージは以下のよう。
\begin{center}
	\begin{tikzpicture}
		\filldraw[fill=gray!40,draw=black] (-0.5,0) -- (2,0) -- (0.75,2) -- cycle;
		\fill \point{(-0.5,0)}{below left}{$v_{0}$} \point{(2,0)}{below right}{$v_{1}$} \point{(0.75,2)}{right}{$v_{2}$};
		\draw[->] (2.5,1) -- node[midway,above]{$\partial_{2}$} (3.5,1);
		\node at (3.875,1) {$+$};
		\fill \point{(7,0)}{below right}{$v_{1}$} \point{(5.75,2)}{right}{$v_{2}$};
		\draw (7,0) -- (5.75,2);
		\node at (7.625,1) {$-$};
		\fill \point{(8.25,0)}{below left}{$v_{0}$} \point{(9.5,2)}{right}{$v_{2}$};
		\draw (8.25,0) -- (9.5,2);
		\node at (11.375,1) {$+$};
		\fill \point{(14.5,0)}{below right}{$v_{1}$} \point{(13.25,2)}{right}{$v_{2}$};
		\draw (14.5,0) -- (13.25,2);
	\end{tikzpicture}
\end{center}

\begin{prop}
  $\partial_{n}$は$C_{n}$から$C_{n-1}$への準同型写像
\end{prop}

\begin{hodai}
  $\partial_{n-1} \circ \partial_{n} = 0$
\end{hodai}

\begin{proof}
  $\sigma_{\alpha} \colon [v_{0},v_{1},\cdots, v_{n}] \longrightarrow X$に対し,
  \begin{equation}
    \partial_{n} \sigma_{\alpha} = \sum_{i=0}^{n} (-1)^{i} \restriction{\sigma_{\alpha}}{[v_{0},\dots,\hat{v_{i}},\dots,v_{n}]}
  \end{equation}
  であり,
  \begin{align}
    (\partial_{n-1}\circ \partial_{n})\sigma_{\alpha} &=
    \sum_{i=0}^{n} (-1)^{i} \partial_{n-1} \left( \restriction{\sigma_{\alpha}}{[v_{0},\dots,\hat{v_{i}},\dots,v_{n}]} \right) \\
    &= \sum_{i=0}^{n} (-1)^{i}
    \left\{ \sum_{j<i} (-1)^{j} \restriction{\sigma_{\alpha}}{[v_{0},\dots,\hat{v_{j}},\dots,\hat{v_{i}},\dots,v_{n}]} + \sum_{j<i} (-1)^{j-1} \restriction{\sigma_{\alpha}}{[v_{0},\dots,\hat{v_{i}},\dots,\hat{v_{j}},\dots,v_{n}]} \right\} \\
    &= \sum_{0 \le j < i \le n} (-1)^{i+j} \restriction{\sigma_{\alpha}}{[v_{0},\dots,\hat{v_{j}},\dots,\hat{v_{i}},\dots,v_{n}]} - \sum_{0 \le i < j \le n} (-1)^{i+j} \restriction{\sigma_{\alpha}}{[v_{0},\dots,\hat{v_{i}},\dots,\hat{v_{j}},\dots,v_{n}]} \\
    &= 0
  \end{align}
  である。結局,$\partial_{n-1} \circ \partial_{n} = 0$である。
\end{proof}

これにより,
\begin{center}
  $\dots \to C_{n+1} \xrightarrow{\partial_{n+1}} C_{n} \xrightarrow{\partial_{n}} C_{n-1} \xrightarrow{\partial_{n-1}} \dots \xrightarrow{\partial_{3}} C_{2} \xrightarrow{\partial_{2}} C_{1} \xrightarrow{\partial_{1}} C_{0} \xrightarrow{\partial_{0}} 0$,

  $\partial_{n-1} \circ \partial_{n} = 0$($n=1,2,\dots$)
\end{center}
なる列が得られた。これを\nw{チェイン複体}という。

\begin{prop}
  $\Image \partial_{n} \subseteq \Kernel \partial_{n-1}$
\end{prop}

\begin{proof}
  $\partial_{n-1} \circ \partial_{n} = 0$より従う。
\end{proof}

\begin{teigi}[サイクル]
  $\Kernel \partial_{n}$の元を\nw{サイクル}という。
\end{teigi}

\begin{rei}[サイクルのイメージ] \\
  \begin{tikzpicture}
    \fill \point{(180:2)}{left}{$v_{0}$} \point{(120:2)}{above}{$v_{1}$} \point{(60:2)}{above}{$v_{2}$} \point{(0:2)}{right}{$v_{3}$} \point{(-60:2)}{}{} \point{(-120:2)}{}{};
    \midwayarrow{(180:2)}{(120:2)} \midwayarrow{(120:2)}{(60:2)} \midwayarrow{(60:2)}{(0:2)} \midwayarrow{(0:2)}{(-60:2)} \midwayarrow{(-60:2)}{(-120:2)} \midwayarrow{(-120:2)}{(-180:2)}
    \node at (3.5,0) [right]{$= [v_{0},v_{1}] + [v_{1},v_{2}] + \dots + [v_{k},v_{0}]$};
  \end{tikzpicture}

  $\xrightarrow{\partial_{1}} [v_{1}] - [v_{0}] + [v_{2}] - [v_{1}] + \dots + [v_{0}] - [v_{k}] = 0$
  
  であり,これはサイクル。
  %モウヒトツレイガアル
\end{rei}

\begin{teigi}[ホモローグ]
  2つのサイクル$z, z' \in \Kernel \partial_{n}$が\nw{ホモローグ(homologue)} $\defines$ $z - z' \in \Image \partial_{n+1}$

  また,このことを$z \sim z'$と書く。
\end{teigi}

\begin{hodai}
  ホモローグの関係$\sim$は同値関係
\end{hodai}

\begin{proof}
  推移律のみ示す。$z \sim z'$,$z' \sim z''$とする。このとき,$z - z'' = (z - z') + (z' - z'') \in \Image \partial_{n+1}$であるから,$z \sim z''$である。
\end{proof}

$\sim$による同値類を\nw{ホモロジー類}という。

\begin{rei}[ホモローグのイメージ]
  %よくわからない図がある
\end{rei}

\begin{hodai}
  $u \sim v$ $\Longleftrightarrow$ $u$と$v$は同じ弧状連結成分に属する
\end{hodai}

\begin{teigi}[ホモロジー群]
  \begin{align}
    H_{n}(X) &:= \Kernel \partial_{n} / \Image \partial_{n+1} \\
    & = \sets{[z] | z \in \Kernel \partial_{n}}
  \end{align}
  とすると,このホモロジー類の全体のなす集合は次の演算で群をなす。:
  \begin{equation}
    [z] + [z'] := [z + z']
  \end{equation}
  (この演算がwell-definedであることの証明は割愛する)この群を\nw{ホモロジー群}という。
  単位元は$\Image \partial_{n+1}$の元によるホモロジー類。
\end{teigi}

\begin{center}
  \begin{tikzpicture}
    \draw (-4,0) circle [x radius=1, y radius=2];
    \draw (0,0) circle [x radius=1, y radius=2];
    \draw (0,0) circle [x radius=0.5, y radius=1];
    \draw (0,0) circle [x radius=0.25, y radius=0.5];
    \draw (4,0) circle [x radius=1, y radius=2];
    \draw[dashed] (-4,2) --node[midway,above]{$\partial_{n+1}$} (0,0.5);
    \draw[dashed] (-4,-2) -- (0,-0.5);
    \draw[dashed] (0,1) --node[midway,above]{$\partial_{n}$} (4,0);
    \draw[dashed] (0,-1) -- (4,0);
    \filldraw \point{(4,0)}{right}{0};
    \draw[<-] (0,0) to[bend left] (-2,0.3) node[above]{$\Image \partial_{n+1}$};
    \draw[<-] (0.25,-0.5) to[bend left] (2,-1) node[below]{$\Kernel \partial_{n}$};
    \node at (-4,-2.5) {$C_{n+1}$} node at (0,-2.5) {$C_{n}$} node at (4,-2.5) {$C_{n-1}$};
  \end{tikzpicture}
\end{center}

\begin{prop}
  $r$($< \infty$)を$X$の弧状連結成分の数とする。このとき,$H_{0} \simeq \mathbb{Z}^{r}$である。
\end{prop}

\begin{proof}
  各弧状連結成分から1つずつ頂点$v_{1},v_{2},\dots,v_{r}$をとると,
  $H_{0}$の元は${\displaystyle \sum_{i=1}^{n} n_{i} [v_{i}]}$と一意的に記述される。
  このことより従う。
\end{proof}

\midashi{注意.} $C_{1} \xrightarrow{\partial_{1}} C_{0} \xrightarrow{\partial_{0}} 0$であるから$\Kernel \partial_{0} = C_{0}$である。

\begin{prop}
  $X$:頂点数$n$,枝数$m$の連結なグラフとする。このとき,$H_{1} \simeq \mathbb{Z}^{m-n}$である。
\end{prop}

\sukima \midashi{\large 位相不変性・ホモトピー不変性}

$X,Y$:位相空間,$\varphi \colon X \longrightarrow Y$:連続写像,
$\sigma \colon [v_{0}, v_{1}, \dots, v_{n}] \longrightarrow X$:$X$上の$n$-単体,とする。
このとき,$\varphi \circ \sigma \colon [v_{0}, v_{1}, \dots, v_{n}] \longrightarrow Y$は$Y$上の$n$-単体になる。
これにより,$\varphi \colon C_{n} (X) \longrightarrow C_{n}(Y)$が誘導される。:
\begin{equation}
  \varphi \left( \sum_{\alpha} n_{\alpha} \sigma_{\alpha} \right) := \sum_{\alpha} n_{\alpha}\varphi (\sigma_{\alpha})
\end{equation}

よって,以下のような図式がかける。

\begin{center}
  \begin{tikzpicture}
    \draw[->] (0,0.5) --node[midway,left]{$\varphi$} (0,-0.5);
    \draw[->] (3,0.5) --node[midway,left]{$\varphi$} (3,-0.5);
    \draw[->] (6,0.5) --node[midway,left]{$\varphi$} (6,-0.5);

    \draw[->] (-2,1) --node[midway,above]{$\partial_{n+2}$} (-1,1);
    \draw[->] (1,1) --node[midway,above]{$\partial_{n+1}$} (2,1);
    \draw[->] (4,1) --node[midway,above]{$\partial_{n}$} (5,1);
    \draw[->] (7,1) --node[midway,above]{$\partial_{n-1}$} (8,1);

    \draw[->] (-2,-1) --node[midway,below]{$\partial_{n+2}$} (-1,-1);
    \draw[->] (1,-1) --node[midway,below]{$\partial_{n+1}$} (2,-1);
    \draw[->] (4,-1) --node[midway,below]{$\partial_{n}$} (5,-1);
    \draw[->] (7,-1) --node[midway,below]{$\partial_{n-1}$} (8,-1);

    \node at (-3,-1) {$\cdots$} node at (0,-1) {$C_{n+1}(Y)$} node at (3,-1) {$C_{n}(Y)$} node at (6,-1) {$C_{n-1}(Y)$} node at (9,-1) {$\cdots$};
    \node at (-3,1) {$\cdots$} node at (0,1) {$C_{n+1}(X)$} node at (3,1) {$C_{n}(X)$} node at (6,1) {$C_{n-1}(X)$} node at (9,1) {$\cdots$};
  \end{tikzpicture}
\end{center}

\begin{hodai}
  \label{homology1.commutative}
  $\varphi \circ \partial_{n} = \partial_{n} \circ \varphi$
\end{hodai}

\begin{proof}
  次の式から従う。
  \begin{equation}
    (\varphi \circ \partial_{n}) \sigma = \varphi \left( \sum_{i=0}^{n} (-1)^{i} \restriction{\sigma}{[v_{0}, \dots, \hat{v_{i}}, \dots, v_{n}]} \right) = \sum_{i=0}^{n} (-1)^{i} \varphi \circ \restriction{\sigma}{[v_{0}, \dots, \hat{v_{i}}, \dots, v_{n}]} = \partial_{n} (\varphi \circ \sigma)
  \end{equation}
\end{proof}

このことから,先の図式が可換になることが従う。

\begin{hodai}
  \label{homology1.cycle}
  $z$:$X$上$n$-サイクル $\Longrightarrow$ $\varphi(z)$:$Y$上$n$-サイクル
\end{hodai}

\begin{proof}
  先の\rref{補題}{homology1.commutative}から$\partial_{n} \varphi(z) = \varphi (\partial_{n} (z)) = 0$であって従う。
\end{proof}

\begin{teigi}
  $\varphi \colon X \longrightarrow Y$:連続とする。写像$\varphi_{\#}$を
  \begin{equation}
    \varphi_{\#} \colon \mapdef{H_{n}(X)}{H_{n}(Y)}{\left[ z \right]}{\left[ \varphi(z) \right]}
  \end{equation}
  で定義する。
\end{teigi}

\begin{hodai}
  $\varphi_{\#}$はwell-definedであり,準同型である。
\end{hodai}

\begin{proof}
  \midashi{[well-definedness]}

  先の\rref{補題}{homology1.cycle}より$\varphi(z)$は$Y$上$n$-サイクルである。よって,$[\varphi(z)] \in H_{n}(Y)$である。いま,
  \renewcommand{\arraystretch}{1}
  \begin{tabular}{l@{\,}l}
    $z \sim z'$ & $\Longleftrightarrow$ $\exists c, \quad z - z' = \partial_{n+1} c$ \\
    & $\longrightarrow$ $\varphi(z) - \varphi(z') = \varphi (\partial_{n+1} c) = \partial_{n+1} (\varphi(c)) \in \Image \partial_{n+1}$ \\
    & $\longrightarrow$ $[\varphi(z)] \sim [\varphi(z')]$
  \end{tabular}
  \renewcommand{\arraystretch}{1.3}
  のようになる。

  \sukima \midashi{[準同型性]}

  これは定義の仕方から明らか。
\end{proof}

\begin{hodai}
  {\bf (1)} $X \xrightarrow{\varphi} Y \xrightarrow{\varphi'} Z$のとき,
  $\varphi_{\#}' \circ \varphi_{\#} = (\varphi' \circ \varphi)_{\#}$

  {\bf (2)} $\mathrm{id}_{\#} = \mathrm{id}$
\end{hodai}

\begin{proof}
  {\bf (1)} $\left( \varphi' \circ \varphi \right)_{\#}([z]) = [(\varphi' \circ \varphi)(z)] = \varphi'_{\#} \left( [\varphi (z)] \right) = (\varphi'_{\#} \circ \varphi_{\#}) ([z])$より従う。

  {\bf (2)} は明らか。
\end{proof}

\begin{teiri}
  $X,Y$:同相 $\Longrightarrow$ $H_{n}(X) \simeq H_{n}(Y)$
\end{teiri}

\begin{proof}
  $\varphi \colon X \longrightarrow Y$:同相写像とする。
  このとき,$\varphi^{-1}_{\#} \circ \varphi_{\#} = \mathrm{id}$,
  $\varphi_{\#} \circ \varphi^{-1}_{\#} = \mathrm{id}$より従う。
\end{proof}

実はさらに強く,次のことが言える。

\begin{teiri}
  $X,Y$:ホモトピー同値 $\Longrightarrow$ $H_{n}(X) \simeq H_{n}(Y)$
\end{teiri}

\begin{proof}[{\bf 証明の気分}]
  $f \colon X \longrightarrow Y$,$g \colon Y \longrightarrow X$をホモトピー同値写像とする。
  つまり,$g \circ f \simeq \mathrm{id}_{X}$,$f \circ g \simeq \mathrm{id}_{Y}$とする。
  このときに,$(g \circ f)_{\#} \simeq \mathrm{id}_{H_{n}(X)}$を示せばよい。
  
  ここで,$f_{\#}$と$g_{\#}$は全単射であるから,$g_{\#} \circ f_{\#}$は全単射であり,$f_{\#} \circ g_{\#}$は全単射であることになる。

  $g \circ f$と$\mathrm{id}_{X}$の間にはホモトピーがある。サイクル$z$はそのホモトピーによって連続的にサイクル$g \circ f (z)$になる。

  サイクルの軌跡が作る「筒」はうまく単体らによって表すことができる。
  したがって,$z \sim (g \circ f)(z)$である。
  よって,$(g \circ f)_{\#} ([z]) = [z]$である。

  一般の場合は,サイクル${\displaystyle z = \sum_{\alpha} n_{\alpha} \sigma_{\alpha}}$の各単体$\sigma_{\alpha}$に対して,プリズムを一次元上の単体に分割する。
\end{proof}

\midashi{演習.} これをきちんと証明せよ。

\expandafter\ifx\csname readornot\endcsname\relax
  \end{document}
\fi % ホモロジー
\expandafter\ifx\csname readornot\endcsname\relax
  \documentclass[uplatex]{jsarticle}
  \usepackage{octopus}
  \usepackage{url}
  \usetikzlibrary{calc}

  \renewcommand{\proofname}{\textsf{証明}}
  \renewcommand{\postpartname}{章}
  \renewcommand{\thesection}{\thepart.\arabic{section}}
  \renewcommand{\thepart}{\arabic{part}}
  \makeatletter\renewcommand{\theequation}{\thesection.\arabic{equation}}\@addtoreset{equation}{section}\makeatother

  \newcommand{\octopuspart}[1]{\newpage\part{#1}\setcounter{section}{0}\vspace{3\baselineskip}}

  \renewcommand{\restriction}[2]{\left. #1 \right|_{#2}}
  \DeclareMathOperator{\dcup}{\dot{\cup}}
  \DeclareMathOperator{\conv}{conv}
  \DeclareMathOperator{\Image}{Im}
  \DeclareMathOperator{\Kernel}{Ker}
  \begin{document}
\fi

\section{ホモロジーの計算 1}

\midashi{\large 複体構造}

特異チェインの集合
${\displaystyle C_{n} = \sets{ \sum_{\alpha} \mu_{\alpha} \sigma_{\alpha} | \sigma_{\alpha} \colon \Delta^{n} \longrightarrow X, \quad \mu_{\alpha} \in \mathbb{Z}}}$
は大きすぎて直接ホモロジー群$H_{n} = \Kernel \partial_{n} / \Image \partial_{n+1}$を計算するのは難しい。

そこで計算に便利な概念を導入する。
$X$:位相空間,とする。

\begin{teigi}[$\Delta$-複体構造]
	\label{homcalc.complex}
	$X$の\nw{$\Delta$-複体構造($\Delta$-complex structure)} $\defines$ 以下の(i)から(iv)を満たす(特異)$n$-単体の族$\sets{\sigma_{\alpha} \colon \Delta^{n} \longrightarrow X}_{\alpha \in \Lambda}$:
	\begin{enumerate}
		\vspace{-0.5\baselineskip}
		\item[(i)] $\forall \alpha \in \Lambda, \quad \restriction{\sigma_{\alpha}}{\mathring{\Delta}^{n}}$は単射
		\item[(ii)] $\forall x \in X, \quad \exists! \alpha \in \Lambda, \quad x \in \sigma_{\alpha} (\mathring{\Delta}^{n})$
		\item[(iii)] $\forall \alpha \in \Lambda, \quad \forall f : \Delta^{n} = [v_{0}v_{1}\cdots v_{n}]$の面$, \quad \exists \beta \in \Lambda, \quad \restriction{\sigma_{\alpha}}{f} = \sigma_{\beta}$
		\item[(iv)] $A \subseteq X$:開 $\Longleftrightarrow$ $\forall \alpha \in \Lambda, \quad \sigma_{\alpha}^{-1} (A)$:$\Delta^{n}$上開
	\end{enumerate}
\end{teigi}

上の\rref{定義}{homcalc.complex}について補足する。
(i)の条件は「$X$が単体の内部の像の非交和(disjoint union)で表される」ことを意味する。
(ii)の条件は,もう少し正確に述べれば,
「標準全単射$\varphi \colon f \longrightarrow \Delta^{n-1}$によって$\restriction{\sigma_{\alpha}}{f} = \sigma_{\beta} \circ \varphi$となる」ことを要請している。
ただし,ここでの$f = [v_{0} \cdots \hat{v_{i}} \cdots v_{n}]$や$\Delta^{n-1}$には向きが付けられているものとする。
これは「複数の単体が面でうまく貼り合わされている」ことを意味している。
また,(iii)の条件は「位相的にも単体を貼り合わせたものである」ことを要請している。
このことは$X$がセル複体${\displaystyle \coprod_{\alpha} \Delta_{\alpha}^{n_{\alpha}} / \sim}$とみなせることに他ならない。

\begin{rei}[トーラスの場合]
	$X = T^{2}$は以下のような$\Delta$-複体構造として捉えられる。
	\begin{center}
		\begin{tikzpicture}
			% 本図
			%% 面を塗る
			\fill[red!20] (0,0) -- (2,2) -- (0,2) -- cycle;
			\fill[blue!20] (0,0) -- (2,2) -- (2,0) -- cycle;
			%% 点の名前もろもろ
			\fill \point{(0,0)}{below left}{$v$} \point{(2,0)}{below right}{$v$} \point{(0,2)}{above left}{$v$} \point{(2,2)}{above right}{$v$};
			\node at (1,0) [below]{$b$} node at (0,1) [left]{$a$} node at (1,2) [above]{$b$} node at (2,1) [right]{$a$} node at (1,1) [left]{$c$};
			\node at (0.5,1.5){$U$} node at (1.5,0.5){$L$};
			%% 矢印つき
			\midwayarrow{(0,0)}{(0,2)} \midwayarrow{(0,0)}{(2,0)} \midwayarrow{(0,2)}{(2,2)} \midwayarrow{(2,0)}{(2,2)} \midwayarrow{(0,0)}{(2,2)}
			% 取り巻き作り
			%% 点
			\fill \point{(-1.5,2.2)}{}{};
			\draw[->] (-1.4,2.1) to[bend right] node[midway,above]{$\sigma_{v}$} (-0.1,1.9);
			%% 辺
			\fill \point{(3.5,1)}{}{} \point{(3.5,2)}{}{} \point{(1.25,-1.5)}{}{} \point{(2.25,-1.5)}{}{} \point{(-0.25,-1.5)}{}{} \point{(0.75,-1.5)}{}{};
			\midwayarrow{(3.5,1)}{(3.5,2)} \midwayarrow{(1.25,-1.5)}{(2.25,-1.5)} \midwayarrow{(-0.25,-1.5)}{(0.75,-1.5)}
			\draw[->] (3.4,1.5) to[bend left] node[midway,above]{$\sigma_{a}$} (2.1,1.5);
			\draw[->] (1.75,-1.4) to[bend right] node[midway,right]{$\sigma_{b}$} (1.5,-0.1);
			\draw[->] (0.25,-1.4) to[bend left] node[midway,left]{$\sigma_{c}$} (0.5,-0.1);
			%% 面
			\fill[red!20] (-3,-2.5) -- (-1,-0.5) -- (-3,-0.5) -- cycle;
			\fill[blue!20] (2.5,-2) -- (4.5,0) -- (4.5,-2) -- cycle;
			\fill \point{(-3,-2.5)}{}{} \point{(-1,-0.5)}{}{} \point{(-3,-0.5)}{}{} \point{(2.5,-2)}{}{} \point{(4.5,0)}{}{} \point{(4.5,-2)}{}{};
			\midwayarrow{(-3,-2.5)}{(-1,-0.5)} \midwayarrow{(-3,-2.5)}{(-3,-0.5)} \midwayarrow{(-3,-0.5)}{(-1,-0.5)} \midwayarrow{(2.5,-2)}{(4.5,0)} \midwayarrow{(2.5,-2)}{(4.5,-2)} \midwayarrow{(4.5,-2)}{(4.5,0)}
			\draw[->] (-1.5,-0.4) -- node[pos=.3, above]{$\sigma_{U}$} (0.4,0.9);
			\draw[->] (4,-0.25) -- node[pos=.3, above]{$\sigma_{L}$} (1.8,0.5);
		\end{tikzpicture}
	\end{center}	
\end{rei}
	
\begin{rei}[射影平面$P^{2}$]
	$X = P^{2}$は以下のような$\Delta$-複体構造として捉えられる。
	\begin{center}
		\begin{tikzpicture}
			\fill[red!20] (0,0) -- (2,2) -- (0,2) -- cycle;
			\fill[blue!20] (0,0) -- (2,2) -- (2,0) -- cycle;
			\fill \point{(0,0)}{below left}{$v$} \point{(2,0)}{below right}{$w$} \point{(0,2)}{above left}{$w$} \point{(2,2)}{above right}{$v$};
			\node at (1,0) [below]{$b$} node at (0,1) [left]{$a$} node at (1,2) [above]{$b$} node at (2,1) [right]{$a$} node at (1,1) [left]{$c$};
			\node at (0.5,1.5){$U$} node at (1.5,0.5){$L$};
			\midwayarrow{(0,0)}{(2,0)} \midwayarrow{(0,0)}{(0,2)} \midwayarrow{(0,0)}{(2,2)} \midwayarrow{(2,2)}{(2,0)} \midwayarrow{(2,2)}{(0,2)}
		\end{tikzpicture}
	\end{center}	
\end{rei}

\sukima \midashi{注意.} いい加減に$\Delta$-複体構造をつくろうとすると失敗しやすいので注意が必要。

\begin{teigi}[単体的複体]
	次の条件を満たすものを\nw{単体的複体(simplicial complex)}という。
	\begin{itemize}
		\vspace{-0.5\baselineskip}
		\item 各$n$-単体(の像)が$n+1$個の異なる頂点からなる。
		\item 異なる$n$-単体(の像)が同じ頂点集合を持つことはない。
	\end{itemize}
\end{teigi}

\sukima \midashi{注意.}
\begin{itemize}
	\vspace{-0.5\baselineskip}
	\item $\Delta$-複体は細分することで単体的複体になる。
	\item 単体的複体では,頂点に番号をつけることで,その順番によって向きを定めることができる。
	\item 完全に組み合わせ的に扱える - - - - $>$ 抽象的単体的複体:(単体をなす頂点集合の族)
	
	頂点集合を$V$として$\mathcal{F} \subseteq 2^{V}$を考え,これが$X \subseteq Y \in \mathcal{F}$ $\Longrightarrow$ $X \in \mathcal{F}$を満たすものとして定める。
	
	例. \begin{tikzpicture}[baseline=-2.5]
			\filldraw[fill=black!30] (0,0) -- (1.4,-1) -- (1.4,1) -- cycle;
			\draw (1.4,1) -- (2.5,0) -- (3.5,0);
			\draw (1.4,-1) -- (2.5,0);
			\fill \point{(0,0)}{left}{0} \point{(1.4,-1)}{below left}{1} \point{(1.4,1)}{above left}{2} \point{(2.5,0)}{below right}{3} \point{(3.5,0)}{below}{4};
		\end{tikzpicture}
	\renewcommand{\arraystretch}{1}
	$\longrightarrow$ $\mathcal{F} = \sets{\begin{array}{c} \sets{0,1,2}, \\ \sets{0,1}, \sets{1,2}, \sets{0,2}, \sets{1,3}, \sets{2,3}, \sets{3,4},\\  \sets{0}, \sets{1}, \sets{2}, \sets{3}, \sets{4},\\  \emptyset \end{array}} \subseteq 2^{\sets{0,1,2,3,4}}$
	\renewcommand{\arraystretch}{1.3}
\end{itemize}

\sukima \midashi{\large 単体的ホモロジー}

$X$:位相空間,$\Delta$-複体構造$\sets{\sigma_{\alpha} \colon \Delta^{n_{\alpha}} \longrightarrow X}_{\alpha \in \Lambda}$,

$\Delta^{n}$:複体内の$n$-単体からなるチェイン$\sum_{\alpha \in \Lambda} \mu_{\alpha} \sigma_{\alpha}$の集合($\subseteq C_{n}$:すべての$n$-単体からなるチェインの集合)

\begin{rei}[射影平面]
	$X = P^{2}$では$\Delta_{n}$では以下のようになる。
	\begin{center}
		\begin{tikzpicture}
			\fill[red!20] (0,0) -- (2,2) -- (0,2) -- cycle;
			\fill[blue!20] (0,0) -- (2,2) -- (2,0) -- cycle;
			\fill \point{(0,0)}{below left}{$v$} \point{(2,0)}{below right}{$w$} \point{(0,2)}{above left}{$w$} \point{(2,2)}{above right}{$v$};
			\node at (1,0) [below]{$b$} node at (0,1) [left]{$a$} node at (1,2) [above]{$b$} node at (2,1) [right]{$a$} node at (1,1) [left]{$c$};
			\node at (0.5,1.5){$U$} node at (1.5,0.5){$L$};
			\midwayarrow{(0,0)}{(2,0)} \midwayarrow{(0,0)}{(0,2)} \midwayarrow{(0,0)}{(2,2)} \midwayarrow{(2,2)}{(2,0)} \midwayarrow{(2,2)}{(0,2)}
		\end{tikzpicture}
	\end{center}
	\begin{itemize}
		\vspace{-0.5\baselineskip}
		\item $\Delta_{0} = \sets{n_{v} \tikz[baseline=0] \fill \point{(0,0.1)}{below}{\footnotesize $v$};
		+ n_{w} \tikz[baseline=0] \fill \point{(0,0.1)}{below}{\footnotesize $w$}; | n_{v}, n_{w} \in \mathbb{Z}}$
		\item $\Delta_{1} = \sets{n_{a} \,
		\begin{tikzpicture}[baseline=-2.5]
			\fill \point{(0,0.3)}{}{} \point{(0,-0.3)}{}{};
			\midwayarrow{(0,0.3)}{(0,-0.3)};
			\node at (0,0) [right]{\footnotesize $a$};
		\end{tikzpicture}
		+ n_{b} \,
		\begin{tikzpicture}[baseline=-2.5]
			\fill \point{(0,0)}{}{} \point{(0.6,0)}{}{};
			\midwayarrow{(0,0)}{(0.6,0)};
			\node at (0.3,0) [below]{\footnotesize $b$};
		\end{tikzpicture} | n_{a}, n_{b} \in \mathbb{Z}}$
		\item $\Delta_{2} = \sets{n_{U} \,
		\begin{tikzpicture}[baseline=-2.5]
			\fill[red!20] (0,-0.3) -- (0.6,0.3) -- (0,0.3) -- cycle;
			\fill \point{(0,-0.3)}{}{} \point{(0.6,0.3)}{}{} \point{(0,0.3)}{}{};
			\midwayarrow{(0,-0.3)}{(0.6,0.3)}; \midwayarrow{(0.6,0.3)}{(0,0.3)}; \midwayarrow{(0,-0.3)}{(0,0.3)};
			\node at (0.18,0.13){\footnotesize $U$};
		\end{tikzpicture}
		+ n_{L} \,
		\begin{tikzpicture}[baseline=-2.5]
			\fill[blue!20] (0,-0.3) -- (0.6,-0.3) -- (0.6,0.3) -- cycle;
			\fill \point{(0,-0.3)}{}{} \point{(0.6,-0.3)}{}{} \point{(0.6,0.3)}{}{};
			\midwayarrow{(0,-0.3)}{(0.6,-0.3)}; \midwayarrow{(0.6,0.3)}{(0.6,-0.3)}; \midwayarrow{(0,-0.3)}{(0.6,0.3)};
			\node at (0.42,-0.13){\footnotesize $L$};
		\end{tikzpicture} | n_{U}, n_{L} \in \mathbb{Z}}$
		\item $\Delta_{k} = \sets{0}$($k \ge 3$)\footnote{編註:ノートでは「$k \ge 0$」とあるが誤植を疑い「$k \ge 3$」とした}
	\end{itemize}
	2-チェインの例:$3\, \begin{tikzpicture}[baseline=-2.5]
		\fill[red!20] (0,-0.3) -- (0.6,0.3) -- (0,0.3) -- cycle;
		\fill \point{(0,-0.3)}{}{} \point{(0.6,0.3)}{}{} \point{(0,0.3)}{}{};
		\midwayarrow{(0,-0.3)}{(0.6,0.3)}; \midwayarrow{(0.6,0.3)}{(0,0.3)}; \midwayarrow{(0,-0.3)}{(0,0.3)};
		\node at (0.18,0.13){\footnotesize $U$};
	\end{tikzpicture}
	-2 \,
	\begin{tikzpicture}[baseline=-2.5]
		\fill[blue!20] (0,-0.3) -- (0.6,-0.3) -- (0.6,0.3) -- cycle;
		\fill \point{(0,-0.3)}{}{} \point{(0.6,-0.3)}{}{} \point{(0.6,0.3)}{}{};
		\midwayarrow{(0,-0.3)}{(0.6,-0.3)}; \midwayarrow{(0.6,0.3)}{(0.6,-0.3)}; \midwayarrow{(0,-0.3)}{(0.6,0.3)};
		\node at (0.42,-0.13){\footnotesize $L$};
	\end{tikzpicture}$
\end{rei}

\begin{prop}
	$\Delta_{n}$は$C_{n}$の部分群
\end{prop}

\begin{teigi}[境界作用素]
	$C_{n}$に対して定義した境界作用素$\partial_{n} \colon C_{n} \longrightarrow C_{n-1}$に対して,
	$\Delta_{n}$に対する境界作用素を$\partial_{n}^{\Delta} := \restriction{\partial_{n}}{\Delta_{n}}$で定める。
\end{teigi}

なお,$\Delta$内の単体の面も$\Delta$の単体として定義してあるから,$\partial_{n}^{\Delta}$は$\Delta_{n}$から$\Delta_{n-1}$への写像となる。

\begin{hodai}
	$\Delta_{n} \xrightarrow{\partial_{n}^{\Delta}} \Delta_{n-1} \xrightarrow{\partial_{n-1}^{\Delta}} \Delta_{n-2}$において,$\partial_{n-1}^{\Delta} \circ \partial_{n}^{\Delta} = 0$
\end{hodai}

\begin{teigi}[($\Delta$に関する)ホモロジー群]
	$H_{n}^{\Delta} (X) := \Kernel \partial_{n}^{\Delta} / \Image \partial_{n+1}^{\Delta}$
	を($\Delta$に関する)ホモロジー群という。
\end{teigi}

\begin{center}
	\begin{tikzpicture}
	  \draw[->] (0,0.5) --node[midway,left]{$i:包含写像$} (0,-0.5);
	  \draw[->] (3,0.5) --node[midway,left]{$i$} (3,-0.5);
	  \draw[->] (6,0.5) --node[midway,left]{$i$} (6,-0.5);
  
	  \draw[->] (-2,1) --node[midway,above]{$\partial_{n+2}^{\Delta}$} (-1,1);
	  \draw[->] (1,1) --node[midway,above]{$\partial_{n+1}^{\Delta}$} (2,1);
	  \draw[->] (4,1) --node[midway,above]{$\partial_{n}^{\Delta}$} (5,1);
	  \draw[->] (7,1) --node[midway,above]{$\partial_{n-1}^{\Delta}$} (8,1);
  
	  \draw[->] (-2,-1) --node[midway,below]{$\partial_{n+2}$} (-1,-1);
	  \draw[->] (1,-1) --node[midway,below]{$\partial_{n+1}$} (2,-1);
	  \draw[->] (4,-1) --node[midway,below]{$\partial_{n}$} (5,-1);
	  \draw[->] (7,-1) --node[midway,below]{$\partial_{n-1}$} (8,-1);
  
	  \node at (-3,1) {$\cdots$} node at (0,1) {$\Delta_{n+1}$} node at (3,1) {$\Delta_{n}$} node at (6,1) {$\Delta_{n-1}$} node at (9,1) {$\cdots$};
	  \node at (-3,-1) {$\cdots$} node at (0,-1) {$C_{n+1}$} node at (3,-1) {$C_{n}$} node at (6,-1) {$C_{n-1}$} node at (9,-1) {$\cdots$};
	\end{tikzpicture}
\end{center}

上のような図式を考えると,
明らかに$i \circ \partial_{n}^{\Delta} = \partial_{n} \circ i$であるので,
ホモロジー群の間の写像$i_{\#} \colon H_{n}^{\Delta} \longrightarrow H_{n}$が誘導される。

\begin{teiri}
	$i_{\#}$は同型写像であり,$n = 0,1,2, \dots$で$H_{n}^{\Delta} \simeq H_{n}$である
\end{teiri}

\begin{proof}
	{\bf 難しい}
\end{proof}

特に,単体ホモロジーは,複体のとり方に依存しない。

\begin{corr}
	$n$が複体の次元より大きいとき,$H_{n}(X) = 0$
\end{corr}

\begin{proof}
	$\Kernel \partial_{n}^{\Delta} \subseteq \Delta_{n} = 0$
\end{proof}

\sukima \midashi{\large ホモロジー群の計算}

\begin{rei}[円周のホモロジー群]
	$X = S^{1}$:円周,とする。
	\begin{center}
		\begin{tikzpicture}
			\draw (0,0) circle [radius=1];
			\draw[<-] (0.1,-1.1) to[bend left] (0.7,-1.7);
			\fill \point{(0,-1)}{below}{$r$} \point{(0.8,-1.8)}{below}{$r$} \point{(2,-0.8)}{}{} \point{(2,0.8)}{}{};
			\node at (2,0) [right]{$e$} node at (0,1) [above right]{$e$} node at (0,1){\rotatebox{90}{\footnotesize $\blacktriangle$}};
			\midwayarrow{(2,-0.8)}{(2,0.8)};
			\draw[<-] (1.25,0) -- (1.75,0);
		\end{tikzpicture}
	\end{center}
	よって,
	\begin{equation}
		H_{n} (X) = \begin{cases}
			\mathbb{Z}, & n = 0, 1 \\
			0, & \text{otherwise}
		\end{cases}
	\end{equation}
\end{rei}

\begin{rei}[円板のホモロジー群]
	$X = D^{2}$:円板,とする。
	\begin{center}
		\begin{tikzpicture}
			\filldraw[fill=black!20] (0,0) circle [radius=1];
			\fill \point{(90:1)}{above}{$u$} \point{(-30:1)}{below right}{$v$} \point{(210:1)}{below left}{$w$};
			\node at (30:1){\rotatebox{210}{\footnotesize $\blacktriangle$}} node at (270:1){\rotatebox{90}{\footnotesize $\blacktriangle$}} node at (150:1){\rotatebox{150}{\footnotesize $\blacktriangle$}};
			\draw[<-] (1.5,0) -- (2.5,0);
			\fill[black!20] ($(4,-0.25) + (90:1)$) -- ($(4,-0.25) + (-30:1)$) -- ($(4,-0.25) + (210:1)$) -- cycle;
			\fill \point{($(4,-0.25) + (90:1)$)}{above}{$u$} \point{($(4,-0.25) + (-30:1)$)}{below right}{$v$} \point{($(4,-0.25) + (210:1)$)}{below left}{$w$};
			\midwayarrow{($(4,-0.25) + (90:1)$)}{($(4,-0.25) + (-30:1)$)} \midwayarrow{($(4,-0.25) + (-30:1)$)}{($(4,-0.25) + (210:1)$)} \midwayarrow{($(4,-0.25) + (90:1)$)}{($(4,-0.25) + (210:1)$)}
		\end{tikzpicture}
	\end{center}
	いま,閉路チェイン$[uv] + [vw] - [uw]$は2-単体$[uvw]$の境界であるから1次ホモロジー群は自明であり,
	2次ホモロジー群は
	\begin{equation}
		H_{2} = \Kernel \partial_{2} / \Image \partial_{3} = \Kernel \partial_{2} = \sets{\alpha [uvw] | \alpha \partial[uvw] = 0} = 0
	\end{equation}
	である。まとめると,
	\begin{equation}
		H_{n}(X) = \begin{cases}
			\mathbb{Z}, & n = 0 \\
			0, & \text{otherwise}
		\end{cases}
	\end{equation}
\end{rei}

\begin{rei}[球面のホモロジー群]
	$X = S^{2}$:球面,とする。
	\begin{center}
		\begin{tikzpicture}
		%% 面を塗る
		\fill[red!20] (0,0) -- (2,2) -- (0,2) -- cycle;
		\fill[blue!20] (0,0) -- (2,2) -- (2,0) -- cycle;
		%% 点の名前もろもろ
		\fill \point{(0,0)}{below left}{$u$} \point{(2,0)}{below right}{$v$} \point{(0,2)}{above left}{$v$} \point{(2,2)}{above right}{$w$};
		\node at (1,0) [below]{$b$} node at (0,1) [left]{$b$} node at (1,2) [above]{$a$} node at (2,1) [right]{$a$} node at (1,1) [left]{$c$};
		\node at (0.5,1.5){$U$} node at (1.5,0.5){$L$};
		%% 矢印つき
		\midwayarrow{(0,0)}{(0,2)} \midwayarrow{(0,0)}{(2,0)} \midwayarrow{(0,2)}{(2,2)} \midwayarrow{(2,0)}{(2,2)} \midwayarrow{(0,0)}{(2,2)}
		\end{tikzpicture}
	\end{center}
	\begin{itemize}
		\vspace{-0.5\baselineskip}
		\item $\Delta_{0} = \mathbb{Z} u + \mathbb{Z} v + \mathbb{Z} w$
		\item $\Delta_{1} = \mathbb{Z} a + \mathbb{Z} b + \mathbb{Z} c$
		\item $\Delta_{2} = \mathbb{Z} U + \mathbb{Z} L$
	\end{itemize}

	0次ホモロジー群は連結成分の個数から求めれば早く,$H_{0}(X) = \mathbb{Z}$。

	1次ホモロジー群$H_{1}(X) = \Kernel \partial_{1} / \Image \partial_{2}$を求める。
	\renewcommand{\arraystretch}{1}
	\begin{center}
		$\begin{array}{lcl}
			\alpha a + \beta b + \gamma c \in \Kernel \partial_{1} & \Longleftrightarrow & \alpha (w-v) + \beta (v-u) + \gamma (w-u) = 0 \\
			& \Longleftrightarrow & \alpha = \beta = - \gamma 
		\end{array}$
	\end{center}
	\renewcommand{\arraystretch}{1.3}
	より,$\Kernel \partial_{1} = \mathbb{Z} (a+b-c)$。
	また,$\partial U = \partial L = a+b-c$より,$\Image \partial_{2} = \mathbb{Z} (a+b-c)$。
	よって,$H_{1}(X) = 0$(自明)\footnote{この自明は計算が自明なのではなく,この群が自明,つまり単位元しか元を持たない群であることを指していて,決してハラスメントではないことに注意されたい。}\ である。

	2次ホモロジー群$H_{2}(X) = \Kernel \partial_{2} / \Image \partial_{3}$を求める。
	\renewcommand{\arraystretch}{1}
	\begin{center}
		$\begin{array}{lcl}
			\alpha U + \beta L \in \Kernel \partial_{2} & \Longleftrightarrow & \alpha (a-c+b) + \beta (a-c+b) = 0 \\
			& \Longleftrightarrow & \alpha = - \beta
		\end{array}$
	\end{center}
	\renewcommand{\arraystretch}{1.3}
	より,$\Kernel \partial_{2} = \mathbb{Z} (U-L)$。
	また,そんなものはないので$\Delta_{3} = 0$であり,$\Image \partial_{3} = 0$である。
	よって,$H_{2}(X) = \mathbb{Z}$である。

	まとめると,
	\begin{equation}
		H_{n}(X) = \begin{cases}
			\mathbb{Z}, & n=0, 2 \\
			0, & \text{otherwise}
		\end{cases}
	\end{equation}
\end{rei}

\begin{rei}[超球面のホモロジー群]
	一般に$X = S^{n}$とすると,複体として2つの$n$-単体$U,L$の貼り合わせがとれて,
	$H_{n} = \mathbb{Z} (U - L) \simeq \mathbb{Z}$となる。
	さらに,他の次元のホモロジー群も計算すると,\footnote{編註:Mayer-Vietoris完全系列(cf. \url{<https://ja.wikipedia.org/wiki/マイヤー・ヴィートリス完全系列>})を使うのが一般的?}\
	\begin{equation}
		H_{k} (X) = \begin{cases}
			\mathbb{Z}, & k = 0, n \\
			0, & \text{otherwise}
		\end{cases}
	\end{equation}
	となる。
\end{rei}

\begin{rei}[トーラスのホモロジー群]
	$X = T^{2} = S^{1} \times S^{1}$:2次元トーラス,とする。
	\begin{center}
		\begin{tikzpicture}    
			%% 面を塗る
			\fill[red!20] (0,0) -- (2,2) -- (0,2) -- cycle;
			\fill[blue!20] (0,0) -- (2,2) -- (2,0) -- cycle;
			%% 点の名前もろもろ
			\fill \point{(0,0)}{below left}{$v$} \point{(2,0)}{below right}{$v$} \point{(0,2)}{above left}{$v$} \point{(2,2)}{above right}{$v$};
			\node at (1,0) [below]{$b$} node at (0,1) [left]{$a$} node at (1,2) [above]{$b$} node at (2,1) [right]{$a$} node at (1,1) [left]{$c$};
			\node at (0.5,1.5){$U$} node at (1.5,0.5){$L$};
			%% 矢印つき
			\midwayarrow{(0,0)}{(0,2)} \midwayarrow{(0,0)}{(2,0)} \midwayarrow{(0,2)}{(2,2)} \midwayarrow{(2,0)}{(2,2)} \midwayarrow{(0,0)}{(2,2)}
		\end{tikzpicture}	
	\end{center}
	\begin{itemize}
		\vspace{-0.5\baselineskip}
		\item $\Delta_{0} = \mathbb{Z} v$
		\item $\Delta_{1} = \mathbb{Z} a + \mathbb{Z} b + \mathbb{Z} c$
		\item $\Delta_{2} = \mathbb{Z} U + \mathbb{Z} L$
	\end{itemize}

	0次ホモロジー群$H_{0}(X) = \Kernel \partial_{0} / \Image \partial_{1}$は連結成分の個数から求めれば早く,$H_{0}(X) = \mathbb{Z}$。

	1次ホモロジー群$H_{1}(X) = \Kernel \partial_{1} / \Image \partial_{2}$を求める。
	\begin{equation}
		\partial_{1} a = \partial_{1} b = \partial_{1} c = v - v = 0
	\end{equation}
	であるから,$\Kernel \partial_{1} = \Delta_{1} = \mathbb{Z} a + \mathbb{Z} b + \mathbb{Z} c$である。
	また,$\Image \partial_{2} = \mathbb{Z} (a-c+b) + \mathbb{Z} (b-c+a) = \mathbb{Z} (a+b-c)$である。
	$c$は$a+b$にホモローグであるので,$[a]$や$[b]$をホモロジー類として
	\begin{equation}
		H_{1}(X) = \mathbb{Z} [a] + \mathbb{Z} [b] \simeq \mathbb{Z}^{2}
	\end{equation}
	である。(これはトーラスに1次元の穴が2つあることを示唆している)

	2次ホモロジー群$H_{2}(X) = \Kernel \partial_{2} / \Image \partial_{3}$を求める。
	計算により,$\Kernel \partial_{2} = \mathbb{Z} (U-L)$,$\Image \partial_{3} = 0$と分かるから,
	\begin{equation}
		H_{2}(X) = \mathbb{Z} (U-L) \simeq \mathbb{Z}^{1}
	\end{equation}
	である。(これはトーラスに2次元の穴が1つあることを示唆している)

	まとめると,
	\begin{equation}
		H_{n}(X) = \begin{cases}
			\mathbb{Z}, & n=0 \\
			\mathbb{Z}^{2}, & n=1 \\
			\mathbb{Z}, & n=2 \\
			0 & n \ge 3
		\end{cases}
	\end{equation}
\end{rei}

\begin{rei}[射影平面のホモロジー群]
	$X = P^{2}$:2次元射影空間(射影平面),とする。
	\begin{center}
		\begin{tikzpicture}
			\fill[red!20] (0,0) -- (2,2) -- (0,2) -- cycle;
			\fill[blue!20] (0,0) -- (2,2) -- (2,0) -- cycle;
			\fill \point{(0,0)}{below left}{$u$} \point{(2,0)}{below right}{$v$} \point{(0,2)}{above left}{$v$} \point{(2,2)}{above right}{$u$};
			\node at (1,0) [below]{$b$} node at (0,1) [left]{$a$} node at (1,2) [above]{$b$} node at (2,1) [right]{$a$} node at (1,1) [left]{$c$};
			\node at (0.5,1.5){$U$} node at (1.5,0.5){$L$};
			\midwayarrow{(0,0)}{(2,0)} \midwayarrow{(0,0)}{(0,2)} \midwayarrow{(0,0)}{(2,2)} \midwayarrow{(2,2)}{(2,0)} \midwayarrow{(2,2)}{(0,2)}
		\end{tikzpicture}
	\end{center}
	\begin{itemize}
		\vspace{-0.5\baselineskip}
		\item $\Delta_{0} = \mathbb{Z} u + \mathbb{Z} v$
		\item $\Delta_{1} = \mathbb{Z} a + \mathbb{Z} b + \mathbb{Z} c$
		\item $\Delta_{2} = \mathbb{Z} U + \mathbb{Z} L$
	\end{itemize}

	0次ホモロジー群$H_{0}(X) = \Kernel \partial_{0} / \Image \partial_{1}$は,同値関係を$x \sim -x$で定義したときに$P^{2} \simeq S^{2} / \sim$と表されることから,連結成分の個数から求めれば$H_{0}(X) = \mathbb{Z}$。

	1次ホモロジー群$H_{1}(X) = \Kernel \partial_{1} / \Image \partial_{2}$を求める。
	\renewcommand{\arraystretch}{1}
	\begin{center}
		$\begin{array}{lcl}
			\alpha a + \beta b + \gamma c \in \Kernel \partial_{1} & \Longleftrightarrow & \alpha (v-u) + \beta (v-u) + \gamma (u-u) = 0 \\
			& \Longleftrightarrow & \alpha = - \beta
		\end{array}$
	\end{center}
	より,$\Kernel \partial_{1} = \mathbb{Z} (a-b) + \mathbb{Z} c = \mathbb{Z} (a-b+c) + \mathbb{Z} c$である。
	また,$\Image \partial_{2} = \mathbb{Z} (b-a+c) + \mathbb{Z} (a-b+c) = \mathbb{Z} (a-b+c) + \mathbb{Z} \cdot 2c$である。
	よって,$H_{1}(X) = \mathbb{Z} / 2 \mathbb{Z}$である。(これはねじれ部分があることを示唆している)

	2次ホモロジー群$H_{2}(X) = \Kernel \partial_{2} / \Image \partial_{3}$を求める。
	\begin{center}
		$\begin{array}{lcl}
			\alpha (b-a+c) + \beta (a-b+c) = 0 & \Longleftrightarrow & \alpha = \beta = 0
		\end{array}$
	\end{center}
	\renewcommand{\arraystretch}{1.3}
	より,$\Kernel \partial_{2} = 0$である。また,3次元単体は含まれないので$\Image \partial_{3} = 0$である。
	よって,$H_{2} (X) = 0$である。

	まとめると,
	\begin{equation}
		H_{n}(X) = \begin{cases}
			\mathbb{Z}, & n=0 \\
			\mathbb{Z} / 2 \mathbb{Z}, & n=1 \\
			0, & \text{otherwise}
		\end{cases}
	\end{equation}
\end{rei}

	

\expandafter\ifx\csname readornot\endcsname\relax
  \end{document}
\fi % ホモロジーの計算
\expandafter\ifx\csname readornot\endcsname\relax
  \documentclass[uplatex]{jsarticle}
  \usepackage{octopus}
  \usepackage{url}
  \usetikzlibrary{calc}

  \renewcommand{\proofname}{\textsf{証明}}
  \renewcommand{\postpartname}{章}
  \renewcommand{\thesection}{\thepart.\arabic{section}}
  \renewcommand{\thepart}{\arabic{part}}
  \makeatletter\renewcommand{\theequation}{\thesection.\arabic{equation}}\@addtoreset{equation}{section}\makeatother

  \newcommand{\octopuspart}[1]{\newpage\part{#1}\setcounter{section}{0}\vspace{3\baselineskip}}

  \renewcommand{\restriction}[2]{\left. #1 \right|_{#2}}
  \DeclareMathOperator{\dcup}{\dot{\cup}}
  \DeclareMathOperator{\conv}{conv}
  \DeclareMathOperator{\Image}{Im}
  \DeclareMathOperator{\Kernel}{Ker}
  \DeclareMathOperator{\diag}{diag}
  \DeclareMathOperator{\rank}{rank}
  \begin{document}
  \fi

\section{ホモロジーの計算 2}
\newcommand{\BigZero}{\kern3pt \hbox{\huge \strut 0}}

\begin{teigi}[自由$\mathbb{Z}$加群]
  $G$が\nw{自由$\mathbb{Z}$加群}であるとは,$\mathbb{Z}$上の加群$G$が,
  $G \simeq \mathbb{Z}^{r}$であり,かつ$r$個の元$e_{1}, e_{2}, \dots, e_{r}$によって$g \in G$が一意に
  ${\displaystyle g = \sum_{i=1}^{r} n_{i} e_{i}}$とかけるものとする。
\end{teigi}

この$e_{1}, e_{2}, \dots, e_{r}$を$\mathbb{Z}$基底といい,$r$をランクと呼ぶ。

\begin{teigi}[ユニモジュラ行列]
  $Q$:$r$次正方整数行列とする。

  $Q$:\nw{ユニモジュラ(unimodular)} $\defines$ $\det Q = \pm 1$
\end{teigi}

ユニモジュラ行列の定義について次のことが成り立つ。
\renewcommand{\arraystretch}{1}
\begin{center}
  \begin{tabular}{lcl}
    $Q$:ユニモジュラ & $\defines$ & $\det Q = \pm 1$ \\
    & $\Longleftrightarrow$ & $e_{1}, e_{2}, \dots, e_{r}$が$\mathbb{Z}$基底ならば
    ${\displaystyle e_{i}' = \sum_{j=1}^{r} Q_{ij} e_{j}}$($i = 1, \dots, r$)も$\mathbb{Z}$基底である 
  \end{tabular}
\end{center}

このことから,ユニモジュラ行列は,自由$\mathbb{Z}$加群の基底の変換行列であることが従う。
なお,$Q$は定義から正則行列であるが,逆行列$Q^{-1}$を余因子行列で書けば分かるように,$Q^{-1}$も整数行列であってユニモジュラ行列である。
したがって,${\displaystyle g = \sum_{i=1}^{n} n_{i} e_{i}}$ならば
\begin{equation}
  g = \sum_{j=1}^{r} \sum_{i=1}^{r} n_{i} (Q^{-1})_{ji} \, e_{j}'
\end{equation}
と新たな基底の下での表現を得ることができる。

\begin{teiri}[Smith標準形(単因子標準形)]
  (正方とは限らない)整数行列$A$は,あるユニモジュラ行列$P,Q$を用いて
  \begin{equation}
    PAQ = \left( \begin{array}{cccc|c}
      \alpha_{1} &            &        &            & \multirow{4}{*}{\BigZero} \\
                 & \alpha_{2} &        &            & \\
                 &            & \ddots &            & \\
                 &            &        & \alpha_{k} & \\ \hline
      \multicolumn{4}{c|}{\BigZero} & \BigZero
    \end{array} \right)
  \end{equation}
  の形にできる。ここで,$\alpha_{1} | \alpha_{2} | \dots | \alpha_{k}$($\alpha_{i} > 0$)である。
  また,$\alpha_{1}, \dots, \alpha_{k}$は一意に定まり,これを$A$の\nw{単因子}と呼ぶ。
\end{teiri}

\begin{proof}
  以下に示すような基本変形を繰り返せば右辺のような形になる。:
  \begin{itemize}
    \vspace{-0.5\baselineskip}
    \item[(1)] $i$行(列)を$-1$倍する。
    \begin{equation}
      \begin{pmatrix}
        1 & \\
          & \ddots \\
          &        & -1 \\
          &        &    & \ddots \\ 
          &        &    &        & 1
      \end{pmatrix}
    \end{equation}
    \item[(2)] 行(列)の入れ替え
    \begin{equation}
      \left( \begin{array}{ccccccccccc}
        1 \\
          & \ddots \\
          &        & 1 \\
          &        &   & 0      & \hdots & \hdots & \hdots & 1      \\ 
          &        &   & \vdots & 1      &        &        & \vdots \\
          &        &   & \vdots &        & \ddots &        & \vdots \\
          &        &   & \vdots &        &        & 1      & \vdots \\
          &        &   & 1      & \hdots & \hdots & \hdots & 0      \\
          &        &   &        &        &        &        &        & 1 \\
          &        &   &        &        &        &        &        &   & \ddots \\
          &        &   &        &        &        &        &        &   &        & 1 
      \end{array} \right)
    \end{equation}
    \item[(3)] $i$行(列)を$c$倍して$j$行(列)に足す
    \begin{equation}
      \begin{pmatrix}
        1 & \\
          & \ddots \\
          &        & 1      \\
          &        & \vdots & \ddots \\
          &        & c      & \hdots & 1 \\ 
          &        &        &        &   & \ddots \\
          &        &        &        &   &        & 1
      \end{pmatrix}
    \end{equation}
  \end{itemize}
  これらの変形は直下に示したユニモジュラ行列による変換として表現できるので定理が従う。
\end{proof}

\begin{rei}
  実際にSmith標準形を求めてみる。
  \begin{center}
    $\begin{array}{ccc}
      \begin{pmatrix}
        -3 & 4 & 4 \\ -2 & 2 & -4 \\ 4 & -4 & 4
      \end{pmatrix} & \xrightarrow[\text{絶対値最小の非零要素を$(1,1)$成分へ}]{(1),(2)}
      & \begin{pmatrix}
        2 & -2 & 4 \\ -3 & 4 & 4 \\ 4 & -4 & 4
      \end{pmatrix} \\ & \xrightarrow[\text{1行,1列に$a_{11}$で割り切れないものがあればそれを「割った余り」にする}]{(3); 2 \text{行目} + 2 \times 1 \text{行目}}
      & \begin{pmatrix}
        2 & -2 & 4 \\ 1 & 0 & 12 \\ 4 & -4 & 4
      \end{pmatrix} \\ & \xrightarrow[\text{絶対値最小の非零要素を$(1,1)$成分へ}]{(1)}
      & \begin{pmatrix}
        1 & 0 & 12 \\ 2 & -2 & 4 \\ 4 & -4 & 4
      \end{pmatrix} \\ & \xrightarrow[\text{1行,1列がすべて$a_{11}$で割り切れる場合は割って0にする}]{(3)}
      & \begin{pmatrix}
        1 & 0 & 0 \\ 0 & -2 & -20 \\ 0 & -4 & -44
      \end{pmatrix} \\ & \xrightarrow[\text{右下の$2 \times 2$の小行列に対して同じアルゴリズムを適用}]{(1),(3)}
      & \begin{pmatrix}
        1 & 0 & 0 \\ 0 & 2 & -20 \\ 0 & 0 & -4
      \end{pmatrix} \\ & \xrightarrow{(1),(3)}
      & \begin{pmatrix}
        1 & 0 & 0 \\ 0 & 2 & 0 \\ 0 & 0 & 4
      \end{pmatrix}
    \end{array}$
  \end{center}
  辿り着いた行列がSmith標準形である。
\end{rei}

すなわち,
\begin{itemize}
  \item 一番左の行と列に左上の要素で割り切れない要素があるときには,
  $a_{11}$で割った余りになるように変形した上でその要素を左上に移動する。
  \item 一番左の行と列に左上の要素で割り切れない要素がないときには,
  その行と列を1にするように変形し,サイズが一回り小さい行列に対して更に変形を続けていく。
\end{itemize}
の手順を踏むことでSmith標準形に辿り着く。

\begin{hodai}
  \label{hodai:homology3.unimod}
  $BA = O$とし,$A$の単因子を$\alpha_{1} | \alpha_{2} | \dots | \alpha_{k}$,$B$の単因子を$\beta_{1} | \beta_{2} | \dots | \beta_{\ell}$とする。
  このとき,あるユニモジュラ行列$P,Q,R$が存在して,
  \begin{equation}
    R^{-1} B P = \begin{pmatrix}
      O_{k' \times k'} & O_{k' \times \ell} \\ O_{\ell \times k'} & \diag (\beta_{1}, \dots, \beta_{\ell})
    \end{pmatrix}, \quad
    P^{-1} A Q = \begin{pmatrix}
      \diag (\alpha_{1}, \dots, \alpha_{k}) & O \\ O & O
    \end{pmatrix}
  \end{equation}
  となる。ただし,$O_{m \times n}$で$m \times n$の大きさの零行列を表すとし,$k' \ge k$とする。
\end{hodai}

\begin{proof}
  まず,$A$をSmith標準形にする。$\tilde{P},Q$をユニモジュラ行列として,
  \begin{equation}
    \tilde{P}^{-1}AQ = \left( \begin{array}{cccc|c}
      \alpha_{1} &            &        &            & \multirow{4}{*}{\BigZero} \\
                 & \alpha_{2} &        &            & \\
                 &            & \ddots &            & \\
                 &            &        & \alpha_{k} & \\ \hline
      \multicolumn{4}{c|}{\BigZero} & \BigZero
    \end{array} \right)
  \end{equation}
  $BAQ = B\tilde{P} \tilde{P}^{-1} AQ = O$より,$B \tilde{P}$の形状は第1行から第$k$行までが零行列となっている。
  そこで,$B \tilde{P}$の第$k+1$行目以降の小行列に対して,Smith標準形をつくる。
  すると,$R,S$をユニモジュラ行列として次のようになる。
  \begin{equation}
    R B \tilde{P} \begin{pmatrix}
      I_{k} & O \\ O & S
    \end{pmatrix} = 
    \begin{array}{l}
      \hspace{1em}\overbrace{}^{k} \\
      \left( \begin{array}{c|c|c}
                &          & \multirow{4}{*}{\BigZero} \\
                &          &                           \\
                &          &                           \\
        \BigZero & \BigZero &                           \\ \cline{3-3}
                &          & \begin{array}{ccc} \beta_{1} & & \\ & \ddots & \\ & & \beta_{\ell} \end{array} \\
      \end{array} \right) \\
      \hspace{1em}\underbrace{\hspace{4em}}_{k'}
  \end{array}
  \end{equation}
  いま,
  \begin{equation*}
    P = \tilde{P} \begin{pmatrix}
      I_{k} & O \\ O & S
    \end{pmatrix}
  \end{equation*}
  とおけば,
  \begin{equation}
    P^{-1} A Q = \begin{pmatrix}
      I_{k} & O \\ O & S^{-1}
    \end{pmatrix}
    \tilde{P}^{-1} A Q =
    \left( \begin{array}{cccc|c}
      \alpha_{1} &            &        &            & \multirow{4}{*}{\BigZero} \\
                 & \alpha_{2} &        &            & \\
                 &            & \ddots &            & \\
                 &            &        & \alpha_{k} & \\ \hline
      \multicolumn{4}{c|}{\BigZero} & \BigZero
    \end{array} \right) 
  \end{equation}
  であり,
  \begin{equation}
    R^{-1} B P = \begin{array}{l}
      \hspace{1em}\overbrace{\hspace{4em}}^{k'} \\
      \left( \begin{array}{c|c|c}
                &          & \multirow{4}{*}{\BigZero} \\
                &          &                           \\
                &          &                           \\
        \BigZero & \BigZero &                           \\ \cline{3-3}
                &          & \begin{array}{ccc} \beta_{1} & & \\ & \ddots & \\ & & \beta_{\ell} \end{array} \\
      \end{array} \right)
    \end{array}
  \end{equation}
  である。
\end{proof}

\sukima \midashi{\large ホモロジーの計算}

$X$:位相空間,$\Delta$-複体構造,$c_{n}$:$n$-単体の個数(有限とする),とする。
このとき,
\begin{center}
  \begin{tabular}{l@{\,}c@{\,}l}
    $\Delta_{n}$ & $:=$ & $n$-チェイン全体($n$-単体の整数結合) \\
    & $\sim$ & ランク$c_{n}$の自由$\mathbb{Z}$加群,特に$n$-単体$\sigma_{1},\sigma_{2}, \dots, \sigma_{c_{n}}$は$\Delta_{n}$の$\mathbb{Z}$基底 \\
    & $\simeq$ & $\mathbb{Z}^{c_{n}}$
  \end{tabular}
\end{center}
である。よって,境界作用素$\partial_{n} \colon \Delta_{n} \longrightarrow \Delta_{n-1}$は$\mathbb{Z}^{c_{n}} \longrightarrow \mathbb{Z}^{c_{n-1}}$と見なすことができ,
これを行列によって表示できる。
\begin{align}
  & \partial_{n} \colon \mapdef{\Delta_{n}}{\Delta_{n-1}}{\displaystyle \sum_{i=1}^{c_{n}} t_{i} \sigma_{i}}{\displaystyle \sum_{i=1}^{c_{n}} t_{i} \partial \sigma_{i}} \\
  & \partial_{n}^{*} \colon \mapdef{\mathbb{Z}^{c_{n}}}{\mathbb{Z}^{c_{n-1}}}{
    \begin{pmatrix}
      \sigma_{1} & \sigma_{2} & \cdots & \sigma_{c_{n}}
    \end{pmatrix}
    \begin{pmatrix}
      t_{1} \\ t_{2} \\ \vdots \\ t_{c_{n}}
    \end{pmatrix}
  }{
    \begin{pmatrix}
      \pi_{1} & \pi_{2} & \cdots & \pi_{c_{n-1}}
    \end{pmatrix} A
    \begin{pmatrix}
      t_{1} \\ t_{2} \\ \vdots \\ t_{c_{n}}
    \end{pmatrix}
  }
\end{align}
ここで,$c_{n-1} \times c_{n}$次行列$A$は$\partial_{n}$の行列表示であり,
$\partial \sigma_{j}$における$\pi_{i}$の係数を$(i,j)$成分にもつものとする。

\begin{rei}[射影平面の場合]
  $X = P^{2}$:2次元射影空間(射影平面)とする。
  \begin{center}
		\begin{tikzpicture}
			\fill[red!20] (0,0) -- (2,2) -- (0,2) -- cycle;
			\fill[blue!20] (0,0) -- (2,2) -- (2,0) -- cycle;
			\fill \point{(0,0)}{below left}{$u$} \point{(2,0)}{below right}{$v$} \point{(0,2)}{above left}{$v$} \point{(2,2)}{above right}{$u$};
			\node at (1,0) [below]{$b$} node at (0,1) [left]{$a$} node at (1,2) [above]{$b$} node at (2,1) [right]{$a$} node at (1,1) [left]{$c$};
			\node at (0.5,1.5){$U$} node at (1.5,0.5){$L$};
			\midwayarrow{(0,0)}{(2,0)} \midwayarrow{(0,0)}{(0,2)} \midwayarrow{(0,0)}{(2,2)} \midwayarrow{(2,2)}{(2,0)} \midwayarrow{(2,2)}{(0,2)}
		\end{tikzpicture}
  \end{center}
  \begin{itemize}
    \item $\Delta_{2} \simeq \sets{\begin{pmatrix}
      t_{U} & t_{L} 
    \end{pmatrix}^{\top} | t_{U}, t_{L} \in \mathbb{Z}} = \mathbb{Z}^{2}$
    \item $\Delta_{1} \simeq \sets{\begin{pmatrix}
      t_{a} & t_{b} & t_{c} 
    \end{pmatrix}^{\top} | t_{a}, t_{b}, t_{c} \in \mathbb{Z}} = \mathbb{Z}^{3}$
    \item $\Delta_{0} \simeq \sets{\begin{pmatrix}
      t_{u} & t_{v} 
    \end{pmatrix}^{\top} | t_{u}, t_{v} \in \mathbb{Z}} = \mathbb{Z}^{2}$
  \end{itemize}
  このとき,境界作用素の行列表示はそれぞれ,
  \begin{equation}
    A_{\partial_{1}} = \begin{array}{c@{\,}c}
      & \begin{array}{ccc}
        a & b & c
      \end{array} \\
      \begin{array}{c}
        u \\ v
      \end{array} & 
      \begin{pmatrix}
        -1 & 1 & 0 \\
        -1 & 1 & 0
      \end{pmatrix}  
    \end{array}
    , \quad 
    A_{\partial_{2}} = \begin{array}{c@{\,}c}
      & \begin{array}{cc}
        U & L
      \end{array} \\
      \begin{array}{c}
        a \\ b \\ c
      \end{array} &
      \begin{pmatrix}
      -1 & 1 \\
      1 & -1 \\
      1 & 1
    \end{pmatrix}
  \end{array}
  \end{equation}
  である。
\end{rei}

以上を踏まえて$H_{n} = \Kernel \partial_{n} / \Image \partial_{n+1}$を計算する流れについて述べる。

\sukima\midashi{注意する事実}
\begin{itemize}
  \item $\Delta_{n+1} \xrightarrow{\partial_{n+1}} \Delta_{n} \xrightarrow{\partial_{n}} \Delta_{n-1}$について$\partial_{n} \circ \partial_{n+1} = 0$である。
  \item 行列表示で言えば,$\partial_{n}$の行列表示を$A_{n}$として,
  \begin{equation}
    A_{n} A_{n+1} = O
  \end{equation}
  \item \rref{補題}{hodai:homology3.unimod}より,あるユニモジュラ行列$P,Q,R$が存在して
  \begin{align}
    P^{-1} A_{n+1} Q &=
    \left( \begin{array}{ccc|c}
      \alpha_{1} &        &                  & \multirow{3}{*}{\BigZero} \\
                 & \ddots &                  & \\
                 &        & \alpha_{k_{n+1}} & \\ \hline
      \multicolumn{3}{c|}{\BigZero} & \BigZero
    \end{array} \right), \\
    \label{eq:homology3.calc}
    R^{-1} A_{n} P &=
    \left( \begin{array}{c|ccc}
      \BigZero                  & \multicolumn{3}{c}{\BigZero} \\ \hline
      \multirow{3}{*}{\BigZero} & \beta_{1} &        & \\
                                &           & \ddots & \\
                                &           &        & \beta_{k_{n}} \\
    \end{array} \right)
  \end{align}
  とできる。なお,式\eqref{eq:homology3.calc}において上2つの零行列の縦のサイズは$c_{n} - k_{n}$である。
\end{itemize}

これを使って計算していく。

\begin{equation}
  \begin{array}{c@{\:}c@{\:}c@{\:}c@{\:}c}
    \begin{pmatrix}
      \rho_{1} & \cdots & \rho_{c_{n+1}}
    \end{pmatrix}
    \begin{pmatrix}
      s_{1} \\ \vdots \\ s_{c_{n+1}}
    \end{pmatrix}
    & \longmapsto &
    \begin{pmatrix}
      \sigma_{1} & \cdots & \sigma_{c_{n}}
    \end{pmatrix}
    A_{n+1}
    \begin{pmatrix}
      s_{1} \\ \vdots \\ s_{c_{n+1}}
    \end{pmatrix} \\
    \rotatebox{270}{$\in$} & & \rotatebox{270}{$\in$} \\
    \Delta_{n+1} & \xrightarrow{\partial_{n+1}} & \Delta_{n} & \xrightarrow{\partial_{n}} & \Delta_{n-1} \\
    & & \vin & & \vin \\
    & & \begin{pmatrix}
      \sigma_{1} & \cdots & \sigma_{c_{n}}
    \end{pmatrix}
    \begin{pmatrix}
      t_{1} \\ \vdots \\ t_{c_{n}}
    \end{pmatrix}
    & \longmapsto &
    \begin{pmatrix}
      \pi_{1} & \cdots & \pi_{c_{n-1}}
    \end{pmatrix}
    A_{n}
    \begin{pmatrix}
      t_{1} \\ \vdots \\ t_{c_{n}}
    \end{pmatrix} \\
  \end{array}
\end{equation}
において,先の\textbf{注意する事実}で述べた行列$P,Q,R$を用いて
\begin{equation}
  \left\{ \begin{split}
    \bm{\rho}' &= Q^{\top} \bm{\rho}, \\
    \bm{\sigma}' &= P^{\top} \bm{\sigma}, \\
    \bm{\pi}' &= R^{\top} \bm{\pi}, \\
    \bm{s}' &= Q^{-1} \bm{s}, \\
    \bm{t}' &= P^{-1} \bm{t}
  \end{split} \right.
\end{equation}
と変換すると
\begin{equation}
  \begin{array}{c@{\:}c@{\:}c@{\:}c@{\:}c}
    \begin{pmatrix}
      \rho'_{1} & \cdots & \rho'_{c_{n+1}}
    \end{pmatrix}
    \begin{pmatrix}
      s'_{1} \\ \vdots \\ s'_{c_{n+1}}
    \end{pmatrix}
    & \longmapsto &
    \begin{pmatrix}
      \sigma'_{1} & \cdots & \sigma'_{c_{n}}
    \end{pmatrix}
    P^{-1}A_{n+1}Q
    \begin{pmatrix}
      s'_{1} \\ \vdots \\ s'_{c_{n+1}}
    \end{pmatrix} \\
    \rotatebox{270}{$\in$} & & \rotatebox{270}{$\in$} \\
    \Delta_{n+1} & \xrightarrow{\partial_{n+1}} & \Delta_{n} & \xrightarrow{\partial_{n}} & \Delta_{n-1} \\
    & & \vin & & \vin \\
    & & \begin{pmatrix}
      \sigma'_{1} & \cdots & \sigma'_{c_{n}}
    \end{pmatrix}
    \begin{pmatrix}
      t'_{1} \\ \vdots \\ t'_{c_{n}}
    \end{pmatrix}
    & \longmapsto &
    \begin{pmatrix}
      \pi'_{1} & \cdots & \pi'_{c_{n-1}}
    \end{pmatrix}
    R^{-1}A_{n}P
    \begin{pmatrix}
      t'_{1} \\ \vdots \\ t'_{c_{n}}
    \end{pmatrix} \\
  \end{array}
\end{equation}
となって,ここに
\begin{align}
  P^{-1} A_{n+1} Q &=
    \left( \begin{array}{ccc|c}
      \alpha_{1} &        &                  & \multirow{3}{*}{\BigZero} \\
                 & \ddots &                  & \\
                 &        & \alpha_{k_{n+1}} & \\ \hline
      \multicolumn{3}{c|}{\BigZero} & \BigZero
    \end{array} \right), \\
    R^{-1} A_{n} P &=
    \left( \begin{array}{c|ccc}
      \BigZero                  & \multicolumn{3}{c}{\BigZero} \\ \hline
      \multirow{3}{*}{\BigZero} & \beta_{1} &        & \\
                                &           & \ddots & \\
                                &           &        & \beta_{k_{n}} \\
    \end{array} \right)
\end{align}
である。
すると,これより直ちに
\begin{equation}
  \Image \partial_{n+1} = \begin{pmatrix}
    \mathbb{Z} \alpha_{1} \\ \vdots \\ \mathbb{Z} \alpha_{k_{n+1}} \\ 0 \\ \vdots \\ 0
  \end{pmatrix}, \quad 
  \Kernel \partial_{n} = \begin{pmatrix}
    \mathbb{Z} \\ \vdots \\ \mathbb{Z} \\ 0 \\ \vdots \\ 0
  \end{pmatrix}
\end{equation}
と計算される。ここに,$\Kernel \partial_{n}$の$\mathbb{Z}$の数は$c_{n} - k_{n}$であることに注意。
いま,さらに
\begin{equation}
  \alpha_{1} = \alpha_{2} = \dots = \alpha_{\ell-1} = 1
\end{equation}
の関係が成り立っているとすると,$\mathbb{Z} / \mathbb{Z} = 0$に注意すれば
\begin{equation}
  H_{n} = \Kernel \partial_{n} / \Image \partial_{n+1}
  = \begin{pmatrix}
    \mathbb{Z} / \alpha_{1} \mathbb{Z} \\ \vdots \\ \mathbb{Z} / \alpha_{k_{n+1}} \mathbb{Z} \\ \mathbb{Z} \\ \vdots \\ \mathbb{Z} \\ 0 \\ \vdots \\ 0
  \end{pmatrix}
  = \mathbb{Z} / \alpha_{\ell} \mathbb{Z} \oplus \dots \oplus \mathbb{Z} / \alpha_{k_{n+1}} \mathbb{Z} \oplus \mathbb{Z}^{c_{n} - k_{n+1} - k_{n}}
\end{equation}
となる。
このうち,

\begin{equation}
  \mathbb{Z} / \alpha_{\ell} \mathbb{Z} \oplus \dots \oplus \mathbb{Z} / \alpha_{k_{n+1}} \mathbb{Z}
\end{equation}
の部分を\nw{ねじれ成分}と呼ぶ。
また,$H_{n}$の\nw{ランク},$\rank H_{n}$を
\begin{equation}
  \rank H_{n} := c_{n} - k_{n} - k_{n+1}
\end{equation}
で定める。つまり,境界作用素のSmith標準形を求めていけば,ホモロジー群$H_{n}$を決定することができる。

\sukima \midashi{\large Euler標数}

\begin{teigi}
  次の式\eqref{eq:homology3.Euler}で定まる量$\chi (X)$を$X$の\nw{Euler標数(Euler characteristic)}という。
  \begin{equation}
    \label{eq:homology3.Euler}
    \chi (X) := \sum_{n=0}^{\dim X} (-1)^{n} c_{n}
  \end{equation}
  ただし,$c_{n}$は$X$に含まれる$n$-単体の個数である。
\end{teigi}

\begin{prop}
  ${\displaystyle \chi (X) = \sum_{n=0}^{\dim X} (-1)^{n} \rank H_{n} (X)}$
\end{prop}

\begin{proof}
  定義から$\rank H_{n} (X) = c_{n} - k_{n} - k_{n+1}$であったので,
  \begin{equation}
    \sum_{n=0}^{\dim X} (-1)^{n} \rank H_{n} (X) = \sum_{n=0}^{\dim X} (-1)^{n} (c_{n} - k_{n} - k_{n+1}) = \sum_{n=0}^{\dim X} (-1)^{n} c_{n} = \chi (X)
  \end{equation}
\end{proof}

\midashi{注意.} ホモロジー群は\textbf{位相不変量}であり,複体構造に依存しない。
したがって,Euler標数も複体構造に依存せずに決まる。

\begin{rei}[射影平面の例]
  $\Delta_{3} \xrightarrow{\partial_{3}} \Delta_{2} \xrightarrow{\partial_{2}} \Delta_{1} \xrightarrow{\partial_{1}} \Delta_{0}$において,境界作用素の行列表示とSmith標準形はそれぞれ,
  \begin{align}
    A_{\partial_{2}} = \begin{array}{c@{\,}c}
      & \begin{array}{ccc}
        a & b & c
      \end{array} \\
      \begin{array}{c}
        U \\ L
      \end{array} & 
      \begin{pmatrix}
        -1 & 1 & 1 \\
        1 & -1 & 1
      \end{pmatrix}  
    \end{array}
    , \quad 
    A_{\partial_{1}} = \begin{array}{c@{\,}c}
      & \begin{array}{cc}
        u & v
      \end{array} \\
      \begin{array}{c}
        a \\ b \\ c
      \end{array} &
      \begin{pmatrix}
      -1 & 1 \\
      -1 & 1 \\
      0 & 0
    \end{pmatrix}
  \end{array}, \\
  \mathrm{Smith}(A_{\partial_{2}}) = \begin{pmatrix}
    1 & 0 & 0 \\ 0 & 2 & 0
  \end{pmatrix}, \quad
  \mathrm{Smith}(A_{\partial_{1}}) = \begin{pmatrix}
    0 & 0 \\
    0 & 0 \\
    1 & 0
  \end{pmatrix}
  \end{align}
  となる。これより,
  \begin{equation}
    H_{2}(P^{2}) = 0, \quad H_{1}(P^{2}) = \mathbb{Z} / 2 \mathbb{Z}, \quad H_{0}(P^{2}) = \mathbb{Z}
  \end{equation}
  であり,Euler標数は
  \begin{equation}
    \chi (P^{2}) = 1
  \end{equation}
  と計算される。
\end{rei}

\begin{rei}[2次元球面の例]
  複体構造の図は
  \begin{center}
		\begin{tikzpicture}
		%% 面を塗る
		\fill[red!20] (0,0) -- (2,2) -- (0,2) -- cycle;
		\fill[blue!20] (0,0) -- (2,2) -- (2,0) -- cycle;
		%% 点の名前もろもろ
		\fill \point{(0,0)}{below left}{$u$} \point{(2,0)}{below right}{$v$} \point{(0,2)}{above left}{$v$} \point{(2,2)}{above right}{$w$};
		\node at (1,0) [below]{$b$} node at (0,1) [left]{$b$} node at (1,2) [above]{$a$} node at (2,1) [right]{$a$} node at (1,1) [left]{$c$};
		\node at (0.5,1.5){$U$} node at (1.5,0.5){$L$};
		%% 矢印つき
		\midwayarrow{(0,0)}{(0,2)} \midwayarrow{(0,0)}{(2,0)} \midwayarrow{(0,2)}{(2,2)} \midwayarrow{(2,0)}{(2,2)} \midwayarrow{(0,0)}{(2,2)}
		\end{tikzpicture}
  \end{center}
  であるから,
  \begin{equation}
    \chi (S^{2}) = c_{0} - c_{1} + c_{2} = 3 - 3 + 2 = 2
  \end{equation}
  と求まる。因みに
  \begin{equation}
    H_{0}(S^{2}) = \mathbb{Z}, \quad H_{1}(S^{2}) = 0, \quad H_{2}(S^{2}) = \mathbb{Z}
  \end{equation}
\end{rei}

\expandafter\ifx\csname readornot\endcsname\relax
  \end{document}
\fi % 特異ホモロジー

\octopuspart{テンソル}
座標変換に対する不変性

\expandafter\ifx\csname readornot\endcsname\relax
  \documentclass[uplatex]{jsarticle}
  \usepackage{octopus}
  \usepackage{url}

  \renewcommand{\proofname}{\textsf{証明}}
  \renewcommand{\postpartname}{章}
  \renewcommand{\thesection}{\thepart.\arabic{section}}
  \renewcommand{\thepart}{\arabic{part}}
  \makeatletter\renewcommand{\theequation}{\thesection.\arabic{equation}}\@addtoreset{equation}{section}\makeatother

  \newcommand{\octopuspart}[1]{\newpage\part{#1}\setcounter{section}{0}\vspace{3\baselineskip}}

  \renewcommand{\restriction}[2]{\left. #1 \right|_{#2}}
  \DeclareMathOperator{\dcup}{\dot{\cup}}
  \DeclareMathOperator{\conv}{conv}
  \DeclareMathOperator{\Image}{Im}
  \DeclareMathOperator{\Kernel}{Ker}
  \begin{document}
\fi

\section{テンソルの定義}
\midashi{ベクトル空間}
\begin{teigi}[ベクトル空間]
  略.
\end{teigi}

\begin{rei}
  $\mathbb{R}^{n}$は$\mathbb{R}$上のベクトル空間。
\end{rei}

\midashi{注意.}環においてベクトル空間に対応する概念は\nw{加群 (module)}という。

\begin{teigi}[基底]
  $B \subseteq V$が\nw{基底} $\iff$ 任意の$v \in V$を${\displaystyle v = \sum_{u \in B: \text{有限和}} \alpha_{u} u}$と一意的に表される。
\end{teigi}

$B \subseteq V$が基底であることは,任意の$v \in V$が${\displaystyle v = \sum_{u \in B: \text{有限和}} \alpha_{u} u}$と表すことができて,任意の有限部分集合$B^{\prime}$が一次独立であることと同値。

\begin{hodai}
  ベクトル空間には基底が存在する\footnote{証明にはZornの補題を用いるがここでは省略する。}。
\end{hodai}

ベクトル空間の基底が有限個の元からなるとき,その個数は基底の取り方によらずに等しいので,次のように次元を定義することができる。
\begin{teigi}
  ベクトル空間$V$が$n$次元 $\iff$ $n$個の元からなる基底が存在\par
  ベクトル空間$V$が無限次元 $\iff$ 任意の$n \in \mathbb{N}$に対してベクトル空間$V$が$n$次元でない
\end{teigi}
ベクトル空間$V$の次元が$n$であることを$\dim V = n$と書く。$V$が無限次元であることは$\dim V = \infty$と書く。

\begin{rei}
  $\mathbb{R}^{n}$は$n$次元。
\end{rei}

\begin{rei}
  $\mathbb{R}[x]$を実係数1変数多項式全体の集合とする。これは$\mathbb{R}$ベクトル空間になる。このときの基底は$B = \sets{1, x, x^{2}, \dots}$であり,したがって$\mathbb{R}[x]$は$\mathbb{R}$上の無限次元ベクトル空間である。
\end{rei}

\midashi{注意.} $L^{2}[0,1]$を$[0,1]$上で2乗可積分な関数全体の集合とすると,これは$\mathbb{R}$ベクトル空間である。このとき,任意の$f \in L^{2}[0,1]$は
\begin{align}
  f(x) = \sum_{k=1}^{\infty} \alpha_{k} \sin (2 \pi k x) + \sum_{k=0}^{\infty} \beta_{k} \cos (2 \pi k x)
\end{align}
と一意的に書けるが,$\sets{\sin (2 \pi k x) | k \in \mathbb{N}_{>0}} \cup \sets{\cos (2 \pi k x) | k \in \mathbb{N}}$は上の意味での「基底」ではないことに注意。この基底のことは「\textbf{代数基底}」や「\textbf{Hamel基底}」と呼んで区別する\footnote{\url{https://ja.wikipedia.org/wiki/基底_(線型代数学)}}。

以降は$V$を$n$次元ベクトル空間とする。

\sukima\midashi{双対空間}

\begin{teigi}[線形汎関数]
  $f\colon V\to K$が線形汎関数 $\defines$ 任意の$\alpha,\beta \in K, u,v \in V$に対して
  \begin{align*}
    f(\alpha u+\beta v) = \alpha f(u) + \beta f(v)
  \end{align*}
\end{teigi}
\begin{teigi}[双対空間]
  $V$の\textbf{双対空間}$V^\ast$を,$V^\ast := \{ f\colon V\to K \mid fは線形汎関数 \}$で定義する.
\end{teigi}
\begin{teigi}[双対基底]
  $\{e_1,\cdots,e_n\}$を$V$の基底とする.このとき,$V^\ast$の基底$\{e^1,\cdots,e^n\}$を
  \begin{align*}
    e^j(e_i) = \delta_i^j := \begin{cases}
      1 & i=j \\
      0 & i\neq j
    \end{cases}
  \end{align*}
  を満たす線形汎関数として定義する.
\end{teigi}
\begin{prop}
  $\{e^1,\cdots,e^n\}$は$V^\ast$の基底.
\end{prop}
\begin{proof}
  $f\in V^\ast$に対して
  \begin{align*}
    f = \sum_i f(e_i) e^i
  \end{align*}
  と表すことが出来る.また,
  \begin{align*}
    \sum_i \beta^i e^i = 0
  \end{align*}
  が成り立つとき(右辺の0は零写像を表している),
  \begin{align*}
    \left(\sum_i \beta^i e^i \right) (e_j) & = \beta_j \\
    0(e_j) & = 0 
  \end{align*}
  よって$\beta_j = 0$なので$\{e_1,\cdots,e_n\}$は基底である.
\end{proof}
\begin{corr}
  $\dim V = n$ならば$\dim V^\ast = n$.
\end{corr}
\begin{prop}
  $V^{\ast \ast} \simeq V$
\end{prop}
\begin{proof}
  $v\in V$を$f\in V^\ast$に対して$f\mapsto f(v)$という対応を考えることで,$v$を写像$v\colon V^\ast \to K$とみなす.このとき,
  \begin{align*}
    v(\alpha f + \beta g) & = (\alpha f + \beta g)(v) \\
    & = \alpha f(v) + \beta g(v) \\
    & = \alpha v(f) + \beta v(g)
  \end{align*}
  より$v$は$V^\ast$上の線形汎関数.したがって$V\subseteq V^{\ast \ast}$.また$\dim V = \dim V^\ast = \dim V^{\ast \ast}$より,$V \simeq V^{\ast \ast}$.
\end{proof}

\sukima\midashi{テンソル}

以下では$U,V$を体$K$上の有限次元ベクトル空間とする.
\begin{teigi}[双線形写像]
  $\Phi\colon U\times V\to K$が任意の$\alpha,\alpha',\beta,\beta'\in K,u,u'\in U ,v,v'\in V$に対して
  \begin{align*}
    \Phi(\alpha u + \alpha' u',v) & = \alpha \Phi(u,v) + \alpha' \Phi(u',v) \\
    \Phi(u,\beta v + \beta' v) & = \beta \Phi(u,v) + \beta' \Phi(u,v') 
  \end{align*}
  を満たすとき,$\Phi$を\textbf{双線形写像}という.
\end{teigi}

\begin{rei}
  $U = \mathbb{R}^n , V = \mathbb{R}^m, A \in \mathbb{R}^{n\times m}$とする.$x \in U , y \in V$に対して
  \begin{align*}
    (x,y) \mapsto x^\top A y
  \end{align*}
  という対応は双線形写像になっている(この双線形写像を2次形式とよぶ).
\end{rei}
\begin{remark}
  $\Phi\colon U\times V\to K$を双線形写像とする.$U$の基底$\{e_1,\cdots,e_n\}$,$V$の基底$\{f_1,\cdots,f_m\}$としたとき,$u = \sum_{i} \alpha^i e_i \in U , v = \sum_j \beta^j  f_j \in V$に対して$\Phi(u,v)$は
  \begin{align*}
    \Phi\left(\sum_{i} \alpha^i e_i,\sum_j \beta^j f_j\right) = \sum_{i,j} \alpha^i \beta^j \Phi(e_i,f_j)
  \end{align*}
  となり,二次形式とみなすことができる.
\end{remark}
\begin{teigi}[多重線形写像]
  $V_i\ (i=1,\cdot,k)$を体$K$上の線形空間とする.$\Phi\colon V_1\times \cdots V_k \to K$で各$i\  ( i=1,\cdots,k)$で線形な写像を多重線形写像という.
\end{teigi}

\sukima\midashi{テンソル積の定義(I)}
\begin{teigi}[テンソル積]
  $U$と$V$のテンソル積$U\otimes V$を
  \begin{align*}
    U\otimes V := \{ \Phi \colon U^\ast \times V^\ast \to K \mid \Phi\mbox{は双線形写像} \}
  \end{align*}
  で定義する.

  また,$u\in U$と$v\in V$のテンソル積 $u\otimes v\in U\times V$を
  \begin{align*}
    (u\otimes v)(f,g) = f(u)\cdot g(v)
  \end{align*}
  で定義する.これは双線形写像になっている.
\end{teigi}
\begin{remark}
  $U\otimes V \not \simeq \{ u\otimes v \mid u \in U, v \in V\}$である.
\end{remark}
\begin{prop}
  $\{e_1,\cdots,e_n\}$を$U$の基底,$\{f_1,\cdots,f_m\}$を$V$の基底とする.このとき,$\{e_i\otimes f_j \mid i=1,\cdots,n,j=1,\cdots,m \}$は$U\otimes V$の基底.
\end{prop}
\begin{proof}
  $\{e^1,\cdots,e^n\}$を$U^\ast$の基底,$\{f^1,\cdots,f^m\}$を$V^\ast$の基底とする.
  \begin{align*}
    \sum_{i,j} \alpha^{i,j} e_i \otimes f_j = 0
  \end{align*}
  のとき,
  \begin{align*}
    (e_i\otimes f_j)(e^\nu ,f^\mu) = e_i(e^\nu)\cdot f_j(f^\mu) 
    = \delta_i^\nu \delta_j^\mu = \begin{cases}
      1 & (i,j) = (\nu,\mu) \\
      0 & \mathrm{otherwise}
    \end{cases}
  \end{align*}
  より,
  \begin{align*}
    \left(\sum_{i,j} \alpha^{i,j} e_i \otimes f_j \right) (e^\nu,f^\mu) = 0
  \end{align*}
  よって,$\alpha^{i j} = 0$.
  また,$\Phi \colon U^\ast \times V^\ast \to K$が双線形写像のとき,
  \begin{align*}
    \Phi = \sum_{i,j} \Phi(e^i,f^j) e_i \otimes f_j
  \end{align*}
  と表すことが出来る.
\end{proof}
\begin{corr}
  $\dim U\otimes V = \dim U \times \dim V$
\end{corr}

\sukima\midashi{テンソル積の定義(II)}
\begin{align*}
  U\otimes V := \left\{ \sum_{\mbox{有限和}} \alpha_i u_i \otimes v_i \relmiddle| u_i \in U, v_i \in V, \alpha_i \in K \right\} \mathbin{/} \sim
\end{align*}
ただし$\sim$は
\begin{align*}
  (u+u')\otimes v & \sim u \otimes v + u'\otimes v \\
  u\otimes (v+v') & \sim u\otimes v + u\otimes v' \\
  \alpha(u\otimes v) & \sim (\alpha u)\otimes v \sim u \otimes (\alpha v)
\end{align*}
を満たす同値関係.

\expandafter\ifx\csname readornot\endcsname\relax
  \end{document}
\fi
 % 双対空間
\expandafter\ifx\csname readornot\endcsname\relax
  \documentclass[uplatex]{jsarticle}
  \usepackage{octopus}
  \usepackage{url}

  \renewcommand{\proofname}{\textsf{証明}}
  \renewcommand{\postpartname}{章}
  \renewcommand{\thesection}{\thepart.\arabic{section}}
  \renewcommand{\thepart}{\arabic{part}}
  \makeatletter\renewcommand{\theequation}{\thesection.\arabic{equation}}\@addtoreset{equation}{section}\makeatother

  \newcommand{\octopuspart}[1]{\newpage\part{#1}\setcounter{section}{0}\vspace{3\baselineskip}}

  \renewcommand{\restriction}[2]{\left. #1 \right|_{#2}}
  \DeclareMathOperator{\dcup}{\dot{\cup}}
  \DeclareMathOperator{\conv}{conv}
  \DeclareMathOperator{\Image}{Im}
  \DeclareMathOperator{\Kernel}{Ker}
  \begin{document}
\fi

\section{テンソル解析 1}
$V$を$\mathbb{R}$上の$n$次元ベクトル空間とし,基底を$\sets{e_{1}, \dots, e_{n}}$とする。このとき,任意の$V$の元$v$は${\displaystyle v = \sum_{i=1}^{n} v^{i}e_{i}}$と書ける。
つまり,基底の組を決めると,任意の$v \in V$は$n$次元ベクトル$(v^{1} , \dots, v^{n})^{\top}$とみなせる。つまり,$V \simeq \mathbb{R}^{n}$である。

\begin{teigi}
  \nw{座標系} $\defines$ 基底の組$(e_{\kappa} \,|\, \kappa = 1,\dots, n)$
\end{teigi}

$v \in V$に対して座標系$(e_{\kappa} \,|\, \kappa = 1,\dots, n)$による表示を$v^{\kappa}$,座標系$(e_{\kappa^{\prime}} \,|\, \kappa^{\prime} = 1^{\prime},\dots, n^{\prime})$による表示を$v^{\kappa^{\prime}}$と表すことにする。

\begin{hodai}[変換則]
  \label{lem:12-1-2}
  ${\displaystyle e_{\kappa} = \sum_{\kappa^{\prime} = 1^{\prime}}^{n^{\prime}} A_{\kappa}^{\kappa^{\prime}} e_{\kappa^{\prime}}}$なら,
  \begin{align}
    v^{\kappa^{\prime}} = \sum_{\kappa=1}^{n} A_{\kappa}^{\kappa^{\prime}} v^{\kappa} =: A_{\kappa}^{\kappa^{\prime}} v^{\kappa}.
  \end{align}
\end{hodai}
最後の和の記号$\sum$を省略する書き方を「\nw{Einsteinの縮約記法}」といい,上と下に同じ添字が出てきた場合,その文字については和をとることにする。
\begin{proof}
  $v = v^{\kappa}e_{\kappa} = v^{\kappa^{\prime}}e_{\kappa^{\prime}}$と2通りの表示を考えると,$v^{\kappa} e_{\kappa} = v^{\kappa} A_{\kappa}^{\kappa^{\prime}} e_{\kappa^{\prime}}.
  $である。これより$v^{\kappa^{\prime}} = A_{\kappa}^{\kappa^{\prime}}v^{\kappa}$を得る。
\end{proof}

\begin{teigi}
  座標系から$\mathbb{R}^{n}$への対応を\nw{(一般化された)ベクトル}という。特に,\textbf{補題\ref{lem:12-1-2}}に示された変換則に従うベクトルを\nw{反変ベクトル}という。
\end{teigi}

$V^{*}$を$V$の双対空間,つまり,$V^{*} := \sets{f \colon V \longrightarrow \mathbb{R} | f \text{は線形写像}}$とする。$V$の座標系を$(e_{\kappa})$,その双対基底を$(e^{\kappa})$とする\footnote{すなわち,$e^{\kappa}(e_{\lambda}) = \delta^{\kappa}_{\lambda}$が成り立つ。}。このとき,$f \in V^{*}$は双対基底によって$f = f_{\kappa} e^{\kappa}$と書き表すことができる。

\begin{hodai}[変換則]
  \label{lem:12-1-4}
  $V$の座標系$(e_{\kappa})$と$(e_{\kappa^{\prime}})$について,$f = f_{\kappa}e^{\kappa} = f_{\kappa^{\prime}} e^{\kappa^{\prime}}$を考える。$e_{\kappa^{\prime}} = A_{\kappa^{\prime}}^{\kappa} e_{\kappa}$であるとき,
  \begin{align}
    f_{\kappa^{\prime}} = A_{\kappa^{\prime}}^{\kappa} f_{\kappa}.
  \end{align}
\end{hodai}

\begin{proof}
  $f_{\kappa^{\prime}} = f(e_{\kappa^{\prime}}) = f(A_{\kappa^{\prime}}^{\kappa}e_{\kappa}) = A_{\kappa^{\prime}}^{\kappa} f(e_{\kappa}) = A_{\kappa^{\prime}}^{\kappa} f_{\kappa}$である。$f$が線形写像であることに注意。
\end{proof}

\begin{teigi}
  \textbf{補題\ref{lem:12-1-4}}に示された変換則に従うベクトルを\nw{共変ベクトル}という。
\end{teigi}

\begin{remark}
  $V$は反変ベクトル空間であり,$V^{*}$は共変ベクトル空間である。
\end{remark}

\begin{teigi}
  座標系に$\mathbb{R}$の元を対応させる写像であって,いかなる座標系においても同じ値に写すものを\nw{スカラー}という。
\end{teigi}

\begin{hodai}
  $v^{\kappa}$を反変ベクトル,$f_{\kappa}$を共変ベクトルとする。このとき,$f_{\kappa} v^{\kappa}$はスカラー。
\end{hodai}

\begin{proof}
  基底が$e_{\kappa} = A_{\kappa}^{\kappa^{\prime}} e_{\kappa^{\prime}}$と変換されるとする。このとき,
  \begin{align}
    f_{\kappa} v^{\kappa} = f_{\kappa} A^{\kappa}_{\kappa^{\prime}} v^{\kappa^{\prime}} = f_{\kappa^{\prime}} v^{\kappa^{\prime}}.
  \end{align}
  よって,$f_{\kappa} v^{\kappa}$は座標系によらず不変である。
\end{proof}

\sukima \midashi{物理的イメージ}\par
\noindent\textbullet\; 反変ベクトルの例\par
\qquad  
\begin{minipage}[t]{0.15\columnwidth}
  位置ベクトル,
  \begin{align*}
    x = \vect{\mathrm{OP}}
  \end{align*}
\end{minipage}
\vspace{0.05\baselineskip}
\begin{minipage}[t]{0.15\columnwidth}
  速度ベクトル,
  \begin{align*}
    \odif{x}{t}
  \end{align*}
\end{minipage}
\vspace{0.05\baselineskip}
\begin{minipage}[t]{0.15\columnwidth}
  加速度ベクトル
  \begin{align*}
    \odif{^{2}x}{t^{2}}
  \end{align*} 
\end{minipage}

\sukima\noindent\textbullet\; 共変ベクトルの例\par
\qquad  
\begin{minipage}[t]{0.3\columnwidth}
  超平面の法線ベクトル$a_{\kappa}$,
  \begin{align*}
    a_{\kappa} x^{\kappa} = 0.
  \end{align*}
\end{minipage}
\vspace{0.05\baselineskip}
\begin{minipage}[t]{0.3\columnwidth}
  勾配ベクトル,
  \begin{align*}
    \left( \pdif{U}{x^{\kappa}} \right)
  \end{align*}
  {\footnotesize
  \begin{align*}
    \pdif{U}{x^{\kappa^{\prime}}} = \pdif{x^{\kappa}}{x^{\kappa^{\prime}}} \pdif{U}{x^{\kappa}} = A_{\kappa^{\prime}}^{\kappa} \pdif{U}{x^{\kappa}}
  \end{align*}
  による。
  }
\end{minipage}
\vspace{0.05\baselineskip}
\begin{minipage}[t]{0.3\columnwidth}
  力\par
  {\footnotesize
  力が位置エネルギー$U$の勾配であること,力と変位の内積である仕事がスカラー量になることから力は共変ベクトルとみなすのが自然。
  }
\end{minipage}

\sukima\midashi{テンソル空間}\par
\noindent\textbf{復習}
\begin{align}
  U \otimes V &= \sets{\Phi \colon U^{*} \times V^{*} \longrightarrow \mathbb{R} | \Phi \text{は双線形写像}} \\
  &= \sets{ \sum_{\alpha} u_{\alpha} \otimes v_{\alpha} | u_{\alpha} \in U, v_{\alpha} \in V} / \sim
\end{align}
ただし$\sim$で定めた同値関係は
\begin{align*}
  (u+u')\otimes v & \sim u \otimes v + u'\otimes v \\
  u\otimes (v+v') & \sim u\otimes v + u\otimes v' \\
  \alpha(u\otimes v) & \sim (\alpha u)\otimes v \sim u \otimes (\alpha v).
\end{align*}

これを拡張して$\underbrace{V \otimes \dots \otimes V}_{p\text{個}}$を考える。これを\nw{$p$階反変テンソル空間}という。
$(e_{\kappa})$を$V$の座標系としたとき,$(e_{\kappa_{1}} \otimes e_{\kappa_{2}} \otimes \dots \otimes e_{\kappa_{p}} \,|\, \kappa_{1}, \dots, \kappa_{p} \in \sets{1,\dots, n})$は$\underbrace{V \otimes \dots \otimes V}_{p\text{個}}$の基底をなす。
このことから,$T \in \underbrace{V \otimes \dots \otimes V}_{p\text{個}}$は$T = T^{\kappa_{1}\kappa_{2} \dots \kappa_{p}}e_{\kappa_{1}} \otimes e_{\kappa_{2}} \otimes \dots \otimes e_{\kappa_{p}}$という表示をもつ。

\begin{hodai}[変換則]
  $(e_{\kappa}),(e_{\kappa^{\prime}})$を$V$の座標系とする。$e_{\kappa} = A_{\kappa}^{\kappa^{\prime}}e_{\kappa^{\prime}}$であるとき,
  \begin{align}
    T^{\kappa^{\prime}_{1} \kappa^{\prime}_{2} \dots \kappa^{\prime}_{p}} = A_{\kappa_{1}}^{\kappa^{\prime}_{1}} A_{\kappa_{2}}^{\kappa^{\prime}_{2}} \dots A_{\kappa_{p}}^{\kappa^{\prime}_{p}} T^{\kappa_{1}\kappa_{2}\dots\kappa_{p}}.
  \end{align}
\end{hodai}

\begin{proof}
  2つの座標系によって表すと,
  \begin{align*}
    T &= T^{\kappa_{1}\kappa_{2}\dots\kappa_{p}} e_{\kappa_{1}} \otimes e_{\kappa_{2}} \otimes \dots \otimes e_{\kappa_{p}} \\
    &= T^{\kappa_{1}\kappa_{2}\dots\kappa_{p}} A_{\kappa_{1}}^{\kappa^{\prime}_{1}} e_{\kappa^{\prime}_{1}} \otimes A_{\kappa_{2}}^{\kappa^{\prime}_{2}} e_{\kappa^{\prime}_{2}} \otimes A_{\kappa_{p}}^{\kappa^{\prime}_{p}} e_{\kappa^{\prime}_{p}} \\
    &= A_{\kappa_{1}}^{\kappa^{\prime}_{1}} A_{\kappa_{2}}^{\kappa^{\prime}_{2}} \dots A_{\kappa_{p}}^{\kappa^{\prime}_{p}} T^{\kappa_{1}\kappa_{2}\dots\kappa_{p}} e_{\kappa^{\prime}_{1}} \otimes e_{\kappa^{\prime}_{2}} \otimes \dots \otimes e_{\kappa^{\prime}_{p}}
  \end{align*}
  となることから従う。
\end{proof}

同様に$\underbrace{V^{*} \otimes \dots \otimes V^{*}}_{q\text{個}}$を考える。これを\nw{$q$階共変テンソル空間}という。
$V$の座標系$(e_{\kappa})$の双対座標系を$(e^{\lambda})$としたとき,$(e^{\lambda_{1}} \otimes e_{\lambda_{2}} \otimes \dots \otimes e_{\lambda_{q}} \,|\, \lambda_{1}, \dots, \lambda_{q} \in \sets{1,\dots, n})$は$\underbrace{V^{*} \otimes \dots \otimes V^{*}}_{q\text{個}}$の基底をなす。
このことから,$T \in \underbrace{V^{*} \otimes \dots \otimes V^{*}}_{q\text{個}}$も$T = T_{\lambda_{1}\lambda_{2} \dots \lambda_{q}}e^{\lambda_{1}} \otimes e^{\lambda_{2}} \otimes \dots \otimes e^{\lambda_{q}}$という表示をもつ。

また,$\underbrace{V \otimes \dots \otimes V}_{p\text{個}} \otimes \underbrace{V^{*} \otimes \dots \otimes V^{*}}_{q\text{個}}$を考える。$r = p+q$とするとき,これを\nw{$r$階混合テンソル空間}という。より詳しく,\nw{反変$p$価 共変$q$価}の混合テンソル空間ともいう。
やはり混合テンソル空間においても$T \in \underbrace{V \otimes \dots \otimes V}_{p\text{個}} \otimes \underbrace{V^{*} \otimes \dots \otimes V^{*}}_{q\text{個}}$は$T = T^{\kappa_{1}\kappa_{2}\dots\kappa_{p}}_{\lambda_{1}\lambda_{2} \dots \lambda_{q}} e_{\kappa_{1}} \otimes \dots \otimes e_{\kappa_{p}} \otimes e^{\lambda_{1}} \otimes \dots \otimes e^{\lambda_{q}}$という表示をもつ。

証明は反変テンソル空間のそれと同様なので省略するが,次の補題が成立する。
\begin{hodai}[変換則]
  \begin{align}
    T^{\kappa^{\prime}_{1}\kappa^{\prime}_{2}\dots\kappa^{\prime}_{p}}_{\lambda^{\prime}_{1}\lambda^{\prime}_{2} \dots \lambda^{\prime}_{q}} = A_{\kappa_{1}}^{\kappa^{\prime}_{1}} \dots A_{\kappa_{p}}^{\kappa^{\prime}_{p}} A^{\lambda_{1}}_{\lambda^{\prime}_{1}} \dots A^{\lambda_{q}}_{\lambda^{\prime}_{q}} T^{\kappa_{1}\kappa_{2}\dots\kappa_{p}}_{\lambda_{1}\lambda_{2} \dots \lambda_{q}}.
  \end{align}
\end{hodai}

以下,特に明示しないときでも座標変換の仕方は$e_{\kappa} = A^{\kappa^{\prime}}_{\kappa} e_{\kappa^{\prime}}$,$e^{\lambda} = A^{\lambda}_{\lambda^{\prime}}e^{\lambda^{\prime}}$のようになっているものとする。

\begin{hodai}
  $T^{\mu\lambda}$を2階反変テンソル,$U_{\mu\lambda}$を2階共変テンソルとすると,$T^{\kappa\mu}U_{\mu\lambda}$は反変1価,共変1価の混合テンソルである。
\end{hodai}
\begin{proof}
  変換後と比較する。
  \begin{align*}
    T^{\kappa^{\prime} \mu^{\prime}} U_{\mu^{\prime} \lambda^{\prime}} &= A_{\kappa}^{\kappa^{\prime}} A_{\mu}^{\mu^{\prime}} T^{\kappa \mu} A_{\mu^{\prime}}^{\sigma} A_{\lambda^{\prime}}^{\lambda} U_{\sigma\lambda} \\
    &= A_{\kappa}^{\kappa^{\prime}} \left( A_{\mu}^{\mu^{\prime}} A_{\mu^{\prime}}^{\sigma} \right) A_{\lambda^{\prime}}^{\lambda} T^{\kappa \mu} U_{\sigma\lambda} \\
    &= A_{\kappa}^{\kappa^{\prime}} \delta_{\mu}^{\sigma} A_{\lambda^{\prime}}^{\lambda} T^{\kappa \mu} U_{\sigma\lambda} \\
    &= A_{\kappa}^{\kappa^{\prime}} A_{\lambda^{\prime}}^{\lambda} T^{\kappa \mu} U_{\mu\lambda}
  \end{align*}
  この変換則に従うものは反変1価,共変1価の混合テンソルであった。
\end{proof}

\sukima\midashi{Newtonの運動方程式}\par
Newtonの運動方程式
\begin{align}
  m \odif{^{2}\bm{x}}{t^{2}} = \bm{f}
\end{align}
において,$\bm{x}$を反変ベクトル$x^{\kappa}$,$\bm{f}$を共変ベクトル$f_{\kappa}$とモデリングする\footnote{このようにモデリングすることの妥当性に関しては先の「物理的イメージ」の部分を参照。}。
\begin{align}
  m \odif{^{2}x^{\kappa}}{t^{2}} = f_{\kappa} \label{eq:12-1-8}
\end{align}
このモデルの下で座標系の変換を行ってみる。

いま,元の座標系は$(e_{\kappa})$であったとして,次に座標系を$(e_{\kappa^{\prime}})$に変換させたとする\footnote{このことは観測者を変えることに対応する。}。このとき$x^{\kappa} = A_{\kappa^{\prime}}^{\kappa} x^{\kappa^{\prime}}$,$f_{\kappa^{\prime}} = A_{\kappa^{\prime}}^{\kappa} f_{\kappa}$と変換するから,Newtonの運動方程式は
\begin{align*}
  m A_{\kappa^{\prime}}^{\kappa} \odif{^{2}x^{\kappa^{\prime}}}{t^{2}} = f_{\kappa}
\end{align*}
となり両辺に$A_{\lambda^{\prime}}^{\kappa}$を乗じて(和をとって)
\begin{align*}
  m \sum_{\kappa=1}^{3} A_{\lambda^{\prime}}^{\kappa} A_{\kappa^{\prime}}^{\kappa} \odif{^{2}x^{\kappa^{\prime}}}{t^{2}} = f_{\lambda^{\prime}}
\end{align*}
となる。 これはNewtonの運動方程式の形式\eqref{eq:12-1-8}と異なってしまう。しかしながら,物理法則を表す式は「可能な」座標変換の下では不変であるべきだと考えられる。

これに対して,2つの解決方法を提示する。まず1つ目の解決方法として,「可能な」座標変換を制限する方法がある。つまり,
\begin{align*}
  \sum_{\kappa=1}^{3} A_{\lambda^{\prime}}^{\kappa} A_{\kappa^{\prime}}^{\kappa} = \begin{cases}
    1, & \lambda^{\prime} = \kappa^{\prime}, \\
    0, & \text{otherwise}.
  \end{cases}
\end{align*}
なる変換だけを考える。この変換はよくみると直交変換になっていて,結局,この解決方法は物理的に自然な座標系どうしの変換のみを考えることにすぎない。

もう1つの解決方法として,質量$m$をただのスカラーとしてみるのではなく,2階の共変テンソル$m_{\kappa\lambda}$とみなす方法がある。つまり,モデリングの仕方から変えてしまう。このときNewtonの運動方程式は\eqref{eq:12-1-8}から\eqref{eq:12-1-10}に変化する。
\begin{align}
  m_{\kappa\lambda} \odif{^{2}x^{\lambda}}{t^{2}} = f_{\kappa}. \label{eq:12-1-10}
\end{align}
すると,別の座標系$(e_{\kappa^{\prime}})$でみたとき,
\begin{align}
  m_{\kappa \lambda} A^{\lambda}_{\lambda^{\prime}} A^{\kappa}_{\kappa^{\prime}} \odif{^{2} x^{\lambda^{\prime}}}{t^{2}} = f_{\kappa^{\prime}}
\end{align}
を得て,$m_{\kappa \lambda} = A_{\kappa}^{\kappa^{\prime}} A_{\lambda}^{\lambda^{\prime}} m_{\kappa^{\prime} \lambda^{\prime}}$であることから
\begin{align}
  m_{\kappa^{\prime}\lambda^{\prime}} \odif{^{2}x^{\lambda^{\prime}}}{t^{2}} = f_{\kappa^{\prime}}
\end{align}
と書き直される。これでNewtonの運動方程式が座標系によらずに同じ形式で記述されるようになった。

\sukima\midashi{問題.} その他の物理法則の不変な形について調べよ。例えば,Maxwell方程式や特殊相対論,一般相対論の式を俎上にあげてみよ。


\sukima\midashi{対称テンソル}\par
$\underbrace{V^{*} \otimes \dots \otimes V^{*}}_{q\text{個}}$を共変$q$価テンソル空間とする。

\begin{teigi}
  $T \in \underbrace{V^{*} \otimes \dots \otimes V^{*}}_{q\text{個}}$と写像$\sigma \colon \sets{1,\dots,n} \longrightarrow \sets{1,\dots,n}$に対し,$\sigma T \in \underbrace{V^{*} \otimes \dots \otimes V^{*}}_{q\text{個}}$を
  \begin{align*}
    \sigma T (v_{1}, \dots, v_{q}) := T(v_{\sigma(1)}, \dots, v_{\sigma(q)})
  \end{align*}
  で定める。
\end{teigi}

\begin{teigi}[対称テンソル]
  $T \in \underbrace{V^{*} \otimes \dots \otimes V^{*}}_{q\text{個}}$が\nw{対称}であるとは,任意の写像$\sigma \colon \sets{1,\dots,n} \longrightarrow \sets{1,\dots,n}$に対して,$\sigma T = T$であることと定義する。
\end{teigi}

対称テンソルの意味をもう少し詳しく見る。テンソル$T$を座標系$(e_{\kappa})$を通してみると,$T = T_{\kappa_{1} \dots \kappa_{q}} e^{\kappa_{1}} \otimes \dots \otimes e^{\kappa_{q}}$であったことから$T_{\kappa_{1} \dots \kappa_{q}} = T(e_{\kappa_{1}}, \dots, e_{\kappa_{q}})$である。これが$\sigma T (e_{\kappa_{1}}, \dots, e_{\kappa_{q}})$に等しいということが対称であることの定義なので,結局,$T_{\kappa_{1} \dots \kappa_{q}} = T_{\kappa_{\sigma(1)}\dots \kappa_{\sigma(q)}}$が成立する。つまり,
\begin{center}
  $T_{\kappa_{1}\dots\kappa_{q}}$が対称テンソル $\iff$ テンソルの成分は入れ替えに関して不変
\end{center}
ということ。

\begin{rei}
  $q = 2$のとき$T \in V^{*} \otimes V^{*}$が対称であるとは$T_{ij} = T_{ji}$が成立することに等しい。これは$T$が対称行列であることと同じとみなすことができる。また,対応する多重線形写像が2次形式になる。
\end{rei}


\sukima\midashi{計量テンソル}\par
\begin{teigi}[計量テンソル]
  次の3つの条件を満たすテンソル$g_{\kappa\lambda}$を\nw{計量テンソル}という。
  \begin{enumerate}
    \item 共変2価
    \item 対称
    \item 正定値,すなわち,$\forall v \neq 0, \quad g_{\kappa\lambda}v^{\kappa}v^{\lambda} > 0.$
  \end{enumerate}
\end{teigi}

\begin{teigi}
  計量テンソル$g_{\kappa\lambda}$の逆行列を$g^{\kappa\lambda}$で表す。
\end{teigi}

\begin{remark}
  計量には反変量と共変量を結びつける働きがある。
  \begin{align}
    v_{\lambda} &= g_{\kappa \lambda} v^{\kappa}, \\
    v^{\lambda} &= g^{\kappa \lambda} v_{\kappa}.
  \end{align}
\end{remark}

\sukima\midashi{反変ベクトル対の内積}\par
\begin{teigi}
  \begin{enumerate}
    \item[(1)] $u,v$を反変ベクトル空間$V$の元とする。このとき,\begin{align}
      \langle u,v \rangle := g_{\kappa\lambda} u^{\kappa} v^{\lambda}
    \end{align}
    で定まるスカラー量を$u$と$v$の\nw{内積}と定める。
    \item[(2)] 反変ベクトル$v$の\nw{長さ}$\left\| v \right\|$を
    \begin{align}
      \left\| v \right\| := \sqrt{\langle v, v \rangle} = \sqrt{g_{\kappa\lambda} v^{\kappa} v^{\lambda}}
    \end{align}
    で定める。
    \item[(3)] 2つの反変ベクトル$u,v \in V$の間の\nw{角度}$\theta$を
    \begin{align}
      \theta := \arccos \frac{\langle u,v \rangle}{\left\| u \right\| \left\| v \right\|} = \arccos \frac{g_{\kappa\lambda} u^{\kappa}v^{\lambda}}{\sqrt{g_{\kappa\lambda} u^{\kappa}u^{\lambda}}\sqrt{g_{\kappa\lambda} v^{\kappa}v^{\lambda}}}
    \end{align}
    で定める。
  \end{enumerate}
\end{teigi}


\sukima\midashi{最適化アルゴリズムの不変性}\par
ここでは目的関数$f \colon \mathbb{R}^{n} \longrightarrow \mathbb{R}$の最小化を行うアルゴリズムである最急降下法とNewton法についてテンソルの観点からそれぞれの手法の善し悪しを評する。

最急降下法のステップはパラメータ$\lambda$を用いて
\begin{align}
  \bm{x} \longleftarrow \bm{x} + \lambda \begin{pmatrix}
    \pdif{f}{x^{1}} \\ \pdif{f}{x^{2}} \\ \vdots \\ \pdif{f}{x^{n}}
  \end{pmatrix}
\end{align}
と表される。これをテンソルで書き直すと
\begin{align}
  x^{\kappa} \longleftarrow x^{\kappa} + \lambda \pdif{f}{x^{\kappa}} \label{eq:12-1-19}
\end{align}
である。$x^{\kappa}$は反変量であるが,$\pdif{f}{x^{\kappa}}$は共変量であるため,更新式の右辺は座標変換に関して不変ではない。

一方,式\eqref{eq:12-1-19}の$\lambda$を反変計量テンソル$g^{\kappa\lambda}$に変えて更新式を
\begin{align}
  x^{\kappa} \longleftarrow x^{\kappa} + g^{\kappa\lambda} \pdif{f}{x^{\lambda}} \label{eq:12-1-20}
\end{align}
とすれば,右辺は全体として反変1価となり座標変換に関して不変になる。
この$g^{\kappa\lambda}$として具体的に$\left( \pdif{f}{x^{\kappa}x^{\lambda}}\right)^{-1}$にしたのがNewton法であり,したがって,Newton法は座標変換に関して不変であることが分かる。

\sukima\midashi{問題.} その他の最適化アルゴリズムの不変性について調べなさい。例えば,共役勾配法,内点法,準Newton法などを考えよ。
  
\expandafter\ifx\csname readornot\endcsname\relax
  \end{document}
\fi % テンソルの定義
\expandafter\ifx\csname readornot\endcsname\relax
  \documentclass[uplatex]{jsarticle}
  \usepackage{octopus}
  \usepackage{url}

  \renewcommand{\proofname}{\textsf{証明}}
  \renewcommand{\postpartname}{章}
  \renewcommand{\thesection}{\thepart.\arabic{section}}
  \renewcommand{\thepart}{\arabic{part}}
  \makeatletter\renewcommand{\theequation}{\thesection.\arabic{equation}}\@addtoreset{equation}{section}\makeatother

  \newcommand{\octopuspart}[1]{\newpage\part{#1}\setcounter{section}{0}\vspace{3\baselineskip}}

  \renewcommand{\restriction}[2]{\left. #1 \right|_{#2}}
  \DeclareMathOperator{\dcup}{\dot{\cup}}
  \DeclareMathOperator{\conv}{conv}
  \DeclareMathOperator{\Image}{Im}
  \DeclareMathOperator{\Kernel}{Ker}
  \DeclareMathOperator{\sgn}{sgn}
  \begin{document}
\fi

\section{テンソル解析 2}
\midashi{交代テンソル}\par
$\underbrace{V^{*} \otimes \dots \otimes V^{*}}_{q\text{個}}$を共変$q$価テンソル空間とする。
$T \in \underbrace{V^{*} \otimes \dots \otimes V^{*}}_{q\text{個}}$と写像$\sigma \colon \sets{1,\dots,n} \longrightarrow \sets{1,\dots,n}$に対し,$\sigma T \in \underbrace{V^{*} \otimes \dots \otimes V^{*}}_{q\text{個}}$を$\sigma T (v_{1}, \dots, v_{q}) := T(v_{\sigma(1)},\dots, v_{\sigma(q)})$で定めた。

\begin{teigi}[交代テンソル]
  $T \in \underbrace{V^{*} \otimes \dots \otimes V^{*}}_{q\text{個}}$が\nw{交代的}であるとは,任意の写像$\sigma \colon \sets{1,\dots,n} \longrightarrow \sets{1,\dots,n}$に対して,$\sigma T = \sgn (\sigma) T$であることと定義する。
\end{teigi}

対称テンソルのときと同様に交代テンソルの意味するところを座標系を通してみてみる。$T = T_{\kappa_{1} \dots \kappa_{q}} e^{\kappa_{1}} \otimes \dots \otimes e^{\kappa_{q}}$であったことから$T_{\kappa_{1} \dots \kappa_{q}} = T(e_{\kappa_{1}}, \dots, e_{\kappa_{q}})$である。これが$\sgn (\sigma) \sigma T (e_{\kappa_{1}}, \dots, e_{\kappa_{q}})$に等しいということが交代的であることの定義なので,結局,$T_{\kappa_{1} \dots \kappa_{q}} = \sgn(\sigma) T_{\kappa_{\sigma(1)}\dots \kappa_{\sigma(q)}}$が成立する。つまり,
\begin{center}
  $T_{\kappa_{1}\dots\kappa_{q}}$が交代テンソル $\iff$ テンソルの成分は互換した回数だけ符号が反転
\end{center}
ということ。

\begin{rei}
  $q = 2$のとき$T \in V^{*} \otimes V^{*}$が交代的であるとは$T_{ij} = -T_{ji}$が成立することに等しい。これは$T$が反対称行列であることと同じとみなすことができる。
\end{rei}

\begin{remark}
  相異なる$i,j$が存在して,$\kappa_{i} = \kappa_{j}$となるとき,$T_{\kappa_{1}\dots\kappa_{q}} = 0$である。
\end{remark}

\begin{hodai}
  $u \wedge v := \dfrac{u \otimes v - v \otimes u}{2}$は交代2価共変テンソルである。
  一般に
  \begin{align}
    u^{(1)} \wedge u^{(2)} \wedge \cdots \wedge u^{(p)}:=\dfrac{1}{p!} \sum_{\sigma} \sgn (\sigma) u^{(\sigma(1))} \otimes u^{(\sigma(2))} \otimes \cdots \otimes u^{(\sigma(p))}
  \end{align}
  は交代$p$価共変テンソルである。
\end{hodai}
\begin{proof}
  $p = 2$のときだけ示す。
  \begin{align*}
    (u \wedge v)(x,y) &= \frac{u \otimes v (x,y) - v \otimes u (x,y)}{2} \\
    &= \frac{u(x)v(y) - v(x)u(y)}{2} \\
    &= \frac{-u \otimes v (y,x) + v \otimes u (y,x)}{2} \\
    &= - (u \wedge v)(y,x).
  \end{align*}
\end{proof}

座標系で見ると,$(u \wedge v)_{\kappa\lambda} = \dfrac{u_{\kappa}v_{\lambda} - u_{\lambda}v_{\kappa}}{2}$となる。

\begin{hodai}
  $u,v,u^{\prime}, v^{\prime} \in V^{*}$,$\alpha, \beta \in \mathbb{R}$とする。このとき,次の4つが成立する。
  \begin{enumerate}
    \item[(1)] $u \wedge v = - v \wedge u$.
    \item[(2)] $u \wedge u = 0$.
    \item[(3)] $u \wedge (\alpha v + \beta v^{\prime}) = \alpha u \wedge v + \beta u \wedge v^{\prime}$.
    \item[(4)] $(\alpha u + \beta u^{\prime}) \wedge v = \alpha u \wedge v + \beta u^{\prime} \wedge v$.   
  \end{enumerate}
\end{hodai}
\begin{proof}
  (1)については定義$u \wedge v := \dfrac{u \otimes v - v \otimes u}{2}$より明らか。
  (2)については(1)の$v$を$u$に置き換えれば得られる。
  (3)と(4)は$\otimes$が線型性を保つことから従う。
\end{proof}

\begin{teigi}[外冪]
  交代$p$価共変テンソルからなる空間を
  \begin{align}
    \wedge^{p} V^{*} := \underbrace{V^{*} \wedge \dots \wedge V^{*}}_{p\text{個}}
  \end{align}
  と書く。この空間の元を\nw{共変$p$-ベクトル}という。
\end{teigi}

\begin{prop}
  $\sets{e^{i} | i = 1,\dots, n}$を$V^{*}$の基底とする。このとき,$\sets{e^{i_{1}} \wedge e^{i_{2}} \wedge \dots \wedge e^{i_{p}} | 1 \le i_{1} < i_{2} < \dots < i_{p} \le n}$は$\wedge^{p} V^{*}$の基底をなす。したがって,$\dim \wedge^{p} V^{*} = \binom{n}{p}$である。
\end{prop}
\begin{proof}
  あとで
\end{proof}

\begin{hodai}
  $u^{(1)} \wedge u^{(2)} \wedge \dots \wedge u^{(n)} = \det [u_{k}^{(i)}] e^{1} \wedge e^{2} \wedge \dots \wedge e^{n}$.
\end{hodai}
\begin{proof}
  あとで
\end{proof}

\begin{rei}
  $n = 3$とすると
  \begin{align*}
    u \wedge v &= (u_{1} e^{1} + u_{2} e^{2} + u_{3} e^{3}) \wedge (v_{1} e^{1} + v_{2} e^{2} + v_{3} e^{3}) \\
    &= (u_{2}v_{3} - u_{3}v_{2}) e^{2} \wedge e^{3} + (u_{3}v_{1} - u_{1}v_{3}) e^{3} \wedge e^{1} + (u_{1}v_{2} - u_{2}v_{1}) e^{1} \wedge e^{2}.
  \end{align*}
\end{rei}

\sukima\midashi{テンソル密度,擬テンソル密度}\par
これまで扱ってきたテンソルは,$V^{*} \otimes V$など,$V$や$V^{*}$から構成される空間の元であった。これらは座標表示しなくても議論ができる。
しかしながら,既に$V$や$V^{*}$からは構成されない「ベクトル」を扱っている。例えば3次元ベクトルの外積や$3 \times 3$行列の行列式など。







\expandafter\ifx\csname readornot\endcsname\relax
  \end{document}
\fi % 反変性・共変性・スカラー・混合テンソル
\input{sub-resume-14} % 交代テンソル・擬テンソル・テンソル密度・Eddintonのε

\end{document}
