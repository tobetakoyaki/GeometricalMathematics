\expandafter\ifx\csname readornot\endcsname\relax
  \documentclass[uplatex]{jsarticle}    
  \usepackage{octopus}
  \usepackage{url}
  %%%% コマンド定義専用のtexファイル
\renewcommand{\postpartname}{章}
\renewcommand{\thepart}{\arabic{part}}
\renewcommand{\thesection}{\thepart.\arabic{section}}
\makeatletter\renewcommand{\theequation}{\thesection.\arabic{equation}}\@addtoreset{equation}{section}\makeatother
\newcommand{\octopuspart}[1]{\newpage\part{#1}\setcounter{section}{0}\vspace{3\baselineskip}}
\renewcommand{\restriction}[2]{\left. #1 \right|_{#2}}
\DeclareMathOperator{\dcup}{\dot{\cup}}
\DeclareMathOperator{\conv}{conv}
\DeclareMathOperator{\Image}{Im}
\DeclareMathOperator{\Kernel}{Ker}
\DeclareMathOperator{\diag}{diag}
\DeclareMathOperator{\rank}{rank}
\DeclareMathOperator{\sgn}{sgn}
\DeclareMathOperator{\rot}{rot}
  \usetikzlibrary{calc}
  \begin{document}
\fi

\section{コンパクト性}
$(X, \mathcal{O})$:位相空間,$A \subseteq X$とする。

\begin{teigi}[被覆]
  $\mathcal{C} \subseteq 2^{X}$:$A$の\nw{被覆} $\defines$ ${\displaystyle A \subseteq \bigcup_{C \in \mathcal{C}} C}$

  特に,$\mathcal{C} \subseteq \mathcal{O}$のとき,$\mathcal{C}$を\nw{開被覆}という。
\end{teigi}

\begin{teigi}[コンパクト]
  \midashi{(1)} $A \subseteq X$:\nw{コンパクト} $\defines$ $\forall \mathcal{C}$:$A$の開被覆,$\exists O_{1}, \dots, O_{k} \in \mathcal{C}$,${\displaystyle A \subseteq \bigcup_{j=1}^{k} O_{j}}$

  \midashi{(2)} $X$がコンパクトのとき,$(X, \mathcal{O})$をコンパクト空間という。
\end{teigi}

$A \subseteq X$がコンパクトであることは,標語的に「任意の開被覆は有限部分開被覆を含む」ということができる。
また,$A \subseteq X$がコンパクトであることは,$(A, \mathcal{O}_{A})$がコンパクト空間であることと同値である。
ただし,$\mathcal{O}_{A}$は相対位相。

\begin{hodai}
  $A_{1}, \dots, A_{k} \subseteq X$ :コンパクト $\Longrightarrow$ $A_{1} \cup \dots \cup A_{k}$:コンパクト
\end{hodai}

\begin{hodai}
  $(X, \mathcal{O})$:コンパクト $\Longrightarrow$ $A \in \mathfrak{A}$:コンパクト
\end{hodai}

\begin{proof}
  $A$の開被覆と$X \setminus A$が$X$の開被覆になっていることから従う。
\end{proof}

\begin{hodai}
  $X$:コンパクト,$f \colon X \longrightarrow Y$:連続 $\Longrightarrow$ $f(X)$:コンパクト
\end{hodai}

\begin{proof}
  $\mathcal{C}$:$f(X)$の開被覆とすると,$f^{-1}(\mathcal{C}) := \sets{f^{-1} (O) | O \in \mathcal{C}}$は$f$の連続性から$f^{-1}(\mathcal{C}) \subseteq \mathcal{O}$を満たし,
  これは$X$の開被覆である。
  $X$のコンパクト性から,ある$O_{1}, \dots, O_{k} \in \mathcal{C}$が存在して,
  $f^{-1}(O_{1}) \cap \dots \cap f^{-1}(O_{k}) = X$を満たす。
  このとき,$f(X) \subseteq O_{1} \cap \dots \cap O_{k}$である。
\end{proof}

\begin{hodai}
  $(X, d)$:距離空間とする。
  $A \subseteq X$:コンパクト $\Longrightarrow$ $A$は有界な閉集合
\end{hodai}

\begin{proof}
  \midashi{[有界性]} $x \in X$とする。開集合族$\sets{N(x,r) | r \in \mathbb{N}}$は$A$の被覆である。
  したがってコンパクト性から有限個の$r_{1} < r_{2} < \dots < r_{k}$が存在して$\sets{N(x,r_{i})}_{i=1}^{k}$は$A$の開被覆である。
  よって,$A \subseteq N(x,r_{k})$であり,$A$は有界。

  \midashi{[閉性]} $X \setminus A$が開集合であることを示す。$y \in X \setminus A$とする。
  ${\displaystyle \sets{X \setminus \overline{N \left( y, \frac{1}{r} \right)} | r \in \mathbb{N}}}$は$A$の開被覆である。
  したがってコンパクト性から有限個の$r_{1} < r_{2} < \dots < r_{k}$が存在して${\displaystyle \sets{X \setminus \overline{N \left( y,\frac{1}{r_{i}} \right)}}_{i=1}^{k}}$は$A$の開被覆である。
  よって,${\displaystyle A \subseteq X \setminus \overline{N \left( y,\frac{1}{r_{1}} \right)}}$であり,
  ${\displaystyle N \left( y, \frac{1}{r_{1}} \right) \subseteq X \setminus A}$であるから,$X \setminus A$は開集合である。
\end{proof}

\begin{corr}
  $X$:コンパクト,$f \colon X \longrightarrow \mathbb{R}$:連続とする。
  このとき,${\displaystyle \max_{x \in X} f(x)}$,${\displaystyle \min_{x \in X} f(x)}$が存在する。
\end{corr}

\begin{proof}
  $f(X)$はコンパクトであり有界閉集合であるから。
\end{proof}

\midashi{演習} 有界閉集合がコンパクトにならない距離空間の例を挙げよ。

\begin{hodai}[Lebesgue数の補題]
  $(X,d)$:コンパクトな距離空間,$\mathcal{C}$:$X$の開被覆とする。このとき
  \begin{equation}
    \exists \delta > 0, \quad \forall x \in X, \quad \exists O \in \mathcal{C}, \quad N(x, \delta) \subseteq O
  \end{equation}
  を満たす。
\end{hodai}

\begin{proof}
  $X$のコンパクト性から有限な部分開被覆$\sets{O_{i}}_{i=1}^{k} \subseteq \mathcal{C}$が存在する。
  \begin{equation}
    f(x) := \frac{1}{k} \sum_{i=1}^{k} d(x, X \setminus O_{i})
  \end{equation}
  とすると$f$は連続である。被覆性から常に$f(x) > 0$である。
  すなわち${\displaystyle \delta := \min_{x \in X} f(x) > 0}$である。よって,$\forall x \in X, \quad f(x) \ge \delta$である。
  これより,すべての$x \in X$に対してある$i \in \sets{1,\dots,k}$が存在して$d(x, X \setminus O_{i}) \ge \delta$である。
  よって,$N(x, \delta) \subseteq O_{i}$である。
\end{proof}

\begin{hodai}[Heine-Borelの被覆定理]
  $\mathbb{R}$の閉区間$I:=[a,b] \subseteq \mathbb{R}$はコンパクト
\end{hodai}

\begin{proof}
  $\mathcal{C}$:$[a,b]$の開被覆,$I':= \sets{x \in I | \mathcal{C}\text{は}[a,x]\text{の有限開被覆を含む}}$とする。
  明らかに$a \in I'$であるから$I' \neq \emptyset$である。
  $I' \subseteq I = [a,b]$であるから$c := \sup I' \le b$である。
  $\exists C_{0} \in \mathcal{C}, \quad c \in C_{0}$であり$\exists \varepsilon > 0, \quad [c-\varepsilon, c+\varepsilon] \subseteq C_{0}$である。
  $c = \sup I'$であるから,ある有限開被覆$\mathcal{C}'$が存在して$\mathcal{C}' \subseteq \mathcal{C}$であり${\displaystyle [a,c-\varepsilon] \subseteq \bigcup_{C' \in \mathcal{C}'} C'}$である。
  このとき,${\displaystyle [a,c+\varepsilon] \subseteq C_{0} \cup \bigcup_{C' \in \mathcal{C}'} C'}$である。
  これより$[a,c+\varepsilon]$には有限部分開被覆が存在する。以上より$c = b$となる。
\end{proof}

\begin{prop}
  $X,Y$:コンパクト $\Longrightarrow$ $X \times Y$:コンパクト
\end{prop}

\begin{proof}
  直積位相$\mathcal{O}_{X \times Y} = \sets{\bigcup B | B \subseteq \mathcal{O}_{X} \times \mathcal{O}_{Y}}$を考える。
  $\mathcal{C}$:$X \times Y$の開被覆とすると,
  \begin{equation}
    \mathcal{C} = \sets{\bigcup B_{\lambda} | B_{\lambda} \subseteq \mathcal{O}_{X} \times \mathcal{O}_{Y}} = \sets{U_{\lambda} \times V_{\lambda} | U_{\lambda} \in \mathcal{O}_{X}, V_{\lambda} \in \mathcal{O}_{Y}, \lambda \in \Lambda}
  \end{equation}
  としてよい。すべての$x \in X$に対して$\sets{x} \times Y$は$Y$と同相であり,これはコンパクトである。
  よって$\exists \Lambda(x)$:$\Lambda$の有限部分集合,${\displaystyle \sets{x} \times Y \subseteq \bigcup_{\lambda \in \Lambda(x)} U_{\lambda} \times V_{\lambda}}$となる。
  また${\displaystyle U_{x} := \bigcap_{\lambda \in \Lambda(x)} U_{\lambda} \in \mathcal{O}_{X}}$であり,${\displaystyle U_{x} \times Y \subseteq \bigcup_{\lambda \in \Lambda(x)} U_{\lambda} \times V_{\lambda}}$である。
  すると$\sets{U_{x}}_{x \in X}$は$X$の開被覆である。
  $X$のコンパクト性から有限個の$x_{1},\dots,x_{n}$が存在して$U_{x_{1}} \cup \dots \cup U_{x_{n}} = X$である。
  よって,${\displaystyle X \times Y = \bigcup_{k=1}^{n} U_{x_{k}} \times Y = \bigcup_{k=1}^{n} \bigcup_{\lambda \in \Lambda(x_{k})} U_{\lambda} \times V_{\lambda}}$であって,$X \times Y$はコンパクトである。
\end{proof}

\begin{teiri}[Tychonoffの定理]
  $X_{\lambda}$:コンパクト($\lambda \in \Lambda$) $\Longrightarrow$ ${\displaystyle \prod_{\lambda} X_{\lambda}}$:コンパクト
\end{teiri}

\begin{prop}
  $X \subseteq \mathbb{R}^{n}$に対して次が成り立つ。
  \begin{center}
    $X$:コンパクト $\Longleftrightarrow$ $X$:有界閉集合
  \end{center}
\end{prop}

\begin{proof}
  $\Longrightarrow$向きは既に示した。

  $\Longrightarrow$向きについて示す。
  $X \subseteq [a_{1}, b_{1}] \times [a_{2}, b_{2}] \times \dots \times [a_{n}, b_{n}] \subseteq \mathbb{R}^{n}$と書くことができる。
  各$[a_{i},b_{i}]$はコンパクトであり,その直積$[a_{1}, b_{1}] \times [a_{2}, b_{2}] \times \dots \times [a_{n}, b_{n}]$もコンパクトである。
  コンパクト集合の部分閉集合はコンパクトであるから$X$はコンパクトである。
\end{proof}

\sukima \midashi{\large コンパクトハウスドルフ空間}

\begin{prop}
  $(X, \mathcal{O})$:Hausdorffとする。
  $A \subseteq X$:コンパクト,$x \in X \setminus A$とすると,
  \begin{equation}
    \exists U, V \in \mathcal{O}, \quad A \subseteq U, \quad x \in V, \quad U \cap V = \emptyset
  \end{equation}
\end{prop}

\begin{proof}
  $X$のHausdorff性から,任意の$a \in A$に対して
  \begin{equation}
    \exists U_{a}, V_{a} \in \mathcal{O}, \quad a \in U_{a}, \quad x \in V_{a}, \quad U_{a} \cap V_{a} = \emptyset
  \end{equation}
  である。$\sets{U_{a} | a \in A}$は$A$の開被覆であるが,コンパクト性からある$a_{1}, \dots, a_{k} \in A$が存在して
  $U := U_{a_{1}} \cup \dots \cup U_{a_{k}} \supseteq A$である。
  $x \in V_{a} \in \mathcal{O}$であるから$i=1, \dots, k$に対して$x \notin \overline{U_{a_{i}}}$である。

  $\overline{U} = \overline{U_{a_{1}}} \cup \overline{U_{a_{2}}} \cup \dots \cup \overline{U_{a_{k}}}$であり,
  $V := X \setminus \overline{U}$とおくとこれは開集合であり,かつ$U \cap V = \emptyset$である。また,$x \in V$である。 
\end{proof}

\begin{corr}
  $X$:Hausdorff,$A \subseteq X$:コンパクトとする。このとき$A$は閉集合である。
\end{corr}

\begin{proof}
  $X \setminus A$の各点$x$は$V \subseteq X \setminus A$となる開近傍をもつので。
\end{proof}

\begin{corr}
  $X$:コンパクト,$Y$:Hausdorff,$f \colon X \longrightarrow Y$:連続全単射とする。このとき,$f$:同相写像。
\end{corr}

\begin{proof}
  逆写像の連続性を示すため,「任意の$X$上の閉集合$A$に対して$f(A)$が$Y$上の閉集合であること」を示す。
  コンパクト空間上の閉集合である$A$はコンパクトであり,コンパクト集合の連続写像による像である$f(A)$はコンパクトである。
  Hausdorff空間上のコンパクト集合は閉集合であるから$f(A)$は$Y$上閉である。
\end{proof}

\begin{teigi}[局所コンパクト]
  $X$:\nw{局所コンパクト}  $\defines$ 任意の$x \in X$に対してある$x$のコンパクトな近傍が存在する
\end{teigi}

\begin{prop}
  コンパクト $\Longrightarrow$ 局所コンパクト
\end{prop}

\begin{proof}
  それ自身をコンパクトな近傍としてとればよい。
\end{proof}

\sukima \midashi{\large 閉曲面}

\begin{teigi}[閉曲面]
  \nw{閉曲面} $\defines$ コンパクトな2次元多様体
\end{teigi}

$S$:閉曲面 $\Longrightarrow$ 有限個の2次元の座標近傍がとれて,これらで$S$を覆い尽くせる。

\begin{rei}
  \begin{itemize}
    \item 球面$S^{2}$(有界閉集合なのでコンパクト)
    \item トーラス$S^{1} \times S^{1}$(コンパクト空間の直積)
    \item 射影平面$P^{2}$($S^{2}/\sim$なので$S^{2}$の連続像)
  \end{itemize}
\end{rei}

\midashi{連結な閉曲面とはどんなものか}
\begin{itemize}
  \vspace{-0.5\baselineskip}
  \item 有限個の多角形をその辺に沿って貼り合わせて得られる。
  \item 1つの辺はちょうど2つの多角形の共通辺。($\leftarrow$多様体の定義から)
\end{itemize}

\begin{rei}
  % なんか楽しそうな絵がある
\end{rei}

\sukima \midashi{\large 分類定理}

(連結な閉曲面) $\simeq$ (穴が$k$個の浮き輪) $\simeq$ (射影平面を$k'$個とりつけたもの)

穴の個数$k$を\nw{種数(genus)}という。

\begin{rei}[種数]
  球の種数は0,トーラスの種数は1である。
\end{rei}

\midashi{演習}これを証明せよ。

\expandafter\ifx\csname readornot\endcsname\relax
  \end{document}
\fi