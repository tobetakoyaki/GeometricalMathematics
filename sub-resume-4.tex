\expandafter\ifx\csname readornot\endcsname\relax
  \documentclass[uplatex]{jsarticle}
  \usepackage{octopus}
  \usepackage{url}

  \renewcommand{\proofname}{\textsf{証明}}
  \renewcommand{\postpartname}{章}
  \renewcommand{\thesection}{\thepart.\arabic{section}}
  \renewcommand{\thepart}{\arabic{part}}
  \makeatletter\renewcommand{\theequation}{\thesection.\arabic{equation}}\@addtoreset{equation}{section}\makeatother

  \newcommand{\octopuspart}[1]{\newpage\part{#1}\setcounter{section}{0}\vspace{3\baselineskip}}

  \DeclareMathOperator{\dcup}{\dot{\cup}}
  \begin{document}
\fi

\section{コンパクト性}
$(X, \mathcal{O})$:位相空間,$A \subseteq X$とする。

\begin{teigi}[被覆]
  $\mathcal{C} \subseteq 2^{X}$:$A$の\nw{被覆} $\defines$ ${\displaystyle A \subseteq \bigcup_{C \in \mathcal{C}} C}$

  特に,$\mathcal{C} \subseteq \mathcal{O}$のとき,$\mathcal{C}$を\nw{開被覆}という。
\end{teigi}

\begin{teigi}[コンパクト]
  \midashi{(1)} $A \subseteq X$:\nw{コンパクト} $\defines$ $\forall \mathcal{C}$:$A$の開被覆,$\exists O_{1}, \dots, O_{k} \in \mathcal{C}$,${\displaystyle A \subseteq \bigcup_{j=1}^{k} O_{j}}$

  \midashi{(2)} $X$がコンパクトのとき,$(X, \mathcal{O})$をコンパクト空間という。
\end{teigi}

$A \subseteq X$がコンパクトであることは,標語的に「任意の開被覆は有限部分開被覆を含む」ということができる。
また,$A \subseteq X$がコンパクトであることは,$(A, \mathcal{O}_{A})$がコンパクト空間であることと同値である。
ただし,$\mathcal{O}_{A}$は相対位相。

\begin{hodai}
  $A_{1}, \dots, A_{k} \subseteq X$ :コンパクト $\Longrightarrow$ $A_{1} \cup \dots \cup A_{k}$:コンパクト
\end{hodai}

\begin{hodai}
  $(X, \mathcal{O})$:コンパクト $\Longrightarrow$ $A \in \mathfrak{A}$:コンパクト
\end{hodai}

\begin{proof}
  $A$の開被覆と$X \setminus A$が$X$の開被覆になっていることから従う。
\end{proof}

\expandafter\ifx\csname readornot\endcsname\relax
  \end{document}
\fi