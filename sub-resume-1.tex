\expandafter\ifx\csname readornot\endcsname\relax
    \documentclass[uplatex]{jsarticle}    
    \usepackage{octopus}
    \usepackage{url}
    %%%% コマンド定義専用のtexファイル
\renewcommand{\postpartname}{章}
\renewcommand{\thepart}{\arabic{part}}
\renewcommand{\thesection}{\thepart.\arabic{section}}
\makeatletter\renewcommand{\theequation}{\thesection.\arabic{equation}}\@addtoreset{equation}{section}\makeatother
\newcommand{\octopuspart}[1]{\newpage\part{#1}\setcounter{section}{0}\vspace{3\baselineskip}}
\renewcommand{\restriction}[2]{\left. #1 \right|_{#2}}
\DeclareMathOperator{\dcup}{\dot{\cup}}
\DeclareMathOperator{\conv}{conv}
\DeclareMathOperator{\Image}{Im}
\DeclareMathOperator{\Kernel}{Ker}
\DeclareMathOperator{\diag}{diag}
\DeclareMathOperator{\rank}{rank}
\DeclareMathOperator{\sgn}{sgn}
\DeclareMathOperator{\rot}{rot}
    \usetikzlibrary{calc}
    \begin{document}
\fi

\section{距離空間}
\subsection{距離空間}
\begin{teigi}[距離空間]
    $X$を非空な集合,$d \colon X \times X \to \mathbb{R}$を実数値関数とする.次の3つの条件\textbf{D1},\textbf{D2},\textbf{D3}を考える.
    
    \textbf{D1. } $\forall x, y \in X, \quad d(x,y) \ge 0$,\qquad $d(x,y) = 0 \iff x = y$\par
    \textbf{D2. } $\forall x, y \in X, \quad d(x,y) = d(y,x)$\par
    \textbf{D3. } $\forall x, y \in X, \quad d(x,y) + d(y,z) \ge d(x,z)$
    
    $d$が\textbf{D1},\textbf{D2},\textbf{D3}の条件を満たすとき,$d$を$X$上の\nw{距離関数}という.また,$X$あるいは$(X,d)$を\nw{距離空間(metric space)}という.
\end{teigi}

\begin{remark}~\vspace{-0.5\baselineskip}
    \begin{itemize}
        \item \textbf{D2.}は対称性を表し,\textbf{D3.}は三角不等式と呼ばれる.
        \item \textbf{D1.},\textbf{D2.},\textbf{D3.}を満たさないような関数として,例えば${d_{2}}^{2}$(2乗距離)がある.
        これは三角不等式を満たさない.
    \end{itemize}
\end{remark}

\begin{rei}
    $X = \mathbb{R}^{n}$とする.
    \begin{align}
        d_{2} (x,y) &:= \sqrt{\sum_{i=1}^{n} (x_{i} - y_{i})^{2}}, \\
        d_{1} (x,y) &:= \sum_{i=1}^{n} |x_{i} - y_{i}|, \\
        d_{\infty} (x,y) &:= \max_{i} |x_{i} - y_{i}|
    \end{align}
    とすると,これらはどれも$X$上の距離関数である.なお,$(\mathbb{R}^{n}, d_{2})$を\nw{$n$次元Euclid空間}と呼ぶ.
\end{rei}
\begin{proof}
    ここでは$d_{2}$が\textbf{D3.}の三角不等式を満たすことのみを示す.
    \begin{align*}
        {d_{2}}^{2}(x,z)
        &= \sum_{i=1}^{n} (x_{i} - z_{i})^{2} = \sum_{i=1}^{n} (x_{i} - y_{i} + y_{i} - z_{i})^{2} \notag \\
        &= \sum_{i=1}^{n} \left( x_{i} - y_{i} \right)^{2} + \sum_{i=1}^{n} \left( y_{i} - z_{i} \right)^{2} + 2 \sum_{i=1}^{n} \left( x_{i} - y_{i} \right) \left( y_{i} - z_{i} \right) \notag \\
      &\le \sum_{i=1}^{n} \left( x_{i} - y_{i} \right)^{2} + \sum_{i=1}^{n} \left( y_{i} - z_{i} \right)^{2} + 2 \sqrt{\left( \sum_{i=1}^{n} \left( x_{i} - y_{i} \right)^{2} \right) \left( \sum_{i=1}^{n} \left( y_{i} - z_{i} \right)^{2} \right)} \\
        &= \left\{ \sqrt{\sum_{i=1}^{n} \left( x_{i} - y_{i} \right)^{2}} + \sqrt{\sum_{i=1}^{n} \left( y_{i} - z_{i} \right)^{2}} \right\}^{2}
         = \left( d_{2}(x,y) + d_{2} (y,z) \right)^{2}
    \end{align*}
    である.ここで,不等号はCauchy-Schwarzの不等式による.
\end{proof}

\begin{rei}
    $X = \mathcal{C} [a,b]$を区間$[a,b] \subseteq \mathbb{R}$上の連続関数全体の集合とする.
    \begin{align}
        d_{2} (f,g) &:= \sqrt{ \int_{a}^{b} | f(t) - g(t) | ^{2} \dx{t}}, \\
        d_{\infty} (f,g) &:= \sup \sets{ \left| f(t) - g(t) \right| | t \in [a,b]}, \\
        d_{1} (f,g) &:= \int_{a}^{b} \left| f(t) - g(t) \right| \dx{t}
    \end{align}
    とすると,これらはどれも$X$上の距離関数である.
\end{rei}

\begin{proof}
    ここでは$d_{1}$が\textbf{D3.}の三角不等式を満たすことと\textbf{D1.}の零点が対角集合に限られることのみを示す.\par
    \midashi{[D3.について]}
    \begin{align*}
        d_{1} (f,h)
        &= \int_{a}^{b} \left| f(t) - h(t) \right| \dx{t} \\
      &\le \int_{a}^{b} \left( \left| f(t) - g(t) \right| + \left| g(t) - h(t) \right| \right) \dx{t} \\
      &\le \int_{a}^{b} \left| f(t) - g(t) \right| \dx{t} +  \int_{a}^{b} \left| g(t) - h(t) \right| \dx{t} \\
        &= d_{1} (f,g) + d_{1} (g,h).
    \end{align*}
    \midashi{[零点が対角集合に限られることについて]}\par
    $f = g$ならば$d_{1} (f,g) = 0$は明らかであるので,問題はその逆である.

    もし,$f(x) \neq g(x)$ならばある$x_{0} \in [a,b]$で$\left| f(x_{0}) - g(x_{0}) \right|> 0$であり,
    $f-g$の連続性からある$\varepsilon > 0$が存在して,$y \in [x_{0} - \varepsilon, x_{0} + \varepsilon]$に対して
    $\left| f(y) - g(y) \right| > 0$である.
    したがって,
    \begin{equation*}
        \int_{a}^{b} \left| f(x) - g(x) \right| \dx{x} \ge \varepsilon \min_{x_{0} - \varepsilon \le y \le x_{0} + \varepsilon} \left| f(y) - g(y) \right| > 0
    \end{equation*}
    である.
\end{proof}

\begin{rei}
    $X = \sets{0,1}^{n}$とする.
    \begin{equation}
        d_{H}(x,y) = \# \sets{i | x_{i} \neq y_{i}}
    \end{equation}
    で定めると,これは$X$上の距離関数である.この距離関数は\nw{Hamming距離}と呼ばれる.
\end{rei}

\begin{rei}
    $G = (X,E)$を無向グラフとする.
    $d_{G}(x,y)$を$x$から$y$への最短路の長さとして定義すると,
    これは$G$上の距離関数である.
\end{rei}

以下,$(X,d)$を距離空間とする.

\begin{teigi}[内点・外点・境界点・触点]~\par
    \midashi{(1) } $a \in X$,$\varepsilon > 0$とする.集合$N(a,\varepsilon) := \sets{x \in X | d(a,x) < \varepsilon}$を,$a$の\nw{$\varepsilon$-近傍}という.

    \midashi{(2) } $A \subseteq X$の\nw{内点}とは,次の条件を満たす点$x \in X$のこと:
    \begin{equation}
        \exists \varepsilon > 0, \quad N(x,\varepsilon) \subseteq A
    \end{equation}
    また,$A$の内点全体の集合$A^{\circ} = \sets{x \in X | N(x, \varepsilon) \subseteq A}$のことを$A$の\nw{内部}という.

    \midashi{(3) } $A \subseteq X$の\nw{外点}とは,次の条件を満たす点$x \in X$のこと:
    \begin{equation}
        \exists \varepsilon > 0, \quad N(x , \varepsilon ) \cap A = \emptyset
    \end{equation}
    言い換えると,$A$の補集合の内点のこと.
    また,$A$の外点全体の集合$(X \setminus A)^{\circ}$を$A$の\nw{外部}という.

    \midashi{(4) } $A \subseteq X$の\nw{境界点}とは,次の条件を満たす点$x \in X$のこと:
    \begin{equation}
        \forall \varepsilon > 0, \quad N(x, \varepsilon) \cap A \neq \emptyset, \, N(x,\varepsilon) \cap (X \setminus A) \neq \emptyset
    \end{equation}
    また,$A$の境界点全体の集合のことを$A$の\nw{境界}といい,記号$\partial A$で表す.

    \midashi{(5) } $A \subseteq X$の\nw{触点}とは,次の条件を満たす点$x \in X$のこと:
    \begin{equation}
        \forall \varepsilon > 0, \quad N (x,\varepsilon) \cap A \neq \emptyset
    \end{equation}
    また,$A$の触点全体の集合のことを$A$の\nw{閉包}といい,記号$\overline{A}$で表す.
\end{teigi}

\begin{figure}[htbp]
    \centering
    \newcommand{\makeaxis}{
    \draw[thick, ->] (-2.5,0) -- (2.5,0) node [below] {$x$};%x軸
    \draw[thick, ->] (0,-2.5) -- (0,2.5) node [left] {$y$};%y軸
    \coordinate (O) at (0,0);
    \node at (O) [above left] {$a$};
    }
    \centering    
    \begin{tikzpicture}
        \makeaxis
        \draw[dashed] (O) circle (1.5);
        \draw[->] (O) -- node[midway, right]{$\varepsilon$} (45:1.5);
        \node at (-2,2) {$(\mathbb{R}^{2}, d_{2})$};
    \end{tikzpicture}
    \begin{tikzpicture}
        \makeaxis
        \draw[dashed] (1.5,0) -- (0,1.5) -- (-1.5,0) -- (0,-1.5) -- cycle;
        \node at (1.5,0) [below] {$a + \varepsilon$};
        \node at (-1.5,0) [below] {$a - \varepsilon$};
        \node at (-2,2) {$(\mathbb{R}^{2}, d_{1})$};
    \end{tikzpicture}
    \begin{tikzpicture}
        \makeaxis
        \draw[dashed] (-1.5,-1.5) -- (1.5,-1.5) -- (1.5,1.5) -- (-1.5,1.5) -- cycle;
        \node at (-1.5,0) [below left] {$a-\varepsilon$};
        \node at (0,-1.5) [below right] {$a-\varepsilon$};
        \node at (1.5,0) [above right] {$a+\varepsilon$};
        \node at (0,1.5) [above right] {$a+\varepsilon$};
        \node at (-2,2) {$(\mathbb{R}^{2}, d_{\infty})$};
    \end{tikzpicture}
    \caption{距離空間$(\mathbb{R}^{2},d)$上の近傍の違い}
    \label{fig:euclid_dist}
\end{figure}

\begin{figure}[htbp]
    \centering
    \begin{tikzpicture}
        \draw[dashed] (-30:2) arc (-30:210:2);
        \fill [black!20] (0,0) circle [x radius=1.97, y radius=1.97];
        \draw[thick] (210:2) arc (210:330:2);
        \draw[<-] (-50:2.1) -- (-50:2.4) node[right]{\textbf{外点}};
        \draw (20:0.5) node{\textbf{内点}};
        \draw[<-] (160:2.1) -- (160:2.6) node[left]{\textbf{境界点}};
        \draw (45:2.5) node {$A$};
    \end{tikzpicture}
    \caption{$A \subseteq X$の内部,外部,境界}
    \label{fig:in_ex_boundary}
\end{figure}

この定義から,内部,外部,境界点について$X = A^{\circ} \dcup \partial A \dcup (X \setminus A)^{\circ}$と互いに非交な集合に分解できる.
また,閉包について,次の二つが成り立つ.
\begin{align}
    & A ^ { \circ } \subseteq A \subseteq \overline{A}, \\
    & \overline { A } = A ^ { \circ } \cup \partial A.
\end{align}

\begin{teigi}[開集合・閉集合]~\par
    \midashi{(1) } $A \subseteq X \text{:\nw{開集合}}  \defines \;\forall x \in A, \quad \exists \varepsilon > 0, \quad  N(x , \varepsilon ) \subseteq A \iff A = A^{\circ}$\par
    \midashi{(2) } $A \subseteq X \text{:\nw{閉集合}} \defines \; 「\forall x \in X, \quad \forall \varepsilon > 0, \quad  N(x , \varepsilon ) \cap A \neq \emptyset \Longrightarrow x \in A」\iff A = \overline{A}$
\end{teigi}

\begin{hodai}
    $N(x,\varepsilon)$は開集合.特に,$N(x,\varepsilon) = N(x,\varepsilon)^{\circ}$.
\end{hodai}

\begin{proof}
    $y \in N ( x, \varepsilon)$をとる.$\delta := \varepsilon - d ( x , y ) > 0$とすると,
    $z \in N(y, \varepsilon)$に対して,三角不等式より
    \begin{equation*}
        d(x,z) \le d(x,y) + d(y,z) < \varepsilon
    \end{equation*}
である.よって,$N(y,\delta) \subseteq N(x,\varepsilon)$となる.
\end{proof}

\begin{prop}
    $\left( A^{\circ} \right)^{\circ} = A^{\circ}$,$\overline{\overline{A}} = \overline{A}$
\end{prop}

\begin{proof}
    $\left( A^{\circ} \right)^{\circ} \subseteq A^{\circ}$より,$\supseteq$向きを示す.
    $x \in A^{\circ}$とする.
    \begin{equation*}
       \exists \varepsilon > 0, \quad N(x, \varepsilon) \subseteq A
    \end{equation*}
    である.$N(x, \varepsilon) = N(x, \varepsilon)^{\circ} \subseteq A^{\circ}$,
    つまり$N(x, \varepsilon) \subseteq A^{\circ}$である.
    $x \in \left( A^{\circ} \right)^{\circ}$である.

    $\overline{\overline{A}} \subseteq \overline{A}$より,$\subseteq$向きを示す.
    $x \in \overline{\overline{A}}$とする.
    \begin{equation}
        \forall \varepsilon > 0, \quad N(x,\varepsilon) \cap \overline{A} \neq \emptyset
    \end{equation}
    である.$y \in N(x, \varepsilon) \cap \overline{A}$,$\delta := \varepsilon - d(x,y) > 0$とする.
    $y \in \overline{A}$より$N(y, \delta) \cap A \neq \emptyset$である.
    $N(y, \delta) \subseteq N(x,\varepsilon)$より,上と同じロジックで
    $\emptyset \neq N(y, \delta) \cap A \subseteq N(x, \varepsilon) \cap A$である.よって,$x \in \overline{A}$である.
\end{proof}

\begin{prop}~\par
    \midashi{(1) } $A$:開 $\Longrightarrow$ $X \setminus A$:閉\par
    \midashi{(2) } $A$:閉 $\Longrightarrow$ $X \setminus A$:開
\end{prop}

\begin{proof}
    \begin{align*}
        A\text{が開でない} & \Longleftrightarrow \exists x \in A, \quad \forall \varepsilon > 0, \quad N(x, \varepsilon) \not\subseteq A \\
        & \Longleftrightarrow \exists x \in A, \quad \forall \varepsilon > 0, \quad N(x, \varepsilon) \cap (X \setminus A) \neq \emptyset \\
        & \Longleftrightarrow X \setminus A \text{が閉でない}
    \end{align*}
    ここで2行目の同値性は,$x$は$X \setminus A$の触点であるが$X \setminus A$の元でないことから従う.
\end{proof}

開集合全体の集合を\nw{開集合系}といい,記号$\mathcal{O}$で表す.
閉集合全体の集合を\nw{閉集合系}といい,記号$\mathfrak{A}$で表す.

\begin{prop}
    \label{prop:open_set_axiom}
    開集合系$\mathcal{O}$について以下が成り立つ.\par
    \midashi{(1) } $X \in \mathcal{O}$,$\emptyset \in \mathcal{O}$\par
    \midashi{(2) } $O_{1}, O_{2}, \cdots, O_{k} \in \mathcal{O} \Longrightarrow O_{1} \cap O_{2} \cap \cdots \cap O_{k} \in \mathcal{O}$\par
    \midashi{(3) } 任意の集合$\Lambda$に対して$O_{\lambda} \in \mathcal{O}$($\forall \lambda \in \Lambda$) ${\displaystyle \Longrightarrow \bigcup_{\lambda \in \Lambda} O_{\lambda} \in \mathcal{O}}$
\end{prop}

\begin{proof}
    \midashi{(1) } obvious.

    \midashi{(2) } $x \in O_{1} \cap O_{2} \cap \dots \cap O_{k}$をとる.$i = 1,2,\dots,k$に対して$\varepsilon_{i} > 0$が存在して
    \begin{equation*}
        N(x_{i}, \varepsilon_{i}) \subseteq O_{i}
    \end{equation*}
    である.${\displaystyle \varepsilon := \min_{i} \varepsilon_{i}}$とすると,
    $N(x, \varepsilon) \subseteq O_{1} \cap O_{2} \cap \dots \cap O_{k}$である.
    $x$は$O_{1} \cap O_{2} \cap \dots \cap O_{k}$の内点である.よって,$\left( O_{1} \cap O_{2} \cap \dots \cap O_{k} \right)^{\circ} = O_{1} \cap O_{2} \cap \dots \cap O_{k}$である.

    \midashi{(3) } ${x \in \displaystyle \bigcup_{\lambda \in \Lambda} O_{\lambda}}$とすると,
    $\exists \lambda \in \Lambda, \quad x \in O_{\lambda}$である.${\displaystyle \exists \varepsilon > 0, \quad N(x, \varepsilon) \subseteq O_{\lambda} \subseteq \bigcup_{\lambda} O_{\lambda}}$とする.
\end{proof}

\begin{prop}
    閉集合系$\mathfrak{A}$について以下が成り立つ.\par
    \midashi{(1) } $X \in \mathfrak{A}$,$\emptyset \in \mathfrak{A}$\par
    \midashi{(2) } $A_{1}, A_{2}, \cdots, A_{k} \in \mathfrak{A} \Longrightarrow A_{1} \cup A_{2} \cup \cdots \cup A_{k} \in \mathfrak{A}$\par
    \midashi{(3) } 任意の集合$\Lambda$に対して$A_{\lambda} \in \mathfrak{A}$($\forall \lambda \in \Lambda$) ${\displaystyle \Longrightarrow \bigcap_{\lambda \in \Lambda} A_{\lambda} \in \mathfrak{A}}$
\end{prop}

\begin{proof}
    \midashi{(1) } $\mathfrak{A} = \sets{ X \setminus O | O \in \mathcal{O}}$に注意すると明らか.

    \midashi{(2) } $A_{i} = X \setminus O_{i}$とかくと,
    \begin{equation*}
        A_{1} \cup A_{2} \cup \dots \cup A_{k}
        = (X \setminus O_{1}) \cup (X \setminus O_{2}) \cup \dots \cup (X \setminus O_{k})
        = X \setminus (O_{1} \cap O_{2} \cap \dots \cap O_{k}) \in \mathfrak{A}
    \end{equation*}
    である.

    \midashi{(3) } ${\displaystyle \bigcap_{\lambda \in \Lambda} A_{\lambda} = \bigcap_{\lambda \in \Lambda} (X \setminus O_{\lambda}) = X \setminus \bigcup_{\lambda \in \Lambda} O_{\lambda}}$であるから従う.
\end{proof}

収束性,連続性なども距離空間で定義できる.

\begin{teigi}[収束性・連続性]
    \midashi{(1) } $a_{i} \in X$($i=1,2,\dots$):\nw{収束}とは,
    \begin{equation}
        \lim_{i \to \infty} a_{i} = a \defines \lim_{i \to \infty} d(a, a_{i}) = 0
    \end{equation}
    
    \midashi{(2) } $f:X \longrightarrow \mathbb{R}$が$a \in X$で\nw{連続}とは,
    \begin{equation}
        \forall \varepsilon > 0, \quad \exists \delta > 0, d(a,x) < \delta \Longrightarrow \left| f(a) - f(x) \right| < \varepsilon
    \end{equation}
\end{teigi}

では,$X$上の異なる距離$d$,$d'$の下で連続性や収束性も変わるのか.

$\longrightarrow$位相空間の考え方へ

実際,開集合系を与える(空間に「位相」を与える)ことによって,
連続性や収束性が定義でき,
開集合系が同じとき,連続性や収束性も等しくなる.

\begin{rei}
    $(\mathbb{R}^{n},d_{2})$と$(\mathbb{R}^{n},d_{\infty})$が定義する開集合系は等しい.(位相は等しい)
    \begin{proof}
        $A$:$d_{2}$の下で開集合とする.
        $x \in A$をとる.$\exists \varepsilon > 0, \quad N_{2}(x,\varepsilon) \subseteq A$である.
        $\exists c > 0, \quad N_{\infty} (x, c \varepsilon) \subseteq N_{2} (x, \varepsilon) \subseteq A$である.
        $A$は$d_{\infty}$の下でも開.逆も同様に言える. 
    \end{proof}
    結局,連続性や収束性はどちらの距離で考えても同じになる.
\end{rei}

\begin{rei}
    $\mathbb{R}^{n}$上のノルム$\left\| \cdot \right\|$とは,
    \begin{itemize}
        \vspace{-0.5\baselineskip}
        \item $\left\| v \right\| \ge 0$,\qquad $\left\| v \right\| = 0 \Longleftrightarrow v = 0$
        \item $\forall a \in \mathbb{R}$,$v \in \mathbb{R}^{n}$,$\left\| av \right\| = \left| a \right| \left\| v \right\|$
        \item $\forall u,v \in \mathbb{R}^{n}$,$\left\| u + v \right\| \le \left\| u \right\| + \left\| v \right\|$
        \vspace{-0.5\baselineskip}
    \end{itemize}
    を満たすものである.このとき,$d(x,y) := \left\| x - y \right\|$で定義するとこれは距離関数である.
\end{rei}

\midashi{知っておいてほしいこと}
\begin{itemize}
    \vspace{-0.5\baselineskip}
    \item $\mathbb{R}^{n}$上のどんなノルムを定める「位相」は等しい.
    \item 無限次元では,そうはいかない(\textbf{らしい}).
\end{itemize}

\expandafter\ifx\csname readornot\endcsname\relax
  \end{document}
\fi