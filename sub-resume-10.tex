\expandafter\ifx\csname readornot\endcsname\relax
  \documentclass[uplatex]{jsarticle}
  \usepackage{octopus}
  \usepackage{url}
  \usetikzlibrary{calc}

  \renewcommand{\proofname}{\textsf{証明}}
  \renewcommand{\postpartname}{章}
  \renewcommand{\thesection}{\thepart.\arabic{section}}
  \renewcommand{\thepart}{\arabic{part}}
  \makeatletter\renewcommand{\theequation}{\thesection.\arabic{equation}}\@addtoreset{equation}{section}\makeatother

  \newcommand{\octopuspart}[1]{\newpage\part{#1}\setcounter{section}{0}\vspace{3\baselineskip}}

  \renewcommand{\restriction}[2]{\left. #1 \right|_{#2}}
  \DeclareMathOperator{\dcup}{\dot{\cup}}
  \DeclareMathOperator{\conv}{conv}
  \DeclareMathOperator{\Image}{Im}
  \DeclareMathOperator{\Kernel}{Ker}
  \DeclareMathOperator{\diag}{diag}
  \DeclareMathOperator{\rank}{rank}
  \begin{document}
  \fi

\section{ホモロジーの計算 2}
\newcommand{\BigZero}{\kern3pt \hbox{\huge \strut 0}}

\begin{teigi}[自由$\mathbb{Z}$加群]
  $G$が\nw{自由$\mathbb{Z}$加群}であるとは,$\mathbb{Z}$上の加群$G$が,
  $G \simeq \mathbb{Z}^{r}$であり,かつ$r$個の元$e_{1}, e_{2}, \dots, e_{r}$によって$g \in G$が一意に
  ${\displaystyle g = \sum_{i=1}^{r} n_{i} e_{i}}$とかけるものとする。
\end{teigi}

この$e_{1}, e_{2}, \dots, e_{r}$を$\mathbb{Z}$基底といい,$r$をランクと呼ぶ。

\begin{teigi}[ユニモジュラ行列]
  $Q$:$r$次正方整数行列とする。

  $Q$:\nw{ユニモジュラ(unimodular)} $\defines$ $\det Q = \pm 1$
\end{teigi}

ユニモジュラ行列の定義について次のことが成り立つ。
\renewcommand{\arraystretch}{1}
\begin{center}
  \begin{tabular}{lcl}
    $Q$:ユニモジュラ & $\defines$ & $\det Q = \pm 1$ \\
    & $\Longleftrightarrow$ & $e_{1}, e_{2}, \dots, e_{r}$が$\mathbb{Z}$基底ならば
    ${\displaystyle e_{i}' = \sum_{j=1}^{r} Q_{ij} e_{j}}$($i = 1, \dots, r$)も$\mathbb{Z}$基底である 
  \end{tabular}
\end{center}

このことから,ユニモジュラ行列は,自由$\mathbb{Z}$加群の基底の変換行列であることが従う。
なお,$Q$は定義から正則行列であるが,逆行列$Q^{-1}$を余因子行列で書けば分かるように,$Q^{-1}$も整数行列であってユニモジュラ行列である。
したがって,${\displaystyle g = \sum_{i=1}^{n} n_{i} e_{i}}$ならば
\begin{equation}
  g = \sum_{j=1}^{r} \sum_{i=1}^{r} n_{i} (Q^{-1})_{ji} \, e_{j}'
\end{equation}
と新たな基底の下での表現を得ることができる。

\begin{teiri}[Smith標準形(単因子標準形)]
  (正方とは限らない)整数行列$A$は,あるユニモジュラ行列$P,Q$を用いて
  \begin{equation}
    PAQ = \left( \begin{array}{cccc|c}
      \alpha_{1} &            &        &            & \multirow{4}{*}{\BigZero} \\
                 & \alpha_{2} &        &            & \\
                 &            & \ddots &            & \\
                 &            &        & \alpha_{k} & \\ \hline
      \multicolumn{4}{c|}{\BigZero} & \BigZero
    \end{array} \right)
  \end{equation}
  の形にできる。ここで,$\alpha_{1} | \alpha_{2} | \dots | \alpha_{k}$($\alpha_{i} > 0$)である。
  また,$\alpha_{1}, \dots, \alpha_{k}$は一意に定まり,これを$A$の\nw{単因子}と呼ぶ。
\end{teiri}

\begin{proof}
  以下に示すような基本変形を繰り返せば右辺のような形になる。:
  \begin{itemize}
    \vspace{-0.5\baselineskip}
    \item[(1)] $i$行(列)を$-1$倍する。
    \begin{equation}
      \begin{pmatrix}
        1 & \\
          & \ddots \\
          &        & -1 \\
          &        &    & \ddots \\ 
          &        &    &        & 1
      \end{pmatrix}
    \end{equation}
    \item[(2)] 行(列)の入れ替え
    \begin{equation}
      \left( \begin{array}{ccccccccccc}
        1 \\
          & \ddots \\
          &        & 1 \\
          &        &   & 0      & \hdots & \hdots & \hdots & 1      \\ 
          &        &   & \vdots & 1      &        &        & \vdots \\
          &        &   & \vdots &        & \ddots &        & \vdots \\
          &        &   & \vdots &        &        & 1      & \vdots \\
          &        &   & 1      & \hdots & \hdots & \hdots & 0      \\
          &        &   &        &        &        &        &        & 1 \\
          &        &   &        &        &        &        &        &   & \ddots \\
          &        &   &        &        &        &        &        &   &        & 1 
      \end{array} \right)
    \end{equation}
    \item[(3)] $i$行(列)を$c$倍して$j$行(列)に足す
    \begin{equation}
      \begin{pmatrix}
        1 & \\
          & \ddots \\
          &        & 1      \\
          &        & \vdots & \ddots \\
          &        & c      & \hdots & 1 \\ 
          &        &        &        &   & \ddots \\
          &        &        &        &   &        & 1
      \end{pmatrix}
    \end{equation}
  \end{itemize}
  これらの変形は直下に示したユニモジュラ行列による変換として表現できるので定理が従う。
\end{proof}

\begin{rei}
  実際にSmith標準形を求めてみる。
  \begin{center}
    $\begin{array}{ccc}
      \begin{pmatrix}
        -3 & 4 & 4 \\ -2 & 2 & -4 \\ 4 & -4 & 4
      \end{pmatrix} & \xrightarrow[\text{絶対値最小の非零要素を$(1,1)$成分へ}]{(1),(2)}
      & \begin{pmatrix}
        2 & -2 & 4 \\ -3 & 4 & 4 \\ 4 & -4 & 4
      \end{pmatrix} \\ & \xrightarrow[\text{1行,1列に$a_{11}$で割り切れないものがあればそれを「割った余り」にする}]{(3); 2 \text{行目} + 2 \times 1 \text{行目}}
      & \begin{pmatrix}
        2 & -2 & 4 \\ 1 & 0 & 12 \\ 4 & -4 & 4
      \end{pmatrix} \\ & \xrightarrow[\text{絶対値最小の非零要素を$(1,1)$成分へ}]{(1)}
      & \begin{pmatrix}
        1 & 0 & 12 \\ 2 & -2 & 4 \\ 4 & -4 & 4
      \end{pmatrix} \\ & \xrightarrow[\text{1行,1列がすべて$a_{11}$で割り切れる場合は割って0にする}]{(3)}
      & \begin{pmatrix}
        1 & 0 & 0 \\ 0 & -2 & -20 \\ 0 & -4 & -44
      \end{pmatrix} \\ & \xrightarrow[\text{右下の$2 \times 2$の小行列に対して同じアルゴリズムを適用}]{(1),(3)}
      & \begin{pmatrix}
        1 & 0 & 0 \\ 0 & 2 & -20 \\ 0 & 0 & -4
      \end{pmatrix} \\ & \xrightarrow{(1),(3)}
      & \begin{pmatrix}
        1 & 0 & 0 \\ 0 & 2 & 0 \\ 0 & 0 & 4
      \end{pmatrix}
    \end{array}$
  \end{center}
  辿り着いた行列がSmith標準形である。
\end{rei}

すなわち,
\begin{itemize}
  \item 一番左の行と列に左上の要素で割り切れない要素があるときには,
  $a_{11}$で割った余りになるように変形した上でその要素を左上に移動する。
  \item 一番左の行と列に左上の要素で割り切れない要素がないときには,
  その行と列を1にするように変形し,サイズが一回り小さい行列に対して更に変形を続けていく。
\end{itemize}
の手順を踏むことでSmith標準形に辿り着く。

\begin{hodai}
  \label{hodai:homology3.unimod}
  $BA = O$とし,$A$の単因子を$\alpha_{1} | \alpha_{2} | \dots | \alpha_{k}$,$B$の単因子を$\beta_{1} | \beta_{2} | \dots | \beta_{\ell}$とする。
  このとき,あるユニモジュラ行列$P,Q,R$が存在して,
  \begin{equation}
    R^{-1} B P = \begin{pmatrix}
      O_{k' \times k'} & O_{k' \times \ell} \\ O_{\ell \times k'} & \diag (\beta_{1}, \dots, \beta_{\ell})
    \end{pmatrix}, \quad
    P^{-1} A Q = \begin{pmatrix}
      \diag (\alpha_{1}, \dots, \alpha_{k}) & O \\ O & O
    \end{pmatrix}
  \end{equation}
  となる。ただし,$O_{m \times n}$で$m \times n$の大きさの零行列を表すとし,$k' \ge k$とする。
\end{hodai}

\begin{proof}
  まず,$A$をSmith標準形にする。$\tilde{P},Q$をユニモジュラ行列として,
  \begin{equation}
    \tilde{P}^{-1}AQ = \left( \begin{array}{cccc|c}
      \alpha_{1} &            &        &            & \multirow{4}{*}{\BigZero} \\
                 & \alpha_{2} &        &            & \\
                 &            & \ddots &            & \\
                 &            &        & \alpha_{k} & \\ \hline
      \multicolumn{4}{c|}{\BigZero} & \BigZero
    \end{array} \right)
  \end{equation}
  $BAQ = B\tilde{P} \tilde{P}^{-1} AQ = O$より,$B \tilde{P}$の形状は第1行から第$k$行までが零行列となっている。
  そこで,$B \tilde{P}$の第$k+1$行目以降の小行列に対して,Smith標準形をつくる。
  すると,$R,S$をユニモジュラ行列として次のようになる。
  \begin{equation}
    R B \tilde{P} \begin{pmatrix}
      I_{k} & O \\ O & S
    \end{pmatrix} = 
    \begin{array}{l}
      \hspace{1em}\overbrace{}^{k} \\
      \left( \begin{array}{c|c|c}
                &          & \multirow{4}{*}{\BigZero} \\
                &          &                           \\
                &          &                           \\
        \BigZero & \BigZero &                           \\ \cline{3-3}
                &          & \begin{array}{ccc} \beta_{1} & & \\ & \ddots & \\ & & \beta_{\ell} \end{array} \\
      \end{array} \right) \\
      \hspace{1em}\underbrace{\hspace{4em}}_{k'}
  \end{array}
  \end{equation}
  いま,
  \begin{equation*}
    P = \tilde{P} \begin{pmatrix}
      I_{k} & O \\ O & S
    \end{pmatrix}
  \end{equation*}
  とおけば,
  \begin{equation}
    P^{-1} A Q = \begin{pmatrix}
      I_{k} & O \\ O & S^{-1}
    \end{pmatrix}
    \tilde{P}^{-1} A Q =
    \left( \begin{array}{cccc|c}
      \alpha_{1} &            &        &            & \multirow{4}{*}{\BigZero} \\
                 & \alpha_{2} &        &            & \\
                 &            & \ddots &            & \\
                 &            &        & \alpha_{k} & \\ \hline
      \multicolumn{4}{c|}{\BigZero} & \BigZero
    \end{array} \right) 
  \end{equation}
  であり,
  \begin{equation}
    R^{-1} B P = \begin{array}{l}
      \hspace{1em}\overbrace{\hspace{4em}}^{k'} \\
      \left( \begin{array}{c|c|c}
                &          & \multirow{4}{*}{\BigZero} \\
                &          &                           \\
                &          &                           \\
        \BigZero & \BigZero &                           \\ \cline{3-3}
                &          & \begin{array}{ccc} \beta_{1} & & \\ & \ddots & \\ & & \beta_{\ell} \end{array} \\
      \end{array} \right)
    \end{array}
  \end{equation}
  である。
\end{proof}

\sukima \midashi{\large ホモロジーの計算}

$X$:位相空間,$\Delta$-複体構造,$c_{n}$:$n$-単体の個数(有限とする),とする。
このとき,
\begin{center}
  \begin{tabular}{l@{\,}c@{\,}l}
    $\Delta_{n}$ & $:=$ & $n$-チェイン全体($n$-単体の整数結合) \\
    & $\sim$ & ランク$c_{n}$の自由$\mathbb{Z}$加群,特に$n$-単体$\sigma_{1},\sigma_{2}, \dots, \sigma_{c_{n}}$は$\Delta_{n}$の$\mathbb{Z}$基底 \\
    & $\simeq$ & $\mathbb{Z}^{c_{n}}$
  \end{tabular}
\end{center}
である。よって,境界作用素$\partial_{n} \colon \Delta_{n} \longrightarrow \Delta_{n-1}$は$\mathbb{Z}^{c_{n}} \longrightarrow \mathbb{Z}^{c_{n-1}}$と見なすことができ,
これを行列によって表示できる。
\begin{align}
  & \partial_{n} \colon \mapdef{\Delta_{n}}{\Delta_{n-1}}{\displaystyle \sum_{i=1}^{c_{n}} t_{i} \sigma_{i}}{\displaystyle \sum_{i=1}^{c_{n}} t_{i} \partial \sigma_{i}} \\
  & \partial_{n}^{*} \colon \mapdef{\mathbb{Z}^{c_{n}}}{\mathbb{Z}^{c_{n-1}}}{
    \begin{pmatrix}
      \sigma_{1} & \sigma_{2} & \cdots & \sigma_{c_{n}}
    \end{pmatrix}
    \begin{pmatrix}
      t_{1} \\ t_{2} \\ \vdots \\ t_{c_{n}}
    \end{pmatrix}
  }{
    \begin{pmatrix}
      \pi_{1} & \pi_{2} & \cdots & \pi_{c_{n-1}}
    \end{pmatrix} A
    \begin{pmatrix}
      t_{1} \\ t_{2} \\ \vdots \\ t_{c_{n}}
    \end{pmatrix}
  }
\end{align}
ここで,$c_{n-1} \times c_{n}$次行列$A$は$\partial_{n}$の行列表示であり,
$\partial \sigma_{j}$における$\pi_{i}$の係数を$(i,j)$成分にもつものとする。

\begin{rei}[射影平面の場合]
  $X = P^{2}$:2次元射影空間(射影平面)とする。
  \begin{center}
		\begin{tikzpicture}
			\fill[red!20] (0,0) -- (2,2) -- (0,2) -- cycle;
			\fill[blue!20] (0,0) -- (2,2) -- (2,0) -- cycle;
			\fill \point{(0,0)}{below left}{$u$} \point{(2,0)}{below right}{$v$} \point{(0,2)}{above left}{$v$} \point{(2,2)}{above right}{$u$};
			\node at (1,0) [below]{$b$} node at (0,1) [left]{$a$} node at (1,2) [above]{$b$} node at (2,1) [right]{$a$} node at (1,1) [left]{$c$};
			\node at (0.5,1.5){$U$} node at (1.5,0.5){$L$};
			\midwayarrow{(0,0)}{(2,0)} \midwayarrow{(0,0)}{(0,2)} \midwayarrow{(0,0)}{(2,2)} \midwayarrow{(2,2)}{(2,0)} \midwayarrow{(2,2)}{(0,2)}
		\end{tikzpicture}
  \end{center}
  \begin{itemize}
    \item $\Delta_{2} \simeq \sets{\begin{pmatrix}
      t_{U} & t_{L} 
    \end{pmatrix}^{\top} | t_{U}, t_{L} \in \mathbb{Z}} = \mathbb{Z}^{2}$
    \item $\Delta_{1} \simeq \sets{\begin{pmatrix}
      t_{a} & t_{b} & t_{c} 
    \end{pmatrix}^{\top} | t_{a}, t_{b}, t_{c} \in \mathbb{Z}} = \mathbb{Z}^{3}$
    \item $\Delta_{0} \simeq \sets{\begin{pmatrix}
      t_{u} & t_{v} 
    \end{pmatrix}^{\top} | t_{u}, t_{v} \in \mathbb{Z}} = \mathbb{Z}^{2}$
  \end{itemize}
  このとき,境界作用素の行列表示はそれぞれ,
  \begin{equation}
    A_{\partial_{1}} = \begin{array}{c@{\,}c}
      & \begin{array}{ccc}
        a & b & c
      \end{array} \\
      \begin{array}{c}
        u \\ v
      \end{array} & 
      \begin{pmatrix}
        -1 & 1 & 0 \\
        -1 & 1 & 0
      \end{pmatrix}  
    \end{array}
    , \quad 
    A_{\partial_{2}} = \begin{array}{c@{\,}c}
      & \begin{array}{cc}
        U & L
      \end{array} \\
      \begin{array}{c}
        a \\ b \\ c
      \end{array} &
      \begin{pmatrix}
      -1 & 1 \\
      1 & -1 \\
      1 & 1
    \end{pmatrix}
  \end{array}
  \end{equation}
  である。
\end{rei}

以上を踏まえて$H_{n} = \Kernel \partial_{n} / \Image \partial_{n+1}$を計算する流れについて述べる。

\sukima\midashi{注意する事実}
\begin{itemize}
  \item $\Delta_{n+1} \xrightarrow{\partial_{n+1}} \Delta_{n} \xrightarrow{\partial_{n}} \Delta_{n-1}$について$\partial_{n} \circ \partial_{n+1} = 0$である。
  \item 行列表示で言えば,$\partial_{n}$の行列表示を$A_{n}$として,
  \begin{equation}
    A_{n} A_{n+1} = O
  \end{equation}
  \item \rref{補題}{hodai:homology3.unimod}より,あるユニモジュラ行列$P,Q,R$が存在して
  \begin{align}
    P^{-1} A_{n+1} Q &=
    \left( \begin{array}{ccc|c}
      \alpha_{1} &        &                  & \multirow{3}{*}{\BigZero} \\
                 & \ddots &                  & \\
                 &        & \alpha_{k_{n+1}} & \\ \hline
      \multicolumn{3}{c|}{\BigZero} & \BigZero
    \end{array} \right), \\
    \label{eq:homology3.calc}
    R^{-1} A_{n} P &=
    \left( \begin{array}{c|ccc}
      \BigZero                  & \multicolumn{3}{c}{\BigZero} \\ \hline
      \multirow{3}{*}{\BigZero} & \beta_{1} &        & \\
                                &           & \ddots & \\
                                &           &        & \beta_{k_{n}} \\
    \end{array} \right)
  \end{align}
  とできる。なお,式\eqref{eq:homology3.calc}において上2つの零行列の縦のサイズは$c_{n} - k_{n}$である。
\end{itemize}

これを使って計算していく。

\begin{equation}
  \begin{array}{c@{\:}c@{\:}c@{\:}c@{\:}c}
    \begin{pmatrix}
      \rho_{1} & \cdots & \rho_{c_{n+1}}
    \end{pmatrix}
    \begin{pmatrix}
      s_{1} \\ \vdots \\ s_{c_{n+1}}
    \end{pmatrix}
    & \longmapsto &
    \begin{pmatrix}
      \sigma_{1} & \cdots & \sigma_{c_{n}}
    \end{pmatrix}
    A_{n+1}
    \begin{pmatrix}
      s_{1} \\ \vdots \\ s_{c_{n+1}}
    \end{pmatrix} \\
    \rotatebox{270}{$\in$} & & \rotatebox{270}{$\in$} \\
    \Delta_{n+1} & \xrightarrow{\partial_{n+1}} & \Delta_{n} & \xrightarrow{\partial_{n}} & \Delta_{n-1} \\
    & & \vin & & \vin \\
    & & \begin{pmatrix}
      \sigma_{1} & \cdots & \sigma_{c_{n}}
    \end{pmatrix}
    \begin{pmatrix}
      t_{1} \\ \vdots \\ t_{c_{n}}
    \end{pmatrix}
    & \longmapsto &
    \begin{pmatrix}
      \pi_{1} & \cdots & \pi_{c_{n-1}}
    \end{pmatrix}
    A_{n}
    \begin{pmatrix}
      t_{1} \\ \vdots \\ t_{c_{n}}
    \end{pmatrix} \\
  \end{array}
\end{equation}
において,先の\textbf{注意する事実}で述べた行列$P,Q,R$を用いて
\begin{equation}
  \left\{ \begin{split}
    \bm{\rho}' &= Q^{\top} \bm{\rho}, \\
    \bm{\sigma}' &= P^{\top} \bm{\sigma}, \\
    \bm{\pi}' &= R^{\top} \bm{\pi}, \\
    \bm{s}' &= Q^{-1} \bm{s}, \\
    \bm{t}' &= P^{-1} \bm{t}
  \end{split} \right.
\end{equation}
と変換すると
\begin{equation}
  \begin{array}{c@{\:}c@{\:}c@{\:}c@{\:}c}
    \begin{pmatrix}
      \rho'_{1} & \cdots & \rho'_{c_{n+1}}
    \end{pmatrix}
    \begin{pmatrix}
      s'_{1} \\ \vdots \\ s'_{c_{n+1}}
    \end{pmatrix}
    & \longmapsto &
    \begin{pmatrix}
      \sigma'_{1} & \cdots & \sigma'_{c_{n}}
    \end{pmatrix}
    P^{-1}A_{n+1}Q
    \begin{pmatrix}
      s'_{1} \\ \vdots \\ s'_{c_{n+1}}
    \end{pmatrix} \\
    \rotatebox{270}{$\in$} & & \rotatebox{270}{$\in$} \\
    \Delta_{n+1} & \xrightarrow{\partial_{n+1}} & \Delta_{n} & \xrightarrow{\partial_{n}} & \Delta_{n-1} \\
    & & \vin & & \vin \\
    & & \begin{pmatrix}
      \sigma'_{1} & \cdots & \sigma'_{c_{n}}
    \end{pmatrix}
    \begin{pmatrix}
      t'_{1} \\ \vdots \\ t'_{c_{n}}
    \end{pmatrix}
    & \longmapsto &
    \begin{pmatrix}
      \pi'_{1} & \cdots & \pi'_{c_{n-1}}
    \end{pmatrix}
    R^{-1}A_{n}P
    \begin{pmatrix}
      t'_{1} \\ \vdots \\ t'_{c_{n}}
    \end{pmatrix} \\
  \end{array}
\end{equation}
となって,ここに
\begin{align}
  P^{-1} A_{n+1} Q &=
    \left( \begin{array}{ccc|c}
      \alpha_{1} &        &                  & \multirow{3}{*}{\BigZero} \\
                 & \ddots &                  & \\
                 &        & \alpha_{k_{n+1}} & \\ \hline
      \multicolumn{3}{c|}{\BigZero} & \BigZero
    \end{array} \right), \\
    R^{-1} A_{n} P &=
    \left( \begin{array}{c|ccc}
      \BigZero                  & \multicolumn{3}{c}{\BigZero} \\ \hline
      \multirow{3}{*}{\BigZero} & \beta_{1} &        & \\
                                &           & \ddots & \\
                                &           &        & \beta_{k_{n}} \\
    \end{array} \right)
\end{align}
である。
すると,これより直ちに
\begin{equation}
  \Image \partial_{n+1} = \begin{pmatrix}
    \mathbb{Z} \alpha_{1} \\ \vdots \\ \mathbb{Z} \alpha_{k_{n+1}} \\ 0 \\ \vdots \\ 0
  \end{pmatrix}, \quad 
  \Kernel \partial_{n} = \begin{pmatrix}
    \mathbb{Z} \\ \vdots \\ \mathbb{Z} \\ 0 \\ \vdots \\ 0
  \end{pmatrix}
\end{equation}
と計算される。ここに,$\Kernel \partial_{n}$の$\mathbb{Z}$の数は$c_{n} - k_{n}$であることに注意。
いま,さらに
\begin{equation}
  \alpha_{1} = \alpha_{2} = \dots = \alpha_{\ell-1} = 1
\end{equation}
の関係が成り立っているとすると,$\mathbb{Z} / \mathbb{Z} = 0$に注意すれば
\begin{equation}
  H_{n} = \Kernel \partial_{n} / \Image \partial_{n+1}
  = \begin{pmatrix}
    \mathbb{Z} / \alpha_{1} \mathbb{Z} \\ \vdots \\ \mathbb{Z} / \alpha_{k_{n+1}} \mathbb{Z} \\ \mathbb{Z} \\ \vdots \\ \mathbb{Z} \\ 0 \\ \vdots \\ 0
  \end{pmatrix}
  = \mathbb{Z} / \alpha_{\ell} \mathbb{Z} \oplus \dots \oplus \mathbb{Z} / \alpha_{k_{n+1}} \mathbb{Z} \oplus \mathbb{Z}^{c_{n} - k_{n+1} - k_{n}}
\end{equation}
となる。
このうち,

\begin{equation}
  \mathbb{Z} / \alpha_{\ell} \mathbb{Z} \oplus \dots \oplus \mathbb{Z} / \alpha_{k_{n+1}} \mathbb{Z}
\end{equation}
の部分を\nw{ねじれ成分}と呼ぶ。
また,$H_{n}$の\nw{ランク},$\rank H_{n}$を
\begin{equation}
  \rank H_{n} := c_{n} - k_{n} - k_{n+1}
\end{equation}
で定める。つまり,境界作用素のSmith標準形を求めていけば,ホモロジー群$H_{n}$を決定することができる。

\sukima \midashi{\large Euler標数}

\begin{teigi}
  次の式\eqref{eq:homology3.Euler}で定まる量$\chi (X)$を$X$の\nw{Euler標数(Euler characteristic)}という。
  \begin{equation}
    \label{eq:homology3.Euler}
    \chi (X) := \sum_{n=0}^{\dim X} (-1)^{n} c_{n}
  \end{equation}
  ただし,$c_{n}$は$X$に含まれる$n$-単体の個数である。
\end{teigi}

\begin{prop}
  ${\displaystyle \chi (X) = \sum_{n=0}^{\dim X} (-1)^{n} \rank H_{n} (X)}$
\end{prop}

\begin{proof}
  定義から$\rank H_{n} (X) = c_{n} - k_{n} - k_{n+1}$であったので,
  \begin{equation}
    \sum_{n=0}^{\dim X} (-1)^{n} \rank H_{n} (X) = \sum_{n=0}^{\dim X} (-1)^{n} (c_{n} - k_{n} - k_{n+1}) = \sum_{n=0}^{\dim X} (-1)^{n} c_{n} = \chi (X)
  \end{equation}
\end{proof}

\midashi{注意.} ホモロジー群は\textbf{位相不変量}であり,複体構造に依存しない。
したがって,Euler標数も複体構造に依存せずに決まる。

\begin{rei}[射影平面の例]
  $\Delta_{3} \xrightarrow{\partial_{3}} \Delta_{2} \xrightarrow{\partial_{2}} \Delta_{1} \xrightarrow{\partial_{1}} \Delta_{0}$において,境界作用素の行列表示とSmith標準形はそれぞれ,
  \begin{align}
    A_{\partial_{2}} = \begin{array}{c@{\,}c}
      & \begin{array}{ccc}
        a & b & c
      \end{array} \\
      \begin{array}{c}
        U \\ L
      \end{array} & 
      \begin{pmatrix}
        -1 & 1 & 1 \\
        1 & -1 & 1
      \end{pmatrix}  
    \end{array}
    , \quad 
    A_{\partial_{1}} = \begin{array}{c@{\,}c}
      & \begin{array}{cc}
        u & v
      \end{array} \\
      \begin{array}{c}
        a \\ b \\ c
      \end{array} &
      \begin{pmatrix}
      -1 & 1 \\
      -1 & 1 \\
      0 & 0
    \end{pmatrix}
  \end{array}, \\
  \mathrm{Smith}(A_{\partial_{2}}) = \begin{pmatrix}
    1 & 0 & 0 \\ 0 & 2 & 0
  \end{pmatrix}, \quad
  \mathrm{Smith}(A_{\partial_{1}}) = \begin{pmatrix}
    0 & 0 \\
    0 & 0 \\
    1 & 0
  \end{pmatrix}
  \end{align}
  となる。これより,
  \begin{equation}
    H_{2}(P^{2}) = 0, \quad H_{1}(P^{2}) = \mathbb{Z} / 2 \mathbb{Z}, \quad H_{0}(P^{2}) = \mathbb{Z}
  \end{equation}
  であり,Euler標数は
  \begin{equation}
    \chi (P^{2}) = 1
  \end{equation}
  と計算される。
\end{rei}

\begin{rei}[2次元球面の例]
  複体構造の図は
  \begin{center}
		\begin{tikzpicture}
		%% 面を塗る
		\fill[red!20] (0,0) -- (2,2) -- (0,2) -- cycle;
		\fill[blue!20] (0,0) -- (2,2) -- (2,0) -- cycle;
		%% 点の名前もろもろ
		\fill \point{(0,0)}{below left}{$u$} \point{(2,0)}{below right}{$v$} \point{(0,2)}{above left}{$v$} \point{(2,2)}{above right}{$w$};
		\node at (1,0) [below]{$b$} node at (0,1) [left]{$b$} node at (1,2) [above]{$a$} node at (2,1) [right]{$a$} node at (1,1) [left]{$c$};
		\node at (0.5,1.5){$U$} node at (1.5,0.5){$L$};
		%% 矢印つき
		\midwayarrow{(0,0)}{(0,2)} \midwayarrow{(0,0)}{(2,0)} \midwayarrow{(0,2)}{(2,2)} \midwayarrow{(2,0)}{(2,2)} \midwayarrow{(0,0)}{(2,2)}
		\end{tikzpicture}
  \end{center}
  であるから,
  \begin{equation}
    \chi (S^{2}) = c_{0} - c_{1} + c_{2} = 3 - 3 + 2 = 2
  \end{equation}
  と求まる。因みに
  \begin{equation}
    H_{0}(S^{2}) = \mathbb{Z}, \quad H_{1}(S^{2}) = 0, \quad H_{2}(S^{2}) = \mathbb{Z}
  \end{equation}
\end{rei}

\expandafter\ifx\csname readornot\endcsname\relax
  \end{document}
\fi