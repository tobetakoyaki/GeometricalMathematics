\expandafter\ifx\csname readornot\endcsname\relax
  \documentclass[uplatex]{jsarticle}    
  \usepackage{octopus}
  \usepackage{url}
  %%%% コマンド定義専用のtexファイル
\renewcommand{\postpartname}{章}
\renewcommand{\thepart}{\arabic{part}}
\renewcommand{\thesection}{\thepart.\arabic{section}}
\makeatletter\renewcommand{\theequation}{\thesection.\arabic{equation}}\@addtoreset{equation}{section}\makeatother
\newcommand{\octopuspart}[1]{\newpage\part{#1}\setcounter{section}{0}\vspace{3\baselineskip}}
\renewcommand{\restriction}[2]{\left. #1 \right|_{#2}}
\DeclareMathOperator{\dcup}{\dot{\cup}}
\DeclareMathOperator{\conv}{conv}
\DeclareMathOperator{\Image}{Im}
\DeclareMathOperator{\Kernel}{Ker}
\DeclareMathOperator{\diag}{diag}
\DeclareMathOperator{\rank}{rank}
\DeclareMathOperator{\sgn}{sgn}
\DeclareMathOperator{\rot}{rot}
  \usetikzlibrary{calc}
  \begin{document}
\fi

\definecolor{mygreen}{rgb}{0,0.75,0}
\section{基本群}

\midashi{\large これからの展望}

\begin{center}
  \begin{tikzpicture}
    \draw (0,0) circle [x radius=1, y radius=0.5] node {空間 $X$};
    \draw (0,-2) circle [x radius=1, y radius=0.5] node {空間 $Y$};
    \node at (3.5,0) {群$\pi_{1}(X)$};
    \node at (3.5,-2) {群$\pi_{1}(Y)$};
    \node[left] at (0,-1) {ホモトピー同値\hspace{1zw}\rotatebox[origin=c]{-90}{$\simeq$}};
    \node[right] at (3.5,-1) {\rotatebox[origin=c]{-90}{$\simeq$}\hspace{1zw}群同型};
    \draw[->] (1.5,0) -- (2.5,0);
    \draw[->] (1.5,-2) -- (2.5,-2);
  \end{tikzpicture}
\end{center}
となるような空間と群の対応を構成したい。

\sukima \midashi{\large 基本群}

$X$:位相空間
\begin{teigi}[パス]
  \nw{パス(path)} $\defines$ $f \colon [0,1] \longrightarrow X$:連続
\end{teigi}

\begin{teigi}[ホモトピー]
  $\sets{f_{t}}$:パスの\nw{ホモトピー} $\defines$ 次の条件を満たす$\sets{f_{t} \colon [0,1] \longrightarrow X}_{t \in [0,1]}$のこと:
  \begin{itemize}
    \vspace{-0.5\baselineskip}
    \item $\forall t \in [0,1],\quad f_{t}(0) = x_{0},\quad f_{t}(1) = x_{1}$
    \item $F \colon \mapdef{[0,1] \times [0,1]}{X}{(s,t)}{f_{t}(s)}$が連続
  \end{itemize}
\end{teigi}

\begin{teigi}[ホモトープ]
  2つのパス$f', f''$が\nw{ホモトープ} $\defines$ $f', f''$をつなぐホモトピー$\sets{f_{t}}$であって,
  $f_{0} = f'$,$f_{1} = f''$であるものが存在する

  このことを$f' \simeq f''$と書く。
\end{teigi}

\begin{prop}
  ホモトープの$\simeq$は同値関係である。
\end{prop}

\begin{proof}
  パスのホモトープは普通のホモトープの特別な場合であるから,そのときの同値性から従う。
\end{proof}

ホモトープの同値関係$\simeq$によるパス$f$の同値類$[f]$を\nw{ホモトピー類}という。

\begin{teigi}[パスの積(合成)]
  $f,g \colon [0,1] \longrightarrow X$:パス,$f(1) = g(0)$を満たすとする。
  パスの\nw{合成} $f \cdot g \colon [0,1] \longrightarrow X$を次で定義する。
  \begin{equation}
    f \cdot g (s) := \begin{cases}
      f(2s) & \left( 0 \le s \le \dfrac{1}{2} \right) \\
      g(2s-1) & \left( \dfrac{1}{2} \le s \le 1 \right)
    \end{cases}
  \end{equation}
\end{teigi}

\sukima \midashi{注意.} $f' \simeq f$,$g' \simeq g$ならば$f' \cdot g' \simeq f \cdot g$である。

なぜならば,$f$と$f'$を結ぶホモトピー$\sets{f_{t}}$,$g$と$g'$を結ぶホモトピー$\sets{g_{t}}$に対して
$\left( f \cdot g \right)_{t} := f_{t} \cdot g_{t}$とすれば,これが$f' \cdot g'$と$f \cdot g$を結ぶホモトピーになるから。

\begin{teigi}[ループ]
  $x_{0}$を\nw{基点(basepoint)とするループ(loop)} $\defines$ パス$f \colon [0,1] \longrightarrow X$であって,$f(0) = f(1) = x_{0}$
\end{teigi}

ループに対するホモトープの同値関係(パスのそれと同じように定められる)による同値類を考える。
集合$\pi_{1}(X, x_{0})$を次で定める。
\begin{equation}
  \pi_{1}(X, x_{0}) := \sets{[f] | f \text{:$x_{0}$を基点とするループ}}
\end{equation}

\begin{prop}
  $\pi_{1}(X,x_{0})$は積$[f] \cdot [g] := [f \cdot g]$の下で
  \footnote{もう少し正確にいうと,$\pi_{1}(X,x_{0})$のホモトピー類$F,G$の積$F \cdot G$を
  $F = [f]$,$G = [g]$となるようなループ$f,g$をとって,
  $F \cdot G := [f \cdot g]$と定義した。}\
  群になる。
\end{prop}

\begin{proof}
  \midashi{[積の定義のwell-defined性]} 

  まず,$x_{0}$を基点とするループの合成も$x_{0}$を基点とするループであることに注意する。
  $f' \in [f]$,$g' \in [g]$をとって$[f' \cdot g'] = [f \cdot g]$を示せばよい。
  これは$f' \simeq f$,$g' \simeq g$であるから,先の{\bf 注意.}より$f' \cdot g' \simeq f \cdot g$であるから従う。

  \sukima \midashi{[準備]}
  
  $\varphi \colon [0,1] \longrightarrow [0,1]$,$\varphi(0) = 0$,$\varphi(1) = 1$とする。
  このとき,ホモトピー$(f \circ \varphi)_{t} := f \circ \left( (1-t) \varphi + t \, \mathrm{id} \right)$を考えれば,
  これは$f \circ \varphi$と$f$を結ぶホモトピーであり,$(f \circ \varphi)_{0} = f \circ \varphi$,$(f \circ \varphi)_{1} = f$である。
  よって$[f \circ \varphi] = [f]$である。

  この$\varphi$を適切に定めることでパスの進むスピードを変換する。

  \sukima \midashi{[単位元の存在]}

  $c \colon [0,1] \longrightarrow X$を$c(s) \equiv x_{0}$(定ループ)で定める。$[c]$は単位元になる。
  \begin{quote}
    $\because)$ $f$を$x_{0}$を基点とするループとして,$c \cdot f$を考える。
    時刻$s$を
    \begin{center}
      \begin{tikzpicture}
        \filldraw (0,0) circle (0.05);
        \filldraw (0,1) circle (0.05);
        \filldraw (0,2) circle (0.05);
        \filldraw (2,0) circle (0.05);
        \filldraw (2,2) circle (0.05);
        \draw[->] (0,0) node[left]{0} -- (0,1) node[left]{1/2} -- (0,2) node[left]{1} -- (0,2.2);
        \draw[->] (2,0) node[right]{0} -- (2,2) node[right]{1} -- (2,2.2);
        \draw[dashed] (0,0) -- (2,0);
        \draw[dashed] (0,1) -- (2,0);
        \draw[dashed] (0,2) -- (2,2);
        \draw[->, mygreen] (0.3,1.8) -- (1.7,1.8);
        \draw[->, mygreen] (0.3,1.2) -- (1.7,0.5);
        \draw[->, mygreen] (0.3,0.75) -- (1.7,0.05);
        \draw[->, mygreen] (0.3,-0.1) -- (1.7,-0.1);
        \node at (1,1.5){\textcolor{mygreen}{$\varphi$}};
      \end{tikzpicture}
    \end{center}
    のように変換する。(つまり時刻1/2まで立ち止まってから2倍のスピードで移動する。)
    このとき,$f \circ \varphi = c \cdot f$である。
    よって,$[f] = [f \circ \varphi] = [c \cdot f] = [c] \cdot [f]$である。
    $[f] \cdot [c] = [f]$も同様。
  \end{quote}

  \sukima \midashi{[結合律]}
  
  $[f \cdot (g \cdot h)] = [(f \cdot g) \cdot h]$を示せばよい。
  時刻のスピードを
  \begin{center}
    \begin{tikzpicture}
      \filldraw (0,0) circle (0.05);
      \filldraw (0,1) circle (0.05);
      \filldraw (0,1.5) circle (0.05);
      \filldraw (0,2) circle (0.05);
      \filldraw (2,0) circle (0.05);
      \filldraw (2,0.5) circle (0.05);
      \filldraw (2,1) circle (0.05);
      \filldraw (2,2) circle (0.05);
      \draw[->] (0,0) node[left]{0} -- (0,1) node[left]{1/2} -- (0,1.5) node[left]{3/4} -- (0,2) node[left]{1} -- (0,2.2);
      \draw[->] (2,0) node[right]{0} -- (2,0.5) node[right]{1/4} -- (2,1) node[right]{1/2} -- (2,2) node[right]{1} -- (2,2.2);
      \draw[dashed] (0,0) -- (2,0);
      \draw[dashed] (0,1) -- (2,0.5);
      \draw[dashed] (0,1.5) -- (2,1);
      \draw[dashed] (0,2) -- (2,2);
      \draw[->, mygreen] (0.3,1.9) -- (1.7,1.9);
      \draw[->, mygreen] (0.3,1.6) -- (1.7,1.25);
      \draw[->, mygreen] (0.3,1.2) -- (1.7,0.85);
      \draw[->, mygreen] (0.3,0.75) -- (1.7,0.4);
      \draw[->, mygreen] (0.3,0.1) -- (1.7,0.1);
      \node at (1,-0.25){\textcolor{mygreen}{$\varphi$}};
      \node at (0.2,0.5){$f$};
      \node at (0.2,1.25){$g$};
      \node at (0.2,1.75){$h$};
      \node at (1.8,0.25){$f$};
      \node at (1.8,0.75){$g$};
      \node at (1.8,1.5){$h$};
    \end{tikzpicture}
  \end{center}
  のように変換する$\varphi$を考える。このとき,$(f \cdot (g \cdot h)) \circ \varphi = (f \cdot g) \cdot h$となる。
  したがって,$[f \cdot (g \cdot h)] = [(f \cdot g) \cdot h]$である。

  \sukima \midashi{[逆元]}
  
  $f \colon [0,1] \longrightarrow X$に対し,$\overline{f} \colon [0,1] \longrightarrow X$を
  $\overline{f}(s) := f(1-s)$で定義すると,$[f][\overline{f}] = [\overline{f}][f] = [c]$である。

  \begin{quote}
    $\because)$ $f \cdot \overline{f} \simeq c$を示す。
    $f_{t}$と$g_{t}$を
    \begin{align}
      & f_{t}(s) := \begin{cases}
        f(s) & (0 \le s \le t) \\
        f(t) & (t \le s \le 1)
      \end{cases}, \\
      & g_{t}(s) := \begin{cases}
        \overline{f}(1-t) = f(t) & (0 \le s \le 1-t) \\
        \overline{f}(s) & (1-t \le s \le 1)
      \end{cases}
    \end{align}
    で定義すると,$f_{t} \cdot g_{t}$は$f \cdot \overline{f}$と$c$を結ぶホモトピーになる。
  \end{quote}
\end{proof}

\begin{teigi}[基本群]
  $\pi_{1}(X,x_{0})$を,$x_{0}$を基点とする$X$の\nw{基本群(fundamental group)}という。
\end{teigi}

\sukima \midashi{\large 基点のとりかえ}

\renewcommand{\arraystretch}{1.0}
\begin{tabular}{lll}
  $x_{0}, x_{1} \in X$, & $h \colon [0,1] \longrightarrow X$,$h(0) = x_{0}$,$h(1) = x_{1}$ &($x_{0}$から$x_{1}$へのパス)\\
  & $\overline{h} \colon [0,1] \longrightarrow X$,$\overline{h}(s) = h(1-s)$ &($x_{1}$から$x_{0}$へのパス)\\
  \multicolumn{2}{l}{$f$:$x_{0}$を基点とするループ} \\
  \multicolumn{2}{l}{$\Rightarrow h \cdot f \cdot \overline{h}$:$x_{0}$を基点とするループ}
\end{tabular}
\renewcommand{\arraystretch}{1.3}

\begin{tikzpicture}
	\filldraw (0,0) circle [radius=0.05];
	\filldraw (2,0.8) circle [radius=0.05];
	\draw (0,0) node[above]{$x_{1}$} .. controls (1,0.35) .. (2,0.8) node[above left]{$x_{0}$};
	\draw (3,1) circle [x radius = 1, y radius = 0.8, rotate=15];
	\draw[->] (0.7,0.5) -- (1.5,0.8);
	\draw[<-] (0.8,0.1) -- (1.6,0.4);
	\draw (2.4,1.5) -- (2.4,1.6) -- (2.3,1.6);
	\draw (3.7,0.4) -- (3.6,0.4) -- (3.6,0.5);
	\draw (4,1.8) node{$f$};
	\draw (1,0.9) node{$h$};
	\draw (1.5,0) node{$\overline{h}$};
\end{tikzpicture}

いま,ホモトピー$\sets{f_{t}}$によって$f' \simeq f$であるとすると,
ホモトピー$\sets{h \cdot f_{t} \cdot \overline{h}}$によって$h \cdot f \cdot \overline{h} \simeq h \cdot f' \cdot \overline{h}$である。
したがって,
\begin{equation}
  \beta_{h} \colon \mapdef{\pi_{1}(X,x_{0})}{\pi_{1}(X,x_{1})}{\left[ f \right]}{\left[ h \cdot f \cdot \overline{h} \, \right]}
\end{equation}
はwell-definedになる。

\begin{prop}
  $\beta_{h}$は$\pi_{1}(X,x_{0})$から$\pi_{1}(X,x_{1})$への同型写像
\end{prop}

\begin{proof}
  \midashi{[準同型性]}
  \begin{equation}
    \beta_{h}([f] \cdot [g]) = [h \cdot f \cdot g \cdot \overline{h}] = [h \cdot f \cdot \overline{h} \cdot h \cdot g \cdot \overline{h}] = [h \cdot f \cdot \overline{h}] \cdot [h \cdot g \cdot \overline{h}] = \beta_{h}([f]) \cdot \beta_{h}([g])
  \end{equation}

  \sukima \midashi{[全単射性]}

  $\beta_{\overline{h}} \colon \pi_{1}(X, x_{0}) \longrightarrow \pi_{1}(X, x_{1})$を考える。
  \begin{equation}
    \beta_{\overline{h}} \circ \beta_{h} ([f]) = \beta_{\overline{h}} ([h \cdot f \cdot \overline{h}]) = [\overline{h} \cdot h \cdot f \cdot \overline{h} \cdot h] = [f]
  \end{equation}
  であるから$\beta_{\overline{h}} \circ \beta_{h} = \mathrm{id}$である。同様に$\beta_{h} \circ \beta_{\overline{h}} = \mathrm{id}$である。
  よって$\beta_{\overline{h}}$が$\beta_{h}$の逆写像である。
\end{proof}

\begin{prop}
  $X$が弧状連結ならば,任意の点$x_{0} \in X$に対して$x_{0}$を基点とする基本群$\pi_{1}(X,x_{0})$はすべて同型。
\end{prop}

そこで,$X$が弧状連結のときには基本群を単に$\pi_{1}(X)$と書く。

\begin{rei}
  $X \subseteq \mathbb{R}^{n}$:凸集合とする。任意のループ$f$は定ループ$c$とホモトープであるから,$\pi_{1}(X) = \sets{e}$(自明な群)である。
  同様に,$X$が可縮空間であるときも$\pi_{1}(X) = \sets{e}$である。
\end{rei}

\begin{teigi}[単連結]
  $X$:\nw{単連結(simply-connected)} $\defines$ $X$:弧状連結 かつ $\pi_{1}(X) = \sets{e}$
\end{teigi}

\begin{prop}
  $X$:単連結 $\Longleftrightarrow$ 「$\forall x, y \in X$,$f, f'$:$x$から$y$へのパス $\Longrightarrow$ $f \simeq f'$」
\end{prop}

\begin{proof}
  $x = y$のときを考えれば$\Longleftarrow$向きは明らかである。
  $\Longrightarrow$向きについては,$f \cdot \overline{f'}$がループであり$\pi_{1}(X) = \sets{e}$であることに注意すれば$f \simeq f \cdot \overline{f'} \cdot f' \simeq f'$である。 
\end{proof}

\begin{teigi}
  $\varphi \colon X \longrightarrow Y$:連続,$y_{0} = \varphi(x_{0})$とする。写像$\varphi_{*}$を次のように定める。
  \begin{equation}
    \varphi_{*} \colon \mapdef{\pi_{1}(X,x_{0})}{\pi_{1}(Y,y_{0})}{\left[ f \right]}{\left[ \varphi \circ f \right]}
  \end{equation}
\end{teigi}

\begin{hodai}
  $\varphi_{*}$はwell-definedであり準同型
\end{hodai}

\begin{proof}
  \midashi{[well-definedness]}

  ホモトピー$\sets{f_{t}}$によって$f \simeq f'$であるならば,ホモトピー$\sets{\varphi \circ f_{t}}$によって$\varphi \circ f \simeq \varphi \circ f'$である。

  \sukima \midashi{[準同型性]}
  \begin{equation}
    \varphi_{*} ([f] \cdot [g]) = \varphi_{*} ([f \cdot g]) = [ \varphi \circ (f \cdot g)] = [(\varphi \circ f) \cdot (\varphi \circ g)] = [\varphi \circ f] \cdot [\varphi \circ g] = \varphi_{*}([f]) \cdot \varphi_{*} ([g])
  \end{equation}
\end{proof}

\begin{hodai}
  \midashi{(1)} $\varphi \colon X \longrightarrow Y$,$\phi \colon Y \longrightarrow Z$に対して,$(\phi \circ \varphi)_{*} = \phi_{*} \circ \varphi_{*}$である。

  \midashi{(2)} $(\mathrm{id}_{X})_{*} = \mathrm{id}_{\pi_{1}(X,x_{0})}$
\end{hodai}

\begin{proof}
  \midashi{(1)} $(\phi \circ \varphi)_{*}([f]) = [\phi \circ \varphi \circ f] = \phi_{*}([\varphi \circ f]) = (\phi_{*} \circ \varphi_{*}) ([f])$

  \midashi{(2)} 明らか
\end{proof}

\begin{teiri}
  $\varphi \colon X \longrightarrow Y$:同相 $\Longrightarrow$ $\varphi_{*} \colon \pi_{1} (X, x_{0}) \longrightarrow \pi_{1} (Y, \varphi (x_{0}))$:同型
\end{teiri}

\begin{proof}
  $\phi_{*}$が準同型写像であることは既に示したことに注意する。

  いま,$\varphi^{-1} \circ \varphi = \mathrm{id}_{X}$から$(\varphi^{-1})_{*} \circ \varphi_{*} = \mathrm{id}_{\pi_{1}(X,x_{0})}$である。
  また,$\varphi \circ \varphi^{-1} = \mathrm{id}_{Y}$から$\varphi_{*} \circ (\varphi^{-1})_{*} = \mathrm{id}_{\pi_{1}(Y,\varphi(x_{0}))}$である。
  よって,$\varphi_{*}$は同型写像。
\end{proof}

\begin{corr}
  $\pi_{1}(X) \not\simeq \pi_{1}(Y)$なら$X$と$Y$は同相でない
\end{corr}

\begin{teiri}
  \label{kihongun.homequiv}
  $\varphi \colon X \longrightarrow Y$がホモトピー同値写像なら
  $\varphi_{*} \colon \pi_{1} (X, x_{0}) \longrightarrow \pi_{1} (Y, \varphi(x_{0}))$は同型写像
\end{teiri}

この\rref{定理}{kihongun.homequiv}の証明に際して,次の補題を用いる。

\begin{hodai}
  $\rho, \rho' \colon X \longrightarrow Y$がホモトピー$\sets{\rho_{t}}$によってホモトピー同値であり,
  $h \colon t \longmapsto \rho_{t}(x_{0})$:パス,とする。
  このとき,$\rho_{*} = \beta_{h} \circ \rho'_{*}$である。
  \begin{tikzpicture}
    \filldraw (0,0) circle [radius=0.05];
    \draw (0,0) node[below]{$x_{0}$} .. controls (2.5,0) and (0,2) .. (0,0);
    \node at (0.6,1.2) {$f$};
    \draw[->] (1.5,0) -- (2,0);
    \node at (1.75,0) [below]{$\rho_{t}$};
    \filldraw (2.8,1) circle [radius=0.05];
    \draw (2.8,1) node[above=8, left=-3]{$\rho'(x_{0})$} .. controls (4.5,0) and (4.3,2) .. (2.8,1);
    \filldraw (2.5,0) circle [radius=0.05];
    \draw (2.5,0) .. controls (4.2,-1) and (4,1) .. (2.5,0);
    \filldraw (2.3,-1) circle [radius=0.05];
    \draw (2.3,-1) node[below=8, left=-3]{$\rho(x_{0})$} .. controls (4,-2) and (3.8,0) .. (2.3,-1);
    \draw (2.3,-1) .. controls (2.5,0) .. (2.8,1);
    \node at (2.3,0.4) {$h$};
    \node at (4.5,1.3) {$\rho' \circ f$};
    \node at (4.2,0.3) {$\rho_{t} \circ f$};
    \node at (4,-0.7) {$\rho \circ f$};
  \end{tikzpicture}
\end{hodai}

\begin{proof}
  \begin{equation}
    (\beta_{h} \circ \rho'_{*}) ([f]) = \beta_{h} ([\rho' \circ f]) = [h \cdot \rho' \circ f \cdot \overline{h}] = [\rho \circ f] = \rho_{*} ([f])
  \end{equation}
  による。なお,後ろから2つ目の等号はホモトピー$\sets{h_{t} \cdot \rho_{t} \circ f \cdot \overline{h}_{t}}$による。
\end{proof}

\begin{proof}[{\bf \rref{定理}{kihongun.homequiv}の証明}]
  ホモトピー$\sets{\rho_{t}}$によって$\phi \circ \varphi \simeq \mathrm{id}$となるならば,あるパス$h$が存在して$\phi_{*} \circ \varphi_{*} = \beta_{h}$となる。
  特に写像$\phi_{*} \circ \varphi_{*}$は$\pi_{1}(X, x_{0})$と$\pi_{1}(X, (\phi \circ \varphi)(x_{0}))$の間の同型写像である。このことから$\phi_{*}$は全射,$\varphi_{*}$は単射である。
  同様に$\varphi_{*} \circ \phi_{*}$は同型写像であり,$\varphi_{*}$は全射,$\phi_{*}$は単射である。よって$\varphi_{*}$は全単射である。
\end{proof}

\sukima \midashi{\large $S^{n}$の基本群について}

\begin{teiri}
  $n \ge 2$に対して,$\pi_{1}(S^{n}) = \sets{e}$である。つまり,2次元以上の球面は単連結である。
\end{teiri}

\begin{proof}
  $f$を$x_{0} \in S^{n}$を基点とするループ,とする。

  このとき,$y \in S^{n}$であって$y \neq x_{0}$,$y \notin f([0,1])$を満たす点をとることができる。({\bf 注意.}へ)

  いま,$S^{n} \setminus \sets{y} \simeq \mathbb{R}^{n}$なので,$f$を$\mathbb{R}^{n}$のループとみると,
  これは1点にホモトピー同値である。
\end{proof}

\midashi{注意.} このような条件を満たす$y$がとれることは直感的には明らかであるが,本来は証明が必要な事柄である。

\midashi{問題.} では,これを証明せよ。

\sukima \midashi{注意.} $S^{1}$では$y \notin f([0,1])$をとることができない。

\sukima \midashi{\large Poincar\'e予想}\footnote{Перельман(Perelman)により証明された}
\begin{teiri}
  単連結なコンパクト3次元多様体は$S^{3}$に同相である
\end{teiri}

このことは2次以上のコンパクト多様体についても同様なことがいえる。

\sukima \midashi{\large $S^{1}$の基本群について}

ループ$f$が円周を$n$回「まわり」,$g$が円周を$m$回「まわる」とする。このとき,$f \cdot g$は円周を$n + m$回「まわる」といえる。
この写像$\pi_{1}(S^{1}) \longrightarrow \mathbb{Z}$は同型写像を与えるのではないかと思える。

このことをあとできちんと示す。

\begin{teiri}
  $\pi_{1}(S^{1}) \simeq \mathbb{Z}$
\end{teiri}

証明は,被覆空間の考え方を用いるので,後に回す。

\sukima \midashi{\large 直積の基本群}

\begin{teiri}
  $\pi_{1} (X \times Y, (x_{0}, y_{0})) \simeq \pi_{1}(X, x_{0}) \times \pi_{1}(Y, y_{0})$
\end{teiri}

\begin{proof}
  $f$:$(x_{0}, y_{0})$を基点とする$X \times Y$上のループ,とする。
  $f \colon [0,1] \longrightarrow X \times Y$であるから

  \renewcommand{\arraystretch}{1.0}
  \begin{tabular}{lll}
    $f = (g,h)$, & $g \colon [0,1] \longrightarrow X$, & $g(0) = g(1) = x_{0}$, \\
    & $h \colon [0,1] \longrightarrow Y$, & $h(0) = h(1) = y_{0}$
  \end{tabular}
  \renewcommand{\arraystretch}{1.3}

  と表せる。$g,h$は$X,Y$上のループである。
  
  写像
  \begin{equation}
    \varphi \colon \mapdef{\pi_{1}(X \times Y, (x_{0}, y_{0}))}{\pi_{1}(X, x_{0}) \times \pi_{1}(Y, y_{0})}{\left[ f \right]}{(\left[ g \right], \left[ h \right])}  
  \end{equation}
  は準同型写像である。(well-defined性も準同型性も示せる)

  同様に
  \begin{equation}
    \overline{\varphi} \colon \mapdef{\pi_{1}(X, x_{0}) \times \pi_{1}(Y, y_{0})}{\pi_{1}(X \times Y, (x_{0}, y_{0}))}{(\left[ g \right], \left[ h \right])}{\left[ (g,h) \right]}
  \end{equation}
  を考える。
  
  すると,$\overline{\varphi} \circ \varphi = \mathrm{id}$,$\varphi \circ \overline{\varphi} = \mathrm{id}$であり,$\simeq$が成り立つ。
\end{proof}

\sukima \midashi{\large トーラスの基本群}
\begin{corr}
  $\pi_{1}(T^{2}) \simeq \mathbb{Z} \times \mathbb{Z}$
\end{corr}

\begin{proof}
  $T^{2} = S^{1} \times S^{1}$であることから従う。
\end{proof}

\sukima \midashi{\large wedge和の基本群}

$X \vee Y := X \sqcup Y / \sim$,$\sim$を$x \in X$と$y \in Y$を同一視するような同値関係($x \sim y$)として定める。
$\pi_{1}(X)$と$\pi_{1}(Y)$から$\pi_{1}(X \vee Y)$が定まるであろうと考えられる。

\begin{teigi}[準備:群の自由積]
  $G,G'$:群とする。群の\nw{自由積(free product)}を
  \begin{equation}
    G * G' := \sets{g_{1}g_{2} \cdots g_{n} | n \ge 0, \quad g_{i} \in G \, \text{or} \, G', \quad g_{i} \neq e}
  \end{equation}
  とする。但し,$g_{1} \cdots g_{i} g_{i+1} \cdots g_{n}$が,$g_{i}, g_{i+1} \in G$または$g_{i}, g_{i+1} \in G'$であって$g_{i} \cdot g_{i+1} = h$であるならば
  これを$g_{1} \cdots g_{i-1} h g_{i+2} \cdots g_{n}$と同じ元であるとみなす。$h = e$ならばさらに$g_{1} \cdots g_{i-1} g_{i+2} \cdots g_{n}$と同じものであると見なす。
  ここで,積$\cdot$は列の連結:
  \begin{equation}
    ( g_{1} g_{2} \cdots g_{n}) \cdot (h_{1} h_{2} \cdots h_{m}) := g_{1} g_{2} \cdots g_{n} h_{1} h_{2} \cdots h_{m}
  \end{equation}
  で定める。
\end{teigi}

\begin{prop}
  $G * G'$は積$\cdot$により群をなす。単位元は空列である。
\end{prop}

この{\bf 命題}は自明ではないが,証明は演習とする。

\begin{teiri}
  \label{kihongun.Seifelt}
  $\pi_{1} (X \vee Y) \simeq \pi_{1} (X) * \pi_{1} (Y)$
\end{teiri}

\begin{corr}
  $\pi_{1} (S^{1} \vee S^{1} \vee \dots \vee S^{1}) \simeq \mathbb{Z} * \mathbb{Z} * \dots * \mathbb{Z}$
\end{corr}

意味を考えれば{\bf 確かにそんな気がする}。

\sukima \midashi{演習.} \rref{定理}{kihongun.Seifelt}はSeifelt-van Kampenの定理と呼ばれるものの特殊ケースである。
この定理について調べ(証明して),いろいろな空間・曲面の基本群を計算せよ。

\expandafter\ifx\csname readornot\endcsname\relax
  \end{document}
\fi