\expandafter\ifx\csname readornot\endcsname\relax
  \documentclass[uplatex]{jsarticle}
  \usepackage{octopus}
  \usepackage{url}

  \renewcommand{\proofname}{\textsf{証明}}
  \renewcommand{\postpartname}{章}
  \renewcommand{\thesection}{\thepart.\arabic{section}}
  \renewcommand{\thepart}{\arabic{part}}
  \makeatletter\renewcommand{\theequation}{\thesection.\arabic{equation}}\@addtoreset{equation}{section}\makeatother

  \newcommand{\octopuspart}[1]{\newpage\part{#1}\setcounter{section}{0}\vspace{3\baselineskip}}

  \DeclareMathOperator{\dcup}{\dot{\cup}}
  \begin{document}
\fi

\definecolor{mygreen}{rgb}{0,0.75,0}
\section{基本群}

\midashi{\large これからの展望}

\begin{center}
  \begin{tikzpicture}
    \draw (0,0) circle [x radius=1, y radius=0.5] node {空間 $X$};
    \draw (0,-2) circle [x radius=1, y radius=0.5] node {空間 $Y$};
    \node at (3.5,0) {群$\pi_{1}(X)$};
    \node at (3.5,-2) {群$\pi_{1}(Y)$};
    \node[left] at (0,-1) {ホモトピー同値\hspace{1zw}\rotatebox[origin=c]{-90}{$\simeq$}};
    \node[right] at (3.5,-1) {\rotatebox[origin=c]{-90}{$\simeq$}\hspace{1zw}群同型};
    \draw[->] (1.5,0) -- (2.5,0);
    \draw[->] (1.5,-2) -- (2.5,-2);
  \end{tikzpicture}
\end{center}
となるような空間と群の対応を構成したい。

\sukima \midashi{\large 基本群}

$X$:位相空間
\begin{teigi}[パス]
  \nw{パス(path)} $\defines$ $f \colon [0,1] \longrightarrow X$:連続
\end{teigi}

\begin{teigi}[ホモトピー]
  $\sets{f_{t}}$:パスの\nw{ホモトピー} $\defines$ 次の条件を満たす$\sets{f_{t} \colon [0,1] \longrightarrow X}_{t \in [0,1]}$のこと:
  \begin{itemize}
    \vspace{-0.5\baselineskip}
    \item $\forall t \in [0,1],\quad f_{t}(0) = x_{0},\quad f_{t}(1) = x_{1}$
    \item $F \colon \mapdef{[0,1] \times [0,1]}{X}{(s,t)}{f_{t}(s)}$が連続
  \end{itemize}
\end{teigi}

\begin{teigi}[ホモトープ]
  2つのパス$f', f''$が\nw{ホモトープ} $\defines$ $f', f''$をつなぐホモトピー$\sets{f_{t}}$であって,
  $f_{0} = f'$,$f_{1} = f''$であるものが存在する

  このことを$f' \simeq f''$と書く。
\end{teigi}

\begin{prop}
  ホモトープの$\simeq$は同値関係である。
\end{prop}

\begin{proof}
  パスのホモトープは普通のホモトープの特別な場合であるから,そのときの同値性から従う。
\end{proof}

ホモトープの同値関係$\simeq$によるパス$f$の同値類$[f]$を\nw{ホモトピー類}という。

\begin{teigi}[パスの積(合成)]
  $f,g \colon [0,1] \longrightarrow X$:パス,$f(1) = g(0)$を満たすとする。
  パスの\nw{合成} $f \cdot g \colon [0,1] \longrightarrow X$を次で定義する。
  \begin{equation}
    f \cdot g (s) := \begin{cases}
      f(2s) & \left( 0 \le s \le \dfrac{1}{2} \right) \\
      g(2s-1) & \left( \dfrac{1}{2} \le s \le 1 \right)
    \end{cases}
  \end{equation}
\end{teigi}

\sukima \midashi{注意.} $f' \simeq f$,$g' \simeq g$ならば$f' \cdot g' \simeq f \cdot g$である。

なぜならば,$f$と$f'$を結ぶホモトピー$\sets{f_{t}}$,$g$と$g'$を結ぶホモトピー$\sets{g_{t}}$に対して
$\left( f \cdot g \right)_{t} := f_{t} \cdot g_{t}$とすれば,これが$f' \cdot g'$と$f \cdot g$を結ぶホモトピーになるから。

\begin{teigi}[ループ]
  $x_{0}$を\nw{基点(basepoint)とするループ(loop)} $\defines$ パス$f \colon [0,1] \longrightarrow X$であって,$f(0) = f(1) = x_{0}$
\end{teigi}

ループに対するホモトープの同値関係(パスのそれと同じように定められる)による同値類を考える。
集合$\pi_{1}(X, x_{0})$を次で定める。
\begin{equation}
  \pi_{1}(X, x_{0}) := \sets{[f] | f \text{:$x_{0}$を基点とするループ}}
\end{equation}

\begin{prop}
  $\pi_{1}(X,x_{0})$は積$[f] \cdot [g] := [f \cdot g]$の下で
  \footnote{もう少し正確にいうと,$\pi_{1}(X,x_{0})$のホモトピー類$F,G$の積$F \cdot G$を
  $F = [f]$,$G = [g]$となるようなループ$f,g$をとって,
  $F \cdot G := [f \cdot g]$と定義した。}\
  群になる。
\end{prop}

\begin{proof}
  \midashi{[積の定義のwell-defined性]} 

  まず,$x_{0}$を基点とするループの合成も$x_{0}$を基点とするループであることに注意する。
  $f' \in [f]$,$g' \in [g]$をとって$[f' \cdot g'] = [f \cdot g]$を示せばよい。
  これは$f' \simeq f$,$g' \simeq g$であるから,先の{\bf 注意.}より$f' \cdot g' \simeq f \cdot g$であるから従う。

  \sukima \midashi{[準備]}
  
  $\varphi \colon [0,1] \longrightarrow [0,1]$,$\varphi(0) = 0$,$\varphi(1) = 1$とする。
  このとき,ホモトピー$(f \circ \varphi)_{t} := f \circ \left( (1-t) \varphi + t \, \mathrm{id} \right)$を考えれば,
  これは$f \circ \varphi$と$f$を結ぶホモトピーであり,$(f \circ \varphi)_{0} = f \circ \varphi$,$(f \circ \varphi)_{1} = f$である。
  よって$[f \circ \varphi] = [f]$である。

  この$\varphi$を適切に定めることでパスの進むスピードを変換する。

  \sukima \midashi{[単位元の存在]}

  $c \colon [0,1] \longrightarrow X$を$c(s) \equiv x_{0}$(定ループ)で定める。$[c]$は単位元になる。
  \begin{quote}
    $\because)$ $f$を$x_{0}$を基点とするループとして,$c \cdot f$を考える。
    時刻$s$を
    \begin{center}
      \begin{tikzpicture}
        \filldraw (0,0) circle (0.05);
        \filldraw (0,1) circle (0.05);
        \filldraw (0,2) circle (0.05);
        \filldraw (2,0) circle (0.05);
        \filldraw (2,2) circle (0.05);
        \draw[->] (0,0) node[left]{0} -- (0,1) node[left]{1/2} -- (0,2) node[left]{1} -- (0,2.2);
        \draw[->] (2,0) node[right]{0} -- (2,2) node[right]{1} -- (2,2.2);
        \draw[dashed] (0,0) -- (2,0);
        \draw[dashed] (0,1) -- (2,0);
        \draw[dashed] (0,2) -- (2,2);
        \draw[->, mygreen] (0.3,1.8) -- (1.7,1.8);
        \draw[->, mygreen] (0.3,1.2) -- (1.7,0.5);
        \draw[->, mygreen] (0.3,0.75) -- (1.7,0.05);
        \draw[->, mygreen] (0.3,-0.1) -- (1.7,-0.1);
        \node at (1,1.5){\textcolor{mygreen}{$\varphi$}};
      \end{tikzpicture}
    \end{center}
    のように変換する。(つまり時刻1/2まで立ち止まってから2倍のスピードで移動する。)
    このとき,$f \circ \varphi = c \cdot f$である。
    よって,$[f] = [f \circ \varphi] = [c \cdot f] = [c] \cdot [f]$である。
    $[f] \cdot [c] = [f]$も同様。
  \end{quote}

  \sukima \midashi{[結合律]}
  
  $[f \cdot (g \cdot h)] = [(f \cdot g) \cdot h]$を示せばよい。
  時刻のスピードを
  \begin{center}
    \begin{tikzpicture}
      \filldraw (0,0) circle (0.05);
      \filldraw (0,1) circle (0.05);
      \filldraw (0,1.5) circle (0.05);
      \filldraw (0,2) circle (0.05);
      \filldraw (2,0) circle (0.05);
      \filldraw (2,0.5) circle (0.05);
      \filldraw (2,1) circle (0.05);
      \filldraw (2,2) circle (0.05);
      \draw[->] (0,0) node[left]{0} -- (0,1) node[left]{1/2} -- (0,1.5) node[left]{3/4} -- (0,2) node[left]{1} -- (0,2.2);
      \draw[->] (2,0) node[right]{0} -- (2,0.5) node[right]{1/4} -- (2,1) node[right]{1/2} -- (2,2) node[right]{1} -- (2,2.2);
      \draw[dashed] (0,0) -- (2,0);
      \draw[dashed] (0,1) -- (2,0.5);
      \draw[dashed] (0,1.5) -- (2,1);
      \draw[dashed] (0,2) -- (2,2);
      \draw[->, mygreen] (0.3,1.9) -- (1.7,1.9);
      \draw[->, mygreen] (0.3,1.6) -- (1.7,1.25);
      \draw[->, mygreen] (0.3,1.2) -- (1.7,0.85);
      \draw[->, mygreen] (0.3,0.75) -- (1.7,0.4);
      \draw[->, mygreen] (0.3,0.1) -- (1.7,0.1);
      \node at (1,-0.25){\textcolor{mygreen}{$\varphi$}};
      \node at (0.2,0.5){$f$};
      \node at (0.2,1.25){$g$};
      \node at (0.2,1.75){$h$};
      \node at (1.8,0.25){$f$};
      \node at (1.8,0.75){$g$};
      \node at (1.8,1.5){$h$};
    \end{tikzpicture}
  \end{center}
  のように変換する$\varphi$を考える。このとき,$(f \cdot (g \cdot h)) \circ \varphi = (f \cdot g) \cdot h$となる。
  したがって,$[f \cdot (g \cdot h)] = [(f \cdot g) \cdot h]$である。

  \sukima \midashi{[逆元]}
  
  $f \colon [0,1] \longrightarrow X$に対し,$\overline{f} \colon [0,1] \longrightarrow X$を
  $\overline{f}(s) := f(1-s)$で定義すると,$[f][\overline{f}] = [\overline{f}][f] = [c]$である。

  \begin{quote}
    $\because)$ $f \cdot \overline{f} \simeq c$を示す。
    $f_{t}$と$g_{t}$を
    \begin{align}
      & f_{t}(s) := \begin{cases}
        f(s) & (0 \le s \le t) \\
        f(t) & (t \le s \le 1)
      \end{cases}, \\
      & g_{t}(s) := \begin{cases}
        \overline{f}(1-t) = f(t) & (0 \le s \le 1-t) \\
        \overline{f}(s) & (1-t \le s \le 1)
      \end{cases}
    \end{align}
    で定義すると,$f_{t} \cdot g_{t}$は$f \cdot \overline{f}$と$c$を結ぶホモトピーになる。
  \end{quote}
\end{proof}

\begin{teigi}[基本群]
  $\pi_{1}(X,x_{0})$を,$x_{0}$を基点とする$X$の\nw{基本群(fundamental group)}という。
\end{teigi}

\sukima \midashi{\large 基点のとりかえ}

\expandafter\ifx\csname readornot\endcsname\relax
  \end{document}
\fi