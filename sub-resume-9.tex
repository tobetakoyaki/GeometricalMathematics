\expandafter\ifx\csname readornot\endcsname\relax
  \documentclass[uplatex]{jsarticle}
  \usepackage{octopus}
  \usepackage{url}
  \usetikzlibrary{calc}

  \renewcommand{\proofname}{\textsf{証明}}
  \renewcommand{\postpartname}{章}
  \renewcommand{\thesection}{\thepart.\arabic{section}}
  \renewcommand{\thepart}{\arabic{part}}
  \makeatletter\renewcommand{\theequation}{\thesection.\arabic{equation}}\@addtoreset{equation}{section}\makeatother

  \newcommand{\octopuspart}[1]{\newpage\part{#1}\setcounter{section}{0}\vspace{3\baselineskip}}

  \renewcommand{\restriction}[2]{\left. #1 \right|_{#2}}
  \DeclareMathOperator{\dcup}{\dot{\cup}}
  \DeclareMathOperator{\conv}{conv}
  \DeclareMathOperator{\Image}{Im}
  \DeclareMathOperator{\Kernel}{Ker}
  \begin{document}
\fi

\section{ホモロジーの計算 1}

\midashi{\large 複体構造}

特異チェインの集合
${\displaystyle C_{n} = \sets{ \sum_{\alpha} \mu_{\alpha} \sigma_{\alpha} | \sigma_{\alpha} \colon \Delta^{n} \longrightarrow X, \quad \mu_{\alpha} \in \mathbb{Z}}}$
は大きすぎて直接ホモロジー群$H_{n} = \Kernel \partial_{n} / \Image \partial_{n+1}$を計算するのは難しい。

そこで計算に便利な概念を導入する。
$X$:位相空間,とする。

\begin{teigi}[$\Delta$-複体構造]
	\label{homcalc.complex}
	$X$の\nw{$\Delta$-複体構造($\Delta$-complex structure)} $\defines$ 以下の(i)から(iv)を満たす(特異)$n$-単体の族$\sets{\sigma_{\alpha} \colon \Delta^{n} \longrightarrow X}_{\alpha \in \Lambda}$:
	\begin{enumerate}
		\vspace{-0.5\baselineskip}
		\item[(i)] $\forall \alpha \in \Lambda, \quad \restriction{\sigma_{\alpha}}{\mathring{\Delta}^{n}}$は単射
		\item[(ii)] $\forall x \in X, \quad \exists! \alpha \in \Lambda, \quad x \in \sigma_{\alpha} (\mathring{\Delta}^{n})$
		\item[(iii)] $\forall \alpha \in \Lambda, \quad \forall f : \Delta^{n} = [v_{0}v_{1}\cdots v_{n}]$の面$, \quad \exists \beta \in \Lambda, \quad \restriction{\sigma_{\alpha}}{f} = \sigma_{\beta}$
		\item[(iv)] $A \subseteq X$:開 $\Longleftrightarrow$ $\forall \alpha \in \Lambda, \quad \sigma_{\alpha}^{-1} (A)$:$\Delta^{n}$上開
	\end{enumerate}
\end{teigi}

上の\rref{定義}{homcalc.complex}について補足する。
(i)の条件は「$X$が単体の内部の像の非交和(disjoint union)で表される」ことを意味する。
(ii)の条件は,もう少し正確に述べれば,
「標準全単射$\varphi \colon f \longrightarrow \Delta^{n-1}$によって$\restriction{\sigma_{\alpha}}{f} = \sigma_{\beta} \circ \varphi$となる」ことを要請している。
ただし,ここでの$f = [v_{0} \cdots \hat{v_{i}} \cdots v_{n}]$や$\Delta^{n-1}$には向きが付けられているものとする。
これは「複数の単体が面でうまく貼り合わされている」ことを意味している。
また,(iii)の条件は「位相的にも単体を貼り合わせたものである」ことを要請している。
このことは$X$がセル複体${\displaystyle \coprod_{\alpha} \Delta_{\alpha}^{n_{\alpha}} / \sim}$とみなせることに他ならない。

\begin{rei}[トーラスの場合]
	$X = T^{2}$は以下のような$\Delta$-複体構造として捉えられる。
	\begin{center}
		\begin{tikzpicture}
			% 本図
			%% 面を塗る
			\fill[red!20] (0,0) -- (2,2) -- (0,2) -- cycle;
			\fill[blue!20] (0,0) -- (2,2) -- (2,0) -- cycle;
			%% 点の名前もろもろ
			\fill \point{(0,0)}{below left}{$v$} \point{(2,0)}{below right}{$v$} \point{(0,2)}{above left}{$v$} \point{(2,2)}{above right}{$v$};
			\node at (1,0) [below]{$b$} node at (0,1) [left]{$a$} node at (1,2) [above]{$b$} node at (2,1) [right]{$a$} node at (1,1) [left]{$c$};
			\node at (0.5,1.5){$U$} node at (1.5,0.5){$L$};
			%% 矢印つき
			\midwayarrow{(0,0)}{(0,2)} \midwayarrow{(0,0)}{(2,0)} \midwayarrow{(0,2)}{(2,2)} \midwayarrow{(2,0)}{(2,2)} \midwayarrow{(0,0)}{(2,2)}
			% 取り巻き作り
			%% 点
			\fill \point{(-1.5,2.2)}{}{};
			\draw[->] (-1.4,2.1) to[bend right] node[midway,above]{$\sigma_{v}$} (-0.1,1.9);
			%% 辺
			\fill \point{(3.5,1)}{}{} \point{(3.5,2)}{}{} \point{(1.25,-1.5)}{}{} \point{(2.25,-1.5)}{}{} \point{(-0.25,-1.5)}{}{} \point{(0.75,-1.5)}{}{};
			\midwayarrow{(3.5,1)}{(3.5,2)} \midwayarrow{(1.25,-1.5)}{(2.25,-1.5)} \midwayarrow{(-0.25,-1.5)}{(0.75,-1.5)}
			\draw[->] (3.4,1.5) to[bend left] node[midway,above]{$\sigma_{a}$} (2.1,1.5);
			\draw[->] (1.75,-1.4) to[bend right] node[midway,right]{$\sigma_{b}$} (1.5,-0.1);
			\draw[->] (0.25,-1.4) to[bend left] node[midway,left]{$\sigma_{c}$} (0.5,-0.1);
			%% 面
			\fill[red!20] (-3,-2.5) -- (-1,-0.5) -- (-3,-0.5) -- cycle;
			\fill[blue!20] (2.5,-2) -- (4.5,0) -- (4.5,-2) -- cycle;
			\fill \point{(-3,-2.5)}{}{} \point{(-1,-0.5)}{}{} \point{(-3,-0.5)}{}{} \point{(2.5,-2)}{}{} \point{(4.5,0)}{}{} \point{(4.5,-2)}{}{};
			\midwayarrow{(-3,-2.5)}{(-1,-0.5)} \midwayarrow{(-3,-2.5)}{(-3,-0.5)} \midwayarrow{(-3,-0.5)}{(-1,-0.5)} \midwayarrow{(2.5,-2)}{(4.5,0)} \midwayarrow{(2.5,-2)}{(4.5,-2)} \midwayarrow{(4.5,-2)}{(4.5,0)}
			\draw[->] (-1.5,-0.4) -- node[pos=.3, above]{$\sigma_{U}$} (0.4,0.9);
			\draw[->] (4,-0.25) -- node[pos=.3, above]{$\sigma_{L}$} (1.8,0.5);
		\end{tikzpicture}
	\end{center}	
\end{rei}
	
\begin{rei}[射影平面$P^{2}$]
	$X = P^{2}$は以下のような$\Delta$-複体構造として捉えられる。
	\begin{center}
		\begin{tikzpicture}
			\fill[red!20] (0,0) -- (2,2) -- (0,2) -- cycle;
			\fill[blue!20] (0,0) -- (2,2) -- (2,0) -- cycle;
			\fill \point{(0,0)}{below left}{$v$} \point{(2,0)}{below right}{$w$} \point{(0,2)}{above left}{$w$} \point{(2,2)}{above right}{$v$};
			\node at (1,0) [below]{$b$} node at (0,1) [left]{$a$} node at (1,2) [above]{$b$} node at (2,1) [right]{$a$} node at (1,1) [left]{$c$};
			\node at (0.5,1.5){$U$} node at (1.5,0.5){$L$};
			\midwayarrow{(0,0)}{(2,0)} \midwayarrow{(0,0)}{(0,2)} \midwayarrow{(0,0)}{(2,2)} \midwayarrow{(2,2)}{(2,0)} \midwayarrow{(2,2)}{(0,2)}
		\end{tikzpicture}
	\end{center}	
\end{rei}

\sukima \midashi{注意.} いい加減に$\Delta$-複体構造をつくろうとすると失敗しやすいので注意が必要。

\begin{teigi}[単体的複体]
	次の条件を満たすものを\nw{単体的複体(simplicial complex)}という。
	\begin{itemize}
		\vspace{-0.5\baselineskip}
		\item 各$n$-単体(の像)が$n+1$個の異なる頂点からなる。
		\item 異なる$n$-単体(の像)が同じ頂点集合を持つことはない。
	\end{itemize}
\end{teigi}

\sukima \midashi{注意.}
\begin{itemize}
	\vspace{-0.5\baselineskip}
	\item $\Delta$-複体は細分することで単体的複体になる。
	\item 単体的複体では,頂点に番号をつけることで,その順番によって向きを定めることができる。
	\item 完全に組み合わせ的に扱える - - - - $>$ 抽象的単体的複体:(単体をなす頂点集合の族)
	
	頂点集合を$V$として$\mathcal{F} \subseteq 2^{V}$を考え,これが$X \subseteq Y \in \mathcal{F}$ $\Longrightarrow$ $X \in \mathcal{F}$を満たすものとして定める。
	
	例. \begin{tikzpicture}[baseline=-2.5]
			\filldraw[fill=black!30] (0,0) -- (1.4,-1) -- (1.4,1) -- cycle;
			\draw (1.4,1) -- (2.5,0) -- (3.5,0);
			\draw (1.4,-1) -- (2.5,0);
			\fill \point{(0,0)}{left}{0} \point{(1.4,-1)}{below left}{1} \point{(1.4,1)}{above left}{2} \point{(2.5,0)}{below right}{3} \point{(3.5,0)}{below}{4};
		\end{tikzpicture}
	\renewcommand{\arraystretch}{1}
	$\longrightarrow$ $\mathcal{F} = \sets{\begin{array}{c} \sets{0,1,2}, \\ \sets{0,1}, \sets{1,2}, \sets{0,2}, \sets{1,3}, \sets{2,3}, \sets{3,4},\\  \sets{0}, \sets{1}, \sets{2}, \sets{3}, \sets{4},\\  \emptyset \end{array}} \subseteq 2^{\sets{0,1,2,3,4}}$
	\renewcommand{\arraystretch}{1.3}
\end{itemize}

\sukima \midashi{\large 単体的ホモロジー}

$X$:位相空間,$\Delta$-複体構造$\sets{\sigma_{\alpha} \colon \Delta^{n_{\alpha}} \longrightarrow X}_{\alpha \in \Lambda}$,

$\Delta^{n}$:複体内の$n$-単体からなるチェイン$\sum_{\alpha \in \Lambda} \mu_{\alpha} \sigma_{\alpha}$の集合($\subseteq C_{n}$:すべての$n$-単体からなるチェインの集合)

\begin{rei}[射影平面]
	$X = P^{2}$では$\Delta_{n}$では以下のようになる。
	\begin{center}
		\begin{tikzpicture}
			\fill[red!20] (0,0) -- (2,2) -- (0,2) -- cycle;
			\fill[blue!20] (0,0) -- (2,2) -- (2,0) -- cycle;
			\fill \point{(0,0)}{below left}{$v$} \point{(2,0)}{below right}{$w$} \point{(0,2)}{above left}{$w$} \point{(2,2)}{above right}{$v$};
			\node at (1,0) [below]{$b$} node at (0,1) [left]{$a$} node at (1,2) [above]{$b$} node at (2,1) [right]{$a$} node at (1,1) [left]{$c$};
			\node at (0.5,1.5){$U$} node at (1.5,0.5){$L$};
			\midwayarrow{(0,0)}{(2,0)} \midwayarrow{(0,0)}{(0,2)} \midwayarrow{(0,0)}{(2,2)} \midwayarrow{(2,2)}{(2,0)} \midwayarrow{(2,2)}{(0,2)}
		\end{tikzpicture}
	\end{center}
	\begin{itemize}
		\vspace{-0.5\baselineskip}
		\item $\Delta_{0} = \sets{n_{v} \tikz[baseline=0] \fill \point{(0,0.1)}{below}{\footnotesize $v$};
		+ n_{w} \tikz[baseline=0] \fill \point{(0,0.1)}{below}{\footnotesize $w$}; | n_{v}, n_{w} \in \mathbb{Z}}$
		\item $\Delta_{1} = \sets{n_{a} \,
		\begin{tikzpicture}[baseline=-2.5]
			\fill \point{(0,0.3)}{}{} \point{(0,-0.3)}{}{};
			\midwayarrow{(0,0.3)}{(0,-0.3)};
			\node at (0,0) [right]{\footnotesize $a$};
		\end{tikzpicture}
		+ n_{b} \,
		\begin{tikzpicture}[baseline=-2.5]
			\fill \point{(0,0)}{}{} \point{(0.6,0)}{}{};
			\midwayarrow{(0,0)}{(0.6,0)};
			\node at (0.3,0) [below]{\footnotesize $b$};
		\end{tikzpicture} | n_{a}, n_{b} \in \mathbb{Z}}$
		\item $\Delta_{2} = \sets{n_{U} \,
		\begin{tikzpicture}[baseline=-2.5]
			\fill[red!20] (0,-0.3) -- (0.6,0.3) -- (0,0.3) -- cycle;
			\fill \point{(0,-0.3)}{}{} \point{(0.6,0.3)}{}{} \point{(0,0.3)}{}{};
			\midwayarrow{(0,-0.3)}{(0.6,0.3)}; \midwayarrow{(0.6,0.3)}{(0,0.3)}; \midwayarrow{(0,-0.3)}{(0,0.3)};
			\node at (0.18,0.13){\footnotesize $U$};
		\end{tikzpicture}
		+ n_{L} \,
		\begin{tikzpicture}[baseline=-2.5]
			\fill[blue!20] (0,-0.3) -- (0.6,-0.3) -- (0.6,0.3) -- cycle;
			\fill \point{(0,-0.3)}{}{} \point{(0.6,-0.3)}{}{} \point{(0.6,0.3)}{}{};
			\midwayarrow{(0,-0.3)}{(0.6,-0.3)}; \midwayarrow{(0.6,0.3)}{(0.6,-0.3)}; \midwayarrow{(0,-0.3)}{(0.6,0.3)};
			\node at (0.42,-0.13){\footnotesize $L$};
		\end{tikzpicture} | n_{U}, n_{L} \in \mathbb{Z}}$
		\item $\Delta_{k} = \sets{0}$($k \ge 3$)\footnote{編註:ノートでは「$k \ge 0$」とあるが誤植を疑い「$k \ge 3$」とした}
	\end{itemize}
	2-チェインの例:$3\, \begin{tikzpicture}[baseline=-2.5]
		\fill[red!20] (0,-0.3) -- (0.6,0.3) -- (0,0.3) -- cycle;
		\fill \point{(0,-0.3)}{}{} \point{(0.6,0.3)}{}{} \point{(0,0.3)}{}{};
		\midwayarrow{(0,-0.3)}{(0.6,0.3)}; \midwayarrow{(0.6,0.3)}{(0,0.3)}; \midwayarrow{(0,-0.3)}{(0,0.3)};
		\node at (0.18,0.13){\footnotesize $U$};
	\end{tikzpicture}
	-2 \,
	\begin{tikzpicture}[baseline=-2.5]
		\fill[blue!20] (0,-0.3) -- (0.6,-0.3) -- (0.6,0.3) -- cycle;
		\fill \point{(0,-0.3)}{}{} \point{(0.6,-0.3)}{}{} \point{(0.6,0.3)}{}{};
		\midwayarrow{(0,-0.3)}{(0.6,-0.3)}; \midwayarrow{(0.6,0.3)}{(0.6,-0.3)}; \midwayarrow{(0,-0.3)}{(0.6,0.3)};
		\node at (0.42,-0.13){\footnotesize $L$};
	\end{tikzpicture}$
\end{rei}

\begin{prop}
	$\Delta_{n}$は$C_{n}$の部分群
\end{prop}

\begin{teigi}[境界作用素]
	$C_{n}$に対して定義した境界作用素$\partial_{n} \colon C_{n} \longrightarrow C_{n-1}$に対して,
	$\Delta_{n}$に対する境界作用素を$\partial_{n}^{\Delta} := \restriction{\partial_{n}}{\Delta_{n}}$で定める。
\end{teigi}

なお,$\Delta$内の単体の面も$\Delta$の単体として定義してあるから,$\partial_{n}^{\Delta}$は$\Delta_{n}$から$\Delta_{n-1}$への写像となる。

\begin{hodai}
	$\Delta_{n} \xrightarrow{\partial_{n}^{\Delta}} \Delta_{n-1} \xrightarrow{\partial_{n-1}^{\Delta}} \Delta_{n-2}$において,$\partial_{n-1}^{\Delta} \circ \partial_{n}^{\Delta} = 0$
\end{hodai}

\begin{teigi}[($\Delta$に関する)ホモロジー群]
	$H_{n}^{\Delta} (X) := \Kernel \partial_{n}^{\Delta} / \Image \partial_{n+1}^{\Delta}$
	を($\Delta$に関する)ホモロジー群という。
\end{teigi}

\begin{center}
	\begin{tikzpicture}
	  \draw[->] (0,0.5) --node[midway,left]{$i:包含写像$} (0,-0.5);
	  \draw[->] (3,0.5) --node[midway,left]{$i$} (3,-0.5);
	  \draw[->] (6,0.5) --node[midway,left]{$i$} (6,-0.5);
  
	  \draw[->] (-2,1) --node[midway,above]{$\partial_{n+2}^{\Delta}$} (-1,1);
	  \draw[->] (1,1) --node[midway,above]{$\partial_{n+1}^{\Delta}$} (2,1);
	  \draw[->] (4,1) --node[midway,above]{$\partial_{n}^{\Delta}$} (5,1);
	  \draw[->] (7,1) --node[midway,above]{$\partial_{n-1}^{\Delta}$} (8,1);
  
	  \draw[->] (-2,-1) --node[midway,below]{$\partial_{n+2}$} (-1,-1);
	  \draw[->] (1,-1) --node[midway,below]{$\partial_{n+1}$} (2,-1);
	  \draw[->] (4,-1) --node[midway,below]{$\partial_{n}$} (5,-1);
	  \draw[->] (7,-1) --node[midway,below]{$\partial_{n-1}$} (8,-1);
  
	  \node at (-3,1) {$\cdots$} node at (0,1) {$\Delta_{n+1}$} node at (3,1) {$\Delta_{n}$} node at (6,1) {$\Delta_{n-1}$} node at (9,1) {$\cdots$};
	  \node at (-3,-1) {$\cdots$} node at (0,-1) {$C_{n+1}$} node at (3,-1) {$C_{n}$} node at (6,-1) {$C_{n-1}$} node at (9,-1) {$\cdots$};
	\end{tikzpicture}
\end{center}

上のような図式を考えると,
明らかに$i \circ \partial_{n}^{\Delta} = \partial_{n} \circ i$であるので,
ホモロジー群の間の写像$i_{\#} \colon H_{n}^{\Delta} \longrightarrow H_{n}$が誘導される。

\begin{teiri}
	$i_{\#}$は同型写像であり,$n = 0,1,2, \dots$で$H_{n}^{\Delta} \simeq H_{n}$である
\end{teiri}

\begin{proof}
	{\bf 難しい}
\end{proof}

特に,単体ホモロジーは,複体のとり方に依存しない。

\begin{corr}
	$n$が複体の次元より大きいとき,$H_{n}(X) = 0$
\end{corr}

\begin{proof}
	$\Kernel \partial_{n}^{\Delta} \subseteq \Delta_{n} = 0$
\end{proof}

\sukima \midashi{\large ホモロジー群の計算}

\begin{rei}[円周のホモロジー群]
	$X = S^{1}$:円周,とする。
	\begin{center}
		\begin{tikzpicture}
			\draw (0,0) circle [radius=1];
			\draw[<-] (0.1,-1.1) to[bend left] (0.7,-1.7);
			\fill \point{(0,-1)}{below}{$r$} \point{(0.8,-1.8)}{below}{$r$} \point{(2,-0.8)}{}{} \point{(2,0.8)}{}{};
			\node at (2,0) [right]{$e$} node at (0,1) [above right]{$e$} node at (0,1){\rotatebox{90}{\footnotesize $\blacktriangle$}};
			\midwayarrow{(2,-0.8)}{(2,0.8)};
			\draw[<-] (1.25,0) -- (1.75,0);
		\end{tikzpicture}
	\end{center}
	よって,
	\begin{equation}
		H_{n} (X) = \begin{cases}
			\mathbb{Z}, & n = 0, 1 \\
			0, & \text{otherwise}
		\end{cases}
	\end{equation}
\end{rei}

\begin{rei}[円板のホモロジー群]
	$X = D^{2}$:円板,とする。
	\begin{center}
		\begin{tikzpicture}
			\filldraw[fill=black!20] (0,0) circle [radius=1];
			\fill \point{(90:1)}{above}{$u$} \point{(-30:1)}{below right}{$v$} \point{(210:1)}{below left}{$w$};
			\node at (30:1){\rotatebox{210}{\footnotesize $\blacktriangle$}} node at (270:1){\rotatebox{90}{\footnotesize $\blacktriangle$}} node at (150:1){\rotatebox{150}{\footnotesize $\blacktriangle$}};
			\draw[<-] (1.5,0) -- (2.5,0);
			\fill[black!20] ($(4,-0.25) + (90:1)$) -- ($(4,-0.25) + (-30:1)$) -- ($(4,-0.25) + (210:1)$) -- cycle;
			\fill \point{($(4,-0.25) + (90:1)$)}{above}{$u$} \point{($(4,-0.25) + (-30:1)$)}{below right}{$v$} \point{($(4,-0.25) + (210:1)$)}{below left}{$w$};
			\midwayarrow{($(4,-0.25) + (90:1)$)}{($(4,-0.25) + (-30:1)$)} \midwayarrow{($(4,-0.25) + (-30:1)$)}{($(4,-0.25) + (210:1)$)} \midwayarrow{($(4,-0.25) + (90:1)$)}{($(4,-0.25) + (210:1)$)}
		\end{tikzpicture}
	\end{center}
	いま,閉路チェイン$[uv] + [vw] - [uw]$は2-単体$[uvw]$の境界であるから1次ホモロジー群は自明であり,
	2次ホモロジー群は
	\begin{equation}
		H_{2} = \Kernel \partial_{2} / \Image \partial_{3} = \Kernel \partial_{2} = \sets{\alpha [uvw] | \alpha \partial[uvw] = 0} = 0
	\end{equation}
	である。まとめると,
	\begin{equation}
		H_{n}(X) = \begin{cases}
			\mathbb{Z}, & n = 0 \\
			0, & \text{otherwise}
		\end{cases}
	\end{equation}
\end{rei}

\begin{rei}[球面のホモロジー群]
	$X = S^{2}$:球面,とする。
	\begin{center}
		\begin{tikzpicture}
		%% 面を塗る
		\fill[red!20] (0,0) -- (2,2) -- (0,2) -- cycle;
		\fill[blue!20] (0,0) -- (2,2) -- (2,0) -- cycle;
		%% 点の名前もろもろ
		\fill \point{(0,0)}{below left}{$u$} \point{(2,0)}{below right}{$v$} \point{(0,2)}{above left}{$v$} \point{(2,2)}{above right}{$w$};
		\node at (1,0) [below]{$b$} node at (0,1) [left]{$b$} node at (1,2) [above]{$a$} node at (2,1) [right]{$a$} node at (1,1) [left]{$c$};
		\node at (0.5,1.5){$U$} node at (1.5,0.5){$L$};
		%% 矢印つき
		\midwayarrow{(0,0)}{(0,2)} \midwayarrow{(0,0)}{(2,0)} \midwayarrow{(0,2)}{(2,2)} \midwayarrow{(2,0)}{(2,2)} \midwayarrow{(0,0)}{(2,2)}
		\end{tikzpicture}
	\end{center}
	\begin{itemize}
		\vspace{-0.5\baselineskip}
		\item $\Delta_{0} = \mathbb{Z} u + \mathbb{Z} v + \mathbb{Z} w$
		\item $\Delta_{1} = \mathbb{Z} a + \mathbb{Z} b + \mathbb{Z} c$
		\item $\Delta_{2} = \mathbb{Z} U + \mathbb{Z} L$
	\end{itemize}

	0次ホモロジー群は連結成分の個数から求めれば早く,$H_{0}(X) = \mathbb{Z}$。

	1次ホモロジー群$H_{1}(X) = \Kernel \partial_{1} / \Image \partial_{2}$を求める。
	\renewcommand{\arraystretch}{1}
	\begin{center}
		$\begin{array}{lcl}
			\alpha a + \beta b + \gamma c \in \Kernel \partial_{1} & \Longleftrightarrow & \alpha (w-v) + \beta (v-u) + \gamma (w-u) = 0 \\
			& \Longleftrightarrow & \alpha = \beta = - \gamma 
		\end{array}$
	\end{center}
	\renewcommand{\arraystretch}{1.3}
	より,$\Kernel \partial_{1} = \mathbb{Z} (a+b-c)$。
	また,$\partial U = \partial L = a+b-c$より,$\Image \partial_{2} = \mathbb{Z} (a+b-c)$。
	よって,$H_{1}(X) = 0$(自明)\footnote{この自明は計算が自明なのではなく,この群が自明,つまり単位元しか元を持たない群であることを指していて,決してハラスメントではないことに注意されたい。}\ である。

	2次ホモロジー群$H_{2}(X) = \Kernel \partial_{2} / \Image \partial_{3}$を求める。
	\renewcommand{\arraystretch}{1}
	\begin{center}
		$\begin{array}{lcl}
			\alpha U + \beta L \in \Kernel \partial_{2} & \Longleftrightarrow & \alpha (a-c+b) + \beta (a-c+b) = 0 \\
			& \Longleftrightarrow & \alpha = - \beta
		\end{array}$
	\end{center}
	\renewcommand{\arraystretch}{1.3}
	より,$\Kernel \partial_{2} = \mathbb{Z} (U-L)$。
	また,そんなものはないので$\Delta_{3} = 0$であり,$\Image \partial_{3} = 0$である。
	よって,$H_{2}(X) = \mathbb{Z}$である。

	まとめると,
	\begin{equation}
		H_{n}(X) = \begin{cases}
			\mathbb{Z}, & n=0, 2 \\
			0, & \text{otherwise}
		\end{cases}
	\end{equation}
\end{rei}

\begin{rei}[超球面のホモロジー群]
	一般に$X = S^{n}$とすると,複体として2つの$n$-単体$U,L$の貼り合わせがとれて,
	$H_{n} = \mathbb{Z} (U - L) \simeq \mathbb{Z}$となる。
	さらに,他の次元のホモロジー群も計算すると,\footnote{編註:Mayer-Vietoris完全系列(cf. \url{<https://ja.wikipedia.org/wiki/マイヤー・ヴィートリス完全系列>})を使うのが一般的?}\
	\begin{equation}
		H_{k} (X) = \begin{cases}
			\mathbb{Z}, & k = 0, n \\
			0, & \text{otherwise}
		\end{cases}
	\end{equation}
	となる。
\end{rei}

\begin{rei}[トーラスのホモロジー群]
	$X = T^{2} = S^{1} \times S^{1}$:2次元トーラス,とする。
	\begin{center}
		\begin{tikzpicture}    
			%% 面を塗る
			\fill[red!20] (0,0) -- (2,2) -- (0,2) -- cycle;
			\fill[blue!20] (0,0) -- (2,2) -- (2,0) -- cycle;
			%% 点の名前もろもろ
			\fill \point{(0,0)}{below left}{$v$} \point{(2,0)}{below right}{$v$} \point{(0,2)}{above left}{$v$} \point{(2,2)}{above right}{$v$};
			\node at (1,0) [below]{$b$} node at (0,1) [left]{$a$} node at (1,2) [above]{$b$} node at (2,1) [right]{$a$} node at (1,1) [left]{$c$};
			\node at (0.5,1.5){$U$} node at (1.5,0.5){$L$};
			%% 矢印つき
			\midwayarrow{(0,0)}{(0,2)} \midwayarrow{(0,0)}{(2,0)} \midwayarrow{(0,2)}{(2,2)} \midwayarrow{(2,0)}{(2,2)} \midwayarrow{(0,0)}{(2,2)}
		\end{tikzpicture}	
	\end{center}
	\begin{itemize}
		\vspace{-0.5\baselineskip}
		\item $\Delta_{0} = \mathbb{Z} v$
		\item $\Delta_{1} = \mathbb{Z} a + \mathbb{Z} b + \mathbb{Z} c$
		\item $\Delta_{2} = \mathbb{Z} U + \mathbb{Z} L$
	\end{itemize}

	0次ホモロジー群$H_{0}(X) = \Kernel \partial_{0} / \Image \partial_{1}$は連結成分の個数から求めれば早く,$H_{0}(X) = \mathbb{Z}$。

	1次ホモロジー群$H_{1}(X) = \Kernel \partial_{1} / \Image \partial_{2}$を求める。
	\begin{equation}
		\partial_{1} a = \partial_{1} b = \partial_{1} c = v - v = 0
	\end{equation}
	であるから,$\Kernel \partial_{1} = \Delta_{1} = \mathbb{Z} a + \mathbb{Z} b + \mathbb{Z} c$である。
	また,$\Image \partial_{2} = \mathbb{Z} (a-c+b) + \mathbb{Z} (b-c+a) = \mathbb{Z} (a+b-c)$である。
	$c$は$a+b$にホモローグであるので,$[a]$や$[b]$をホモロジー類として
	\begin{equation}
		H_{1}(X) = \mathbb{Z} [a] + \mathbb{Z} [b] \simeq \mathbb{Z}^{2}
	\end{equation}
	である。(これはトーラスに1次元の穴が2つあることを示唆している)

	2次ホモロジー群$H_{2}(X) = \Kernel \partial_{2} / \Image \partial_{3}$を求める。
	計算により,$\Kernel \partial_{2} = \mathbb{Z} (U-L)$,$\Image \partial_{3} = 0$と分かるから,
	\begin{equation}
		H_{2}(X) = \mathbb{Z} (U-L) \simeq \mathbb{Z}^{1}
	\end{equation}
	である。(これはトーラスに2次元の穴が1つあることを示唆している)

	まとめると,
	\begin{equation}
		H_{n}(X) = \begin{cases}
			\mathbb{Z}, & n=0 \\
			\mathbb{Z}^{2}, & n=1 \\
			\mathbb{Z}, & n=2 \\
			0 & n \ge 3
		\end{cases}
	\end{equation}
\end{rei}

\begin{rei}[射影平面のホモロジー群]
	$X = P^{2}$:2次元射影空間(射影平面),とする。
	\begin{center}
		\begin{tikzpicture}
			\fill[red!20] (0,0) -- (2,2) -- (0,2) -- cycle;
			\fill[blue!20] (0,0) -- (2,2) -- (2,0) -- cycle;
			\fill \point{(0,0)}{below left}{$u$} \point{(2,0)}{below right}{$v$} \point{(0,2)}{above left}{$v$} \point{(2,2)}{above right}{$u$};
			\node at (1,0) [below]{$b$} node at (0,1) [left]{$a$} node at (1,2) [above]{$b$} node at (2,1) [right]{$a$} node at (1,1) [left]{$c$};
			\node at (0.5,1.5){$U$} node at (1.5,0.5){$L$};
			\midwayarrow{(0,0)}{(2,0)} \midwayarrow{(0,0)}{(0,2)} \midwayarrow{(0,0)}{(2,2)} \midwayarrow{(2,2)}{(2,0)} \midwayarrow{(2,2)}{(0,2)}
		\end{tikzpicture}
	\end{center}
	\begin{itemize}
		\vspace{-0.5\baselineskip}
		\item $\Delta_{0} = \mathbb{Z} u + \mathbb{Z} v$
		\item $\Delta_{1} = \mathbb{Z} a + \mathbb{Z} b + \mathbb{Z} c$
		\item $\Delta_{2} = \mathbb{Z} U + \mathbb{Z} L$
	\end{itemize}

	0次ホモロジー群$H_{0}(X) = \Kernel \partial_{0} / \Image \partial_{1}$は,同値関係を$x \sim -x$で定義したときに$P^{2} \simeq S^{2} / \sim$と表されることから,連結成分の個数から求めれば$H_{0}(X) = \mathbb{Z}$。

	1次ホモロジー群$H_{1}(X) = \Kernel \partial_{1} / \Image \partial_{2}$を求める。
	\renewcommand{\arraystretch}{1}
	\begin{center}
		$\begin{array}{lcl}
			\alpha a + \beta b + \gamma c \in \Kernel \partial_{1} & \Longleftrightarrow & \alpha (v-u) + \beta (v-u) + \gamma (u-u) = 0 \\
			& \Longleftrightarrow & \alpha = - \beta
		\end{array}$
	\end{center}
	より,$\Kernel \partial_{1} = \mathbb{Z} (a-b) + \mathbb{Z} c = \mathbb{Z} (a-b+c) + \mathbb{Z} c$である。
	また,$\Image \partial_{2} = \mathbb{Z} (b-a+c) + \mathbb{Z} (a-b+c) = \mathbb{Z} (a-b+c) + \mathbb{Z} \cdot 2c$である。
	よって,$H_{1}(X) = \mathbb{Z} / 2 \mathbb{Z}$である。(これはねじれ部分があることを示唆している)

	2次ホモロジー群$H_{2}(X) = \Kernel \partial_{2} / \Image \partial_{3}$を求める。
	\begin{center}
		$\begin{array}{lcl}
			\alpha (b-a+c) + \beta (a-b+c) = 0 & \Longleftrightarrow & \alpha = \beta = 0
		\end{array}$
	\end{center}
	\renewcommand{\arraystretch}{1.3}
	より,$\Kernel \partial_{2} = 0$である。また,3次元単体は含まれないので$\Image \partial_{3} = 0$である。
	よって,$H_{2} (X) = 0$である。

	まとめると,
	\begin{equation}
		H_{n}(X) = \begin{cases}
			\mathbb{Z}, & n=0 \\
			\mathbb{Z} / 2 \mathbb{Z}, & n=1 \\
			0, & \text{otherwise}
		\end{cases}
	\end{equation}
\end{rei}

	

\expandafter\ifx\csname readornot\endcsname\relax
  \end{document}
\fi