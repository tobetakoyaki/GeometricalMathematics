\expandafter\ifx\csname readornot\endcsname\relax
  \documentclass[uplatex]{jsarticle}
  \usepackage{octopus}
  \usepackage{url}
  \usetikzlibrary{calc}

  \renewcommand{\proofname}{\textsf{証明}}
  \renewcommand{\postpartname}{章}
  \renewcommand{\thesection}{\thepart.\arabic{section}}
  \renewcommand{\thepart}{\arabic{part}}
  \makeatletter\renewcommand{\theequation}{\thesection.\arabic{equation}}\@addtoreset{equation}{section}\makeatother

  \newcommand{\octopuspart}[1]{\newpage\part{#1}\setcounter{section}{0}\vspace{3\baselineskip}}

  \renewcommand{\restriction}[2]{\left. #1 \right|_{#2}}
  \DeclareMathOperator{\dcup}{\dot{\cup}}
  \DeclareMathOperator{\conv}{conv}
  \DeclareMathOperator{\Image}{Im}
  \DeclareMathOperator{\Kernel}{Ker}
  \begin{document}
\fi

\section{ホモロジーの計算}

%%%%%%%%%% 図 Δ複体構造の例 X = トーラス %%%%%%%%%%  %% 文章を入れたら消してください
\begin{center}
	\begin{tikzpicture}
		% 本図
		%% 面を塗る
		\fill[red!20] (0,0) -- (2,2) -- (0,2) -- cycle;
		\fill[blue!20] (0,0) -- (2,2) -- (2,0) -- cycle;
		%% 点の名前もろもろ
		\fill \point{(0,0)}{below left}{$v$} \point{(2,0)}{below right}{$v$} \point{(0,2)}{above left}{$v$} \point{(2,2)}{above right}{$v$};
		\node at (1,0) [below]{$b$} node at (0,1) [left]{$a$} node at (1,2) [above]{$b$} node at (2,1) [right]{$a$} node at (1,1) [left]{$c$};
		\node at (0.5,1.5){$U$} node at (1.5,0.5){$L$};
		%% 矢印つき
		\midwayarrow{(0,0)}{(0,2)} \midwayarrow{(0,0)}{(2,0)} \midwayarrow{(0,2)}{(2,2)} \midwayarrow{(2,0)}{(2,2)} \midwayarrow{(0,0)}{(2,2)}
		% 取り巻き作り
		%% 点
		\fill \point{(-1.5,2.2)}{}{};
		\draw[->] (-1.4,2.1) to[bend right] node[midway,above]{$\sigma_{v}$} (-0.1,1.9);
		%% 辺
		\fill \point{(3.5,1)}{}{} \point{(3.5,2)}{}{} \point{(1.25,-1.5)}{}{} \point{(2.25,-1.5)}{}{} \point{(-0.25,-1.5)}{}{} \point{(0.75,-1.5)}{}{};
		\midwayarrow{(3.5,1)}{(3.5,2)} \midwayarrow{(1.25,-1.5)}{(2.25,-1.5)} \midwayarrow{(-0.25,-1.5)}{(0.75,-1.5)}
		\draw[->] (3.4,1.5) to[bend left] node[midway,above]{$\sigma_{a}$} (2.1,1.5);
		\draw[->] (1.75,-1.4) to[bend right] node[midway,right]{$\sigma_{b}$} (1.5,-0.1);
		\draw[->] (0.25,-1.4) to[bend left] node[midway,left]{$\sigma_{c}$} (0.5,-0.1);
		%% 面
		\fill[red!20] (-3,-2.5) -- (-1,-0.5) -- (-3,-0.5) -- cycle;
		\fill[blue!20] (2.5,-2) -- (4.5,0) -- (4.5,-2) -- cycle;
		\fill \point{(-3,-2.5)}{}{} \point{(-1,-0.5)}{}{} \point{(-3,-0.5)}{}{} \point{(2.5,-2)}{}{} \point{(4.5,0)}{}{} \point{(4.5,-2)}{}{};
		\midwayarrow{(-3,-2.5)}{(-1,-0.5)} \midwayarrow{(-3,-2.5)}{(-3,-0.5)} \midwayarrow{(-3,-0.5)}{(-1,-0.5)} \midwayarrow{(2.5,-2)}{(4.5,0)} \midwayarrow{(2.5,-2)}{(4.5,-2)} \midwayarrow{(4.5,-2)}{(4.5,0)}
		\draw[->] (-1.5,-0.4) -- node[pos=.3, above]{$\sigma_{U}$} (0.4,0.9);
		\draw[->] (4,-0.25) -- node[pos=.3, above]{$\sigma_{L}$} (1.8,0.5);
	\end{tikzpicture}
\end{center}
%%%%%%%%%% %% 文章を入れたら消してください

%%%%%%%%%% 図 Δ複体構造の例 X = P^{2} %%%%%%%%%%  %% 文章を入れたら消してください
\begin{center}
	\begin{tikzpicture}
		\fill[red!20] (0,0) -- (2,2) -- (0,2) -- cycle;
		\fill[blue!20] (0,0) -- (2,2) -- (2,0) -- cycle;
		\fill \point{(0,0)}{below left}{$v$} \point{(2,0)}{below right}{$w$} \point{(0,2)}{above left}{$w$} \point{(2,2)}{above right}{$v$};
		\node at (1,0) [below]{$b$} node at (0,1) [left]{$a$} node at (1,2) [above]{$b$} node at (2,1) [right]{$a$} node at (1,1) [left]{$c$};
		\node at (0.5,1.5){$U$} node at (1.5,0.5){$L$};
		\midwayarrow{(0,0)}{(2,0)} \midwayarrow{(0,0)}{(0,2)} \midwayarrow{(0,0)}{(2,2)} \midwayarrow{(2,2)}{(2,0)} \midwayarrow{(2,2)}{(0,2)}
	\end{tikzpicture}
\end{center}
%%%%%%%%%% %% 文章を入れたら消してください

%%%%%%%%%% 図 抽象的単体的複体 %%%%%%%%%%  %% 文章を入れたら消してください
\begin{center}
	\begin{tikzpicture}
		\filldraw[fill=black!30] (0,0) -- (1.4,-1) -- (1.4,1) -- cycle;
		\draw (1.4,1) -- (2.5,0) -- (3.5,0);
		\draw (1.4,-1) -- (2.5,0);
		\fill \point{(0,0)}{left}{0} \point{(1.4,-1)}{below left}{1} \point{(1.4,1)}{above left}{2} \point{(2.5,0)}{below right}{3} \point{(3.5,0)}{below}{4};
	\end{tikzpicture}
\end{center}
%%%%%%%%%% %% 文章を入れたら消してください

%%%%%%%%%% 図 単体的ホモロジーの例 P^{2} %%%%%%%%%%  %% 文章を入れたら消してください
\begin{center}
	\begin{tikzpicture}
		\fill[red!20] (0,0) -- (2,2) -- (0,2) -- cycle;
		\fill[blue!20] (0,0) -- (2,2) -- (2,0) -- cycle;
		\fill \point{(0,0)}{below left}{$v$} \point{(2,0)}{below right}{$w$} \point{(0,2)}{above left}{$w$} \point{(2,2)}{above right}{$v$};
		\node at (1,0) [below]{$b$} node at (0,1) [left]{$a$} node at (1,2) [above]{$b$} node at (2,1) [right]{$a$} node at (1,1) [left]{$c$};
		\node at (0.5,1.5){$U$} node at (1.5,0.5){$L$};
		\midwayarrow{(0,0)}{(2,0)} \midwayarrow{(0,0)}{(0,2)} \midwayarrow{(0,0)}{(2,2)} \midwayarrow{(2,2)}{(2,0)} \midwayarrow{(2,2)}{(0,2)}
	\end{tikzpicture}
\end{center}

$\Delta_{0} = \sets{n_{v} \tikz[baseline=0] \fill \point{(0,0.1)}{below}{\footnotesize $v$};
+ n_{w} \tikz[baseline=0] \fill \point{(0,0.1)}{below}{\footnotesize $w$}; | n_{v}, n_{w} \in \mathbb{Z}}$

$\Delta_{1} = \sets{n_{a} \,
\begin{tikzpicture}[baseline=-2.5]
	\fill \point{(0,0.3)}{}{} \point{(0,-0.3)}{}{};
	\midwayarrow{(0,0.3)}{(0,-0.3)};
	\node at (0,0) [right]{\footnotesize $a$};
\end{tikzpicture}
+ n_{b} \,
\begin{tikzpicture}[baseline=-2.5]
	\fill \point{(0,0)}{}{} \point{(0.6,0)}{}{};
	\midwayarrow{(0,0)}{(0.6,0)};
	\node at (0.3,0) [below]{\footnotesize $b$};
\end{tikzpicture} | n_{a}, n_{b} \in \mathbb{Z}}$

$\Delta_{2} = \sets{n_{U} \,
\begin{tikzpicture}[baseline=-2.5]
	\fill[red!20] (0,-0.3) -- (0.6,0.3) -- (0,0.3) -- cycle;
	\fill \point{(0,-0.3)}{}{} \point{(0.6,0.3)}{}{} \point{(0,0.3)}{}{};
	\midwayarrow{(0,-0.3)}{(0.6,0.3)}; \midwayarrow{(0.6,0.3)}{(0,0.3)}; \midwayarrow{(0,-0.3)}{(0,0.3)};
	\node at (0.18,0.13){\footnotesize $U$};
\end{tikzpicture}
+ n_{L} \,
\begin{tikzpicture}[baseline=-2.5]
	\fill[blue!20] (0,-0.3) -- (0.6,-0.3) -- (0.6,0.3) -- cycle;
	\fill \point{(0,-0.3)}{}{} \point{(0.6,-0.3)}{}{} \point{(0.6,0.3)}{}{};
	\midwayarrow{(0,-0.3)}{(0.6,-0.3)}; \midwayarrow{(0.6,0.3)}{(0.6,-0.3)}; \midwayarrow{(0,-0.3)}{(0.6,0.3)};
	\node at (0.42,-0.13){\footnotesize $L$};
\end{tikzpicture} | n_{U}, n_{L} \in \mathbb{Z}}$

2-チェインの例:$3\, \begin{tikzpicture}[baseline=-2.5]
	\fill[red!20] (0,-0.3) -- (0.6,0.3) -- (0,0.3) -- cycle;
	\fill \point{(0,-0.3)}{}{} \point{(0.6,0.3)}{}{} \point{(0,0.3)}{}{};
	\midwayarrow{(0,-0.3)}{(0.6,0.3)}; \midwayarrow{(0.6,0.3)}{(0,0.3)}; \midwayarrow{(0,-0.3)}{(0,0.3)};
	\node at (0.18,0.13){\footnotesize $U$};
\end{tikzpicture}
-2 \,
\begin{tikzpicture}[baseline=-2.5]
	\fill[blue!20] (0,-0.3) -- (0.6,-0.3) -- (0.6,0.3) -- cycle;
	\fill \point{(0,-0.3)}{}{} \point{(0.6,-0.3)}{}{} \point{(0.6,0.3)}{}{};
	\midwayarrow{(0,-0.3)}{(0.6,-0.3)}; \midwayarrow{(0.6,0.3)}{(0.6,-0.3)}; \midwayarrow{(0,-0.3)}{(0.6,0.3)};
	\node at (0.42,-0.13){\footnotesize $L$};
\end{tikzpicture}$
%%%%%%%%%% %% 文章を入れたら消してください

%%%%%%%%%% 図 ホモロジー群の定義の後の図式 %%%%%%%%%%  %% 文章を入れたら消してください
\begin{center}
	\begin{tikzpicture}
	  \draw[->] (0,0.5) --node[midway,left]{$i:包含写像$} (0,-0.5);
	  \draw[->] (3,0.5) --node[midway,left]{$i$} (3,-0.5);
	  \draw[->] (6,0.5) --node[midway,left]{$i$} (6,-0.5);
  
	  \draw[->] (-2,1) --node[midway,above]{$\partial_{n+2}^{\Delta}$} (-1,1);
	  \draw[->] (1,1) --node[midway,above]{$\partial_{n+1}^{\Delta}$} (2,1);
	  \draw[->] (4,1) --node[midway,above]{$\partial_{n}^{\Delta}$} (5,1);
	  \draw[->] (7,1) --node[midway,above]{$\partial_{n-1}^{\Delta}$} (8,1);
  
	  \draw[->] (-2,-1) --node[midway,below]{$\partial_{n+2}$} (-1,-1);
	  \draw[->] (1,-1) --node[midway,below]{$\partial_{n+1}$} (2,-1);
	  \draw[->] (4,-1) --node[midway,below]{$\partial_{n}$} (5,-1);
	  \draw[->] (7,-1) --node[midway,below]{$\partial_{n-1}$} (8,-1);
  
	  \node at (-3,1) {$\cdots$} node at (0,1) {$\Delta_{n+1}$} node at (3,1) {$\Delta_{n}$} node at (6,1) {$\Delta_{n-1}$} node at (9,1) {$\cdots$};
	  \node at (-3,-1) {$\cdots$} node at (0,-1) {$C_{n+1}$} node at (3,-1) {$C_{n}$} node at (6,-1) {$C_{n-1}$} node at (9,-1) {$\cdots$};
	\end{tikzpicture}
\end{center}
%%%%%%%%%% %% 文章を入れたら消してください

%%%%%%%%%% 図 Cor. Hn(X) = 0 の後の例の図 %%%%%%%%%%  %% 文章を入れたら消してください
\begin{tikzpicture}
	\draw (0,0) circle [radius=1];
	\draw[<-] (0.1,-1.1) to[bend left] (0.7,-1.7);
	\fill \point{(0,-1)}{below}{$r$} \point{(0.8,-1.8)}{below}{$r$} \point{(2,-0.8)}{}{} \point{(2,0.8)}{}{};
	\node at (2,0) [right]{$e$} node at (0,1) [above right]{$e$} node at (0,1){\rotatebox{90}{\footnotesize $\blacktriangle$}};
	\midwayarrow{(2,-0.8)}{(2,0.8)};
	\draw[<-] (1.25,0) -- (1.75,0);
\end{tikzpicture}

\begin{tikzpicture}
  \filldraw[fill=black!20] (0,0) circle [radius=1];
	\fill \point{(90:1)}{above}{$u$} \point{(-30:1)}{below right}{$v$} \point{(210:1)}{below left}{$w$};
	\node at (30:1){\rotatebox{210}{\footnotesize $\blacktriangle$}} node at (270:1){\rotatebox{90}{\footnotesize $\blacktriangle$}} node at (150:1){\rotatebox{150}{\footnotesize $\blacktriangle$}};
  \draw[<-] (1.5,0) -- (2.5,0);
  \fill[black!20] ($(4,-0.25) + (90:1)$) -- ($(4,-0.25) + (-30:1)$) -- ($(4,-0.25) + (210:1)$) -- cycle;
	\fill \point{($(4,-0.25) + (90:1)$)}{above}{$u$} \point{($(4,-0.25) + (-30:1)$)}{below right}{$v$} \point{($(4,-0.25) + (210:1)$)}{below left}{$w$};
	\midwayarrow{($(4,-0.25) + (90:1)$)}{($(4,-0.25) + (-30:1)$)} \midwayarrow{($(4,-0.25) + (-30:1)$)}{($(4,-0.25) + (210:1)$)} \midwayarrow{($(4,-0.25) + (90:1)$)}{($(4,-0.25) + (210:1)$)}
\end{tikzpicture}
%%%%%%%%%% %% 文章を入れたら消してください




%%%%%%%%%% 図 Cor. Hn(X) = 0 の後の例の図 S^{2} %%%%%%%%%%  %% 文章を入れたら消してください
\begin{tikzpicture}
  %% 面を塗る
  \fill[red!20] (0,0) -- (2,2) -- (0,2) -- cycle;
  \fill[blue!20] (0,0) -- (2,2) -- (2,0) -- cycle;
  %% 点の名前もろもろ
  \fill \point{(0,0)}{below left}{$u$} \point{(2,0)}{below right}{$v$} \point{(0,2)}{above left}{$v$} \point{(2,2)}{above right}{$w$};
  \node at (1,0) [below]{$b$} node at (0,1) [left]{$b$} node at (1,2) [above]{$a$} node at (2,1) [right]{$a$} node at (1,1) [left]{$c$};
  \node at (0.5,1.5){$U$} node at (1.5,0.5){$L$};
  %% 矢印つき
  \midwayarrow{(0,0)}{(0,2)} \midwayarrow{(0,0)}{(2,0)} \midwayarrow{(0,2)}{(2,2)} \midwayarrow{(2,0)}{(2,2)} \midwayarrow{(0,0)}{(2,2)}
\end{tikzpicture}
%%%%%%%%%% %% 文章を入れたら消してください




%%%%%%%%%% 図 Cor. Hn(X) = 0 の後の例の図 トーラス %%%%%%%%%%  %% 文章を入れたら消してください
\begin{tikzpicture}    
  %% 面を塗る
  \fill[red!20] (0,0) -- (2,2) -- (0,2) -- cycle;
  \fill[blue!20] (0,0) -- (2,2) -- (2,0) -- cycle;
  %% 点の名前もろもろ
  \fill \point{(0,0)}{below left}{$v$} \point{(2,0)}{below right}{$v$} \point{(0,2)}{above left}{$v$} \point{(2,2)}{above right}{$v$};
  \node at (1,0) [below]{$b$} node at (0,1) [left]{$a$} node at (1,2) [above]{$b$} node at (2,1) [right]{$a$} node at (1,1) [left]{$c$};
  \node at (0.5,1.5){$U$} node at (1.5,0.5){$L$};
  %% 矢印つき
  \midwayarrow{(0,0)}{(0,2)} \midwayarrow{(0,0)}{(2,0)} \midwayarrow{(0,2)}{(2,2)} \midwayarrow{(2,0)}{(2,2)} \midwayarrow{(0,0)}{(2,2)}
\end{tikzpicture}
%%%%%%%%%% %% 文章を入れたら消してください

%%%%%%%%%% 図 X = P^{2} %%%%%%%%%%  %% 文章を入れたら消してください
\begin{tikzpicture}
\fill[red!20] (0,0) -- (2,2) -- (0,2) -- cycle;
\fill[blue!20] (0,0) -- (2,2) -- (2,0) -- cycle;
\fill \point{(0,0)}{below left}{$u$} \point{(2,0)}{below right}{$v$} \point{(0,2)}{above left}{$v$} \point{(2,2)}{above right}{$u$};
\node at (1,0) [below]{$b$} node at (0,1) [left]{$a$} node at (1,2) [above]{$b$} node at (2,1) [right]{$a$} node at (1,1) [left]{$c$};
\node at (0.5,1.5){$U$} node at (1.5,0.5){$L$};
\midwayarrow{(0,0)}{(2,0)} \midwayarrow{(0,0)}{(0,2)} \midwayarrow{(0,0)}{(2,2)} \midwayarrow{(2,2)}{(2,0)} \midwayarrow{(2,2)}{(0,2)}
\end{tikzpicture}
%%%%%%%%%% %% 文章を入れたら消してください

\expandafter\ifx\csname readornot\endcsname\relax
  \end{document}
\fi