\expandafter\ifx\csname readornot\endcsname\relax
  \documentclass[uplatex]{jsarticle}
  \usepackage{octopus}
  \usepackage{url}

  \renewcommand{\proofname}{\textsf{証明}}
  \renewcommand{\postpartname}{章}
  \renewcommand{\thesection}{\thepart.\arabic{section}}
  \renewcommand{\thepart}{\arabic{part}}
  \makeatletter\renewcommand{\theequation}{\thesection.\arabic{equation}}\@addtoreset{equation}{section}\makeatother

  \newcommand{\octopuspart}[1]{\newpage\part{#1}\setcounter{section}{0}\vspace{3\baselineskip}}

  \renewcommand{\restriction}[2]{\left. #1 \right|_{#2}}
  \DeclareMathOperator{\dcup}{\dot{\cup}}
  \DeclareMathOperator{\conv}{conv}
  \DeclareMathOperator{\Image}{Im}
  \DeclareMathOperator{\Kernel}{Ker}
  \begin{document}
\fi

\section{テンソル解析 1}
$V$を$\mathbb{R}$上の$n$次元ベクトル空間とし,基底を$\sets{e_{1}, \dots, e_{n}}$とする。このとき,任意の$V$の元$v$は${\displaystyle v = \sum_{i=1}^{n} v^{i}e_{i}}$と書ける。
つまり,基底の組を決めると,任意の$v \in V$は$n$次元ベクトル$(v^{1} , \dots, v^{n})^{\top}$とみなせる。つまり,$V \simeq \mathbb{R}^{n}$である。

\begin{teigi}
  \nw{座標系} $\defines$ 基底の組$(e_{\kappa} \,|\, \kappa = 1,\dots, n)$
\end{teigi}

$v \in V$に対して座標系$(e_{\kappa} \,|\, \kappa = 1,\dots, n)$による表示を$v^{\kappa}$,座標系$(e_{\kappa^{\prime}} \,|\, \kappa^{\prime} = 1^{\prime},\dots, n^{\prime})$による表示を$v^{\kappa^{\prime}}$と表すことにする。

\begin{hodai}[変換則]
  \label{lem:12-1-2}
  ${\displaystyle e_{\kappa} = \sum_{\kappa^{\prime} = 1^{\prime}}^{n^{\prime}} A_{\kappa}^{\kappa^{\prime}} e_{\kappa^{\prime}}}$なら,
  \begin{align}
    v^{\kappa^{\prime}} = \sum_{\kappa=1}^{n} A_{\kappa}^{\kappa^{\prime}} v^{\kappa} =: A_{\kappa}^{\kappa^{\prime}} v^{\kappa}.
  \end{align}
\end{hodai}
最後の和の記号$\sum$を省略する書き方を「\nw{Einsteinの縮約記法}」といい,上と下に同じ添字が出てきた場合,その文字については和をとることにする。
\begin{proof}
  $v = v^{\kappa}e_{\kappa} = v^{\kappa^{\prime}}e_{\kappa^{\prime}}$と2通りの表示を考えると,$v^{\kappa} e_{\kappa} = v^{\kappa} A_{\kappa}^{\kappa^{\prime}} e_{\kappa^{\prime}}.
  $である。これより$v^{\kappa^{\prime}} = A_{\kappa}^{\kappa^{\prime}}v^{\kappa}$を得る。
\end{proof}

\begin{teigi}
  座標系から$\mathbb{R}^{n}$への対応を\nw{(一般化された)ベクトル}という。特に,\textbf{補題\ref{lem:12-1-2}}に示された変換則に従うベクトルを\nw{反変ベクトル}という。
\end{teigi}

$V^{*}$を$V$の双対空間,つまり,$V^{*} := \sets{f \colon V \longrightarrow \mathbb{R} | f \text{は線形写像}}$とする。$V$の座標系を$(e_{\kappa})$,その双対基底を$(e^{\kappa})$とする\footnote{すなわち,$e^{\kappa}(e_{\lambda}) = \delta^{\kappa}_{\lambda}$が成り立つ。}。このとき,$f \in V^{*}$は双対基底によって${\displaystyle f = \sum_{\kappa} f_{\kappa} e^{\kappa}}$と書き表すことができる。

\begin{hodai}[変換則]
  \label{lem:12-1-4}
  $V$の座標系$(e_{\kappa})$と$(e_{\kappa^{\prime}})$について,${\displaystyle f = \sum_{\kappa} f_{\kappa} e^{\kappa} = \sum_{\kappa^{\prime}} f_{\kappa^{\prime}} e^{\kappa^{\prime}}}$を考える。$e_{\kappa^{\prime}} = A_{\kappa^{\prime}}^{\kappa} e_{\kappa}$であるとき,
  \begin{align}
    f_{\kappa^{\prime}} = A_{\kappa^{\prime}}^{\kappa} f_{\kappa}.
  \end{align}
\end{hodai}

\begin{proof}
  $f_{\kappa^{\prime}} = f(e_{\kappa^{\prime}}) = f(A_{\kappa^{\prime}}^{\kappa}e_{\kappa}) = A_{\kappa^{\prime}}^{\kappa} f(e_{\kappa}) = A_{\kappa^{\prime}}^{\kappa} f_{\kappa}$である。$f$が線形写像であることに注意。
\end{proof}

\begin{teigi}
  \textbf{補題\ref{lem:12-1-4}}に示された変換則に従うベクトルを\nw{共変ベクトル}という。
\end{teigi}

\begin{remark}
  $V$は反変ベクトル空間であり,$V^{*}$は共変ベクトル空間である。
\end{remark}

\begin{teigi}
  座標系に$\mathbb{R}$の元を対応させる写像であって,いかなる座標系においても同じ値に写すものを\nw{スカラー}という。
\end{teigi}

\begin{hodai}
  $v^{\kappa}$を反変ベクトル,$f_{\kappa}$を共変ベクトルとする。このとき,$f_{\kappa} v^{\kappa}$はスカラー。
\end{hodai}

\begin{proof}
  基底が$e_{\kappa} = A_{\kappa}^{\kappa^{\prime}} e_{\kappa^{\prime}}$と変換されるとする。このとき,
  \begin{align}
    f_{\kappa} v^{\kappa} = f_{\kappa} A^{\kappa}_{\kappa^{\prime}} v^{\kappa^{\prime}} = f_{\kappa^{\prime}} v^{\kappa^{\prime}}.
  \end{align}
  よって,$f_{\kappa} v^{\kappa}$は座標系によらず不変である。
\end{proof}

\sukima \midashi{物理的イメージ}\par
\noindent\textbullet\; 反変ベクトルの例\par
\qquad  
\begin{minipage}[t]{0.15\columnwidth}
  位置ベクトル,
  \begin{align*}
    x = \vect{\mathrm{OP}}
  \end{align*}
\end{minipage}
\vspace{0.05\baselineskip}
\begin{minipage}[t]{0.15\columnwidth}
  速度ベクトル,
  \begin{align*}
    \odif{x}{t}
  \end{align*}
\end{minipage}
\vspace{0.05\baselineskip}
\begin{minipage}[t]{0.15\columnwidth}
  加速度ベクトル
  \begin{align*}
    \odif{^{2}x}{t^{2}}
  \end{align*} 
\end{minipage}

\sukima\noindent\textbullet\; 共変ベクトルの例\par
\qquad  
\begin{minipage}[t]{0.3\columnwidth}
  超平面の法線ベクトル$a_{\kappa}$,
  \begin{align*}
    a_{\kappa} x^{\kappa} = 0.
  \end{align*}
\end{minipage}
\vspace{0.05\baselineskip}
\begin{minipage}[t]{0.3\columnwidth}
  勾配ベクトル,
  \begin{align*}
    \left( \pdif{U}{x^{\kappa}} \right)
  \end{align*}
  {\footnotesize
  \begin{align*}
    \pdif{U}{x^{\kappa^{\prime}}} = \pdif{x^{\kappa}}{x^{\kappa^{\prime}}} \pdif{U}{x^{\kappa}} = A_{\kappa^{\prime}}^{\kappa} \pdif{U}{x^{\kappa}}
  \end{align*}
  による。
  }
\end{minipage}
\vspace{0.05\baselineskip}
\begin{minipage}[t]{0.3\columnwidth}
  力\par
  {\footnotesize
  力が位置エネルギー$U$の勾配であること,力と変位の内積である仕事がスカラー量になることから力は共変ベクトルとみなすのが自然。
  }
\end{minipage}

\expandafter\ifx\csname readornot\endcsname\relax
  \end{document}
\fi