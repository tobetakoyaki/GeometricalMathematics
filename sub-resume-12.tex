\expandafter\ifx\csname readornot\endcsname\relax
  \documentclass[uplatex]{jsarticle}
  \usepackage{octopus}
  \usepackage{url}

  \renewcommand{\proofname}{\textsf{証明}}
  \renewcommand{\postpartname}{章}
  \renewcommand{\thesection}{\thepart.\arabic{section}}
  \renewcommand{\thepart}{\arabic{part}}
  \makeatletter\renewcommand{\theequation}{\thesection.\arabic{equation}}\@addtoreset{equation}{section}\makeatother

  \newcommand{\octopuspart}[1]{\newpage\part{#1}\setcounter{section}{0}\vspace{3\baselineskip}}

  \renewcommand{\restriction}[2]{\left. #1 \right|_{#2}}
  \DeclareMathOperator{\dcup}{\dot{\cup}}
  \DeclareMathOperator{\conv}{conv}
  \DeclareMathOperator{\Image}{Im}
  \DeclareMathOperator{\Kernel}{Ker}
  \begin{document}
\fi

\section{テンソル解析 1}
$V$を$\mathbb{R}$上の$n$次元ベクトル空間とし,基底を$\sets{e_{1}, \dots, e_{n}}$とする。このとき,任意の$V$の元$v$は${\displaystyle v = \sum_{i=1}^{n} v^{i}e_{i}}$と書ける。
つまり,基底の組を決めると,任意の$v \in V$は$n$次元ベクトル$(v^{1} , \dots, v^{n})^{\top}$とみなせる。つまり,$V \simeq \mathbb{R}^{n}$である。

\begin{teigi}
  \nw{座標系} $\defines$ 基底の組$(e_{\kappa} \,|\, \kappa = 1,\dots, n)$
\end{teigi}

$v \in V$に対して座標系$(e_{\kappa} \,|\, \kappa = 1,\dots, n)$による表示を$v^{\kappa}$,座標系$(e_{\kappa^{\prime}} \,|\, \kappa^{\prime} = 1^{\prime},\dots, n^{\prime})$による表示を$v^{\kappa^{\prime}}$とする。

\expandafter\ifx\csname readornot\endcsname\relax
  \end{document}
\fi