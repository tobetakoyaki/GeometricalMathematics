\expandafter\ifx\csname readornot\endcsname\relax
  \documentclass[uplatex]{jsarticle}
  \usepackage{octopus}
  \usepackage{url}
  \usetikzlibrary{calc}

  \renewcommand{\proofname}{\textsf{証明}}
  \renewcommand{\postpartname}{章}
  \renewcommand{\thesection}{\thepart.\arabic{section}}
  \renewcommand{\thepart}{\arabic{part}}
  \makeatletter\renewcommand{\theequation}{\thesection.\arabic{equation}}\@addtoreset{equation}{section}\makeatother

  \newcommand{\octopuspart}[1]{\newpage\part{#1}\setcounter{section}{0}\vspace{3\baselineskip}}

  \renewcommand{\restriction}[2]{\left. #1 \right|_{#2}}
  \DeclareMathOperator{\dcup}{\dot{\cup}}
  \DeclareMathOperator{\conv}{conv}
  \DeclareMathOperator{\Image}{Im}
  \DeclareMathOperator{\Kernel}{Ker}
  
  \begin{document}
\fi

\section{ホモロジー}

\midashi{\large ホモロジー群の導入}\\
\renewcommand{\arraystretch}{1.0}
\begin{tabular}{ll}
  基本群 & ・「2次元の穴」が検出できる \\
  & ・非可換 \\
  & ・計算は一般に難しい \\
  ホモロジー群 & ・「高次元の穴」が検出できる \\
  & ・可換 \\
  & ・計算が比較的容易(計算機にも相性が良い)
\end{tabular}
\renewcommand{\arraystretch}{1.3}

\sukima \midashi{\large 準備:単体}

\begin{teigi}[アフィン独立]
  $v_{0}, v_{1}, \dots, v_{k} \in \mathbb{R}^{n}$が\nw{アフィン独立} $\defines$
  ${\displaystyle \sum_{i=0}^{k} \lambda_{i} v_{i} = 0}$ かつ ${\displaystyle \sum_{i=0}^{k} \lambda_{i} = 0}$ならば,$\lambda_{0} = \lambda_{1} = \dots = \lambda_{k} = 0$
\end{teigi}

アフィン独立であることは,
\begin{center}
  $v_{1} - v_{0}, v_{2} - v_{0}, \dots, v_{k} - v_{0}$が一次独立
\end{center}
であることと同値である。

\begin{rei}
  $\mathbb{R}^{2}$上で,

  \begin{center}
    \begin{tikzpicture}
      \filldraw (0,0) circle [radius=0.05] node[left] {$v_{0}$};
      \filldraw (1.5,-0.5) circle [radius=0.05] node[left] {$v_{1}$};
      \filldraw (0.7,1.2) circle [radius=0.05] node[left] {$v_{2}$};
      \node at (0.7,-2) {アフィン独立};

      \filldraw (5,-1) circle [radius=0.05] node[left] {$v_{0}$};
      \filldraw (5.9,0.2) circle [radius=0.05] node[left] {$v_{1}$};
      \filldraw (6.8,1.4) circle [radius=0.05] node[left] {$v_{2}$};
      \draw[dashed] (4.7,-1.4) -- (7.1,1.8);
      \node at (5.9,-2) {アフィン独立でない};

      \filldraw (11.5,0.3) circle [radius=0.05] node[above right] {$v_{0}$};
      \filldraw (10.5,-0.5) circle [radius=0.05] node[below] {$v_{1}$};
      \filldraw (12.5,-0.5) circle [radius=0.05] node[below] {$v_{2}$};
      \filldraw (11.5,1.3) circle [radius=0.05] node[above] {$v_{3}$};
      \node at (11.5,-2) {アフィン独立でない};
    \end{tikzpicture}
  \end{center}
\end{rei}

\begin{teigi}[$n$次元単体]
  $\Delta^{n}$:\nw{$n$次元単体(simplex)} $\defines$ 
  アフィン独立な$v_{0},v_{1},\dots,v_{n}$が存在して次を満たすような$\Delta^{n}$のこと:
  \begin{equation}
    \Delta^{n} = \conv \sets{v_{0}, v_{1}, \dots, v_{n}} = \sets{\sum_{i=0}^{n} \lambda_{i} v_{i} | \sum_{i=0}^{n} \lambda_{i} = 1, \quad \lambda_{i} \ge 0}
  \end{equation}
\end{teigi}

$n$次元単体を\nw{$n$--単体}ともいう。
また,定義中に登場した$\conv\sets{v_{0}, \dots, v_{n}}$は\nw{凸包}を表しており,
\nw{凸結合}と呼ばれる${\displaystyle \sum_{i=0}^{n} \lambda_{i} = 1}$かつ$\lambda_{i} \ge 0$を満たす$\lambda_{i}$らによって
${\displaystyle \sum_{i=0}^{n} \lambda_{i} v_{i}}$と表される点全体のなす集合のことをいう。

\begin{rei}
  0--単体から3--単体までを以下に示す。
  \begin{center}
    \begin{tikzpicture}
      % 0-単体
      \filldraw (0,0) circle [radius=0.05];
      % 1-単体
      \filldraw (1.9, -0.3) circle [radius=0.05] -- (2.1,0.3) circle [radius=0.05];
      % 2-単体
      \filldraw[fill=gray!50] (3.7, -0.3) -- (4.3, -0.2) -- (4,0.3) -- (3.7, -0.3) -- cycle;
      \fill (3.7, -0.3) circle [radius=0.05] (4.3,-0.2) circle [radius=0.05] (4,0.3) circle [radius=0.05];
      \filldraw[fill=gray!50, opacity=0.5,draw=black, opacity=1] (5.5,-0.3) -- (6.3,-0.5) -- (6.5,0.2) -- (5.9,0.5) -- (5.5,-0.3) -- cycle;
      % 3-単体
      \draw (6.3,-0.5) -- (5.9,0.5);
      \draw[dashed] (5.5,-0.3) -- (6.5,0.2);
      \fill (5.5,-0.3) circle [radius=0.05] (6.3,-0.5) circle [radius=0.05] (6.5,0.2) circle [radius=0.05] (5.9,0.5) circle [radius=0.05];
      \node at (0,-1) {$\Delta^{0}$};
      \node at (2,-1) {$\Delta^{1}$};
      \node at (4,-1) {$\Delta^{2}$};
      \node at (6,-1) {$\Delta^{3}$};
    \end{tikzpicture}
  \end{center}
\end{rei}

以下,頂点$v_{0}, v_{1}, \dots, v_{n}$による単体を$[v_{0}, v_{1}, \dots, v_{n}]$と書く。

また,\nw{面}$[v_{0}, \dots, \hat{v_{i}}, \dots, v_{n}]$を
\begin{equation}
  [v_{0}, \dots, \hat{v_{i}}, \dots, v_{n}] := [v_{0}, \dots, v_{i-1}, v_{i+1}, \dots, v_{n}]
\end{equation}
で定める。

\begin{center}
  \begin{tikzpicture}
    \filldraw[fill=gray!50] (-0.5, -0.5) node[left]{$v_{0}$} -- (0.5, -0.4) node[right]{$v_{1}$} -- (0,0.5) node[above]{$v_{2}$} -- (-0.5, -0.5) -- cycle;
    \fill (-0.5, -0.5) circle [radius=0.05] (0.5,-0.4) circle [radius=0.05] (0,0.5) circle [radius=0.05];
    \draw[<-] node at (0,-1) {$[v_{0}, v_{1}, v_{2}]$} (0.4,0.1) -- (0.8,0.4) node[right] {面$[v_{1},v_{2}]$};
    \filldraw[fill=gray!50, opacity=0.5,draw=black, opacity=1] (7.3,-0.8) node[below]{$v_{0}$} -- (7.7,0) node[right]{$v_{1}$} -- (6.9,0.5) node[above]{$v_{3}$} -- (7.3,-0.8) -- cycle;
    \draw (6.3,-0.4) node[left]{$v_{2}$} -- (7.3,-0.8) (6.3,-0.4) -- (6.9,0.5);
    \draw[dashed] (6.3,-0.4) -- (7.7,0);
    \fill (6.3,-0.4) circle [radius=0.05] (7.3,-0.8) circle [radius=0.05] (7.7,0) circle [radius=0.05] (6.9,0.5) circle [radius=0.05];
    \draw[<-] (7.4,0) -- (8,0.5) node[right]{面$[v_{0}, v_{1}, v_{3}]$};
  \end{tikzpicture}
\end{center}

\sukima \midashi{\large 向き付けられた単体}

\nw{向き付けられた単体}は,頂点の順番が指定されている単体のこと。
$[v_{0}, v_{1}, \dots, v_{n}]$は頂点に$v_{0} \to v_{1} \to \dots \to v_{n}$の順番が指定されているものとする。

\newcommand{\midwayarrow}[3][black]{
	\draw[-latex,#1] #2 -- ($#2!0.6!#3$);
	\draw[#1] ($#2!0.41!#3$) -- #3;
}

\newcommand{\point}[3]{
	#1 node[#2]{#3} circle [radius=0.05]
}

\begin{center}
	\begin{tikzpicture}
		\fill \point{(0,0)}{left}{$v_{0}$} \point{(0,2.5)}{left}{$v_{1}$};
		\midwayarrow{(0,0)}{(0,2.5)};
	\end{tikzpicture}
	\qquad
	\begin{tikzpicture}
		\fill \point{(0,0)}{below left}{$v_{0}$} \point{(2.5,0)}{below right}{$v_{1}$} \point{(1.25,2)}{right}{$v_{2}$};
		\midwayarrow{(0,0)}{(2.5,0)} \midwayarrow{(2.5,0)}{(1.25,2)} \midwayarrow{(0,0)}{(1.25,2)}
		\midwayarrow[red]{(0.4,0.2)}{(2.1,0.2)} \midwayarrow[red]{(2.1,0.2)}{(1.3,1.5)}
	\end{tikzpicture}
	\qquad
	\begin{tikzpicture}
		\coordinate (v0) at (2,-0.5) coordinate (v1) at (2.5,1) coordinate (v2) at (0,0) coordinate (v3) at (1,2);
		\fill \point{(v2)}{left}{$v_{2}$} \point{(v0)}{below}{$v_{0}$} \point{(v1)}{right}{$v_{1}$} \point{(v3)}{right}{$v_{3}$};
		\midwayarrow{(v0)}{(v1)} \midwayarrow{(v2)}{(v0)} \midwayarrow{(v0)}{(v3)} \midwayarrow{(v3)}{(v1)} \midwayarrow[dashed]{(v2)}{(v1)} \midwayarrow{(v2)}{(v3)}
		\midwayarrow[red]{(0.2,-0.2)}{(1.8,-0.6)} \midwayarrow[red]{(1.8,-0.6)}{(0.8,1.9)} \midwayarrow[red]{(0.8,1.9)}{(2.3,0.9)}
	\end{tikzpicture}
\end{center}

このとき,面にも向きが誘導される。すなわち,面$[v_{0}, \dots, \hat{v_{i}}, \dots, v_{n}]$には
頂点に$v_{0} \to v_{1} \to \dots \to v_{i-1} \to v_{i+1} \to \dots \to v_{n}$の順番が指定される。

するとさらに,向き付けられた$n$--単体どうしの間に次のような自然な全単射が定まる。
\begin{equation}
  \mapdef{[v_{0}, \dots, v_{n}]}{[u_{0}, \dots, u_{n}]}{\displaystyle \sum_{i=0}^{n} \lambda_{i} v_{i}}{\displaystyle \sum_{i=0}^{n} \lambda_{i} u_{i}}
\end{equation}

\begin{teigi}[境界,内部]
  単体の\nw{境界} $\partial \Delta^{n}$と\nw{内部}$\mathring{\Delta}^{n}$をそれぞれ次で定める。
  \begin{align}
    & \partial \Delta^{n} := \bigcup_{i=0}^{n} [v_{0}, \dots, \hat{v_{i}}, \dots, v_{n}], \\
    & \mathring{\Delta}^{n} := \Delta^{n} \setminus \partial \Delta^{n}
  \end{align}
\end{teigi}

\sukima \midashi{\large 特異ホモロジー}

$X$:位相空間

\begin{teigi}[特異$n$-単体]
  $\sigma$:\nw{特異$n$-単体(singular n-simplex)} $\defines$ $\sigma$:向き付けられた$n$-単体から$X$への連続写像
\end{teigi}

ものすごい特異的なものも許される。
例えば,$X$を一点集合にとったりKleinの壺にとったりしてもよい。

\begin{teigi}
  特異$n$-単体$\sigma \colon [v_{0}, \dots, v_{n}] \longrightarrow X$の\nw{面}を
  $\restriction{\sigma}{[v_{0}, \dots,\hat{v_{i}}, \dots, v_{n}]}$とする。
  これは特異$n-1$-単体である。
\end{teigi}

\begin{teigi}[特異$n$-チェイン]
  \nw{特異$n$-チェイン} $\defines$ 特異$n$-単体たちの有限な形式的整数結合
\end{teigi}

すなわち,有限個の$\alpha$以外では$n_{\alpha}=0$であるような$\sets{n_{\alpha} \in \mathbb{Z}}_{\alpha}$を用いて
\begin{equation}
  \sum_{\alpha} n_{\alpha} \sigma_{\alpha}
\end{equation}
と表されるものを特異$n$-チェインという。

特異$n$-チェイン全体のなす集合を$C_{n}$とする。
\begin{equation}
  C_{n} := \sets{\sigma | \sigma\text{:特異$n$-チェイン}}
\end{equation}
ここに,
\begin{equation}
    \sum_{\alpha} n_{\alpha} \sigma_{\alpha} + \sum_{\alpha} n'_{\alpha} \sigma_{\alpha} := \sum_{\alpha} (n_{\alpha} + n'_{\alpha}) \sigma_{\alpha}
\end{equation}
で演算を定めると,
単位元を${\displaystyle \sum_{\alpha} 0 \cdot \sigma := 0}$として$C_{n}$は可換群になる。

\begin{teigi}[境界作用素]
  \nw{境界作用素}$\partial_{n} \colon C_{n} \longrightarrow C_{n-1}$を次にようにして定める。
  \begin{itemize}
    \vspace{-0.5\baselineskip}
    \item $n$-単体$\sigma \colon [v_{0},\dots,v_{n}] \longrightarrow X$に対し,
    \begin{equation}
      \partial_{n} \sigma := \sum_{i=0}^{n} (-1)^{i} \restriction{\sigma}{[v_{0}, \dots, \hat{v_{i}}, \dots, v_{n}]}
    \end{equation}
    \item $n$-チェインに対し,
    \begin{equation}
      \partial_{n} \left( \sum_{\alpha} n_{\alpha} \sigma_{\alpha} \right) := \sum_{\alpha} n_{\alpha} \partial_{n} \sigma_{\alpha}
    \end{equation}
  \end{itemize}
\end{teigi}

イメージは以下のよう。
\begin{center}
	\begin{tikzpicture}
		\filldraw[fill=gray!40,draw=black] (-0.5,0) -- (2,0) -- (0.75,2) -- cycle;
		\fill \point{(-0.5,0)}{below left}{$v_{0}$} \point{(2,0)}{below right}{$v_{1}$} \point{(0.75,2)}{right}{$v_{2}$};
		\draw[->] (2.5,1) -- node[midway,above]{$\partial_{2}$} (3.5,1);
		\node at (3.875,1) {$+$};
		\fill \point{(7,0)}{below right}{$v_{1}$} \point{(5.75,2)}{right}{$v_{2}$};
		\draw (7,0) -- (5.75,2);
		\node at (7.625,1) {$-$};
		\fill \point{(8.25,0)}{below left}{$v_{0}$} \point{(9.5,2)}{right}{$v_{2}$};
		\draw (8.25,0) -- (9.5,2);
		\node at (11.375,1) {$+$};
		\fill \point{(14.5,0)}{below right}{$v_{1}$} \point{(13.25,2)}{right}{$v_{2}$};
		\draw (14.5,0) -- (13.25,2);
	\end{tikzpicture}
\end{center}

\begin{prop}
  $\partial_{n}$は$C_{n}$から$C_{n-1}$への準同型写像
\end{prop}

\begin{hodai}
  $\partial_{n-1} \circ \partial_{n} = 0$
\end{hodai}

\begin{proof}
  $\sigma_{\alpha} \colon [v_{0},v_{1},\cdots, v_{n}] \longrightarrow X$に対し,
  \begin{equation}
    \partial_{n} \sigma_{\alpha} = \sum_{i=0}^{n} (-1)^{i} \restriction{\sigma_{\alpha}}{[v_{0},\dots,\hat{v_{i}},\dots,v_{n}]}
  \end{equation}
  であり,
  \begin{align}
    (\partial_{n-1}\circ \partial_{n})\sigma_{\alpha} &=
    \sum_{i=0}^{n} (-1)^{i} \partial_{n-1} \left( \restriction{\sigma_{\alpha}}{[v_{0},\dots,\hat{v_{i}},\dots,v_{n}]} \right) \\
    &= \sum_{i=0}^{n} (-1)^{i}
    \left\{ \sum_{j<i} (-1)^{j} \restriction{\sigma_{\alpha}}{[v_{0},\dots,\hat{v_{j}},\dots,\hat{v_{i}},\dots,v_{n}]} + \sum_{j<i} (-1)^{j-1} \restriction{\sigma_{\alpha}}{[v_{0},\dots,\hat{v_{i}},\dots,\hat{v_{j}},\dots,v_{n}]} \right\} \\
    &= \sum_{0 \le j < i \le n} (-1)^{i+j} \restriction{\sigma_{\alpha}}{[v_{0},\dots,\hat{v_{j}},\dots,\hat{v_{i}},\dots,v_{n}]} - \sum_{0 \le i < j \le n} (-1)^{i+j} \restriction{\sigma_{\alpha}}{[v_{0},\dots,\hat{v_{i}},\dots,\hat{v_{j}},\dots,v_{n}]} \\
    &= 0
  \end{align}
  である。結局,$\partial_{n-1} \circ \partial_{n} = 0$である。
\end{proof}

これにより,
\begin{center}
  $\dots \to C_{n+1} \xrightarrow{\partial_{n+1}} C_{n} \xrightarrow{\partial_{n}} C_{n-1} \xrightarrow{\partial_{n-1}} \dots \xrightarrow{\partial_{3}} C_{2} \xrightarrow{\partial_{2}} C_{1} \xrightarrow{\partial_{1}} C_{0} \xrightarrow{\partial_{0}} 0$,

  $\partial_{n-1} \circ \partial_{n} = 0$($n=1,2,\dots$)
\end{center}
なる列が得られた。これを\nw{チェイン複体}という。

\begin{prop}
  $\Image \partial_{n} \subseteq \Kernel \partial_{n-1}$
\end{prop}

\begin{proof}
  $\partial_{n-1} \circ \partial_{n} = 0$より従う。
\end{proof}

\begin{teigi}[サイクル]
  $\Kernel \partial_{n}$の元を\nw{サイクル}という。
\end{teigi}

\begin{rei}[サイクルのイメージ] \\
  \begin{tikzpicture}
    \fill \point{(180:2)}{left}{$v_{0}$} \point{(120:2)}{above}{$v_{1}$} \point{(60:2)}{above}{$v_{2}$} \point{(0:2)}{right}{$v_{3}$} \point{(-60:2)}{}{} \point{(-120:2)}{}{};
    \midwayarrow{(180:2)}{(120:2)} \midwayarrow{(120:2)}{(60:2)} \midwayarrow{(60:2)}{(0:2)} \midwayarrow{(0:2)}{(-60:2)} \midwayarrow{(-60:2)}{(-120:2)} \midwayarrow{(-120:2)}{(-180:2)}
    \node at (3.5,0) [right]{$= [v_{0},v_{1}] + [v_{1},v_{2}] + \dots + [v_{k},v_{0}]$};
  \end{tikzpicture}

  $\xrightarrow{\partial_{1}} [v_{1}] - [v_{0}] + [v_{2}] - [v_{1}] + \dots + [v_{0}] - [v_{k}] = 0$
  
  であり,これはサイクル。
  %モウヒトツレイガアル
\end{rei}

\begin{teigi}[ホモローグ]
  2つのサイクル$z, z' \in \Kernel \partial_{n}$が\nw{ホモローグ(homologue)} $\defines$ $z - z' \in \Image \partial_{n+1}$

  また,このことを$z \sim z'$と書く。
\end{teigi}

\begin{hodai}
  ホモローグの関係$\sim$は同値関係
\end{hodai}

\begin{proof}
  推移律のみ示す。$z \sim z'$,$z' \sim z''$とする。このとき,$z - z'' = (z - z') + (z' - z'') \in \Image \partial_{n+1}$であるから,$z \sim z''$である。
\end{proof}

$\sim$による同値類を\nw{ホモロジー類}という。

\begin{rei}[ホモローグのイメージ]
  %よくわからない図がある
\end{rei}

\begin{hodai}
  $u \sim v$ $\Longleftrightarrow$ $u$と$v$は同じ弧状連結成分に属する
\end{hodai}

\begin{teigi}[ホモロジー群]
  \begin{align}
    H_{n}(X) &:= \Kernel \partial_{n} / \Image \partial_{n+1} \\
    & = \sets{[z] | z \in \Kernel \partial_{n}}
  \end{align}
  とすると,このホモロジー類の全体のなす集合は次の演算で群をなす。:
  \begin{equation}
    [z] + [z'] := [z + z']
  \end{equation}
  (この演算がwell-definedであることの証明は割愛する)この群を\nw{ホモロジー群}という。
  単位元は$\Image \partial_{n+1}$の元によるホモロジー類。
\end{teigi}

\begin{center}
  \begin{tikzpicture}
    \draw (-4,0) circle [x radius=1, y radius=2];
    \draw (0,0) circle [x radius=1, y radius=2];
    \draw (0,0) circle [x radius=0.5, y radius=1];
    \draw (0,0) circle [x radius=0.25, y radius=0.5];
    \draw (4,0) circle [x radius=1, y radius=2];
    \draw[dashed] (-4,2) --node[midway,above]{$\partial_{n+1}$} (0,0.5);
    \draw[dashed] (-4,-2) -- (0,-0.5);
    \draw[dashed] (0,1) --node[midway,above]{$\partial_{n}$} (4,0);
    \draw[dashed] (0,-1) -- (4,0);
    \filldraw \point{(4,0)}{right}{0};
    \draw[<-] (0,0) to[bend left] (-2,0.3) node[above]{$\Image \partial_{n+1}$};
    \draw[<-] (0.25,-0.5) to[bend left] (2,-1) node[below]{$\Kernel \partial_{n}$};
    \node at (-4,-2.5) {$C_{n+1}$} node at (0,-2.5) {$C_{n}$} node at (4,-2.5) {$C_{n-1}$};
  \end{tikzpicture}
\end{center}

\begin{prop}
  $r$($< \infty$)を$X$の弧状連結成分の数とする。このとき,$H_{0} \simeq \mathbb{Z}^{r}$である。
\end{prop}

\begin{proof}
  各弧状連結成分から1つずつ頂点$v_{1},v_{2},\dots,v_{r}$をとると,
  $H_{0}$の元は${\displaystyle \sum_{i=1}^{n} n_{i} [v_{i}]}$と一意的に記述される。
  このことより従う。
\end{proof}

\midashi{注意.} $C_{1} \xrightarrow{\partial_{1}} C_{0} \xrightarrow{\partial_{0}} 0$であるから$\Kernel \partial_{0} = C_{0}$である。

\begin{prop}
  $X$:頂点数$n$,枝数$m$の連結なグラフとする。このとき,$H_{1} \simeq \mathbb{Z}^{m-n}$である。
\end{prop}

\sukima \midashi{\large 位相不変性・ホモトピー不変性}

$X,Y$:位相空間,$\varphi \colon X \longrightarrow Y$:連続写像,
$\sigma \colon [v_{0}, v_{1}, \dots, v_{n}] \longrightarrow X$:$X$上の$n$-単体,とする。
このとき,$\varphi \circ \sigma \colon [v_{0}, v_{1}, \dots, v_{n}] \longrightarrow Y$は$Y$上の$n$-単体になる。
これにより,$\varphi \colon C_{n} (X) \longrightarrow C_{n}(Y)$が誘導される。:
\begin{equation}
  \varphi \left( \sum_{\alpha} n_{\alpha} \sigma_{\alpha} \right) := \sum_{\alpha} n_{\alpha}\varphi (\sigma_{\alpha})
\end{equation}

よって,以下のような図式がかける。

\begin{center}
  \begin{tikzpicture}
    \draw[->] (0,0.5) --node[midway,left]{$\varphi$} (0,-0.5);
    \draw[->] (3,0.5) --node[midway,left]{$\varphi$} (3,-0.5);
    \draw[->] (6,0.5) --node[midway,left]{$\varphi$} (6,-0.5);

    \draw[->] (-2,1) --node[midway,above]{$\partial_{n+2}$} (-1,1);
    \draw[->] (1,1) --node[midway,above]{$\partial_{n+1}$} (2,1);
    \draw[->] (4,1) --node[midway,above]{$\partial_{n}$} (5,1);
    \draw[->] (7,1) --node[midway,above]{$\partial_{n-1}$} (8,1);

    \draw[->] (-2,-1) --node[midway,below]{$\partial_{n+2}$} (-1,-1);
    \draw[->] (1,-1) --node[midway,below]{$\partial_{n+1}$} (2,-1);
    \draw[->] (4,-1) --node[midway,below]{$\partial_{n}$} (5,-1);
    \draw[->] (7,-1) --node[midway,below]{$\partial_{n-1}$} (8,-1);

    \node at (-3,-1) {$\cdots$} node at (0,-1) {$C_{n+1}(Y)$} node at (3,-1) {$C_{n}(Y)$} node at (6,-1) {$C_{n-1}(Y)$} node at (9,-1) {$\cdots$};
    \node at (-3,1) {$\cdots$} node at (0,1) {$C_{n+1}(X)$} node at (3,1) {$C_{n}(X)$} node at (6,1) {$C_{n-1}(X)$} node at (9,1) {$\cdots$};
  \end{tikzpicture}
\end{center}

\begin{hodai}
  \label{homology1.commutative}
  $\varphi \circ \partial_{n} = \partial_{n} \circ \varphi$
\end{hodai}

\begin{proof}
  次の式から従う。
  \begin{equation}
    (\varphi \circ \partial_{n}) \sigma = \varphi \left( \sum_{i=0}^{n} (-1)^{i} \restriction{\sigma}{[v_{0}, \dots, \hat{v_{i}}, \dots, v_{n}]} \right) = \sum_{i=0}^{n} (-1)^{i} \varphi \circ \restriction{\sigma}{[v_{0}, \dots, \hat{v_{i}}, \dots, v_{n}]} = \partial_{n} (\varphi \circ \sigma)
  \end{equation}
\end{proof}

このことから,先の図式が可換になることが従う。

\begin{hodai}
  \label{homology1.cycle}
  $z$:$X$上$n$-サイクル $\Longrightarrow$ $\varphi(z)$:$Y$上$n$-サイクル
\end{hodai}

\begin{proof}
  先の\rref{補題}{homology1.commutative}から$\partial_{n} \varphi(z) = \varphi (\partial_{n} (z)) = 0$であって従う。
\end{proof}

\begin{teigi}
  $\varphi \colon X \longrightarrow Y$:連続とする。写像$\varphi_{\#}$を
  \begin{equation}
    \varphi_{\#} \colon \mapdef{H_{n}(X)}{H_{n}(Y)}{\left[ z \right]}{\left[ \varphi(z) \right]}
  \end{equation}
  で定義する。
\end{teigi}

\begin{hodai}
  $\varphi_{\#}$はwell-definedであり,準同型である。
\end{hodai}

\begin{proof}
  \midashi{[well-definedness]}

  先の\rref{補題}{homology1.cycle}より$\varphi(z)$は$Y$上$n$-サイクルである。よって,$[\varphi(z)] \in H_{n}(Y)$である。いま,
  \renewcommand{\arraystretch}{1}
  \begin{tabular}{l@{\,}l}
    $z \sim z'$ & $\Longleftrightarrow$ $\exists c, \quad z - z' = \partial_{n+1} c$ \\
    & $\longrightarrow$ $\varphi(z) - \varphi(z') = \varphi (\partial_{n+1} c) = \partial_{n+1} (\varphi(c)) \in \Image \partial_{n+1}$ \\
    & $\longrightarrow$ $[\varphi(z)] \sim [\varphi(z')]$
  \end{tabular}
  \renewcommand{\arraystretch}{1.3}
  のようになる。

  \sukima \midashi{[準同型性]}

  これは定義の仕方から明らか。
\end{proof}

\begin{hodai}
  {\bf (1)} $X \xrightarrow{\varphi} Y \xrightarrow{\varphi'} Z$のとき,
  $\varphi_{\#}' \circ \varphi_{\#} = (\varphi' \circ \varphi)_{\#}$

  {\bf (2)} $\mathrm{id}_{\#} = \mathrm{id}$
\end{hodai}

\begin{proof}
  {\bf (1)} $\left( \varphi' \circ \varphi \right)_{\#}([z]) = [(\varphi' \circ \varphi)(z)] = \varphi'_{\#} \left( [\varphi (z)] \right) = (\varphi'_{\#} \circ \varphi_{\#}) ([z])$より従う。

  {\bf (2)} は明らか。
\end{proof}

\begin{teiri}
  $X,Y$:同相 $\Longrightarrow$ $H_{n}(X) \simeq H_{n}(Y)$
\end{teiri}

\begin{proof}
  $\varphi \colon X \longrightarrow Y$:同相写像とする。
  このとき,$\varphi^{-1}_{\#} \circ \varphi_{\#} = \mathrm{id}$,
  $\varphi_{\#} \circ \varphi^{-1}_{\#} = \mathrm{id}$より従う。
\end{proof}

実はさらに強く,次のことが言える。

\begin{teiri}
  $X,Y$:ホモトピー同値 $\Longrightarrow$ $H_{n}(X) \simeq H_{n}(Y)$
\end{teiri}

\begin{proof}[{\bf 証明の気分}]
  $f \colon X \longrightarrow Y$,$g \colon Y \longrightarrow X$をホモトピー同値写像とする。
  つまり,$g \circ f \simeq \mathrm{id}_{X}$,$f \circ g \simeq \mathrm{id}_{Y}$とする。
  このときに,$(g \circ f)_{\#} \simeq \mathrm{id}_{H_{n}(X)}$を示せばよい。
  
  ここで,$f_{\#}$と$g_{\#}$は全単射であるから,$g_{\#} \circ f_{\#}$は全単射であり,$f_{\#} \circ g_{\#}$は全単射であることになる。

  $g \circ f$と$\mathrm{id}_{X}$の間にはホモトピーがある。サイクル$z$はそのホモトピーによって連続的にサイクル$g \circ f (z)$になる。

  サイクルの軌跡が作る「筒」はうまく単体らによって表すことができる。
  したがって,$z \sim (g \circ f)(z)$である。
  よって,$(g \circ f)_{\#} ([z]) = [z]$である。

  一般の場合は,サイクル${\displaystyle z = \sum_{\alpha} n_{\alpha} \sigma_{\alpha}}$の各単体$\sigma_{\alpha}$に対して,プリズムを一次元上の単体に分割する。
\end{proof}

\midashi{演習.} これをきちんと証明せよ。

\expandafter\ifx\csname readornot\endcsname\relax
  \end{document}
\fi