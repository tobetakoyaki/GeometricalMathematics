\expandafter\ifx\csname readornot\endcsname\relax
  \documentclass[uplatex]{jsarticle}
  \usepackage{octopus}
  \usepackage{url}

  \renewcommand{\proofname}{\textsf{証明}}
  \renewcommand{\postpartname}{章}
  \renewcommand{\thesection}{\thepart.\arabic{section}}
  \renewcommand{\thepart}{\arabic{part}}
  \makeatletter\renewcommand{\theequation}{\thesection.\arabic{equation}}\@addtoreset{equation}{section}\makeatother

  \newcommand{\octopuspart}[1]{\newpage\part{#1}\setcounter{section}{0}\vspace{3\baselineskip}}

  \DeclareMathOperator{\dcup}{\dot{\cup}}
  \DeclareMathOperator{\conv}{conv}
  \begin{document}
\fi

\section{ホモロジー}

\midashi{\large ホモロジー群の導入}\\
\renewcommand{\arraystretch}{1.0}
\begin{tabular}{ll}
  基本群 & ・「2次元の穴」が検出できる \\
  & ・非可換 \\
  & ・計算は一般に難しい \\
  ホモロジー群 & ・「高次元の穴」が検出できる \\
  & ・可換 \\
  & ・計算が比較的容易(計算機にも相性が良い)
\end{tabular}
\renewcommand{\arraystretch}{1.3}

\sukima \midashi{\large 準備:単体}

\begin{teigi}[アフィン独立]
  $v_{0}, v_{1}, \dots, v_{k} \in \mathbb{R}^{n}$が\nw{アフィン独立} $\defines$
  ${\displaystyle \sum_{i=0}^{k} \lambda_{i} v_{i} = 0}$ かつ ${\displaystyle \sum_{i=0}^{k} \lambda_{i} = 0}$ならば,$\lambda_{0} = \lambda_{1} = \dots = \lambda_{k} = 0$
\end{teigi}

アフィン独立であることは,
\begin{center}
  $v_{1} - v_{0}, v_{2} - v_{0}, \dots, v_{k} - v_{0}$が一次独立
\end{center}
であることと同値である。

\begin{rei}
  $\mathbb{R}^{2}$上で,

  \begin{center}
    \begin{tikzpicture}
      \filldraw (0,0) circle [radius=0.05] node[left] {$v_{0}$};
      \filldraw (1.5,-0.5) circle [radius=0.05] node[left] {$v_{1}$};
      \filldraw (0.7,1.2) circle [radius=0.05] node[left] {$v_{2}$};
      \node at (0.7,-2) {アフィン独立};

      \filldraw (5,-1) circle [radius=0.05] node[left] {$v_{0}$};
      \filldraw (5.9,0.2) circle [radius=0.05] node[left] {$v_{1}$};
      \filldraw (6.8,1.4) circle [radius=0.05] node[left] {$v_{2}$};
      \draw[dashed] (4.7,-1.4) -- (7.1,1.8);
      \node at (5.9,-2) {アフィン独立でない};

      \filldraw (11.5,0.3) circle [radius=0.05] node[above right] {$v_{0}$};
      \filldraw (10.5,-0.5) circle [radius=0.05] node[below] {$v_{1}$};
      \filldraw (12.5,-0.5) circle [radius=0.05] node[below] {$v_{2}$};
      \filldraw (11.5,1.3) circle [radius=0.05] node[above] {$v_{3}$};
      \node at (11.5,-2) {アフィン独立でない};
    \end{tikzpicture}
  \end{center}
\end{rei}

\begin{teigi}[$n$次元単体]
  $\Delta^{n}$:\nw{$n$次元単体(simplex)} $\defines$ 
  アフィン独立な$v_{0},v_{1},\dots,v_{n}$が存在して次を満たすような$\Delta^{n}$のこと:
  \begin{equation}
    \Delta^{n} = \conv \sets{v_{0}, v_{1}, \dots, v_{n}} = \sets{\sum_{i=0}^{n} \lambda_{i} v_{i} | \sum_{i=0}^{n} \lambda_{i} = 1, \quad \lambda_{i} \ge 0}
  \end{equation}
\end{teigi}

$n$次元単体を\nw{$n$--単体}ともいう。
また,定義中に登場した$\conv\sets{v_{0}, \dots, v_{n}}$は\nw{凸包}を表しており,
\nw{凸結合}と呼ばれる${\displaystyle \sum_{i=0}^{n} \lambda_{i} = 1}$かつ$\lambda_{i} \ge 0$を満たす$\lambda_{i}$らによって
${\displaystyle \sum_{i=0}^{n} \lambda_{i} v_{i}}$と表される点全体のなす集合のことをいう。

\begin{rei}
  0--単体から3--単体までを以下に示す。
  \begin{center}
    \begin{tikzpicture}
      % 0-単体
      \filldraw (0,0) circle [radius=0.05];
      % 1-単体
      \filldraw (1.9, -0.3) circle [radius=0.05] -- (2.1,0.3) circle [radius=0.05];
      % 2-単体
      \filldraw[fill=gray!50] (3.7, -0.3) -- (4.3, -0.2) -- (4,0.3) -- (3.7, -0.3) -- cycle;
      \fill (3.7, -0.3) circle [radius=0.05] (4.3,-0.2) circle [radius=0.05] (4,0.3) circle [radius=0.05];
      \filldraw[fill=gray!50, opacity=0.5,draw=black, opacity=1] (5.5,-0.3) -- (6.3,-0.5) -- (6.5,0.2) -- (5.9,0.5) -- (5.5,-0.3) -- cycle;
      % 3-単体
      \draw (6.3,-0.5) -- (5.9,0.5);
      \draw[dashed] (5.5,-0.3) -- (6.5,0.2);
      \fill (5.5,-0.3) circle [radius=0.05] (6.3,-0.5) circle [radius=0.05] (6.5,0.2) circle [radius=0.05] (5.9,0.5) circle [radius=0.05];
      \node at (0,-1) {$\Delta^{0}$};
      \node at (2,-1) {$\Delta^{1}$};
      \node at (4,-1) {$\Delta^{2}$};
      \node at (6,-1) {$\Delta^{3}$};
    \end{tikzpicture}
  \end{center}
\end{rei}

以下,頂点$v_{0}, v_{1}, \dots, v_{n}$による単体を$[v_{0}, v_{1}, \dots, v_{n}]$と書く。

また,\nw{面} $[v_{0}, \dots, \hat{v_{i}}, \dots, v_{n}]$を
\begin{equation}
  [v_{0}, \dots, \hat{v_{i}}, \dots, v_{n}] := [v_{0}, \dots, v_{i-1}, v_{i+1}, \dots, v_{n}]
\end{equation}
で定める。

\begin{center}
  \begin{tikzpicture}
    \filldraw[fill=gray!50] (-0.5, -0.5) node[left]{$v_{0}$} -- (0.5, -0.4) node[right]{$v_{1}$} -- (0,0.5) node[above]{$v_{2}$} -- (-0.5, -0.5) -- cycle;
    \fill (-0.5, -0.5) circle [radius=0.05] (0.5,-0.4) circle [radius=0.05] (0,0.5) circle [radius=0.05];
    \draw[<-] node at (0,-1) {$[v_{0}, v_{1}, v_{2}]$} (0.4,0.1) -- (0.8,0.4) node[right] {面$[v_{1},v_{2}]$};
    \filldraw[fill=gray!50, opacity=0.5,draw=black, opacity=1] (7.3,-0.8) node[below]{$v_{0}$} -- (7.7,0) node[right]{$v_{1}$} -- (6.9,0.5) node[above]{$v_{3}$} -- (7.3,-0.8) -- cycle;
    \draw (6.3,-0.4) node[left]{$v_{2}$} -- (7.3,-0.8) (6.3,-0.4) -- (6.9,0.5);
    \draw[dashed] (6.3,-0.4) -- (7.7,0);
    \fill (6.3,-0.4) circle [radius=0.05] (7.3,-0.8) circle [radius=0.05] (7.7,0) circle [radius=0.05] (6.9,0.5) circle [radius=0.05];
    \draw[<-] (7.4,0) -- (8,0.5) node[right]{面$[v_{0}, v_{1}, v_{3}]$};
  \end{tikzpicture}
\end{center}

\expandafter\ifx\csname readornot\endcsname\relax
  \end{document}
\fi